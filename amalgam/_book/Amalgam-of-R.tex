% Options for packages loaded elsewhere
\PassOptionsToPackage{unicode}{hyperref}
\PassOptionsToPackage{hyphens}{url}
\PassOptionsToPackage{dvipsnames,svgnames,x11names}{xcolor}
%
\documentclass[
  12pt,
]{book}
\usepackage{amsmath,amssymb}
\usepackage{iftex}
\ifPDFTeX
  \usepackage[T1]{fontenc}
  \usepackage[utf8]{inputenc}
  \usepackage{textcomp} % provide euro and other symbols
\else % if luatex or xetex
  \usepackage{unicode-math} % this also loads fontspec
  \defaultfontfeatures{Scale=MatchLowercase}
  \defaultfontfeatures[\rmfamily]{Ligatures=TeX,Scale=1}
\fi
\usepackage{lmodern}
\ifPDFTeX\else
  % xetex/luatex font selection
\fi
% Use upquote if available, for straight quotes in verbatim environments
\IfFileExists{upquote.sty}{\usepackage{upquote}}{}
\IfFileExists{microtype.sty}{% use microtype if available
  \usepackage[]{microtype}
  \UseMicrotypeSet[protrusion]{basicmath} % disable protrusion for tt fonts
}{}
\makeatletter
\@ifundefined{KOMAClassName}{% if non-KOMA class
  \IfFileExists{parskip.sty}{%
    \usepackage{parskip}
  }{% else
    \setlength{\parindent}{0pt}
    \setlength{\parskip}{6pt plus 2pt minus 1pt}}
}{% if KOMA class
  \KOMAoptions{parskip=half}}
\makeatother
\usepackage{xcolor}
\usepackage{color}
\usepackage{fancyvrb}
\newcommand{\VerbBar}{|}
\newcommand{\VERB}{\Verb[commandchars=\\\{\}]}
\DefineVerbatimEnvironment{Highlighting}{Verbatim}{commandchars=\\\{\}}
% Add ',fontsize=\small' for more characters per line
\usepackage{framed}
\definecolor{shadecolor}{RGB}{248,248,248}
\newenvironment{Shaded}{\begin{snugshade}}{\end{snugshade}}
\newcommand{\AlertTok}[1]{\textcolor[rgb]{0.94,0.16,0.16}{#1}}
\newcommand{\AnnotationTok}[1]{\textcolor[rgb]{0.56,0.35,0.01}{\textbf{\textit{#1}}}}
\newcommand{\AttributeTok}[1]{\textcolor[rgb]{0.13,0.29,0.53}{#1}}
\newcommand{\BaseNTok}[1]{\textcolor[rgb]{0.00,0.00,0.81}{#1}}
\newcommand{\BuiltInTok}[1]{#1}
\newcommand{\CharTok}[1]{\textcolor[rgb]{0.31,0.60,0.02}{#1}}
\newcommand{\CommentTok}[1]{\textcolor[rgb]{0.56,0.35,0.01}{\textit{#1}}}
\newcommand{\CommentVarTok}[1]{\textcolor[rgb]{0.56,0.35,0.01}{\textbf{\textit{#1}}}}
\newcommand{\ConstantTok}[1]{\textcolor[rgb]{0.56,0.35,0.01}{#1}}
\newcommand{\ControlFlowTok}[1]{\textcolor[rgb]{0.13,0.29,0.53}{\textbf{#1}}}
\newcommand{\DataTypeTok}[1]{\textcolor[rgb]{0.13,0.29,0.53}{#1}}
\newcommand{\DecValTok}[1]{\textcolor[rgb]{0.00,0.00,0.81}{#1}}
\newcommand{\DocumentationTok}[1]{\textcolor[rgb]{0.56,0.35,0.01}{\textbf{\textit{#1}}}}
\newcommand{\ErrorTok}[1]{\textcolor[rgb]{0.64,0.00,0.00}{\textbf{#1}}}
\newcommand{\ExtensionTok}[1]{#1}
\newcommand{\FloatTok}[1]{\textcolor[rgb]{0.00,0.00,0.81}{#1}}
\newcommand{\FunctionTok}[1]{\textcolor[rgb]{0.13,0.29,0.53}{\textbf{#1}}}
\newcommand{\ImportTok}[1]{#1}
\newcommand{\InformationTok}[1]{\textcolor[rgb]{0.56,0.35,0.01}{\textbf{\textit{#1}}}}
\newcommand{\KeywordTok}[1]{\textcolor[rgb]{0.13,0.29,0.53}{\textbf{#1}}}
\newcommand{\NormalTok}[1]{#1}
\newcommand{\OperatorTok}[1]{\textcolor[rgb]{0.81,0.36,0.00}{\textbf{#1}}}
\newcommand{\OtherTok}[1]{\textcolor[rgb]{0.56,0.35,0.01}{#1}}
\newcommand{\PreprocessorTok}[1]{\textcolor[rgb]{0.56,0.35,0.01}{\textit{#1}}}
\newcommand{\RegionMarkerTok}[1]{#1}
\newcommand{\SpecialCharTok}[1]{\textcolor[rgb]{0.81,0.36,0.00}{\textbf{#1}}}
\newcommand{\SpecialStringTok}[1]{\textcolor[rgb]{0.31,0.60,0.02}{#1}}
\newcommand{\StringTok}[1]{\textcolor[rgb]{0.31,0.60,0.02}{#1}}
\newcommand{\VariableTok}[1]{\textcolor[rgb]{0.00,0.00,0.00}{#1}}
\newcommand{\VerbatimStringTok}[1]{\textcolor[rgb]{0.31,0.60,0.02}{#1}}
\newcommand{\WarningTok}[1]{\textcolor[rgb]{0.56,0.35,0.01}{\textbf{\textit{#1}}}}
\usepackage{longtable,booktabs,array}
\usepackage{calc} % for calculating minipage widths
% Correct order of tables after \paragraph or \subparagraph
\usepackage{etoolbox}
\makeatletter
\patchcmd\longtable{\par}{\if@noskipsec\mbox{}\fi\par}{}{}
\makeatother
% Allow footnotes in longtable head/foot
\IfFileExists{footnotehyper.sty}{\usepackage{footnotehyper}}{\usepackage{footnote}}
\makesavenoteenv{longtable}
\usepackage{graphicx}
\makeatletter
\def\maxwidth{\ifdim\Gin@nat@width>\linewidth\linewidth\else\Gin@nat@width\fi}
\def\maxheight{\ifdim\Gin@nat@height>\textheight\textheight\else\Gin@nat@height\fi}
\makeatother
% Scale images if necessary, so that they will not overflow the page
% margins by default, and it is still possible to overwrite the defaults
% using explicit options in \includegraphics[width, height, ...]{}
\setkeys{Gin}{width=\maxwidth,height=\maxheight,keepaspectratio}
% Set default figure placement to htbp
\makeatletter
\def\fps@figure{htbp}
\makeatother
\setlength{\emergencystretch}{3em} % prevent overfull lines
\providecommand{\tightlist}{%
  \setlength{\itemsep}{0pt}\setlength{\parskip}{0pt}}
\setcounter{secnumdepth}{5}

\usepackage{booktabs}
\usepackage{amsthm}
\usepackage{longtable}
\usepackage{pdflscape}
\usepackage{animate}
\usepackage{wasysym}
\usepackage{pdfpages}
\usepackage{graphicx}

\usepackage{fontsetup}
\usepackage{import}
\usepackage[margin=1in]{geometry}
\usepackage{imakeidx}
\makeindex[name=terms,title={Index of Terms},intoc]
\makeindex[name=R,title={Index of R Operators and Functions},intoc]

\makeatletter
\def\thm@space@setup{%
  \thm@preskip=8pt plus 2pt minus 4pt
  \thm@postskip=\thm@preskip
}
\makeatother

\setmainfont[Ligatures=TeX,BoldFont={cambriab.ttf},ItalicFont={cambriai.ttf}]{CAMBRIA.TTC}
%\DeclareMathAlphabet{\mathbfit}{OML}{cmm}{b}{it}

\usepackage{hyperref} 
% draft turns off -- hyperlinks\usepackage[draft, colorlinks=false, linkcolor = black]{hyperref}
\hypersetup{
    colorlinks = true,
    linkcolor={red!50!black},
    citecolor={blue!50!black},
    urlcolor={blue!80!black},
		%pdftitle={Amalgam of R},
		pdfpagemode = UseNone,
		pdfauthor = {Ken Aho},
		pdfstartview = FitB 
}

\usepackage{titling}
\pretitle{
\author{}
\title{}
\begin{center}
\includegraphics[width=6in,height=9in]{cover.jpg}\text{ }\\
\vspace{0.5in}}
\posttitle{\end{center}}

\usepackage{tcolorbox}
\newtcolorbox{blackbox}{
  colback=lightgray,
  colframe=red,
  coltext=black,
  boxsep=5pt,
  arc=4pt}
	


\frontmatter

% bookdown::render_book('index.Rmd', 'bookdown::pdf_book') # creates entire PDF book with correct links
% bookdown::render_book('11-Ch11.Rmd', 'bookdown::pdf_book', preview = T) # only creates PDF for Ch 11. Book links and section/fig/table numbers will be incorect.
% bookdown::render_book('11-Ch11.Rmd', 'bookdown::bs4_book', preview = T) # only creates PDF for Ch 11. Book links and section/fig/table numbers will be incorect.
% bookdown::render_book('index.Rmd', 'bookdown::html_book') # creates entire HTML book with correct links (R Markdown will complain)
\usepackage{float}
\ifLuaTeX
  \usepackage{selnolig}  % disable illegal ligatures
\fi
\usepackage[]{natbib}
\bibliographystyle{apalike}
\usepackage{bookmark}
\IfFileExists{xurl.sty}{\usepackage{xurl}}{} % add URL line breaks if available
\urlstyle{same}
\hypersetup{
  pdftitle={An Amalgam of R},
  pdfauthor={Ken Aho},
  colorlinks=true,
  linkcolor={Maroon},
  filecolor={Maroon},
  citecolor={Blue},
  urlcolor={blue},
  pdfcreator={LaTeX via pandoc}}

\title{An Amalgam of R}
\usepackage{etoolbox}
\makeatletter
\providecommand{\subtitle}[1]{% add subtitle to \maketitle
  \apptocmd{\@title}{\par {\large #1 \par}}{}{}
}
\makeatother
\subtitle{A Primer and Reference for Biologists}
\author{Ken Aho}
\date{\vspace{-2.5em}2025-01-28}

\usepackage{amsthm}
\newtheorem{theorem}{Theorem}[chapter]
\newtheorem{lemma}{Lemma}[chapter]
\newtheorem{corollary}{Corollary}[chapter]
\newtheorem{proposition}{Proposition}[chapter]
\newtheorem{conjecture}{Conjecture}[chapter]
\theoremstyle{definition}
\newtheorem{definition}{Definition}[chapter]
\theoremstyle{definition}
\newtheorem{example}{Example}[chapter]
\theoremstyle{definition}
\newtheorem{exercise}{Exercise}[chapter]
\theoremstyle{definition}
\newtheorem{hypothesis}{Hypothesis}[chapter]
\theoremstyle{remark}
\newtheorem*{remark}{Remark}
\newtheorem*{solution}{Solution}
\begin{document}
\maketitle

{
\hypersetup{linkcolor=}
\setcounter{tocdepth}{1}
\tableofcontents
}
\chapter*{Preface}\label{preface}
\addcontentsline{toc}{chapter}{Preface}

\section*{What this book is about}\label{what-this-book-is-about}
\addcontentsline{toc}{section}{What this book is about}

This book explores the ever expanding universe of \textbf{R}. Specifically, it considers:

\begin{itemize}
\item
  The historical development of the \textbf{R} language, the \textbf{R}
  engine, and the installation of \textbf{R} (Ch \ref{ch1})
\item
  The creation of \textbf{R} objects and their fundamental characteristics
  (Ch \ref{ch2})
\item
  \textbf{R} data storage entities, and the import and export of user data
  files (Ch \ref{ch3})
\item
  Data management approaches using base \textbf{R} (Ch \ref{ch4}) and the \emph{tidyverse} (Ch \ref{ch5})
\item
  \textbf{R} approaches to graphics, including base plotting methods (Ch \ref{ch6})
  and the \emph{ggplot2} package (Ch \ref{ch7})
\item
  \textbf{R} functions (Ch \ref{ch8}) including loops, and the creation of
  user-defined classes and generic methods
\item
  Interfacing other languages (e.g., C, Fortran, C++, Python) and software
  environments to and from \textbf{R} (Ch \ref{ch9})
\item
  Building custom \textbf{R} packages (Ch \ref{ch10})
\item
  \textbf{R} Interactive interfaces and web applications including
  approaches from the packages \emph{tcltk}, \emph{plotly} and \emph{shiny} (Ch \ref{ch11})
\item
  The fundamental ways that \textbf{R} interacts with your computer (Ch \ref{ch12})
\end{itemize}

While this book covers a lot of ground, clearly many other topics could
be considered. Subjects explored are those I have found to be
particularly useful or interesting during my 20+ years of using \textbf{R} as a biologist and statistician. Chapters concerning advanced topics (i.e., Chs 8-12) are intended to be starting points for further exploration, and the reader is directed to additional resources when necessary.

The book emphasizes \textbf{R} as an important computer language. While ignored in many phenologies of computer languages \citep[e.g.,][]{boutin2002}, \textbf{R} has had a large, devoted following for decades and its computational engine and language can be clearly linked to seminal concepts and advances in computer science. Further, from its inception \textbf{R} has been a tool for \emph{metaprogramming} wherein code is shared and modified programmatically. For instance \textbf{R} has a wide variety of widely used APIs for languages like C, Fortran, C++, Java, Python, and many other others.

Individuals from the natural sciences, particularly biologists, are
likely to find this book more useful than individuals from other
backgrounds because coding examples and applications are generally biological. Non-biologists may find, however, that examples readily extend to other settings.

\section*{What this book is not about}\label{what-this-book-is-not-about}
\addcontentsline{toc}{section}{What this book is not about}

Notably, although statistics is the primary focus/purpose of \textbf{R}, the
primary emphasis of this book is \emph{not} statistics. Instead I focus on the
\textbf{R} language, and the characteristics, capabilities, and
extensions of the \textbf{R} system. I take this approach because: 1)
coverage of non-statistical topics is challenging in and of itself, and
2) the responsible introduction of statistical algorithms from any
program or language (including \textbf{R}) should be accompanied by detailed
information concerning the statistical procedures. Many pedagogic resources exist for the statistical application of \textbf{R}. These include: \citet{aho2014} (the pedagogic statistical companion to this book), \citet{venables2002}, \citep{faraway2004, faraway2016}, \citet{crawley2012}, and \citet{fox2019}, among others. It should be noted that while this text does not focus on inferential statistical methods, it \emph{does} emphasize methods for handling, summarizing and displaying empirical data, and these steps serve as a companion and prerequisite to formal inferential analyses.

\section*{Distinguishing Characteristics of This Book}\label{distinguishing-characteristics-of-this-book}
\addcontentsline{toc}{section}{Distinguishing Characteristics of This Book}

Many other sources have emphasized fundamental programming aspects of \textbf{R}, while largely ignoring statistics, including seminal texts \citep[e.g.,][]{chambers2008, chambers2020, wickham2016, wickham2021mastering}, and definitive CRAN manuals \citep{team2024rint, team2024r, team2024rext}, or have focused on particular, potentially non-statistical \textbf{R} attributes, including graphics \citep{wickham2016, murrell2019} and web-based applications \citep{wickham2021mastering, sievert2020}. This book is a brave/foolish attempt to \emph{amalgamize} and distill this disparate information, while occasionally emphasizing topics earlier works have ignored. For instance, \citet{wickham2019advanced} admirably emphasizes many foundational and advanced programming ideas in \textbf{R}, but does not thoroughly consider some important programming extensions, including powerful syntheses with Python and Tcl. Unlike many other texts, this book also adheres to the format of a textbook, with numerous worked (often biological) examples, and exercises at the end each chapter.

\section*{Conventions}\label{conventions}
\addcontentsline{toc}{section}{Conventions}

This document has been created with Windows users of \textbf{R} in mind. In the vast majority of cases, however, instructions and examples will be extendable to other operating systems. In cases when this is not true I note steps to address these inconsistencies.

Several conventions are followed throughout the text. \textbf{R} package names and important terms are \emph{italicized}. \textbf{R} function names, function arguments and objects are written in blocked \texttt{Courier\ font}. Functions and operations are often written into ``chunks'' whose contents are readily copied to a clipboard using an icon located at the top right of the chunk (HTML version of text only). For example:
\vspace{0.1in}

\begin{Shaded}
\begin{Highlighting}[]
\FunctionTok{print}\NormalTok{(}\StringTok{"hello world"}\NormalTok{)}
\end{Highlighting}
\end{Shaded}

The \emph{output} from an evaluated chunk is generally printed immediately below. For example:
\vspace{0.1in}

\begin{verbatim}
[1] "hello world"
\end{verbatim}

\section*{Acknowledgements and Corrections}\label{acknowledgements-and-corrections}
\addcontentsline{toc}{section}{Acknowledgements and Corrections}

I thank individuals who have reviewed/edited this book in various forms including Lauren Tucker and Adam Zambie. Corrections and comments are welcome, and can be sent using the book's GitHub \href{https://github.com/moondog1969/Amalgam/issues}{site}.

\section*{Copyright}\label{copyright}
\addcontentsline{toc}{section}{Copyright}

Copyright © 2025 -- Ken Aho\\

Permission is granted to copy, distribute, and/or modify this document under the terms of
the GNU Free Documentation License, Version 1.3 or any later version published by the Free
Software Foundation; with no Invariant Sections, no Front-Cover Texts, and no Back-Cover
Texts. A copy of the license is included in the section entitled \href{https://www.gnu.org/licenses/fdl-1.3.en.html}{``GNU Free Documentation
License''}.

\mainmatter

\chapter{Welcome to R}\label{ch1}

\begin{quote}
\emph{"I believe that \textbf{R} currently represents the best medium
for quality software in support of data analysis."}
\end{quote}

- \textbf{John Chambers}, \emph{Developer of \textbf{S} }

\begin{quote}
\emph{"\textbf{R} is a real demonstration of the power of
collaboration, and I don't think you could construct something like
this any other way."}
\end{quote}

- \textbf{Ross Ihaka}, \emph{original co-developer of \textbf{R} }

\section{What is R?}\label{what-is-r}

\textbf{R}\index[terms]{R!introduction to} is a computer language and an open source setting for statistics, data management, computation, and graphics. The outward mien of the
\textbf{R}-environment is minimalistic, with few menu-driven interactive
facilities (\emph{no} menus exist for some implementations of \textbf{R}). This is
in contrast to conventional statistical software consisting of black
box, menu-dominated, often inflexible tools. The simplicity of \textbf{R}
allows one to easily evaluate, edit, and build procedures for data
analysis, and many other purposes.

\section{R and Biology}\label{r-and-biology}

I am a statistical ecologist, so this book was written with natural scientists, particularly biologists, in mind. \textbf{R} is useful to biologists for three major reasons. First, it provides access to a large number of cutting edge statistical, graphical, and organizational procedures, many of which have been designed specifically for biological research. Second, biological datasets, including those from genetic and spatiotemporal research can be extensive and complex. \textbf{R} can readily manage and analyze such data. Third, analysis of biological data often requires analytical and computational flexibility. \textbf{R} allows one to ``get under the hood'', look at the code, and check to see what algorithms are doing. If, after examining an \textbf{R}-algorithm we are unsatisfied, we can generally modify its code or create new code to meet our specific needs.

\section{Popularity of R}\label{popularity-of-r}

Because of its freeware status, the overall number of people using \textbf{R}
is difficult to determine\index[terms]{R!popularity of}. Nonetheless, the \textbf{R}-consortium \href{https://www.r-consortium.org/}{website} estimates that there are currently more than two million active \textbf{R} users. The \href{https://r4stats.com/articles/popularity/}{r4stats
website} houses up-to-date
surveys concerning the popularity of analytical software. These surveys
(accessed 10/23/2024) indicate that \textbf{R} is often preferred among
data scientists for big data projects and data mining. \textbf{R} is also one
of the most frequently cited statistical environments in scholarly
articles, one of the most frequently used languages on the
\href{https://github.com/}{GitHub} repository, and one of the most frequently
discussed languages on \href{https://stackoverflow.com/}{Stack Overflow}. In
2024 the \textbf{R} language \href{https://spectrum.ieee.org/top-programming-languages-2022}{was ranked 20th in the
world} by the
Institute of Electrical and Electronics Engineers
(\href{https://www.ieee.org/}{IEEE})\index[terms]{IEEE (Institute of Electrical and Electronics Engineers)}, and was recently (10/23/2024) \href{https://pypl.github.io/PYPL.html}{ranked 6th} by the PopularitY of Programming Language (PYPL) Index, which uses the search string \texttt{\textquotesingle{}X\ tutorial\textquotesingle{}}, as an indicator of future language usage. Further, in a \href{https://stackoverflow.blog/2017/10/31/disliked-programming-languages/}{2017 survey}, based on Stack Overflow queries, \textbf{R} was the ``least disliked"
programming language.

The growth and popularity of \textbf{R} can be partially tied to its
relatively straightforward extendability via user generated functions
and packages. This characteristic prompts a strong sense of community
among \textbf{R}-users, along with a practical need for the perpetuation and
upkeep of the \textbf{R} system. While trailing Python, there are
currently over 20000 formally contributed \textbf{R}-packages at the
Comprehensive \textbf{R} Archive Network
(\href{https://cran.r-project.org/}{CRAN})\index[terms]{R!CRAN (archive network)}\index[terms]{R!package}.

\section{A Brief History}\label{Rhist}

\textbf{R}\index[terms]{R!history of} was created in the early 1990s by Australian computational statisticians Ross Ihaka and Robert Gentleman (Fig \ref{fig:ig}) to address \emph{scope}\footnote{In computer science, \emph{scope} refers to the degree of binding between an identifier of an entity (e.g., an object name) and the entity itself (e.g., an object).}\index[terms]{Scope (computer science)} and memory use deficiencies in its primary progenitor language, \textbf{S} \citep{ihaka1996}.\index[terms]{S (programming language)} Ihaka and Gentleman used the name \textbf{R} both to acknowledge the influence of \textbf{S} (because \emph{r} and \emph{s} are juxtaposed in the alphabet), and to celebrate their own personal efforts (because R was the first letter of their first names).

\begin{figure}

{\centering \includegraphics[width=0.4\linewidth]{figs1/ihaka_gent} 

}

\caption{Ross Ihaka (1954 - ) (L) and Robert Gentleman (1959 - ) (R), the co-creators of $\textbf{R}$.}\label{fig:ig}
\end{figure}

At the insistence of Swiss statistician Martin Maechler (Fig \ref{fig:core}l), Ihaka and Gentleman distributed the \textbf{R} source code in 1995 under the Free Software Foundation's GNU general license \citep{ihaka1998r}. Because of its relatively easy-to-learn language, \textbf{R} was quickly extended with code and packages developed by its users. The rapid growth of \textbf{R} gave rise to the need for a group to guide its progress. This led, in 1997, to the establishment of the \textbf{R}-development core team\footnote{The first \textbf{R}-core consisted of: Douglas Bates, Peter Dalgaard, Robert Gentleman, Kurt Hornik, Ross Ihaka, Friedrich Leisch, Thomas Lumley, Martin Mächler, Paul Murrell, Heiner Schwarte, and Luke Tierney. Several of these individuals remain in the current \textbf{R}-core (Fig \ref{fig:core}).}, an international panel that modifies, troubleshoots, and manages source code (Fig \ref{fig:core}).
\index[terms]{R!development core team}

\begin{figure}

{\centering \includegraphics[width=0.9\linewidth]{figs1/core} 

}

\caption{A recent version of the $\textbf{R}$-core development team.}\label{fig:core}
\end{figure}

\subsection{Development of the R Language}\label{development-of-the-r-language}

\index[terms]{Lisp (programming language)}
The \textbf{R} language is based on older languages, particularly \textbf{S}, developed at Bell Laboratories \citep{becker1978, becker1981, becker1988}, and Lisp\footnote{Lisp, an abbreviation of ``LISt Processor'', is the second-oldest (after Fortran) high-level programming language still in common use. Lisp can be viewed as a related family of dialects rooted in MaCarthy's general programming approach, which included fully parenthesized prefix notation (wherein a function \texttt{f} with three arguments would be called using: \texttt{(f\ arg1\ arg2\ arg3))} \citet{reilly2003milestones}. Seminal Lisp dialects include \href{https://common-lisp.net/}{Common Lisp}, \href{https://www.scheme.org/}{Scheme}, and \href{https://en.wikipedia.org/wiki/AutoLISP}{AutoLISP} (built for the drafting software AutoCAD). Recent Lisp additions include: \href{https://hylang.org/}{Hy} (a Lisp dialect embedded in Python), \href{https://clojure.org/}{Clojure} (a dialect of Lisp for Java), and \href{https://lfe.io/}{Lisp Flavored Erlang (LFE)}. Many important programming ideas were pioneered by Lisp, including conditional statements based on Boolean decision rules, higher-order functions (which contain other functions as arguments), recursion (which allows a function to be called by its own code), and the read--eval--print loop.} \citep{mccarthy1978} and Scheme, a dialect of Lisp \citep{sussman1998, steele1978}, which were developed at the MIT artificial intelligence laboratory in the late 1970s (Fig \ref{fig:mc}).\\
\index[terms]{Lisp (programming language)!Clojure}
\index[terms]{Lisp (programming language)!AutoLISP}
\index[terms]{Lisp (programming language)!Common Lisp}
\index[terms]{Lisp (programming language)!Hy}
\index[terms]{Lisp (programming language)!Lisp Flavored Erlang}

\begin{figure}

{\centering \includegraphics[width=0.5\linewidth]{figs1/mccarthy} 

}

\caption{John McCarthy (1927-2011), creator of the Lisp language, and the first to coin the term 'artifical intelligence', working at the MIT AI laboratory.}\label{fig:mc}
\end{figure}

In the appendix to his book \emph{Software for Data Analysis}, John M. Chambers (Fig \ref{fig:core}b), a primary developer of \textbf{S}, recounts the unique evolution and goals of \textbf{S} from its inception in 1976 \citep{chambers2008}. Chambers notes that \textbf{S} was originally intended to be an analysis toolbox solely for the statistics research group at Bell Labs, consisting of roughly 20 people at the time. It was decided that \textbf{S} (initially known as ``the system'') would have fundamental \emph{extensibility}\footnote{In software engineering, \emph{extensibility} is a design principle that provides for future growth. This allows developers to easily expand the software capabilities.}, reflecting the Bell Labs' philosophy that ``collaborations could actually enhance research'' \citep{chambers2008}\footnote{Notably, although \textbf{S} was originally designed to support statistical analysis, \citet{chambers2020} asserted that its actual usage at Bell Labs would be viewed today as \emph{data science}; defined as ``techniques and their application to derive and communicate scientifically valid inferences and predictions based on relevant data.''}. The \textbf{S} language definition, and details concerning the fitting and application of \textbf{S} statistical models are given in \citet{becker1988} and \citet{chambers1992}, respectively\footnote{\citet{becker1988} described the third version of \textbf{S}, \textbf{S3}. \citet{chambers1992} introduced formula-notation using the \texttt{\textasciitilde{}} operator, dataframe objects, and modifications to object methods and classes \citep{wikiS}. These publications were often referred to as the \emph{blue book} and the \emph{white book} by \textbf{S}-users, due to color of their covers.}.\\
\textbf{S} was designed to diminish inner functional details of its underlying Fortran\footnote{Fortran (FORmula TRANslator) is a computer language developed by IBM in the 1950s for science and engineering applications. Remarkably, it remains useful for many applications, including speeding up slow looping routines in interpreted languages like \textbf{R}. Fortran/\textbf{R} interfacing is formally considered in Ch 9 (Section \ref{FC})} algorithms while making important higher-level processes more readily accessible and interactive. The inspiration for these goals was the exploratory data analysis approach of John Tukey (Fig \ref{fig:tukey}), who was a contemporary of Chambers and other \textbf{S} developers\footnote{Other important contributors to \textbf{S} included Rick Becker, Trevor Hastie, William Cleveland, and Allan Wilks of Bell Labs.} at Bell Labs \citep{chambers2020}. In a 1965 Bell Labs memo (15 years before the release of \textbf{S}) Tukey noted that modern statistician found themselves in a ``peaceful collision of computing and data analysis'' \citep{chambers1999}.

\begin{figure}

{\centering \includegraphics[width=0.3\linewidth]{figs1/Tukey} 

}

\caption{John Tukey (1915-2000), widely known for achievements in mathematical statistics, including the fast Fourier transform \citep{cooley1965}, tools in exploratory data analysis, including the boxplot \citep{tukey1977}, and computer science, where he coined the term \textit{bit}, as a unit of binary infomration and memory \citep{shannon1948}.}\label{fig:tukey}
\end{figure}

An adherence of \textbf{S} to exploratory data analysis was evident in the high-quality and interactivity of its graphics devices and its easily-accessible function documentation. The initial programmatic objectives of \textbf{S} are apparent in an early design sketch that describes an outer `user interface' layer to core Fortran algorithms that ultimately produces an \textbf{S} object (Fig \ref{fig:Ssketch}). The underlying philosophical principles and programmatic foundations of \textbf{S} have strongly affected and guided the development of \textbf{R} \citep{chambers2020}.

\begin{figure}

{\centering \includegraphics[width=0.5\linewidth]{figs1/Ssketch} 

}

\caption{First designs for the S statistical system, \textit{circa} 1976 \citep{chambers2008}). Written in the lower lefthand corner is the important note: 'Names are meaningful to algorithm, not necessarily to language.' }\label{fig:Ssketch}
\end{figure}

\index[terms]{Lisp (programming language)}
\index[terms]{Fortran (programming language)}

\textbf{S} evolved alongside the Unix operating system (also developed at Bell
Labs) which currently underlies Macintosh and Linux (free-Unix)
operating systems\footnote{Unix itself was originally written in \emph{assembly language}\index[terms]{Assembly (programming language)} (a low-level programming language with a very strong correspondence between language instructions and machine/operating system instructions). Unix was later re-written in C.}. An early inception \textbf{S} was written for Unix, allowing \textbf{S} to be \emph{portable} to any machine using Unix. Both \textbf{S} and Unix were quickly commercially licensed by AT\&T for university and third party retailers. The academic licensing and distribution of \textbf{S} attracted a large number user groups in 1980s. However, the lack of a clear open source strategy caused many early users to switch from \textbf{S} to \textbf{R} in the 1990s. \textbf{S} was purchased by Insightful\(^{\circledR}\) software 2004 to run the commercial software \textbf{S}-Plus\(^{\circledR}\). In 2021 \textbf{S}-Plus\(^{\circledR}\) morphed to include \href{https://www.tibco.com/connected-intelligence}{TIBCO} connected intelligence software, with some \textbf{R} open source applications.

\subsubsection{R is Born}\label{r-is-born}

The original Scheme-inspired \textbf{R} interpreter consisted of roughly 1000 lines of C\footnote{C is a portable, general purpose language, initially developed by Dennis Ritchie \citep{kernighan2002}. C, in turn, evolved from the language B, created by Ken Thompson \citep{thompson1972}\index[terms]{B (programming language)}, which, in turn, was inspired by work on early operating system called Multics \index[terms]{Multics} \citep{corbato1965}. C/\textbf{R} interfacing is formally considered in Ch 9 (Section \ref{FC}).} code which was driven by a command line interface that used a syntax corresponding to \textbf{S}, resulting in ``a free implementation of something `close to' version 3 of the \textbf{S} language (\textbf{S3})'' \citep{ihaka1998r}.\index[terms]{C (programming language)} The \textbf{R} and \textbf{S} languages remain very similar, and code written in \textbf{S} can generally be run unaltered in \textbf{R}. The method of function implementation in \textbf{R}, however, remains more similar to Scheme. The official language definition of the current version of \textbf{R} can be found at the \href{https://cran.r-project.org/doc/manuals/r-release/R-lang.html\#FOOT1}{CRAN website}, along with other sources of complementary information.

\subsubsection{Differences of R and S}\label{differences-of-r-and-s}

\textbf{S3} and the initial release of \textbf{R} differed in two important ways \citep{ihaka1996}\footnote{For additional demonstrations of the technical differences of \textbf{R} and \textbf{S} see \citep{hornik}}. First, the \textbf{R}-environment was given a fixed amount of memory at start up. This was in contrast to \textbf{S}-engines which adjusted available memory to session needs. Among other things, this difference meant more available pre-reserved computer memory, and fewer \emph{virtual pagination}\footnote{\emph{Virtual pagination} is a memory management scheme that allows a computer to store and retrieve data from secondary storage for use in main memory.} problems in \textbf{R} \citep{ihaka1996}. It also made \textbf{R} faster than \textbf{S} for many applications \citep{hornik}. Second, \textbf{R} variable locations are \emph{lexically scoped}. In computer science, \emph{variables} are storage areas with identifiers, and \emph{scope} defines the context in which a variable name is recognized. So-called \emph{global variables} are accessible in every scope (for instance, both inside and outside functions). In contrast, \emph{local variables} may only exist within particular localized scopes. Formal parameters defined in \textbf{R} functions, including arguments, are (generally) local variables, whereas variables created outside of functions are global variables. In contrast to \textbf{S}, lexical scoping allowed functions in \textbf{R} access to variables that were in effect when the function was defined in a session. The characteristics of \textbf{R} functions and details concerning lexical scoping are further addressed in Ch \ref{ch8}.

\subsection{Recent Developments}\label{recent-developments}

According to \citet{thieme2018}, a growing component of the \textbf{R} culture includes individuals who are ``Less interested in the mechanics of \textbf{R}
than in what \textbf{R} allowed them to do.'' This group, which often includes individuals from non-\textbf{R} backgrounds (but with expertise in other languages C, Java, HTML, etc.), and those ``who may have little interest in becoming computer scientists'', has been championed by Hadley Wickham (Fig \ref{fig:hw}), creator of the important \emph{ggplot2} and \emph{dplyr} \textbf{R} packages, and author of many useful books on \textbf{R} programming. A larger collection of packages supported by Wickham is referred to as the \emph{tidyverse} \citep{tidyverse} (see Ch \ref{ch5}).
\index[terms]{R!package!\textit{ggplot2}} \index[terms]{R!package!\textit{dplyr}}
\index[terms]{R!package!\textit{tidyverse}}

\begin{figure}

{\centering \includegraphics[width=0.4\linewidth]{figs1/wickham} 

}

\caption{Hadley Wickham (1979 - ) chief scientist at Rstudio.}\label{fig:hw}
\end{figure}

\subsection{The Future of R}\label{the-future-of-r}

It is apparent that \textbf{R} can be tied (particularly via linkages
with Fortran and Lisp) to early examples of software engineering, and (via John Tukey and others) to foundational figures in data science. The future of \textbf{R} will be determined by the formal and informal community of users who have donated years of their lives to its development without monetary compensation. Importantly, the continued growth of \textbf{R} will require adaptation to the changing demography of \textbf{R}-users. Like most software endeavors, \textbf{R} has been male dominated (Fig \ref{fig:core}). However, this has been changing rapidly. As an example, the \href{https://rladies.org/}{\textbf{R} Ladies group}, founded in 2012 by Gabriela de Queiroz (Fig \ref{fig:gabriela}), currently (8/6/2024) has 225 chapters in 65 countries, and more than 39,000 members worldwide.

\begin{figure}

{\centering \includegraphics[width=0.45\linewidth]{figs1/gabriela} 

}

\caption{Gabriela de Queiroz, chief data scientist at IBM.}\label{fig:gabriela}
\end{figure}

\section{Copyrights and Licenses}\label{copyrights-and-licenses}

\textbf{R} is intentionally open-source and free. Thus, there are no
warranties on its environment or packages. As its copyright framework \textbf{R} uses the \href{https://www.gnu.org/home.en.html}{GNU} (a recursive acronym for GNU is not Unix) General Public License (GPL). This allows users to share and change \textbf{R} and its functions. The associated legalese can read after typing \texttt{RShowDoc("COPYING")} in the \textbf{R}-console. Because its functions can be legally (and easily) recycled and altered we should always give credit to developers, package maintainers, or whomever wrote the \textbf{R} functions or code we are using.

\section{R and Reliability}\label{r-and-reliability}

The lack of an \textbf{R} warranty has frightened away some scientists. But be assured, with few exceptions, \textbf{R} works as well or better than ``top of the line'' analytical commercial software. Indeed, statistical software giants SAS\(^{\circledR}\) and SPSS\(^{\circledR}\) have made \textbf{R} applications accessible from within their products \citep{fox2009}, and \textbf{R} processes and files can be shared directly with Microsoft Excel\(^{\circledR}\) and other proprietary software. For specialized or advanced statistical techniques \textbf{R} often outperforms other alternatives because of its diverse array of continually updated applications.

The computing engines and packages that come with a conventional \textbf{R} download (see Section \ref{package}) meet or exceed U.S. federal analytical standards for clinical trial research \citep{schwartz2008}. In addition, core algorithms used in \textbf{R} are based on seminal and well-trusted functions. For instance, \textbf{R} random number generators include some of the most venerated and thoroughly tested functions in computer history \citep{chambers2008}. Similarly, the Linear Algebra PACKage (LAPACK) algorithms \citep{anderson1999}, used by \textbf{R}, are among the world's most stable and best-tested software.

\section{Installation}\label{installation}

To install \textbf{R}, first go to the website (\url{http://www.r-project.org/}).
On this page specify which platform you are using (Fig \ref{fig:Rinstall}, step 1). \textbf{R} can currently be used on Unix/Linux, Windows and Mac operating
systems. Once an operating system has been selected, one can click on the ``base'' link to download the precompiled base binaries if \textbf{R} currently exists on your computer. If \textbf{R} has not been previously installed on your computer click on ``Install \textbf{R} for the first time'' (Fig \ref{fig:Rinstall}, step 2). You will delivered to a window containing a link to download the most recent version of \textbf{R} by clicking on the ``Download'' link (Fig \ref{fig:Rinstall}, step 3). Two versions of \textbf{R} are generally released each year, one in April and one in October. Archived, older versions of \textbf{R} and \textbf{R} packages are also available from CRAN.

\begin{figure}

{\centering \includegraphics[width=0.9\linewidth]{figs1/CaptureCRAN} 

}

\caption{Method for installing {\textbf{R}} for Windows for the first time.}\label{fig:Rinstall}
\end{figure}

\section*{Exercises}\label{exercises}
\addcontentsline{toc}{section}{Exercises}

\begin{enumerate}
\def\labelenumi{\arabic{enumi}.}
\tightlist
\item
  The following questions concern the history and general characteristics of \textbf{R}.

  \begin{enumerate}
  \def\labelenumii{(\alph{enumii})}
  \tightlist
  \item
    Who were the creators of \textbf{R}?
  \item
    What are some major developmental events in the history of \textbf{R}?
  \item
    What languages is \textbf{R} derived from and/or most similar to?
  \item
    What features distinguish \textbf{R} from other languages and statistical software?
  \item
    What are the three operating systems \textbf{R} works with?
    \vspace{0.1in}
  \end{enumerate}
\item
  Briefly consider \textbf{R} in the context of major historical events in computer software and artificial intelligence.
\end{enumerate}

\chapter{Some Basics}\label{ch2}

\begin{quote}
\emph{``Learning to write programs stretches your mind,
and helps you think better."}
\end{quote}

- \textbf{Bill Gates}, \emph{1955-}

\section{First Steps}\label{first-steps}

Upon opening \textbf{R} in Windows, two things will appear in the console of the \textbf{R} \emph{Graphical User Interface} (\textbf{R}-GUI)\index[terms]{R!R-GUI}\footnote{Unix/Linux operating systems require \textbf{R} to be launched from the shell command line by typing: \texttt{R}. This will begin an interactive \textbf{R} session on the system shell command line itself.}. These are the \emph{license disclaimer} (blue text at the top of the console) and the \emph{command line prompt}, i.e., {\(\boldsymbol{>}\)} \index[terms]{R!command line prompt} (Fig \ref{fig:rconsole}). The prompt indicates that \textbf{R} is ready for a command. All commands in \textbf{R} must begin at {\(\boldsymbol{>}\)}.

The appearance of this simple interface will vary slightly among operating systems. In the Windows \textbf{R}-GUI, the command line prompt and user commands are colored red, and output, including errors and warnings, are colored blue. In Mac OS, the command line prompt will be purple, user inputs will be blue, and output will be black. In Unix/Linux, wherein \textbf{R} will generally run from a shell command line, absent of any menus, all three will be black\footnote{A Unix/Linux GUI, similar to those in Windows and Mac OS, can be initiated by opening \textbf{R} with the commands: \texttt{R\ -g\ Tk\ \&}.}.\index[terms]{Mac (operating system)}\index[terms]{Linux/Unix (operating system)}

We can exit \textbf{R} at any time by typing \texttt{q()}\index[R]{\texttt{q()}} in the console, closing the GUI window (non-Linux only), or by selecting \textbf{Exit} from the pulldown File menu (non-Linux only).

\begin{figure}[H]

{\centering \includegraphics[width=0.7\linewidth,]{figs2/Capture} 

}

\caption{An aged, but still recognizable {\textbf{R}} console: {\textbf{R}} version 2.15.1, 'Roasted Marshmallows', ca. 2012.}\label{fig:rconsole}
\end{figure}

\section{First Operations}\label{first-operations}

As an introduction we can use \textbf{R} to evaluate a simple mathematical expression. Type \texttt{2\ +\ 2} and press Enter.

\vspace{0.1in}

\begin{Shaded}
\begin{Highlighting}[]
\DecValTok{2} \SpecialCharTok{+} \DecValTok{2}
\end{Highlighting}
\end{Shaded}

\begin{verbatim}
[1] 4
\end{verbatim}

The output term \texttt{{[}1{]}} means, ``this is the first requested element.'' In this case there is just one requested element, \(4\), the solution to \(2 + 2\). If the output elements cannot be held on a single console line, then \textbf{R} would begin the second line of output with the element number comprising the first element of the new line. For instance, the command \texttt{rnorm(20)}\index[R]{\texttt{rnorm()}} will take 20 pseudo-random samples (see footnote in Section \ref{pvR}) from a standard normal distribution (see Ch 3 in \citet{aho2014}). We have:

\vspace{0.1in}

\begin{Shaded}
\begin{Highlighting}[]
\FunctionTok{rnorm}\NormalTok{(}\DecValTok{20}\NormalTok{)}
\end{Highlighting}
\end{Shaded}

\begin{verbatim}
 [1] -1.014669993 -0.453660252  0.378558998 -0.785172370 -1.606244042
 [6] -1.643340003  0.063482751  0.600236047  0.730878220 -2.758189778
[11] -0.540611915  0.793614685  0.182480430  0.198782519 -0.053528382
[16] -0.896046658  0.652802335  0.473566867 -0.501028595 -0.003658943
\end{verbatim}

\index[terms]{Pseudo-random number}
The reappearance of the command line prompt indicates that \textbf{R} is ready for another command. Multiple commands can be entered on a single line, separated by semicolons\index[R]{*semi@\texttt{;}}. Note, however, that this is considered poor programming style, as it may make your code more difficult to understand by a third party.
\vspace{0.1in}

\begin{Shaded}
\begin{Highlighting}[]
\DecValTok{2} \SpecialCharTok{+} \DecValTok{2}\NormalTok{; }\DecValTok{3} \SpecialCharTok{+} \DecValTok{2}
\end{Highlighting}
\end{Shaded}

\begin{verbatim}
[1] 4
\end{verbatim}

\begin{verbatim}
[1] 5
\end{verbatim}

\textbf{R} commands are generally insensitive to spaces. This allows the use of spaces to make code more legible. To my eyes, the command \texttt{2\ +\ 2} is simply easier to read and debug than \texttt{2+2}.

\subsection{Use Your Scroll Keys}\label{use-your-scroll-keys}

As with many other command line environments, the scroll keys (Fig \ref{fig:scroll}) provide an important shortcut in \textbf{R}. Instead of editing a line of code by tediously mouse-searching for an earlier command to copy, paste and then modify, you can simply scroll back through your earlier work using the upper scroll key, i.e., \(\uparrow\) . Accordingly, scrolling down using \(\downarrow\) will allow you to move forward through earlier commands.

\begin{figure}[H]

{\centering \includegraphics[width=0.3\linewidth,]{figs2/keyboard} 

}

\caption{Typical scroll direction keys on a keyboard.}\label{fig:scroll}
\end{figure}

\subsection{\texorpdfstring{Note to Self: \texttt{\#}}{Note to Self: \#}}\label{note-to-self}

\textbf{R} will not recognize commands preceded by \texttt{\#}\index[R]{*a@\texttt{\#}}. As a result this is a good way for us to leave messages to ourselves.
\vspace{0.1in}

\begin{Shaded}
\begin{Highlighting}[]
\CommentTok{\# Note at beginning of line}
\DecValTok{2} \SpecialCharTok{+} \DecValTok{2}
\end{Highlighting}
\end{Shaded}

\begin{verbatim}
[1] 4
\end{verbatim}

We can even place comments in the middle of an expression, as long the expression is finished on a new line.
\vspace{0.1in}

\begin{Shaded}
\begin{Highlighting}[]
\DecValTok{2} \SpecialCharTok{+} \CommentTok{\# Note in middle of line}
\SpecialCharTok{+} \DecValTok{2}
\end{Highlighting}
\end{Shaded}

\begin{verbatim}
[1] 4
\end{verbatim}

In the ``best'' code writing style it is recommended that one place a space after \texttt{\#} before beginning a comment, and to insert two spaces following code before placing \texttt{\#} in the middle of a line. This convention is followed above.

\subsection{Unfinished Commands}\label{unfinished-commands}

\textbf{R} will be unable to move on to a new task when a command line is unfinished. For example, type
\vspace{0.1in}

\begin{Shaded}
\begin{Highlighting}[]
\DecValTok{2} \SpecialCharTok{+}
\end{Highlighting}
\end{Shaded}

and press Enter. We note that the \emph{continuation prompt}\index[terms]{R!continuation prompt}, \texttt{+}, is now where the command prompt should be. \textbf{R} is telling us the command is unfinished. We can get back to the command prompt by finishing the function, clicking \textbf{Misc\(>\)Stop current computation} or \textbf{Misc\(>\)Stop all computations} from the \textbf{R}-toolbar (non-Linux only), typing Ctrl + C (Linux), or by pressing the Esc key (all OS).

\section{Expressions, Assignments and Objects}\label{expressions-assignments-and-objects}

All entries in \textbf{R} are either \emph{expressions}\index[terms]{Expression (programming)} or \emph{assignments}\index[terms]{Assignment (programming)}. If a command is an expression it will be evaluated, printed, and discarded. Examples include: \texttt{2\ +\ 2}. Conversely, an assignment evaluates an expression, and assigns a label to the output, but does not automatically print the result.

To convert an expression to an assignment we use the \emph{assignment operator}, \texttt{\textless{}-},\index[R]{*ao@\texttt{\textless-}} \index[terms]{R!assignment operator} which represents an arrow that points to the label of the expression. The assignment operator can go on either side of an expression.

\begin{example}
\protect\hypertarget{exm:exm-2obj1}{}\label{exm:exm-2obj1}

\(\text{}\)\\
If I type:
\vspace{0.1in}

\begin{Shaded}
\begin{Highlighting}[]
\NormalTok{x }\OtherTok{\textless{}{-}} \DecValTok{2} \SpecialCharTok{+} \DecValTok{2}
\end{Highlighting}
\end{Shaded}

or
\vspace{0.1in}

\begin{Shaded}
\begin{Highlighting}[]
\DecValTok{2} \SpecialCharTok{+} \DecValTok{2} \OtherTok{{-}\textgreater{}}\NormalTok{ x}
\end{Highlighting}
\end{Shaded}

then an \textbf{R}-\emph{object}\index[terms]{R!object} is created named \texttt{x} that contains the result of the expression \texttt{2\ +\ 2}. In fact, the code: \texttt{x\ \textless{}-\ 2\ +\ 2} literally means: ``\texttt{x} is \(2 + 2\).'' The \texttt{\textless{}-} operator is generally used (instead of \texttt{-\textgreater{}}) because it is easier to conceptualize the relationship \texttt{object\ name\ -\textgreater{}\ object\ definiton}, than its reversal.

To print the result of the assignment above (to \emph{see} \texttt{x}), I simply type:
\vspace{0.1in}

\begin{Shaded}
\begin{Highlighting}[]
\NormalTok{x}
\end{Highlighting}
\end{Shaded}

\begin{verbatim}
[1] 4
\end{verbatim}

or
\vspace{0.1in}

\begin{Shaded}
\begin{Highlighting}[]
\FunctionTok{print}\NormalTok{(x)}
\end{Highlighting}
\end{Shaded}

\begin{verbatim}
[1] 4
\end{verbatim}

\end{example}

\(\blacksquare\)

In Example \ref{exm:exm-2obj1} above we could have typed \texttt{x\ =\ 2\ +\ 2} with the same assignment results.
\vspace{0.1in}

\begin{Shaded}
\begin{Highlighting}[]
\NormalTok{x }\OtherTok{=} \DecValTok{2} \SpecialCharTok{+} \DecValTok{2}
\NormalTok{x}
\end{Highlighting}
\end{Shaded}

\begin{verbatim}
[1] 4
\end{verbatim}

However, for this document, I will continue to use the arrow operator, \texttt{\textless{}-}, for object assignments, and save the equals sign, \texttt{=}, for specifying arguments in functions (Ch \ref{ch8}).

Note that the \textbf{R}-console can quickly become cluttered and confusing. To remove clutter on the console (without actually getting rid of any of the objects created in a session) press Ctrl + L or, from the \textbf{Edit} pulldown menu, click on \textbf{Clear console} (non-Linux only).

\subsection{Naming Objects}\label{naming-objects}

When assigning names to \textbf{R}-objects we should try to keep the names simple, and avoid names that already represent important definitions and functions. These include: \texttt{TRUE,\ FALSE,\ NULL,\ NA,\ NaN,} and \texttt{Inf}. In addition, we cannot have names:

\begin{itemize}
\tightlist
\item
  beginning with a numeric value,
\item
  containing spaces, colons, and semicolons,
\item
  containing mathematical operators (e.g., \texttt{*}, \texttt{+}, \texttt{-}, \texttt{\^{}}, \texttt{/}, \texttt{=}),
\item
  containing important \textbf{R} \emph{metacharacters} (e.g., \texttt{@}, \texttt{\#}, \texttt{?}, \texttt{!}, \texttt{\%}, \texttt{\&}, \texttt{\textbar{}}).
\end{itemize}

\index[terms]{R!object!names}

However, even these ``forbidden'' names and characters can be used if one encloses them in backticks, also called \emph{accent grave} characters. For example, the code, \texttt{\textasciigrave{}?\textasciigrave{}\ \textless{}-\ 2\ +\ 2} will create an object named \texttt{\textasciigrave{}?\textasciigrave{}}, containing the number 4.

\index[R]{*+a0@\texttt{\`}}

Names should, if possible, be descriptive. Thus, for a object containing 20 random observations from a normal distribution, the name \texttt{rN20} may be superior to the easily-typed, but anonymous name, \texttt{x}. Finally, we should remember that \textbf{R} is case sensitive. That is, each of the following \(2^4\) combinations will be recognized as distinct: \texttt{name,\ Name,\ nAme,\ naMe,\ namE,\ NAme,\ nAMe,\ naME,\ NaMe,\ nAmE,\ NamE,\ naME,\ NAMe,\ nAME,\ NaME,\ NAmE,\ NAME}.

\subsection{Combining Data}\label{combining-data}

To define a collection of numbers (or other data or objects) as a single entity one can use the important \textbf{R} function \texttt{c}\index[R]{\texttt{c()}}, which means ``combine''.

\begin{example}
\protect\hypertarget{exm:exm-2c1}{}\label{exm:exm-2c1}\(\text{}\)\\
To define the numbers 23, 34, and 10 collectively as an object named \texttt{x}, I would type:
\vspace{0.1in}

\begin{Shaded}
\begin{Highlighting}[]
\NormalTok{x }\OtherTok{\textless{}{-}} \FunctionTok{c}\NormalTok{(}\DecValTok{23}\NormalTok{, }\DecValTok{34}\NormalTok{, }\DecValTok{10}\NormalTok{)}
\end{Highlighting}
\end{Shaded}

We could then do something like:
\vspace{0.1in}

\begin{Shaded}
\begin{Highlighting}[]
\NormalTok{x }\SpecialCharTok{+} \DecValTok{7}
\end{Highlighting}
\end{Shaded}

\begin{verbatim}
[1] 30 41 17
\end{verbatim}

Note that seven was added to each element in \texttt{x}.
\end{example}

\(\blacksquare\)

\subsection{Object Classes}\label{oc}

We can view everything created or loaded in \textbf{R} as an \emph{object}\footnote{Although we can view everything created or loaded in \textbf{R} as an object, not all \textbf{R} objects fit neatly into the OOP perspective of ``object-oriented.'' This is true because \textbf{R} base objects (which are \emph{not} object oriented) come from S, which was developed before anyone considered the need for an \textbf{S} OOP system (see \citet{wickham2019advanced} and \citet{chambers2008}).}. Under the idiom of \emph{object oriented programming} (\emph{OOP})\index[terms]{Object oriented programming}, an object may have attributes that allow it to be evaluated correctly, and associated methods appropriate for those attributes (e.g., specific functions for plotting, printing, etc.)\footnote{There are many OOP languages including \textbf{R}, C\#, C++, Objective-C, Smalltalk, Java, Perl, Python and PHP. C is not considered an OOP language.}.
\index[terms]{R!object}
\index[terms]{R!classes}
\index[terms]{R!classes!numeric}
\index[terms]{R!classes!integer}
\index[terms]{R!classes!factor}
\index[terms]{R!classes!complex}
\index[terms]{R!classes!matrix}
\index[terms]{R!classes!list}
\index[terms]{R!classes!array}
\index[terms]{R!classes!data.frame}
\index[terms]{R!classes!function}

I can list objects available in my \textbf{R} session using the function \texttt{objects()} or \texttt{ls()}. Currently, I only have \texttt{x} (which has been applied and modified several times) in my session (global environment), by typing:
\vspace{0.1in}

\begin{Shaded}
\begin{Highlighting}[]
\FunctionTok{objects}\NormalTok{()}
\end{Highlighting}
\end{Shaded}

\begin{verbatim}
[1] "x"
\end{verbatim}

or\\

\begin{Shaded}
\begin{Highlighting}[]
\FunctionTok{ls}\NormalTok{()}
\end{Highlighting}
\end{Shaded}

\begin{verbatim}
[1] "x"
\end{verbatim}

\textbf{R} objects will generally have a \emph{class}, identifiable with the function \texttt{class()}\index[terms]{R!class}\index[R]{\texttt{class()}}.
\vspace{0.1in}

\begin{Shaded}
\begin{Highlighting}[]
\FunctionTok{class}\NormalTok{(x)}
\end{Highlighting}
\end{Shaded}

\begin{verbatim}
[1] "numeric"
\end{verbatim}

Objects in class \texttt{numeric} and several other common classes can be evaluated mathematically. Common \textbf{R} classes are shown in Table \ref{tab:Rclasses}. We will create objects from all of these classes, and learn about their characteristics, over the next few chapters.

\begin{table}[ht]
\centering
\caption{Common \textbf{R} classes for some object \texttt{x}. The listed class would be printed if one created the assignment for \texttt{x} shown in the Example, and typed \texttt{class(x)}\vspace{0.1in}} 
\label{tab:Rclasses}
\begin{tabular}{ll}
  \toprule
Class & Example \\ 
  \midrule
\texttt{logical} & \texttt{x \texttt{\textless-} TRUE} \\ 
  \texttt{numeric} & \texttt{x \texttt{\textless-} 2 + 2} \\ 
  \texttt{integer} & \texttt{x \texttt{\textless-} 1:3} \\ 
  \texttt{character} & \texttt{x \texttt{\textless-} c("a","b","c")} \\ 
  \texttt{factor} & \texttt{x \texttt{\textless-} factor("a","a","b")} \\ 
  \texttt{complex} & \texttt{x \texttt{\textless-} 5i} \\ 
  \texttt{expression} & \texttt{x \texttt{\textless-} expression(x * 4)} \\ 
  \texttt{function} & \texttt{x \texttt{\textless-} function(y)y + 1} \\ 
  \texttt{matrix} & \texttt{x \texttt{\textless-} matrix(nrow = 2, rnorm(4))} \\ 
  \texttt{array} & \texttt{x \texttt{\textless-} array(rnorm(8), c(2, 2, 2))} \\ 
  \texttt{data.frame} & \texttt{x \texttt{\textless-} data.frame(v1 = c(1,2), v2 = c("a","b"))} \\ 
  \texttt{list} & \texttt{x \texttt{\textless-} list()} \\ 
   \bottomrule
\end{tabular}
\end{table}

\subsection{Object Base Types}\label{bt}

All \textbf{R} objects will have so-called \emph{base types}\index[terms]{R!object!base types} that define their underlying C language data structures\footnote{Specifically, \textbf{R} base types correspond to an underlying C-codified \emph{typedef}, i.e., an alias framework for C data types. This internal algorithm is referred to by the \textbf{R}-core development team as \texttt{SEXPTYPE}, meaning \textbf{S}-expression (\texttt{SEXP}) type \citep{team2024rint}. There are currently 24 \texttt{SEXPTYPE} variants, each corresponding to one of the 24 \textbf{R} base types.}. There are currently 24 base types used by \textbf{R} \citep{team2024rint}, and it is unlikely that more will be developed in the near future \citep{wickham2019advanced}. These entities are listed in Table \ref{tab:Rbtypes}. The meaning and usage of some of the base types may seem clear, for instance, \texttt{integer} and \texttt{character}, which are also class designations (Table \ref{tab:Rclasses}). Several base types are specifically addressed in later chapters, including \texttt{list}, \texttt{logical}, \texttt{integer}, and \texttt{NULL} (Chs \ref{ch3} and \ref{ch7}), \texttt{character} (Chs \ref{ch4} and \ref{ch5}), \texttt{closure}, \texttt{special}, \texttt{builtin}, \texttt{environment}, \texttt{pairlist}, \texttt{S4}, \texttt{promise}, and \texttt{symbol} (Ch \ref{ch8}) and \texttt{raw} (Ch \ref{ch12}). Base types meant for C-internal processes, i.e., \texttt{any}, \texttt{bytecode}, \texttt{promise}, \texttt{...}, \texttt{weakref}, \texttt{externalptr}, and \texttt{char}, are not easily accessible with interpreted \textbf{R} code \citep{team2024r}.

\begin{table}[ht]
\centering
\caption{\textbf{R} base types. The listed base type would be printed if one created the assignment for \texttt{x} shown in the Example and typed \texttt{typeof(x)}. The function \texttt{mle}, used to create the Example for base type \texttt{S4}, requires the package \textit{stats4}.\vspace{0.1in}} 
\label{tab:Rbtypes}
\begin{tabular}{lll}
  \toprule
Base type & Example & Application \\ 
  \midrule
\texttt{NULL} & \texttt{x \textless- NULL} & vectors \\ 
  \texttt{logical} & \texttt{x \textless- TRUE} & vectors \\ 
  \texttt{integer} & \texttt{x \textless- 1L} & vectors \\ 
  \texttt{complex} & \texttt{x \textless- 1i} & vectors \\ 
  \texttt{double} & \texttt{x \textless- 1} & vectors \\ 
  \texttt{list} & \texttt{x \textless- list()} & vectors \\ 
  \texttt{character} & \texttt{x \textless- "a"} & vectors \\ 
  \texttt{raw} & \texttt{x \textless- raw(2)} & vectors \\ 
  \texttt{closure} & \texttt{x \textless- function(y)y + 1} & closure functions \\ 
  \texttt{special} & \texttt{x \textless- `[`} & special functions \\ 
  \texttt{builtin} & \texttt{x \textless- sum} & builtin functions \\ 
  \texttt{expression} & \texttt{x \textless- expression(x * 4)} & expressions \\ 
  \texttt{environment} & \texttt{x \textless- globalenv()} & environments \\ 
  \texttt{symbol} & \texttt{x \textless- quote(a)} & language components \\ 
  \texttt{language} & \texttt{x \textless- quote(a + 1)} & language components \\ 
  \texttt{pairlist} & \texttt{x \textless- formals(mean)} & language components \\ 
  \texttt{S4} & \texttt{x \textless- stats4::mle(function(x=1)x$\wedge$2)} & non-simple objects \\ 
  \texttt{any} & No example & C-internal \\ 
  \texttt{bytecode} & No example & C-internal \\ 
  \texttt{promise} & No example & C-internal \\ 
  \texttt{...} & No example & C-internal \\ 
  \texttt{weakref} & No example & C-internal \\ 
  \texttt{externalptr} & No example & C-internal \\ 
  \texttt{char} & No example & C-internal \\ 
   \bottomrule
\end{tabular}
\end{table}

\index[terms]{R!base type}
\index[terms]{R!base type!NULL}
\index[terms]{R!base type!logical}
\index[terms]{R!base type!integer}
\index[terms]{R!base type!complex}
\index[terms]{R!base type!double}
\index[terms]{R!base type!environment}
\index[terms]{R!base type!list}
\index[terms]{R!base type!character}
\index[terms]{R!base type!raw}
\index[terms]{R!base type!S4}
\index[terms]{R!base type!closure}
\index[terms]{R!base type!special}
\index[terms]{R!base type!builtin}
\index[terms]{R!base type!expression}
\index[terms]{R!base type!promise}
\index[terms]{R!base type!...}
\index[terms]{R!base type!weakref}
\index[terms]{R!base type!externalptr}
\index[terms]{R!base type!char}
\index[terms]{R!base type!symbol}
\index[terms]{R!base type!language}
\index[terms]{R!base type!pairlist}
\index[terms]{R!base type!any}
\index[terms]{R!base type!bytecode}

Base types of numeric objects define their \emph{storage mode}, i.e., the way \textbf{R} caches them in its primary memory\footnote{The functions \texttt{mode()} and \texttt{storage.mode()} are generally not appropriate for identifying \textbf{R} base types and
  storage modes \citep{wickham2019advanced}. In particular, the function \texttt{mode()} gives the mode of an object with respect to the \textbf{S3} system
  (see \citet{becker1988}), whereas \texttt{storage.mode()} is generally used when interfacing with algorithms written in other languages, primarily C or Fortran, to check that \textbf{R} objects have the correct type for the interfaced
  language.}. Base types can be identified using the function \texttt{typeof()}\index[R]{\texttt{typeof()}}.
\vspace{0.1in}

\begin{Shaded}
\begin{Highlighting}[]
\FunctionTok{typeof}\NormalTok{(x)}
\end{Highlighting}
\end{Shaded}

\begin{verbatim}
[1] "double"
\end{verbatim}

We see that \texttt{x} has storage mode \texttt{"double"}, meaning that its numeric values are stored using up to 53 bits, resulting in recognizable and distinguishable values between approximately \(5 \times 10^{-323}\) and \(2 \times 10^{307}\) (see Ch \ref{ch12} for more information).

\subsection{Object Attributes}\label{object-attributes}

Many \textbf{R}-objects will also have \emph{attributes} (i.e., characteristics particular to the object or object class). Typing:
\vspace{0.1in}

\begin{Shaded}
\begin{Highlighting}[]
\FunctionTok{attributes}\NormalTok{(x)}
\end{Highlighting}
\end{Shaded}

\begin{verbatim}
NULL
\end{verbatim}

indicates that \texttt{x} does not have additional attributes. However, using \emph{coercion} (Section \ref{s3coer}) we can define \texttt{x} to be an object of class \texttt{matrix} (a collection of data in a row and column format (see Section \ref{Sec3M})).
\vspace{0.1in}

\begin{Shaded}
\begin{Highlighting}[]
\FunctionTok{attributes}\NormalTok{(}\FunctionTok{as.matrix}\NormalTok{(x))}
\end{Highlighting}
\end{Shaded}

\begin{verbatim}
$dim
[1] 3 1
\end{verbatim}

Now \texttt{x} has the attribute \texttt{dim} (i.e., dimension). Specifically,
\texttt{x} is a three-celled matrix. It has three rows and one column.

Amazingly, classes and attributes allow \textbf{R} to simultaneously store and distinguish objects with the same name. For instance:
\vspace{0.1in}

\begin{Shaded}
\begin{Highlighting}[]
\NormalTok{mean }\OtherTok{\textless{}{-}} \FunctionTok{mean}\NormalTok{(}\FunctionTok{c}\NormalTok{(}\DecValTok{1}\NormalTok{, }\DecValTok{2}\NormalTok{, }\DecValTok{3}\NormalTok{))}
\NormalTok{mean}
\end{Highlighting}
\end{Shaded}

\begin{verbatim}
[1] 2
\end{verbatim}

\vspace{0.1in}

\begin{Shaded}
\begin{Highlighting}[]
\FunctionTok{mean}\NormalTok{(}\FunctionTok{c}\NormalTok{(}\DecValTok{1}\NormalTok{, }\DecValTok{2}\NormalTok{, }\DecValTok{3}\NormalTok{))}
\end{Highlighting}
\end{Shaded}

\begin{verbatim}
[1] 2
\end{verbatim}

In general, it is not advisable to name objects after frequently used functions. Nonetheless, the function \texttt{mean()}\index[R]{\texttt{mean()}}, which calculates the arithmetic mean of a collection of data, is distinguishable from the new user-created object \texttt{mean}, because these objects have different identifiable class characteristics. We can remove the user-created object \texttt{mean}, with the function \texttt{rm()}\index[R]{\texttt{rm()}}. This leaves behind only the function \texttt{mean()}.
\vspace{0.1in}

\begin{Shaded}
\begin{Highlighting}[]
\FunctionTok{rm}\NormalTok{(mean)}
\NormalTok{mean}
\end{Highlighting}
\end{Shaded}

\begin{verbatim}
function (x, ...) 
UseMethod("mean")
<bytecode: 0x0000026e3c53be88>
<environment: namespace:base>
\end{verbatim}

The process of how these objects are distinguished by \textbf{R} is further elaborated in Section \ref{howfunctionswork}.

\section{Getting Help}\label{help}

There is no single perfect source for information/documentation for all aspects of \textbf{R}. Detailed manuals from CRAN are available concerning the \textbf{R} \href{https://cran.r-project.org/doc/manuals/r-release/R-lang.html}{language definition}, \href{https://cran.r-project.org/doc/manuals/r-release/R-intro.html}{basic operations}, and \href{https://cran.r-project.org/doc/manuals/r-release/R-exts.html}{package development}. These resources, however, often assume a familiarity with Unix/Linux operating systems and computer science terminology. Thus, they may not be particularly helpful to biologists who are new to \textbf{R}.

\subsection{\texorpdfstring{\texttt{help()} and \texttt{?}}{help() and ?}}\label{help-and}

A comprehensive help system is available for many \textbf{R} components including operators, and loaded package dataframes and functions. The system can be accessed via the question mark, \texttt{?}, operator\index[R]{*q@\texttt{?}} and the function \texttt{help()}\index[R]{\texttt{help()}}. For instance, if I wanted to know more about the \texttt{plot()}\index[R]{\texttt{plot()}} function, I could type:
\vspace{0.1in}

\begin{Shaded}
\begin{Highlighting}[]
\NormalTok{?plot}
\end{Highlighting}
\end{Shaded}

or
\vspace{0.1in}

\begin{Shaded}
\begin{Highlighting}[]
\FunctionTok{help}\NormalTok{(plot)}
\end{Highlighting}
\end{Shaded}

Documentation for packaged \textbf{R} functions (Section \ref{package}) must include an annotated description of function arguments, along with other pertinent information, and documentation for packaged datasets must include descriptions of dataset variables\footnote{Chapter \ref{ch10} provides instructions on how to develop documentation files for your own packages.}. The quality of documentation will generally be excellent for functions from packages in the default \textbf{R} download (i.e., the \textbf{R}-distribution packages, see Section \ref{package}), but will vary from package to package otherwise. A list of arguments for a function, and their default values, can (often) be obtained using the function \texttt{formals()}.
\vspace{0.1in}

\begin{Shaded}
\begin{Highlighting}[]
\FunctionTok{formals}\NormalTok{(plot)}
\end{Highlighting}
\end{Shaded}

\begin{verbatim}
$x


$y


$...
\end{verbatim}

For help and documentation concerning programming metacharacters used in \textbf{R} (for instance \texttt{@}, \texttt{\#}, \texttt{?}, \texttt{!}, \texttt{\%}, \texttt{\&}, \texttt{\textbar{}}), one would enclose the metacharacters with quotes. For example, to find out more information about the logical operator \texttt{\&} I could type \texttt{help("\&")} or \texttt{?\ "\&"}. Placing two question marks in front of a topic will cause \textbf{R} to search for help files concerning with respect to all packages in a workstation. For instance, type:
\vspace{0.1in}

\begin{Shaded}
\begin{Highlighting}[]
\NormalTok{??lm}
\end{Highlighting}
\end{Shaded}

or, alternatively
\vspace{0.1in}

\begin{Shaded}
\begin{Highlighting}[]
\FunctionTok{help.search}\NormalTok{(lm)}
\end{Highlighting}
\end{Shaded}

for a huge number of help files on linear model functions identified through fuzzy matching. \index[terms]{Fuzzy matching} Help for particular \textbf{R}-questions can often be found online using the search engine at \url{http://search.r-project.org/}. This link is provided in the Help pulldown menu in the \textbf{R} console (non-Linux only). Helpful online discussions can also be found at \href{https://stackoverflow.com/questions/tagged/r}{Stack Overflow}, and \href{https://stats.stackexchange.com/}{Stats Exchange}.

\subsection{\texorpdfstring{\texttt{demo()} and \texttt{example()}}{demo() and example()}}\label{demo-and-example}

The function \texttt{demo()} \index[R]{\texttt{demo()}} allows one access to coded examples that developers have worked out for a particular function or topic. For instance, type:
\vspace{0.1in}

\begin{Shaded}
\begin{Highlighting}[]
\FunctionTok{demo}\NormalTok{(graphics)}
\end{Highlighting}
\end{Shaded}

for a brief demonstration of \textbf{R} graphics. Typing
\vspace{0.1in}

\begin{Shaded}
\begin{Highlighting}[]
\FunctionTok{demo}\NormalTok{(persp)}
\end{Highlighting}
\end{Shaded}

will provide a demonstration of 3D perspective plots. And, typing:
\vspace{0.1in}

\begin{Shaded}
\begin{Highlighting}[]
\FunctionTok{demo}\NormalTok{(Hershey)}
\end{Highlighting}
\end{Shaded}

will provide a demonstration of available modifiable symbols from the \emph{Hershey family of fonts} (see Ch 6 in \citet{hershey1967}). Finally, typing:
\vspace{0.1in}

\begin{Shaded}
\begin{Highlighting}[]
\FunctionTok{demo}\NormalTok{()}
\end{Highlighting}
\end{Shaded}

lists all of the demos available in the loaded libraries for a particular workstation. The function \texttt{example()}\index[R]{\texttt{example()}} usually provides less involved demonstrations from the \texttt{man} package directories (short for user manual, see Ch \ref{ch10}) in an \textbf{R} package. For instance, type:
\vspace{0.1in}

\begin{Shaded}
\begin{Highlighting}[]
\FunctionTok{example}\NormalTok{(plotmath)}
\end{Highlighting}
\end{Shaded}

for a coded demonstration of mathematical graphics.

\subsection{Vignettes}\label{vignettes}

\textbf{R} packages often contain \emph{vignettes}\index[terms]{R!vignette}. These are short documents that generally describe the theory underlying algorithms and guidance on how to correctly use package functions. Vignettes can be accessed with the function
\texttt{vignette()}\index[R]{\texttt{vignette()}}. To view all vignettes for all \emph{installed} packages (Section \ref{packagei}), type:
\vspace{0.1in}

\begin{Shaded}
\begin{Highlighting}[]
\FunctionTok{vignette}\NormalTok{(}\AttributeTok{all =} \ConstantTok{TRUE}\NormalTok{)}
\end{Highlighting}
\end{Shaded}

To view all vignettes available for \emph{loaded} packages (see Section \ref{packagel}), type:
\vspace{0.1in}

\begin{Shaded}
\begin{Highlighting}[]
\FunctionTok{vignette}\NormalTok{(}\AttributeTok{all =} \ConstantTok{FALSE}\NormalTok{)}
\end{Highlighting}
\end{Shaded}

To view vignettes for the \textbf{R} contributed package \emph{asbio} (following its installation), type:
\vspace{0.1in}

\begin{Shaded}
\begin{Highlighting}[]
\FunctionTok{vignette}\NormalTok{(}\AttributeTok{package =} \StringTok{"asbio"}\NormalTok{)}
\end{Highlighting}
\end{Shaded}

To see the vignette \texttt{simpson} in package \emph{asbio}, type:
\vspace{0.1in}

\begin{Shaded}
\begin{Highlighting}[]
\FunctionTok{vignette}\NormalTok{(}\StringTok{"simpson"}\NormalTok{, }\AttributeTok{package =} \StringTok{"asbio"}\NormalTok{)}
\end{Highlighting}
\end{Shaded}

The function \texttt{browseVignettes()} \index[R]{\texttt{browseVignettes()}} provides an HTML-browser that allows interactive vignette searches.

\section{Options}\label{options}

To enhance an \textbf{R} session, we can adjust the appearance of the \textbf{R}-console and customize options that affect expression output. These include the characteristics of the graphics devices, the width of print output in the \textbf{R}-console, and the number of print lines and print digits. Changes to some of these parameters can be made by going to \textbf{Edit\(>\)GUI Preferences} in the \textbf{R}-toolbar. Many other parameters can be changed using the \texttt{options()}\index[R]{\texttt{options()}} function. To see all alterable options one can type:
\index[R]{\texttt{options()}}
\vspace{0.1in}

\begin{Shaded}
\begin{Highlighting}[]
\FunctionTok{options}\NormalTok{()}
\end{Highlighting}
\end{Shaded}

The resulting list is extensive. To modify options, one would simply define the desired change within parentheses following a call to \texttt{options}. For instance, to see the default number of digits, I would type:

\vspace{0.1in}

\begin{Shaded}
\begin{Highlighting}[]
\FunctionTok{options}\NormalTok{(}\StringTok{"digits"}\NormalTok{)}
\end{Highlighting}
\end{Shaded}

\begin{verbatim}
$digits
[1] 7
\end{verbatim}

To change the default number of digits in output from 7 to 5 in the current session, I would type:
\vspace{0.1in}

\begin{Shaded}
\begin{Highlighting}[]
\FunctionTok{options}\NormalTok{(}\AttributeTok{digits =} \DecValTok{5}\NormalTok{)}
\CommentTok{\# demonstration using pi}
\NormalTok{pi}
\end{Highlighting}
\end{Shaded}

\begin{verbatim}
[1] 3.1416
\end{verbatim}

One can revert back to default options by restarting an \textbf{R} session.

\subsection{Advanced Options}\label{advanced-options}

To store user-defined options and start up procedures, an\texttt{.Rprofile} file \index[terms]{R!R-profile} will exist in your \textbf{R} program \textbf{etc} directory. This location would be something like: \(\ldots\)\textbf{R/R-version/etc}. \textbf{R} will silently run commands in the \texttt{.Rprofile} file upon opening. Thus, by customizing the \texttt{.Rprofile} file one can ``permanently'' set session options, load installed packages, define your favorite package repository (Section \ref{package}), and even create aliases and defaults for frequently used functions.

The \texttt{.Rprofile} file located in the \textbf{etc} directory is the so-called \texttt{.Rprofile.site} file. Additional \texttt{.Rprofile} files can be placed in the working directory (see below). \textbf{R} will check for these and run them after running the \texttt{.Rprofile.site} file.

\begin{example}
\protect\hypertarget{exm:exm-2options1}{}\label{exm:exm-2options1}\(\text{}\)\\
Here is the content of one of my current \texttt{.Rprofile} files.
\vspace{0.1in}

\begin{Shaded}
\begin{Highlighting}[numbers=left,,]
\FunctionTok{options}\NormalTok{(}\AttributeTok{repos =} \FunctionTok{structure}\NormalTok{(}\FunctionTok{c}\NormalTok{(}\StringTok{"http://ftp.osuosl.org/pub/cran/"}\NormalTok{)))}
\NormalTok{.First }\OtherTok{\textless{}{-}} \ControlFlowTok{function}\NormalTok{()\{}
\FunctionTok{library}\NormalTok{(asbio)}
\FunctionTok{cat}\NormalTok{(}\StringTok{"}\SpecialCharTok{\textbackslash{}n}\StringTok{Welcome to R Ken! "}\NormalTok{, }\FunctionTok{date}\NormalTok{(), }\StringTok{"}\SpecialCharTok{\textbackslash{}n}\StringTok{"}\NormalTok{)}
\NormalTok{\}}
\NormalTok{.Last }\OtherTok{\textless{}{-}} \ControlFlowTok{function}\NormalTok{()\{}
\FunctionTok{cat}\NormalTok{(}\StringTok{"}\SpecialCharTok{\textbackslash{}n}\StringTok{Goodbye Ken"}\NormalTok{, }\FunctionTok{date}\NormalTok{(), }\StringTok{"}\SpecialCharTok{\textbackslash{}n}\StringTok{"}\NormalTok{)}
\NormalTok{\}}
\end{Highlighting}
\end{Shaded}

The command \texttt{options(repos\ =\ structure(c("http://ftp.osuosl.org/pub/cran/")))} (Line 1) defines my preferred CRAN repository mirror site (Section \ref{package}). The function \texttt{.First(\ )} (Lines 2-5) will be run at the start of the \textbf{R} session and \texttt{.Last(\ )} (Lines 6-8) will be run at the end of the session. \textbf{R} functions will formally introduced in Ch \ref{ch8}. As we go through this book it will become clear that these lines of code force \textbf{R} to say hello, and to load the package \emph{asbio}, and print the date/time (using the function \texttt{date()}) when it opens, and to say goodbye, and print the date/time when it closes (although the farewell will only be seen when running \textbf{R} from a shell interface, e.g., the Windows Command Prompt).
\end{example}

\(\blacksquare\)

One can create \texttt{.Rprofile} files, and many other types of \textbf{R} extension files using the function \texttt{file.create()}\index[R]{\texttt{file.create()}}. For instance, the code:
\vspace{0.1in}

\begin{Shaded}
\begin{Highlighting}[]
\FunctionTok{file.create}\NormalTok{(}\StringTok{"defaults.Rprofile"}\NormalTok{)}
\end{Highlighting}
\end{Shaded}

will place an empty, editable,\texttt{.Rprofile} file called \texttt{defaults} in the working directory.

\index[R]{\texttt{.First()}}
\index[R]{\texttt{.Last()}}
\index[R]{\texttt{date()}}
\index[R]{\texttt{library()}}
\index[R]{\texttt{cat()}}
\index[R]{\texttt{file.create()}}
\index[R]{\texttt{function()}}

\section{The Working Directory}\label{the-working-directory}

By default, the \textbf{R} \emph{working directory}\index[terms]{Working directory} is set to be the home directory of the workstation. The command \texttt{getwd()} \index[R]{\texttt{getwd()}} shows the current file path for the working directory.
\vspace{0.1in}

\begin{Shaded}
\begin{Highlighting}[]
\FunctionTok{getwd}\NormalTok{()}
\end{Highlighting}
\end{Shaded}

\begin{verbatim}
[1] "C:/Users/ahoken/Documents/GitHub/Amalgam"
\end{verbatim}

The working directory can be changed with the command \texttt{setwd(filepath)}\index[R]{\texttt{setwd()}}, where \texttt{filepath} is the location of the desired directory, or by using pulldown menus, i.e., \textbf{File\(>\)Change dir} (non-Linux only). Because \textbf{R} developed under Unix, we must specify directory hierarchies using forward slashes or by doubling backslashes.

\begin{example}
\protect\hypertarget{exm:exm-2wd1}{}\label{exm:exm-2wd1}

\(\text{}\)\\
To establish a working directory file path to the Windows directory:
\textbf{C:\textbackslash Users\textbackslash User\textbackslash Documents}, I would type:
\vspace{0.1in}

\begin{Shaded}
\begin{Highlighting}[]
\FunctionTok{setwd}\NormalTok{(}\StringTok{"C:/Users/User/Documents"}\NormalTok{)}
\end{Highlighting}
\end{Shaded}

or
\vspace{0.1in}

\begin{Shaded}
\begin{Highlighting}[]
\FunctionTok{setwd}\NormalTok{(}\StringTok{"C:}\SpecialCharTok{\textbackslash{}\textbackslash{}}\StringTok{Users}\SpecialCharTok{\textbackslash{}\textbackslash{}}\StringTok{User}\SpecialCharTok{\textbackslash{}\textbackslash{}}\StringTok{Documents"}\NormalTok{)}
\end{Highlighting}
\end{Shaded}

\end{example}

\(\blacksquare\)

\section{Saving and Loading Your Work}\label{saving-and-loading-your-work}

As noted in Ch \ref{ch1}, an \textbf{R} session is allocated with a fixed amount of memory that is managed in an on-the-fly manner. An unfortunate consequence of this is that if \textbf{R} crashes, all unsaved information from the work session will be lost. Thus, session work should be saved often. Note that \textbf{R} will not give a warning if you are writing over session files from the \textbf{R} console. The old file will simply be replaced. Three general approaches for saving non-graphics data are possible. These are: 1) saving the history, 2) saving objects, and 3) saving \textbf{R} script. All three of these operations can be greatly facilitated by using an \textbf{R} \emph{integrated development environment} (\emph{IDE})\index[terms]{Integrated development environment (IDE)} like RStudio (Section \ref{RStudio}).

\subsection{R History}\label{r-history}

To view the \emph{history} (i.e., the commands that have been used in a session) one can use \texttt{history(n)}\index[R]{\texttt{history()}} where \texttt{n} is the number of previous command lines one wishes to see\footnote{Importantly, the functions \texttt{savehistory()}, \texttt{loadhistory()}, and \texttt{history()} are not currently supported for Mac OS. There are ways around this. For instance, in RStudio (Section \ref{RStudio}), the Mac OS command history can be obtained by clicking the \textbf{History} icon that appears on the tool bar at the top of the console window. As an additional issue, Windows and Unix-alike platforms have different implementations for \texttt{savehistory()} and \texttt{loadhistory()}. See help pages for these functions within your platform for particulars.}. For instance, to see the last three commands, one would type\footnote{This command will not work in an embedded Windows R GUI, like the one in RStudio.}:
\vspace{0.1in}

\begin{Shaded}
\begin{Highlighting}[]
\FunctionTok{history}\NormalTok{(}\DecValTok{3}\NormalTok{)}
\end{Highlighting}
\end{Shaded}

To save the session history in Windows one can use \textbf{File\(>\)Save History} or the function \texttt{savehistory()}\index[R]{\texttt{savehistory()}}. For instance, to save the session history to the working directory under the name \texttt{history1}, I could type:
\vspace{0.1in}

\begin{Shaded}
\begin{Highlighting}[]
\FunctionTok{savehistory}\NormalTok{(}\AttributeTok{file =} \StringTok{"history1.Rhistory"}\NormalTok{)}
\end{Highlighting}
\end{Shaded}

We can view the code in this file from any text editor. To load the history from a previous session one can use \textbf{File\(>\)Load History} (non-Linux only) or the function
\texttt{loadhistory()}\index[R]{\texttt{loadhistory()}}. For instance, to load \texttt{history1} I would type:
\vspace{0.1in}

\begin{Shaded}
\begin{Highlighting}[]
\FunctionTok{loadhistory}\NormalTok{(}\AttributeTok{file =} \StringTok{"history1.Rhistory"}\NormalTok{)}
\end{Highlighting}
\end{Shaded}

To save the history at the end of (almost) every interactive Windows or Unix-alike \textbf{R} session, one can alter the \texttt{.Rprofile} file \texttt{.Last} function to include:
\vspace{0.1in}

\begin{Shaded}
\begin{Highlighting}[]
\NormalTok{.Last }\OtherTok{\textless{}{-}} \ControlFlowTok{function}\NormalTok{() }\ControlFlowTok{if}\NormalTok{(}\FunctionTok{interactive}\NormalTok{()) }\FunctionTok{try}\NormalTok{(}\FunctionTok{savehistory}\NormalTok{(}\StringTok{"\textasciitilde{}/.Rhistory"}\NormalTok{))}
\end{Highlighting}
\end{Shaded}

\subsection{R Objects}\label{saveRobjects}

To save all of the objects available in the current \textbf{R}-session one can use \textbf{File\(>\)Save Workspace} (non-Linux only), or simply type:\index[R]{\texttt{save.image()}}
\vspace{0.1in}

\begin{Shaded}
\begin{Highlighting}[]
\FunctionTok{save.image}\NormalTok{()}
\end{Highlighting}
\end{Shaded}

This procedure saves session objects to the working directory as a nameless file using an \texttt{.RData} extension. The file will be opened, silently, with the inception of the next \textbf{R}- session, and cause objects used or created in the previous session to be available. Indeed, \textbf{R} will automatically execute all \texttt{.RData} files in the working directory for use in a session. Stored \texttt{.RData} files can also be loaded using \textbf{File\(>\)Load Workspace} (non-Linux only). One can also save \texttt{.RData} objects to a specific directory location and use a specific file name using: \textbf{File\(>\)Save Workspace}, or with flexible function \texttt{save()}\index[R]{\texttt{save()}}.
\textbf{R} data file formats, including .rda, and .RData, (extensions for \textbf{R} data files), and .R (the format for \textbf{R} scripts), can be read into \textbf{R} using the function \texttt{load()}\index[R]{\texttt{load()}}. \index[terms]{R!.RData file}\index[terms]{R!.rda file} Users new to a command line environment will be reassured by typing:
\vspace{0.1in}

\begin{Shaded}
\begin{Highlighting}[]
\FunctionTok{load}\NormalTok{(}\FunctionTok{file.choose}\NormalTok{())}
\end{Highlighting}
\end{Shaded}

The function \texttt{file.choose()}\index[R]{\texttt{file.choose()}} will allow one to browse interactively for files to load using dialog boxes. Detailed procedures for importing (reading) and exporting (saving) data with a row and column format, and an explicit delimiter (e.g.~.csv files) are described in Ch \ref{ch3}.

\subsection{R Scripts}\label{r-scripts}

To save an \textbf{R} \emph{script} as an executable source file, it is best to use an integrated development environment (IDE)\index[terms]{Integrated development environment (IDE)} compatible with \textbf{R}. \textbf{R} contains its own IDE, the \textbf{R}-editor, which is useful for writing, editing, and saving scripts as .r extension files\index[terms]{R!.r file}. To access the \textbf{R}-editor go to \textbf{File\(>\)New script} (non-Linux only) or type the shortcut Ctrl + F + N (Fig \ref{fig:Reditor}). Code written in the \textbf{R} IDE can be sent directly to the \textbf{R}-console by copying and pasting or by selecting code and using the shortcut Ctrl + R.\index[terms]{R!R-editor}

\begin{figure}[H]

{\centering \includegraphics[width=0.7\linewidth,]{figs2/REditor} 

}

\caption{The \textbf{R}-editor providing code for a famous computational exercise.}\label{fig:Reditor}
\end{figure}

Aside from the \textbf{R}-editor, a number of other IDEs outside of allow straightforward generation of \textbf{R} script files, and a direct link between text editors, that provide syntax highlighting for \textbf{R} code, and the \textbf{R}-console itself. These include RWinEdt (an \textbf{R} package plugin for \href{https://www.winedt.com/}{WinEdt} )\index[terms]{rwinedt@RWinEdt}, \href{http://www.sciviews.org/Tinn-R}{Tinn-R}, a recursive acronym for Tinn is not Notepad\index[terms]{tinn-r@Tinn-R}, \href{http://ess.r-project.org}{ESS} (Emacs Speaks Statistics)\index[terms]{ess@ESS}, \href{https://jupyter.org/}{Jupyter Notebook}\index[terms]{Jupyter notebook}, a web-based IDE originally designed for Python, but useful for many languages, and particularly \href{http://rstudio.org}{RStudio}, which will be introduced later in this chapter\footnote{Other text editors with at least some IDE support for \textbf{R} include, but are not limited to, NppToR in \href{http://sourceforge.net/projects/npptor}{Notepad++}, \href{http://bluefish.openoffice.nl/index.htm}{Bluefish}, \href{http://www.crimsoneditor.com/}{Crimson Editor}, \href{http://www.contexteditor.org/}{ConTEXT}, \href{http://www.eclipse.org/eclipse/}{Eclipse}, \href{http://www.vim.org/}{Vim}, \href{http://www.geany.org/}{Geany}, \href{http://www.jedit.org/}{jEdit}, \href{http://kate-editor.org/}{Kate}, \href{http://macromates.com/}{TextMat}, \href{http://projects.gnome.org/gedit/}{gedit}, and \href{http://www.scintilla.org/SciTE.html}{SciTE}.}.

Saved \textbf{R} scripts can be called and executed using the function \texttt{source()}\index[R]{\texttt{source()}}. To browse interactively for source code files, one can type:
\vspace{0.1in}

\begin{Shaded}
\begin{Highlighting}[]
\FunctionTok{source}\NormalTok{(}\FunctionTok{file.choose}\NormalTok{())}
\end{Highlighting}
\end{Shaded}

or go to \textbf{File\(>\)Source R code}.

\section{Basic Mathematics}\label{basic-mathematics}

A large number of mathematical operators and functions are available with a conventional download of \textbf{R}.

Elementary mathematical operators, common mathematical constants, trigonometric functions, derivative functions, integration approaches, and basic statistical functions are shown in shown in Tables \ref{tab:commath} - \ref{tab:confint}.

\subsection{Elementary Operations}\label{elementary-operations}

\index[R]{*+a0@\texttt{+}}
\index[R]{*+a01@\texttt{-}}
\index[R]{*+a1@\texttt{*}}
\index[R]{*+a2@\texttt{\scalebox{.8}{\textsuperscript{$\wedge$}}}}
\index[R]{\texttt{sqrt()}}
\index[R]{\texttt{log()}}
\index[R]{\texttt{factorial()}}
\index[R]{\texttt{gamma()}}
\index[R]{\texttt{choose()}}
\index[R]{\texttt{floor()}}
\index[R]{\texttt{ceiling()}}
\index[R]{\texttt{round()}}
\index[R]{\texttt{abs()}}
\index[R]{*f@\texttt{\%\%}}
\index[R]{*f@\texttt{\%/\%}}
\index[terms]{R!mathematics}
\index[R]{\texttt{sum()}}
\index[R]{\texttt{prod()}}
\index[R]{\texttt{cumsum()}}

\begin{landscape}
\begin{table}[ht]
\centering
\caption{Elementary mathematical operators and functions in \textbf{R}.  For all functions \texttt{x} represents a scalar or a numeric vector.\vspace{0.1in}} 
\label{tab:commath}
\begin{tabular}{llll}
  \toprule
Operator & Operation & To find: & We type: \\ 
  \midrule
\texttt{+} & addition & $2 + 2$ & \texttt{2 + 2} \\ 
  \texttt{-} & subtraction & $2 - 2$ & \texttt{2 - 2} \\ 
  \texttt{*} & multiplication & $2 \times 2$ & \texttt{2 * 2} \\ 
  \texttt{/} & division & $\frac{2}{3}$ & \texttt{2/3} \\ 
  \texttt{\%\%} & modulo & remainder of $\frac{5}{2}$ & \texttt{5\%\%2} \\ 
  \texttt{\%/\%} & integer division & $\frac{5}{2}$ without remainder & \texttt{5\%/\%2} \\ 
  \texttt{$\wedge$} & exponentiation & $2^3$ & \texttt{2$\wedge$3} \\ 
  \texttt{abs(x)} & $\mid x \mid$ & $\mid -23.7 \mid$ & \texttt{abs(-23.7)} \\ 
  \texttt{round(x, digits = d)} & round $x$ to $d$ digits & round $-23.71$ to 1 digit & \texttt{round(-23.71, 1)} \\ 
  \texttt{ceiling(x)} & round $x$ up to closest whole num. & ceiling(2.3) & \texttt{ceiling(2.3)} \\ 
  \texttt{floor(x)} & round $x$ down to closest whole num. & floor(2.3) & \texttt{floor(2.3)} \\ 
  \texttt{sqrt(x)} & $\sqrt{x}$ & $\sqrt{2}$ & \texttt{sqrt(2)} \\ 
  \texttt{log(x)} & $\log_e{x}$ & $\log_e{5}$ & \texttt{log(5)} \\ 
  \texttt{log(x, base = b)} & $\log_b{x}$ & $\log_{10}{5}$ & \texttt{log(5, base = 10)} \\ 
  \texttt{factorial(x)} & $x!$ & $5!$ & \texttt{factorial(5)} \\ 
  \texttt{gamma(x)} & $\Gamma(x)$ & $\Gamma(3.2)$ & \texttt{gamma(3.2)} \\ 
  \texttt{choose(n,x)} & $\binom{n}{x}$ & $\binom{5}{2}$ & \texttt{choose(5,2)} \\ 
  \texttt{sum(x)} & $\sum_{i=1}^{n}x_i$ & sum of \texttt{x} & \texttt{sum(x)} \\ 
  \texttt{cumsum(x)} & cumulative sum & cum. sum of \texttt{x} & \texttt{cumsum(x)} \\ 
  \texttt{prod(x)} & $\prod_{i=1}^{n}x_i$ & product of \texttt{x} & \texttt{prod(x)} \\ 
  \texttt{cumprod(x)} & cumulative product & cum. prod. of \texttt{x} & \texttt{cumprod(x)} \\ 
   \bottomrule
\end{tabular}
\end{table}
\end{landscape}

\subsection{Associativity and Precedence}\label{associativity-and-precedence}

Note that the operation:
\vspace{0.1in}

\begin{Shaded}
\begin{Highlighting}[]
\DecValTok{2} \SpecialCharTok{+} \DecValTok{6} \SpecialCharTok{*} \DecValTok{5}
\end{Highlighting}
\end{Shaded}

\begin{verbatim}
[1] 32
\end{verbatim}

is equivalent to \(2 + (6 \cdot 5) = 32\). This is because the \texttt{*} operator gets higher priority (precedence) than \texttt{+}. Evaluation precedence can be modified with parentheses:
\vspace{0.1in}

\begin{Shaded}
\begin{Highlighting}[]
\NormalTok{(}\DecValTok{2} \SpecialCharTok{+} \DecValTok{6}\NormalTok{) }\SpecialCharTok{*} \DecValTok{5}
\end{Highlighting}
\end{Shaded}

\begin{verbatim}
[1] 40
\end{verbatim}

In the absence of operator \emph{precedence}, mathematical operations in \textbf{R} are (generally) read from left to right (that is, their \emph{associativity} is from left to right) (Table \ref{tab:prec}). This corresponds to the conventional order of operations in mathematics. For instance:
\vspace{0.1in}

\begin{Shaded}
\begin{Highlighting}[]
\DecValTok{2} \SpecialCharTok{+} \DecValTok{2}\SpecialCharTok{\^{}}\NormalTok{(}\DecValTok{2} \SpecialCharTok{+} \DecValTok{1}\NormalTok{)}
\end{Highlighting}
\end{Shaded}

\begin{verbatim}
[1] 10
\end{verbatim}

\begin{table}[ht]
\centering
\caption{Precedence and associativity of mathematical operators.  Operators are listed from highest to lowest precendece in operations.\vspace{0.1in}} 
\label{tab:prec}
\begin{tabular}{llll}
  \toprule
Precedent & Operator & Description & Associativity \\ 
  \midrule
1 & \texttt{$\wedge$} & exponent & right to left \\ 
  2 & \texttt{\%\%} & modulo & left to right \\ 
  3 & \texttt{* /} & multiplication, division & left to right \\ 
  4 & \texttt{+ -} & addition, subtraction & left to right \\ 
   \bottomrule
\end{tabular}
\end{table}

\index[terms]{Operator associativity}
\index[terms]{Operator precedence}

\subsection{Function Arguments}\label{function-arguments}

\textbf{R} functions generally require a user to specify \emph{arguments} (in parentheses) following the function name. For instance, \texttt{sqrt()} and \texttt{factorial()} each require one argument, a call to data itself. Thus, to solve \(1/\sqrt{22!}\), I could type:
\vspace{0.1in}

\begin{Shaded}
\begin{Highlighting}[]
\DecValTok{1}\SpecialCharTok{/}\FunctionTok{sqrt}\NormalTok{(}\FunctionTok{factorial}\NormalTok{(}\DecValTok{22}\NormalTok{))}
\end{Highlighting}
\end{Shaded}

\begin{verbatim}
[1] 2.9827e-11
\end{verbatim}

To solve \(\Gamma \left( \sqrt[3]{23\pi} \right)\), I could type:
\vspace{0.1in}

\begin{Shaded}
\begin{Highlighting}[]
\FunctionTok{gamma}\NormalTok{((}\DecValTok{23} \SpecialCharTok{*}\NormalTok{ pi)}\SpecialCharTok{\^{}}\NormalTok{(}\DecValTok{1}\SpecialCharTok{/}\DecValTok{3}\NormalTok{))}
\end{Highlighting}
\end{Shaded}

\begin{verbatim}
[1] 7.411
\end{verbatim}

By default the function \texttt{log()}\index[R]{\texttt{log()}} computes natural logarithms, i.e.,
\vspace{0.1in}

\begin{Shaded}
\begin{Highlighting}[]
\FunctionTok{log}\NormalTok{(}\FunctionTok{exp}\NormalTok{(}\DecValTok{1}\NormalTok{))}
\end{Highlighting}
\end{Shaded}

\begin{verbatim}
[1] 1
\end{verbatim}

The \texttt{log()} function can also compute logarithms to a particular base by specifying the base in an optional second argument called \texttt{base}. For instance, to solve the operation: \(\log_{10}3 + \log_{3}5\), one could type:
\vspace{0.1in}

\begin{Shaded}
\begin{Highlighting}[]
\FunctionTok{log}\NormalTok{(}\DecValTok{3}\NormalTok{, }\DecValTok{10}\NormalTok{) }\SpecialCharTok{+} \FunctionTok{log}\NormalTok{(}\DecValTok{5}\NormalTok{, }\DecValTok{3}\NormalTok{)}
\end{Highlighting}
\end{Shaded}

\begin{verbatim}
[1] 1.9421
\end{verbatim}

Arguments\index[terms]{R!function!argument} can be specified by the order that they occur in the list of arguments in the function code, or by calling the argument by name. In the code above I know that the first argument in \texttt{log()} is a call to data, and the second argument defines the base. I may not, however, remember the argument order in a function, or may wish to only change certain arguments from a large allotment. In this case it is better to specify an argument by calling its name and defining its value with an equals sign.
\vspace{0.1in}

\begin{Shaded}
\begin{Highlighting}[]
\FunctionTok{log}\NormalTok{(}\AttributeTok{x =} \DecValTok{3}\NormalTok{, }\AttributeTok{base =} \DecValTok{10}\NormalTok{) }\SpecialCharTok{+} \FunctionTok{log}\NormalTok{(}\AttributeTok{x =} \DecValTok{5}\NormalTok{, }\AttributeTok{base =} \DecValTok{3}\NormalTok{)}
\end{Highlighting}
\end{Shaded}

\begin{verbatim}
[1] 1.9421
\end{verbatim}

\index[R]{\texttt{pi}}\index[R]{\texttt{exp()}}\index[R]{\texttt{Inf}}

\subsection{Constants}\label{constants}

\textbf{R} allows easy access to most conventional constants (Table \ref{tab:c}).
\index[terms]{R!mathematics!constants}

\begin{table}[ht]
\centering
\caption{Conventional constants in \textbf{R}.\vspace{0.2in}} 
\label{tab:c}
\begin{tabular}{llll}
  \toprule
Operator & Operation & To find: & We type: \\ 
  \midrule
\texttt{-Inf} & $-\infty$ & $-\infty$ & \texttt{-Inf} \\ 
  \texttt{Inf} & $\infty$ & $\infty$ & \texttt{Inf} \\ 
  \texttt{pi} & $\pi = 3.141593 \dots$ & $\pi$ & \texttt{pi} \\ 
  \texttt{exp(1)} & $e = 2.718282 \dots$ & $e$ & \texttt{exp(1)} \\ 
  \texttt{exp(x)} & $e^x$ & $e^3$ & \texttt{exp(3)} \\ 
   \bottomrule
\end{tabular}
\end{table}

\subsection{Trigonometry}\label{trigonometry}

\index[terms]{R!mathematics!trigonometry}
\textbf{R} assumes that the inputs for trigonometric functions are in radians. Of course degrees can be obtained from
radians using \(Degrees = Radians \times 180/\pi\), or conversely \(Radians = Degrees \times \pi /180\) (Table \ref{tab:trigmath}).
\index[R]{\texttt{sin()}}\index[R]{\texttt{cos()}}\index[R]{\texttt{tan()}}\index[R]{\texttt{asin()}}\index[R]{\texttt{acos()}}\index[R]{\texttt{atan()}}

\begin{table}[ht]
\centering
\caption{Trigonometric functions in \textbf{R}. For all functions \texttt{x} represents a scalar or a numeric vector.} 
\label{tab:trigmath}
\begin{tabular}{llll}
  \toprule
Operator & Operation & To find: & We type: \\ 
  \midrule
\texttt{cos(x)} & $\text{cos}(x)$ & $\text{cos}(3 \text{ rad.})$ & \texttt{cos(3)} \\ 
  \texttt{sin(x)} & $\text{sin}(x)$ & $\text{sin}(45^{\circ})$ & \texttt{sin(45 * pi/180)} \\ 
  \texttt{tan(x)} & $\text{tan}(x)$ & $\text{tan}(3 \text{ rad.})$ & \texttt{tan(3)} \\ 
  \texttt{acos(x)} & $\text{acos}(x)$ & $\text{acos}(45^{\circ})$ & \texttt{acos(45 * pi/180)} \\ 
  \texttt{asin(x)} & $\text{asin}(x)$ & $\text{asin}(3 \text{ rad.})$ & \texttt{asin(3)} \\ 
  \texttt{atan(x)} & $\text{atan}(x)$ & $\text{atan}(45^{\circ})$ & \texttt{atan(45 * pi/180)} \\ 
   \bottomrule
\end{tabular}
\end{table}

\subsection{Derivatives}\label{derivatives}

The function \texttt{D()}\index[R]{\texttt{D()}}\index[terms]{R!mathematics!derivatives} finds symbolic and numerical derivatives of simple expressions. It requires two arguments, a mathematical function specified as an expression (i.e., an object of class and base type \texttt{expression}, created using the function \texttt{expression()}, that can be evaluated with the function \texttt{eval()}\index[R]{\texttt{eval()}}), and the denominator in the difference quotient. Here is an example of how functions \texttt{expression}\index[R]{\texttt{expression()}} and \texttt{eval()} are used:
\vspace{0.1in}

\begin{Shaded}
\begin{Highlighting}[]
\FunctionTok{eval}\NormalTok{(}\FunctionTok{expression}\NormalTok{(}\DecValTok{2} \SpecialCharTok{+} \DecValTok{2}\NormalTok{))}
\end{Highlighting}
\end{Shaded}

\begin{verbatim}
[1] 4
\end{verbatim}

Of course we wouldn't bother to use \texttt{expression()} and \texttt{eval()} in such simple applications. Table \ref{tab:Dmath} contains specific examples using \texttt{D()}.
\index[terms]{R!classes!expression}
\index[terms]{R!base type!expression}

\begin{table}[ht]
\centering
\caption{Conventional constants in \textbf{R}.\vspace{0.1in}} 
\label{tab:Dmath}
\begin{tabular}{ll}
  \toprule
To find: & We type: \\ 
  \midrule
$\frac{d}{dx}5x$ & \texttt{D(expression(5 * x), "x")} \\ 
  $\frac{d^2}{dx^2} 5x^2$ & \texttt{D(D(expression(5 * x$\wedge$2), "x"), "x")} \\ 
  $\frac{\partial}{\partial x} 5xy + y$ & \texttt{D(expression(5 * x * y + y), "x")} \\ 
   \bottomrule
\end{tabular}
\end{table}

\subsection{Integration}\label{integration}

The function \texttt{integrate}\index[R]{\texttt{integrate()}}\index[terms]{R!mathematics!integrals} solves definite integrals. It requires three arguments. The first is an \textbf{R} function defining the integrand. The second and third are the lower and upper bounds of integration.

\begin{example}
\protect\hypertarget{exm:exm-2I1}{}\label{exm:exm-2I1}

\(\text{}\)\\
To solve:
\[\int^4_2 3x^2dx\]
we could type:
\vspace{0.1in}

\begin{Shaded}
\begin{Highlighting}[]
\NormalTok{f }\OtherTok{\textless{}{-}} \ControlFlowTok{function}\NormalTok{(x)\{}\DecValTok{3} \SpecialCharTok{*}\NormalTok{ x}\SpecialCharTok{\^{}}\DecValTok{2}\NormalTok{\}}
\FunctionTok{integrate}\NormalTok{(f, }\DecValTok{2}\NormalTok{, }\DecValTok{4}\NormalTok{)}
\end{Highlighting}
\end{Shaded}

\begin{verbatim}
56 with absolute error < 6.2e-13
\end{verbatim}

\end{example}

\textbf{R} functions are explicitly addressed in Ch \ref{ch8}.
\index[R]{\texttt{function()}}
\#\#\# Statistics
\textbf{R}, of course, contains a huge number of statistical functions\index[terms]{R!mathematics!statistics}. These will generally require sample data for summarization. Data can be brought into \textbf{R} from spreadsheet files or other data storage files (we will learn how to do this shortly). As we have learned, data can also be assembled in \textbf{R}. For instance,
\vspace{0.1in}

\begin{Shaded}
\begin{Highlighting}[]
\NormalTok{x }\OtherTok{\textless{}{-}} \FunctionTok{c}\NormalTok{(}\DecValTok{1}\NormalTok{, }\DecValTok{2}\NormalTok{, }\DecValTok{3}\NormalTok{)}
\end{Highlighting}
\end{Shaded}

Statistical estimators can be separated into \emph{point estimators}, which estimate an underlying parameter that has a single true value (from a Frequentist viewpoint), \index[terms]{Point estimators} and \emph{intervallic estimators}, which estimate the bounds of an interval that is expected, preceding sampling, to contain a parameter at some probability\index[terms]{Intervallic estimator} \citep{aho2014}. Point estimators can be further classified as estimators of location, scale, shape, and order statistics (Table \ref{tab:statmath}). Measures of \emph{location} estimate the typical or central value from a sample. \index[terms]{Point estimator!location estimator} Examples include the arithmetic mean and the sample median\index[terms]{Point estimator!location estimator! sample mean}\index[terms]{Point estimator!location estimator!sample median}. Measures of \emph{scale} quantify data variability or dispersion.\index[terms]{Point estimator!scale estimator} Examples include the sample standard deviation and the sample interquartile range (IQR). \index[terms]{Point estimator!scale estimator!sample variance}\index[terms]{Point estimator!scale estimator!sample IQR} \emph{Shape estimators} describe the shape (i.e., symmetry and peakedness) of a data distribution. Examples include the sample skewness and sample kurtosis.\index[terms]{Point estimator!shape estimator!sample skewness}\index[terms]{Point estimator!shape estimator!sample kurtosis} Finally, the \(k\)th order statistic of a sample is equal to its \(k\)th-smallest value. Examples include the data minimum, the data maximum, and other quantiles (including the median).\index[terms]{Point estimator!order statistic!min}\index[terms]{Point estimator!order statistic!max} Intervallic estimators include confidence intervals (Table \ref{tab:confint}). A huge number of other statistical estimating, modelling, and hypothesis testing algorithms are also available for the \textbf{R} environment. For guidance, see \citet{venables2002}, \citet{aho2014}, and \citet{fox2019}, among others.

\begin{landscape}
\begin{table}[ht]
\centering
\caption{Simple point estimators in \textbf{R}. The term \texttt{x} represents a numeric data vector, and \texttt{y} represents a numeric data vector whose
  elements are paired with those in \texttt{x}. The cipher \texttt{asbio::} indicates that the function is located in the package \textit{asbio}.\vspace{0.1in}} 
\label{tab:statmath}
\begin{tabular}{llll}
  \toprule
Function & Acronym & Description & Estimator type \\ 
  \midrule
\texttt{mean(x)} & $\bar{x}$ & arithmetic mean of $x$ & location \\ 
  \texttt{mean(x, trim = t)} &  & trimmed mean of $x$ for $0 \leq t \leq 1$. & location \\ 
  \texttt{asbio::G.mean(x)} & $GM$ & geometric mean of $x$ & location \\ 
  \texttt{asbio::H.mean(x)} & $HM$ & harmonic mean of $x$ & location \\ 
  \texttt{median(x)} & $\tilde{x}$ & median of $x$ & location order statistic \\ 
  \texttt{asbio::Mode(x)} & $mode(x)$ & mode of $x$ & location \\ 
  \texttt{sd(x)} & $s$ & standard deviation of $x$ & scale \\ 
  \texttt{var(x)} & $s^2$ & variance of $x$ & scale \\ 
  \texttt{cov(x, y)} & $cov(x,y)$ & covariance of $x$ and $y$ & scale \\ 
  \texttt{cor(x, y)} & $r_{x,y}$ & Pearson correlation of $x$ and $y$ & scale \\ 
  \texttt{IQR(x)} & $IQR$ & interquartile range of $x$ & scale order statistic \\ 
  \texttt{mad(x)} & $MAD$ & median absolute deviation of $x$ & scale \\ 
  \texttt{asbio::skew(x)} & $g_1$ & skew of $x$ & shape \\ 
  \texttt{asbio::kurt(x)} & $g_2$ & kurtosis of $x$ & shape \\ 
  \texttt{min(x)} & $min(x)$ & min of $x$ & order statistic \\ 
  \texttt{max(x)} & $max(x)$ & max of $x$ & order statistic \\ 
  \texttt{quantile(x, prob = p)} & $\hat{F}^{-1}(p)$ & quantile of $x$ at lower-tailed probability $p$ & order statistic \\ 
   \bottomrule
\end{tabular}
\end{table}



\index[R]{\texttt{mean()}}
\index[R]{\texttt{min()}}
\index[R]{\texttt{median()}}
\index[R]{\texttt{max()}}
\index[R]{\texttt{sd()}}
\index[R]{\texttt{var()}}
\index[R]{\texttt{IQR()}}
\index[R]{\texttt{asbio::G.mean()}}
\index[R]{\texttt{asbio::H.mean()}}
\index[R]{\texttt{asbio::Mode()}}
\index[R]{\texttt{asbio::skew()}}
\index[R]{\texttt{asbio::kurt()}}
\index[R]{\texttt{quantile()}}
\index[R]{\texttt{cov()}}
\index[R]{\texttt{cor()}}


\end{landscape}

\begin{table}[ht]
\centering
\caption{Some intervallic estimators in \textbf{R}. The term \texttt{x} represents a numeric vector. The cipher \texttt{asbio::} indicates that the function is located in the package \textit{asbio}\vspace{0.1in}} 
\label{tab:confint}
\begin{tabular}{ll}
  \toprule
Function & Description \\ 
  \midrule
\texttt{asbio::ci.mu.z(x, conf, sigma)} & Conf. int. for $\mu$ at level \texttt{conf}. True SD = \texttt{sigma}. \\ 
  \texttt{asbio::ci.mu.t(x, conf)} & Conf. int. for $\mu$ at level \texttt{conf}. $\sigma$ unknown. \\ 
  \texttt{asbio::ci.median(x, conf)} & Conf. int. for true median at level \texttt{conf}. \\ 
   \bottomrule
\end{tabular}
\end{table}

\section{RStudio}\label{RStudio}

\emph{RStudio} is an open source IDE for \textbf{R} (Fig \ref{fig:Rstudio1})\index[terms]{RStudio}\index[terms]{Integrated development environment (IDE)}. RStudio greatly facilitates writing \textbf{R} code, saving and examining \textbf{R} objects and history, and many other processes. These include, but are not limited to, documenting session workflows, writing \textbf{R} package documentation, calling and receiving code from other languages, and even developing web-based graphical user interfaces. RStudio can currently be downloaded at (\url{https://posit.co/products/open-source/rstudio/}). Like \textbf{R} itself, RStudio can be used with Windows, Mac, and Unix/Linux operating systems, RStudio has both freeware and commercial versions\footnote{On 7/27/2022 RStudio announced it was shifting to a new name, \href{https://posit.co/products/open-source/rstudio//index}{Posit}, to acknowledge its growth beyond a simple IDE for \textbf{R}. The RStudio name will be retained for RStudio Desktop, and the RStudio Server, but it will be changed for other applications including the RStudio Workbench (now Posit Workbench) and the RStudio Package Manager (now Posit Package Manager).}. We will use the former here.\index[terms]{Posit (new RStudio name)}

\begin{figure}[H]

{\centering \includegraphics[width=0.5\linewidth,]{figs2/Rstudio} 

}

\caption{The RStudio logo.}\label{fig:Rstudio1}
\end{figure}

RStudio is generally implemented using a four pane workspace (Fig \ref{fig:Rstudio2}). These are: 1) the code editor, 2) the \textbf{R}-console, 3) the environment and histories panel, and 4) the plots and other miscellany panel.

\begin{figure}[H]

{\centering \includegraphics[width=0.9\linewidth,]{figs2/rstudio_summary} 

}

\caption{Interfaces for RStudio 2023.06.2 Build 561.}\label{fig:Rstudio2}
\end{figure}

\begin{itemize}
\item
  The RStudio code editor panel (Fig \ref{fig:Rstudio2}, Panel 1) allows you to create \textbf{R} scripts and scripts for other languages that can be called to and from \textbf{R}. The code panel can also be used to create and edit session documentation files (see Section \ref{workflow} below) and other important \textbf{R} file types. A new \textbf{R} script can be created for editing within the code editor by going to \textbf{File\(>\)New\(>\)R Script}. Commands from an \textbf{R} script can be sent to the \textbf{R} console using the shortcut Ctrl + Enter (Windows and Linux) or Cmd + Enter (Mac).
\item
  The \textbf{R} console panel (Fig \ref{fig:Rstudio2}, Panel 2) by default, is identical in functionality to the \textbf{R} console of the most recent version of \textbf{R} on your workstation (assuming that all of the paths and environments are set up correctly on your computer). Thus, the console panel can be used directly for typing and executing \textbf{R} code, or for receiving commands from the code editor (Panel 1).
\item
  The environments and history panel (Fig \ref{fig:Rstudio2}, Panel 3) can be used to: 1) show a list of \textbf{R} objects available in your \textbf{R} session (the \textbf{Environment} tab), or 2) show, search, and select from the \emph{history} of all previous commands (\textbf{History} tab). This panel also provides an interface for point and click import of data files including .csv, .xls, and many other file formats (\textbf{Import Dataset} pulldown within the \textbf{Environment} tab).
\item
  The plots and files panel (Fig \ref{fig:Rstudio2}, Panel 4) can be used to show: 1) files in the working directory, 2) a scrollable history of plots and image files, and 3) a list of available packages (via the \textbf{Packages} tab), with facilities for \emph{updating} and \emph{installing} packages. If a package is in the GUI list, then the package is currently \emph{loaded}. Packages and their installation, updating, and loading are formally introduced in Section \ref{package}. The panel's \textbf{Files} pulldown tab allows straightforward establishment of working directories (although this can still be done at the command line using \texttt{setwd()})\index[R]{\texttt{setwd()}} (Fig \ref{fig:RstudioMenu1}). The panel's \textbf{Help} tap opens automatically when uses \texttt{?} or \texttt{help} for particular \textbf{R} topics (Section \ref{help}).
\end{itemize}

\begin{blackbox}

\begin{center}
\textbf{CAUTION!}

\end{center}

Be very careful when managing files in the plots and files panel, as you can permanently delete files without (currently) the possibility of recovery from a Recycling Bin.

\end{blackbox}

\subsection{RStudio Project}\label{rstudio-project}

An RStudio \emph{project} can be be created via the \textbf{File} pulldown menu (Fig \ref{fig:RstudioMenu1}). A project allows all related files (data, figures, summaries, etc.) to be easily organized together by setting the working directory to be the location of the project .Rproj file.
\index[terms]{RStudio!project}

\subsection{Workflow Documentation}\label{workflow}

We can document workflow and simultaneously run/test \textbf{R} session code by either:

\begin{enumerate}
\def\labelenumi{\arabic{enumi}.}
\tightlist
\item
  creating an \textbf{R} Markdown\index[terms]{RStudio!R Markdown} .rmd file\index[terms]{R!.rmd file} that can be compiled to make an .html, .pdf, or MS Word\(^{\circledR}\) .doc document\footnote{Markdown is a highly flexible language for creating formatted text using a plain-text editor. HyperText Markup Language or HTML\index[terms]{HTML} is the standard markup language for documents designed for web browser display.}, or
\item
  using Sweave\index[terms]{R!.rnw file}, an approach that implements the LaTeX\index[terms]{latex@\LaTeX} (pronounced lay-tek) document preparation system\index[terms]{Sweave}.
\end{enumerate}

\subsubsection{R Markdown}\label{RMarkdown}

The \textbf{R} Markdown document processing workflow in RStudio is shown Fig \ref{fig:RMarkdown}. These steps are highly modifiable, but can also be run in a more or less automated manner, requiring little understanding of underlying processes.

\begin{figure}[H]

{\centering \includegraphics[width=0.95\linewidth,]{figs2/Markdown_process} 

}

\caption{The process of document creation in \textbf{R} Markdown. Functions in the package \textit{rmarkdown} control conversion of .rmd files to Markdown .md files, using utilities in the package \textit{knitr}. The Pandoc program first creates a .tex file when rendering LaTeX PDF documents.}\label{fig:RMarkdown}
\end{figure}

Use of \textbf{R} Markdown and .rmd files requires the package \emph{rmarkdown} \citep{allaire}, which comes pre-installed in RStudio.

As an initial step, all underlying .rmd files must include a brief YAML\footnote{YAML is a data serialization language. The YAML acronym was originally intended to mean ``Yet Another Markdown Language,'' but more recently has been given the recursive acronym: ``YAML Ain't Markup Language.'' \textbf{R} Markdown uses the YAML format header to communicate with Pandoc, a document converter, written in the Haskell language\index[terms]{Haskell (programming language)}, embedded in RStudio, to create the desired document output.} header (see below) containing document metadata. The remainder of the .rmd document will contain text written in \href{https://en.wikipedia.org/wiki/Markdown}{Markdown} syntax, and code chunks. The \texttt{knit()} function from package \emph{knitr} \citet{xie2015md}, also installed with RStudio, executes all evaluable code within chunks, and formats the code and output for processing within \href{https://en.wikipedia.org/wiki/Pandoc}{Pandoc}, a program for converting markup files from one language to another\footnote{Pandoc can convert Markdown .md files, into many formats including, .rtf, .doc, and .pdf}. Pandoc uses the YAML header to guide this conversion. As an example, if one has requested HTML output, the simple Markdown text: \texttt{This\ is\ a\ script} will be converted to the HTML formatted: \texttt{\textless{}p\textgreater{}This\ is\ a\ script\textless{}/p\textgreater{}}. One can also write HTML script or CSS code\footnote{\emph{Cascading Style Sheets} (CSS) is a language for styling output for files written in markup languages including HTML and Extensible Markup Language (XML), .}\index[terms]{CSS} directly into an .rmd document (see Section \ref{shiny}).
\index[terms]{R!package!\textit{knitr}}
\index[terms]{R!package!\textit{rmarkdown}}
If the desired output is PDF, Pandoc will convert the .md file into an intermediate .tex file. This file is then processed by \href{https://www.latex-project.org/}{LaTeX}, an open source, high-quality scientific typesetting system\footnote{Support for LaTeX can be found at the and at a large number of informal user-driven venues, including \href{https://tex.stackexchange.com}{Stack Exchange} and \href{https://www.overleaf.com/}{Overleaf}, an online LaTeX application}. LaTeX compiles the .tex file into a .pdf file. In this process, the \emph{tinytex} package \citep{xie2024tiny}, which installs the stripped-down LaTeX distribution \href{https://yihui.org/tinytex/}{TinyTex}, can be used.
\index[terms]{R!package!\textit{tinytex}}

Creating an \textbf{R} Markdown document is simple in RStudio. We first open an empty .rmd document by navigating to \textbf{File \(>\)} \textbf{New File \(>\)R Markdown} (Fig \ref{fig:RstudioMenu1}).

\begin{figure}[H]

{\centering \includegraphics[width=0.7\linewidth,]{figs2/menu1_1} 

}

\caption{Part of the RStudio {\textbf{File}} pulldown menu.}\label{fig:RstudioMenu1}
\end{figure}

You will delivered to the GUI shown in Fig \ref{fig:RstudioMenu2}. Note that by default Markdown compilation generates an HTML document.

\begin{figure}[H]

{\centering \includegraphics[width=0.7\linewidth,]{figs2/menu2_11} 

}

\caption{RStudio GUI for creating an {\textbf{R}} Markdown document.}\label{fig:RstudioMenu2}
\end{figure}

The GUI opens a \textbf{R} Markdown (.rmd) skeleton document with a tentative YAML header.

\begin{figure}[H]

{\centering \includegraphics[width=0.3\linewidth,]{figs2/YAML1} 

}

\caption{YAML header to an {\textbf{R}} Markdown (.rmd) skeleton document.}\label{fig:YAML}
\end{figure}

\noindent  The HTML output can be changed to one of:
\vspace{0.1in}

\begin{Shaded}
\begin{Highlighting}[]
\NormalTok{output}\SpecialCharTok{:}\NormalTok{ pdf\_document}
\end{Highlighting}
\end{Shaded}

\noindent to create a LaTex \(\rightarrow\) PDF, or
\vspace{0.1in}

\begin{Shaded}
\begin{Highlighting}[]
\NormalTok{output}\SpecialCharTok{:}\NormalTok{ word\_document}
\end{Highlighting}
\end{Shaded}

to create a Word\(^{\circledR}\) document\footnote{Many other \emph{rmarkdown} output types are possible including: \texttt{html\_vignette}, which provides vignette formatting appropriate for inclusion in \textbf{R} packages, \texttt{ioslides\_presentation}, \texttt{slidy\_presentation}, for an HTML-styled slideshow layout, \texttt{beamer\_presentation}, for a PDF slideshow using LaTex \(\rightarrow\) \href{https://ctan.org/pkg/beamer?lang=en}{Beamer} formatting, and \texttt{powerpoint\_presentation} for a MS Powerpoint\(^{\circledR}\) slideshow.}.
\index[R]{\texttt{rmarkdown::html\_vignette()}}
\index[R]{\texttt{rmarkdown::ioslides\_presentation()}}
\index[R]{\texttt{rmarkdown::slidy\_presentation()}}
\index[R]{\texttt{rmarkdown::beamer\_presentation()}}
\index[R]{\texttt{rmarkdown::powerpoint\_presentation()}}

The \emph{knitr} package facilitates report building in both HTML and LaTeX \(\rightarrow\) PDF formats, within the framework of \emph{rmarkdown} (Fig \ref{fig:RMarkdown}). Under \emph{knitr}, \textbf{R} Markdown lines beginning \texttt{\textasciigrave{}\textasciigrave{}\textasciigrave{}\{r\ \}} and ending \texttt{\textasciigrave{}\textasciigrave{}\textasciigrave{}} delimit an \textbf{R} code ``chunk'' to be potentially run\index[terms]{RStudio!chunk} in the \textbf{R} environment. The chunk header, \texttt{\textasciigrave{}\textasciigrave{}\textasciigrave{}\{r\ \}}, can contain additional options. For a complete list of chunk options, run

\vspace{0.1in}

\begin{Shaded}
\begin{Highlighting}[]
\FunctionTok{str}\NormalTok{(knitr}\SpecialCharTok{::}\NormalTok{opts\_chunk}\SpecialCharTok{$}\FunctionTok{get}\NormalTok{())}
\end{Highlighting}
\end{Shaded}

\index[R]{\texttt{knitr::opts\_chunk()}}
\index[terms]{R!package!\textit{knitr}}
\index[terms]{R!package!\textit{rmarkdown}}
Code chunks can be generated by going to \textbf{Code\(>\)Insert Chunk} or by using the RStudio shortcut Ctrl + Alt + I (Windows and Linux) or Cmd + Alt + I (Mac). \textbf{R} code can also be invoked inline in a \textbf{R} Markdown document using the format:

\vspace{0.1in}

\begin{Shaded}
\begin{Highlighting}[]
\StringTok{\textasciigrave{}}\AttributeTok{r some code}\StringTok{\textasciigrave{}}
\end{Highlighting}
\end{Shaded}

For instance, I could seamlessly place three random numbers generated from a the continuous uniform distribution, \(f(x) = UNIF(0,1)\), inline into text using:\\
\vspace{0.1in}

\begin{Shaded}
\begin{Highlighting}[]
\StringTok{\textasciigrave{}}\AttributeTok{r runif(3)}\StringTok{\textasciigrave{}}
\end{Highlighting}
\end{Shaded}

Here I run an iteration using ``hidden'' inline \textbf{R} code: 0.93582, 0.20912, 0.96909.

In Markdown, pound signs (e.g., \texttt{\#}, \texttt{\#\#}, \texttt{\#\#\#}) can be used as (increasingly nested) hierarchical section delimiters. Italic, bold, and monospace code fonts can be specified by enclosing text in asterisks, double asterisks, and back ticks, respectively. That is: *italic*, and **bold** result in: \emph{italic} and \textbf{bold}. A brief introduction to \textbf{R} Markdown can be found at: \url{http://rmarkdown.rstudio.com}. A thorough description of \textbf{R} Markdown is given in \citet{xie2018r} and \citet{xie2020}. The latter text is currently available as an \href{https://bookdown.org/yihui/rmarkdown-cookbook/}{online resource}.

Inline equations for both Markdown and Sweave (discussed below) can be specified under the LaTeX system, which uses dollar signs, \texttt{\$}, to delimit equations. For instance, to obtain the inline equation: \(P(\theta|y) = \frac{P(y|\theta)P(\theta)}{P(y)}\), i.e., Bayes theorem, I could type the LaTeX script:

\texttt{\$(\textbackslash{}theta\textbar{}y)\ =\ \textbackslash{}frac\{P(y\textbar{}\textbackslash{}theta)P(\textbackslash{}theta)\}\{P(y)\}\$}

\noindent A cheatsheet for LaTeX equation writing can be found \href{https://quickref.me/latex}{here}.

The \textbf{R} Markdown (.rmd) skeleton file has example documentation text, interspersed with example \textbf{R} code in chunks. These been have been modified below to create a simple summary document for the dataset \texttt{Loblolly} from the package \emph{datasets} (Fig \ref{fig:RStudiolob1}), which describes growth characteristics of loblolly pine trees (\emph{Pinus taeda}).

\begin{figure}[H]

{\centering \includegraphics[width=0.8\linewidth,]{figs2/loblollyCode1} 

}

\caption{An {\textbf{R}} Markdown (.rmd) file with documentation text and interspersed {\textbf{R}} code in chunks.}\label{fig:RStudiolob1}
\end{figure}

Note the use of \texttt{echo\ =\ FALSE} in the final chunk to suppress printing of \textbf{R} code. A snapshot of the knitted HTML is shown in Fig \ref{fig:RStudiolob2}.

\begin{figure}[H]

{\centering \includegraphics[width=0.8\linewidth,]{figs2/loblolly1} 

}

\caption{An HTML document knit from Markdown code in the previous figure. Note that  code is displayed (by default) as well as executed.}\label{fig:RStudiolob2}
\end{figure}

I generally use the function \texttt{knitr::kable()} to create \textbf{R} Markdown \(\rightarrow\) Pandoc \(\rightarrow\) HTML tables because it is relatively simple to use. The code below was used to create Table \ref{tab:lobtab}.
\vspace{0.1in}

\begin{Shaded}
\begin{Highlighting}[]
\NormalTok{knitr}\SpecialCharTok{::}\FunctionTok{kable}\NormalTok{(}\FunctionTok{head}\NormalTok{(Loblolly))}
\end{Highlighting}
\end{Shaded}

\begin{table}

\caption{\label{tab:lobtab}Loblolly pine data}
\centering
\begin{tabular}[t]{l|r|r|l}
\hline
  & height & age & Seed\\
\hline
1 & 4.51 & 3 & 301\\
\hline
15 & 10.89 & 5 & 301\\
\hline
29 & 28.72 & 10 & 301\\
\hline
43 & 41.74 & 15 & 301\\
\hline
57 & 52.70 & 20 & 301\\
\hline
71 & 60.92 & 25 & 301\\
\hline
\end{tabular}
\end{table}

I often use functions in the package \emph{xtable} to build \textbf{R} Markdown \(\rightarrow\) Pandoc \(\rightarrow\) LaTeX \(\rightarrow\) PDF tables. Under this approach, one could create Table \ref{tab:lobtab} using:

\begin{Shaded}
\begin{Highlighting}[]
\FunctionTok{print}\NormalTok{(xtable}\SpecialCharTok{::}\FunctionTok{xtable}\NormalTok{(}\FunctionTok{head}\NormalTok{(Loblolly)))}
\end{Highlighting}
\end{Shaded}

\index[R]{\texttt{xtable::xtable()}}
This method would also require that one use the command \texttt{results\ =\ \textquotesingle{}asis\textquotesingle{}} in the chunk options. One can even call for different table approaches on the fly. For instance, I could use the command \texttt{eval\ =\ knitr::is\_html\_output())}, in the options of a Markdown chunk when using table code that optimizes HTML formatting, and use \texttt{eval\ =\ knitr::is\_latex\_output())} to create a table that optimizes LaTeX formatting.
\index[R]{\texttt{knitr::is\_html\_output()}}
\index[R]{\texttt{knitr::is\_latex\_output()}}
Aside from \texttt{knitr::kable()} and \emph{xtable}, there are many other \textbf{R} functions and packages that can be used to create \textbf{R} Markdown tables, particularly for HTML output. These include:

\begin{itemize}
\tightlist
\item
  The \emph{kableExtra} \citep{zhu2022} package extends \texttt{knitr::kable()} by including styles for fonts, features for specific rows, columns, and cells, and straightforward merging and grouping of rows and/or columns. Most \emph{kableExtra} features extend to both HTML and PDF formats.
\item
  \emph{DT} \citep{xieDT}, a wrapper for HTML tables that uses the JavaScript (see Section \ref{JS}) library \emph{DataTables}. Among other features, \emph{DT} allows straightforward implementation in interactive Shiny apps (Section \ref{shiny}).
\item
  Like \emph{DT}, the \emph{reactable} package \citep{lin2023} creates flexible, interactive HTML embedded tables. As with \emph{DT}, \emph{reactable} tables add complications when those interactives are considered as conventional tables in \textbf{R} markdown, with captions and referable labels.
\end{itemize}

\citet{xie2020} discuss several other alternatives.
\index[terms]{R!package!\textit{kableExtra}}
\index[terms]{R!package!\textit{DT}}
\index[terms]{R!package!\textit{reactable}}
\index[terms]{R!package!\textit{xtable}}

\index[terms]{JavaScript (JS) (programming language)}
\index[terms]{JavaScript (JS) (programming language)!package!\textit{DataTable}}

\paragraph{Bookdown}\label{bookdown}

\index[terms]{R!package!\textit{bookdown}}
A large number of useful auxiliary features are available for \textbf{R} Markdown, through the \textbf{R} package \emph{bookdown} (\citet{xie2023}). These include an extended capacity for figure, table, and section numbering and referencing. To use \emph{bookdown} we must modify the \texttt{output:} designation in the YAML header to have a \emph{bookdown}-specific output. For instance,
\vspace{0.1in}

\begin{Shaded}
\begin{Highlighting}[]
\NormalTok{output}\SpecialCharTok{:}\NormalTok{ bookdown}\SpecialCharTok{::}\NormalTok{html\_document2}
\end{Highlighting}
\end{Shaded}

to create an HTML document, or\\
\vspace{0.1in}

\begin{Shaded}
\begin{Highlighting}[]
\NormalTok{output}\SpecialCharTok{:}\NormalTok{ bookdown}\SpecialCharTok{::}\NormalTok{pdf\_document2}
\end{Highlighting}
\end{Shaded}

to create LaTeX \(\rightarrow\) PDF document. or
\vspace{0.1in}

\begin{Shaded}
\begin{Highlighting}[]
\NormalTok{output}\SpecialCharTok{:}\NormalTok{ bookdown}\SpecialCharTok{::}\NormalTok{word\_document2}
\end{Highlighting}
\end{Shaded}

to create an MS Word\(^{\circledR}\) document\footnote{Many other \emph{bookdown} extensions to \emph{rmarkdown} output types are possible including: \texttt{html\_vignette2}, \texttt{ioslides\_presentation2}, \texttt{slidy\_presentation2}, \texttt{beamer\_presentation2}, and \texttt{powerpoint\_presentation2}.}.
\index[R]{\texttt{bookdown::html\_vignette2()}}
\index[R]{\texttt{bookdown::ioslides\_presentation2()}}
\index[R]{\texttt{bookdown::slidy\_presentation2()}}
\index[R]{\texttt{bookdown::beamer\_presentation2()}}
\index[R]{\texttt{bookdown::powerpoint\_presentation2()}}

Numbering \textbf{R}-generated plots and tables in \textbf{R} Markdown or Bookdown requires specification of a chunk label after the language reference \texttt{r} in the chunk generating the plot. In the chunk below I use the label \texttt{lobplot}. Note that a space is included after \texttt{r}. Captions are specified in the chunk header using the chunk option \texttt{fig.cap} or \texttt{tab.cap} for figures and tables, respectively. For instance,

\texttt{\textasciigrave{}\textasciigrave{}\textasciigrave{}\{r\ lobplot,\ echo=FALSE,\ fig.cap=\ "Loblolly\ pine\ height\ versus\ age."\}}

Cross-references within the text can be made using the syntax \texttt{\textbackslash{}@ref(type:label)}, where \texttt{label} is the chunk label and \texttt{type} is the environment being referenced (e.g., \texttt{fig}, \texttt{tab}, or \texttt{eq}). For the current example, we might want to type something like: ``see Figure \texttt{\textbackslash{}@\ ref(fig:lobplot)}''. in some non-chunk component of the Markdown document.

Specification of a bookdown output format, will result in automated numbering of sections. To turn this numbering off, one could modify the YAML output to be:
\vspace{0.1in}

\begin{Shaded}
\begin{Highlighting}[]
\NormalTok{output}\SpecialCharTok{:}
\NormalTok{  bookdown}\SpecialCharTok{::}\NormalTok{html\_document2}\SpecialCharTok{:}
\NormalTok{    number\_sections}\SpecialCharTok{:}\NormalTok{ false}
\end{Highlighting}
\end{Shaded}

The code indents shown above are important because YAML, like the language Python, uses \emph{significant indentation}\index[terms]{Significant indentation}. To omit numbering for \emph{certain} sections, one would retain the bookdown output, and add \texttt{\{-\}} after the unnumbered section heading, e.g.,

\texttt{\#\ This\ section\ is\ unnumbered\ \{-\}}

Thorough guidance for \emph{bookdown} is provided in \citet{xie2016}, which can be viewed as an open-source \href{https://bookdown.org/yihui/bookdown/}{online document}.

\subsubsection{Sweave}\label{Sweave}

Under the Sweave documentation approach, high quality .pdf documents are generated from LaTeX .tex files, which in turn are created from Sweave .rnw files. This can also be facilitated with RStudio. A skeleton .rnw document can be generated by going to \textbf{File\(>\)New File\(>\)R Sweave}\footnote{The document you are reading was either knitted from an \textbf{R} Markdown .rmd file (using \emph{bookdown}) or a Sweave .rnw file, created in RStudio.}. In Fig \ref{fig:sweave1} I create an .rnw file with the text and analyses used in the Markdown example above (Figs \ref{fig:RStudiolob1}-\ref{fig:RStudiolob2}). We note that instead of the Markdown YAML header, we now have lines in the preamble defining the type of desired document (e.g., article) and the LaTeX packages needed for document compilation (e.g, \emph{amsmath}). Note that \textbf{R} code chunks are now initiated by \texttt{\textless{}\textless{}\textgreater{}\textgreater{}=}, which serves as a chunk header and can contain options, and closed with \texttt{@}. Non-code text, including figure and table captions and cross-referencing should follow LaTeX guidelines.

\begin{figure}[H]

{\centering \includegraphics[width=0.7\linewidth,]{figs2/sweavecode} 

}

\caption{A Sweave (.rnw) file with documentation text and interspersed code in chunks.}\label{fig:sweave1}
\end{figure}

Fig \ref{fig:sweave2} shows a snapshot of the .pdf result, following Sweave/LaTeX compilation.

\begin{figure}[H]

{\centering \includegraphics[width=0.8\linewidth,]{figs2/sweavedoc3} 

}

\caption{A .pdf document resulting from compilation of Sweave code in the previous figure.}\label{fig:sweave2}
\end{figure}

\subsubsection{Purl}\label{purl}

\textbf{R} code can be extracted from an .rmd or or an .rnw file using the function \texttt{knitr::purl()}. For instance, assume that the \textbf{R} Markdown loblolly pine summary shown in Fig \ref{fig:RStudiolob1} is saved in the working directory under the name \texttt{lob.rmd}. Code from the file will be extracted to a script file called \texttt{lob.R}, located in the working directory, if one types:
\vspace{0.1in}

\begin{Shaded}
\begin{Highlighting}[]
\FunctionTok{purl}\NormalTok{(}\StringTok{"lob.rmd"}\NormalTok{)}
\end{Highlighting}
\end{Shaded}

\index[R]{\texttt{knitr::purl()}}

\section*{Exercises}\label{exercises-1}
\addcontentsline{toc}{section}{Exercises}

\begin{enumerate}
\def\labelenumi{\arabic{enumi}.}
\tightlist
\item
  Create an \textbf{R} Markdown document to contain your homework assignment. Modify the YAML header to allow numbering of figures and tables, but not sections. To test the formatting, perform the following steps:

  \begin{enumerate}
  \def\labelenumii{(\alph{enumii})}
  \tightlist
  \item
    Create section header called \texttt{Question\ 1} and a subsection header called (a). Under (a) type \texttt{"completed"}.
  \item
    Under the subsection header (b), insert a chunk, and create a simple plot of points at the coordinates: \(\{1,1\}\), \(\{2,2\}\), \(\{3,3\}\), by typing the code: \texttt{plot(1:3)} in the chunk. Create a label for the chunk, and a create caption for plot using the \emph{knitr} chunk option, \texttt{fig.cap}.
  \item
    Under the subsection header (c), create a cross reference for the plot from (b).
  \item
    Under the subsection header (d), write the equation, \(y_i = \hat{\beta}_0 + \hat{\beta}_1x_i + \hat{\varepsilon_i}\), using LaTeX. As noted earlier, a LaTeX equation cheatsheet can be found \href{https://quickref.me/latex}{here}.
  \item
    Render (knit) the final document as either an .html file or a .doc file. \emph{Include other assigned exercises for this Chapter as directed, using the general formatting approach given in Question 1.}
    \vspace{0.1in}
  \end{enumerate}
\item
  Perform the following operations.

  \begin{enumerate}
  \def\labelenumii{(\alph{enumii})}
  \tightlist
  \item
    Leave a note to yourself.
  \item
    Create and examine an object called \texttt{x} that contains the numeric entries 1, 2, and 3.
  \item
    Make a copy of \texttt{x} called \texttt{y}.
  \item
    Show the class of \texttt{y}.
  \item
    Show the base type of \texttt{y}.
  \item
    Show the attributes of \texttt{y}.
  \item
    List the current objects in your work session.
  \item
    Identify your working directory.
    \vspace{0.1in}
  \end{enumerate}
\item
  Distinguish \textbf{R} expressions and assignments.
  \vspace{0.1in}
\item
  Sometimes \textbf{R} reports unexpected results for its classes and base types.

  \begin{enumerate}
  \def\labelenumii{(\alph{enumii})}
  \tightlist
  \item
    Create \texttt{x\ \textless{}-\ factor("a","a","b")} and show the class of \texttt{x}.
  \item
    Type \texttt{?factor}. What is a \texttt{factor} in \textbf{R}?
  \item
    Show the base type of \texttt{x}? Is this surprising? Why? Type \texttt{?integer}. What is an \texttt{integer} in \textbf{R}?
    \vspace{0.1in}
  \end{enumerate}
\item
  Solve the following mathematical operations using \textbf{R}.

  \begin{enumerate}
  \def\labelenumii{(\alph{enumii})}
  \tightlist
  \item
    \(1 + 3/10 + 2\)
  \item
    \((1 + 3)/10 + 2\)
  \item
    \(\left(4 \cdot \frac{(3 - 4)}{23}\right)^2\)
  \item
    \(\log_2(3^{1/2})\)
  \item
    \(3\boldsymbol{x}^3 + 3\boldsymbol{x}^2 + 2\) where \(\boldsymbol{x} = \{0, 1.5, 4, 6, 8, 10\}\)
  \item
    \(4(\boldsymbol{x} + \boldsymbol{y})\) where \(\boldsymbol{x} = \{0, 1.5, 4, 6, 8\}\) and \(\boldsymbol{y} = \{-2, 0.5, 3, 5, 8\}\).
  \item
    \(\frac{d}{dx} \tan(x) 2.3 \cdot e^{3x}\)
  \item
    \(\frac{d^2}{dx^2} \frac{3}{4x^4}\)
  \item
    \(\int_3^{12} 24x + \ln(x)dx\)
  \item
    \(\int_{-\infty}^{\infty} \frac{1}{\sqrt{2\pi}}e^{-\frac{x^2}{2}}dx\) (i.e., find the area under a standard normal pdf).
  \item
    \(\int_{-\infty}^{\infty}\frac{x}{\sqrt{2\pi}}e^{-\frac{x^2}{2}}dx\) (i.e., find \(E(X)\) for a standard normal pdf).
  \item
    \(\int_{-\infty}^{\infty}\frac{x^2}{\sqrt{2\pi}}e^{-\frac{x^2}{2}}dx\) (i.e., find \(E(X^2)\) for a standard normal pdf).
  \item
    Find the sum, cumulative sum, product, cumulative product, arithmetic mean, median and variance of the data \texttt{x\ =\ c(0,\ 1.5,\ 4,\ 6,\ 8,\ 10)}.
    \vspace{0.1in}
  \end{enumerate}
\item
  The velocity of the earth's rotation on its axis at the equator, \(E\), is approximately 1674.364 km/h, or 1040.401 m/h\footnote{The circumference of the earth at the equator is 40,075.02 km (24,901.5 mi). The earth completes one full rotation on its axis with respect to distant stars in 23 hours 56 minutes 4.091 seconds (a sidereal day). This means that in 24 hours, the earth rotates \(\frac{24}{23 + (56/60) + (4.091/60)/60} = 1.002738\) times. And this means that the velocity of the earth at the equator is \(\frac{1.002738 \times 40075.02}{24} = 1674.364\) k\(\cdot\)h\(^{-1}\), or \(0.621371 \times 1674.364 = 1040.401\) m\(\cdot\)h\(^{-1}\).}. We can calculate the velocity of the rotation of the earth at any latitude with the equation, \(V = \cos(\)latitude\(^\text{o}) \times E\). Using \textbf{R}, simultaneously calculate rotational velocities for latitudes of 0,30,60, and 90 degrees north, or south, latitude (they will be the same). Remember, the function \texttt{cos()} assumes inputs are in radians, not degrees.
\end{enumerate}

\chapter{Data Objects, Packages, and Datasets}\label{ch3}

\begin{quote}
\emph{"In God we trust. All others {[}must{]} have data."}
\end{quote}

- \textbf{Edwin R. Fisher}, \emph{cancer pathologist}

\section{Data Storage Objects}\label{data-storage-objects}

There are five primary types of data storage objects in \textbf{R}. These are: (atomic) vectors, matrices, arrays, dataframes, and lists\footnote{Note that distinctions of these objects are not always clear or consistent. For instance, a \texttt{names} attribute can be given to elements of vectors and lists, and columns of dataframes. However, only names from dataframes and lists can be made visible using \texttt{attach}, or called using \texttt{\$}. See additional examples \href{https://adv-r.hadley.nz/vectors-chap.html}{here}.}.

\subsection{Atomic Vectors}\label{vectors}

\emph{Atomic vectors} contain data with order and length, but no dimension. This is clearly different from the linear algebra conception of a vector.\index[terms]{R!vector} In linear algebra, a row vector with \(n\) elements has dimension \(1 \times n\) (1 row and \(n\) columns), whereas a column vector has dimension \(n \times 1\).

We can create atomic vectors with the function \texttt{c}, which means \emph{combine}.
\index[R]{\texttt{c()}}
\index[R]{\texttt{is.vector()}}
\index[R]{\texttt{dim()}}

\vspace{0.1in}

\begin{Shaded}
\begin{Highlighting}[]
\NormalTok{x }\OtherTok{\textless{}{-}} \FunctionTok{c}\NormalTok{(}\DecValTok{1}\NormalTok{, }\DecValTok{2}\NormalTok{, }\DecValTok{3}\NormalTok{)}
\FunctionTok{is.vector}\NormalTok{(x)}
\end{Highlighting}
\end{Shaded}

\begin{verbatim}
[1] TRUE
\end{verbatim}

\vspace{0.1in}

\begin{Shaded}
\begin{Highlighting}[]
\FunctionTok{length}\NormalTok{(x)}
\end{Highlighting}
\end{Shaded}

\begin{verbatim}
[1] 3
\end{verbatim}

\vspace{0.1in}

\begin{Shaded}
\begin{Highlighting}[]
\FunctionTok{dim}\NormalTok{(x)}
\end{Highlighting}
\end{Shaded}

\begin{verbatim}
NULL
\end{verbatim}

We can add a names attribute to vector elements. For example,

\vspace{0.1in}

\begin{Shaded}
\begin{Highlighting}[]
\NormalTok{x }\OtherTok{\textless{}{-}} \FunctionTok{c}\NormalTok{(}\AttributeTok{a =} \DecValTok{1}\NormalTok{, }\AttributeTok{b =} \DecValTok{2}\NormalTok{, }\AttributeTok{c =} \DecValTok{3}\NormalTok{)}
\end{Highlighting}
\end{Shaded}

\vspace{0.1in}

\begin{Shaded}
\begin{Highlighting}[]
\NormalTok{x}
\end{Highlighting}
\end{Shaded}

\begin{verbatim}
a b c 
1 2 3 
\end{verbatim}

\vspace{0.1in}

\begin{Shaded}
\begin{Highlighting}[]
\FunctionTok{attributes}\NormalTok{(x)}
\end{Highlighting}
\end{Shaded}

\begin{verbatim}
$names
[1] "a" "b" "c"
\end{verbatim}

\vspace{0.1in}

\begin{Shaded}
\begin{Highlighting}[]
\FunctionTok{attr}\NormalTok{(x, }\StringTok{"names"}\NormalTok{) }\CommentTok{\# or  names(x)}
\end{Highlighting}
\end{Shaded}

\begin{verbatim}
[1] "a" "b" "c"
\end{verbatim}

\index[R]{\texttt{attributes()}} \index[R]{\texttt{names()}} \index[R]{\texttt{attr()}}
The function \texttt{as.matrix(x)} (see Section \ref{s3coer}) can be used to coerce \texttt{x} to have a matrix structure with dimension \(3 \times 1\) (3 rows and 1 column). Thus, in \textbf{R} a matrix has dimension, but a vector does not.

\vspace{0.1in}

\begin{Shaded}
\begin{Highlighting}[]
\FunctionTok{dim}\NormalTok{(}\FunctionTok{as.matrix}\NormalTok{(x))}
\end{Highlighting}
\end{Shaded}

\begin{verbatim}
[1] 3 1
\end{verbatim}

Recall (Section \ref{bt}) that an object's base type defines the (\textbf{R} internal) type or storage mode of any object. Recall further that the 25 base types include \texttt{"integer"}, \texttt{"double"}, \texttt{"complex"}, and \texttt{"character"}. Elements of vectors must have a single data storage mode. Thus, a vector cannot contain both numeric and character data.
\index[terms]{R!base type}

Importantly, when an element-wise operation is applied to two unequal length vectors, \textbf{R} will generate a warning and automatically recycle elements of the shorter vector. For instance:

\vspace{0.1in}

\begin{Shaded}
\begin{Highlighting}[]
\FunctionTok{c}\NormalTok{(}\DecValTok{1}\NormalTok{, }\DecValTok{2}\NormalTok{, }\DecValTok{3}\NormalTok{) }\SpecialCharTok{+} \FunctionTok{c}\NormalTok{(}\DecValTok{1}\NormalTok{, }\DecValTok{0}\NormalTok{, }\DecValTok{4}\NormalTok{, }\DecValTok{5}\NormalTok{, }\DecValTok{13}\NormalTok{)}
\end{Highlighting}
\end{Shaded}

\begin{verbatim}
Warning in c(1, 2, 3) + c(1, 0, 4, 5, 13): longer object length is not a
multiple of shorter object length
\end{verbatim}

\begin{verbatim}
[1]  2  2  7  6 15
\end{verbatim}

In this case, the result of the addition of the two vectors is: \(1 + 1, 2 + 0, 3 + 4, 1 + 5\), and \(2 + 13\). Thus, the first two elements in the first object are recycled in the vector-wise addition.

\subsection{Matrices}\label{Sec3M}

\emph{Matrices} are two-dimensional (row and column) data structures whose elements all have the same data storage mode (typically \texttt{"double"}).

The function \texttt{matrix()}\index[R]{\texttt{matrix()}} can be used to create matrices.

\vspace{0.1in}

\begin{Shaded}
\begin{Highlighting}[]
\NormalTok{A }\OtherTok{\textless{}{-}} \FunctionTok{matrix}\NormalTok{(}\AttributeTok{ncol =} \DecValTok{2}\NormalTok{, }\AttributeTok{nrow =} \DecValTok{2}\NormalTok{, }\AttributeTok{data =} \FunctionTok{c}\NormalTok{(}\DecValTok{1}\NormalTok{, }\DecValTok{2}\NormalTok{, }\DecValTok{3}\NormalTok{, }\DecValTok{2}\NormalTok{))}
\NormalTok{A}
\end{Highlighting}
\end{Shaded}

\begin{verbatim}
     [,1] [,2]
[1,]    1    3
[2,]    2    2
\end{verbatim}

Note that \texttt{matrix()} enters data ``by column.'' That is, the first two entries in the \texttt{data} argument are placed in column one, and the last two entries are placed in column two. One can enter data ``by row'' by adding the argument \texttt{byrow\ =\ TRUE}.\\
\vspace{0.1in}

\begin{Shaded}
\begin{Highlighting}[]
\NormalTok{B }\OtherTok{\textless{}{-}} \FunctionTok{matrix}\NormalTok{(}\AttributeTok{ncol =} \DecValTok{2}\NormalTok{, }\AttributeTok{nrow =} \DecValTok{2}\NormalTok{, }\AttributeTok{data =} \FunctionTok{c}\NormalTok{(}\DecValTok{1}\NormalTok{, }\DecValTok{2}\NormalTok{, }\DecValTok{3}\NormalTok{, }\DecValTok{2}\NormalTok{), }\AttributeTok{byrow =} \ConstantTok{TRUE}\NormalTok{)}
\NormalTok{B}
\end{Highlighting}
\end{Shaded}

\begin{verbatim}
     [,1] [,2]
[1,]    1    2
[2,]    3    2
\end{verbatim}

\emph{Matrix algebra} operations can be applied directly to \textbf{R} matrices (Table \ref{tab:boo1}). More complex matrix analyses are also possible, including spectral decomposition (function \texttt{eigen()}), and single value, QR, and Cholesky decompositions (the functions \texttt{svd()}, \texttt{qr()}, \texttt{chol()}, respectively).\\
\index[R]{\texttt{eigen()}}
\index[R]{\texttt{svd()}}
\index[R]{\texttt{qr()}}
\index[R]{\texttt{chol()}}
\index[terms]{R!classes!matrix}
\index[terms]{R!classes!array}
\index[terms]{Matrix algebra}

\begin{table}[ht]
\centering
\caption{Simple matrix algebra operations in \textbf{R}. In all operations $\boldsymbol{A}$ (and correspondingly, \texttt{A}) is a matrix} 
\label{tab:boo1}
\begin{tabular}{llll}
  \toprule
Operator & Operation & To.find. & We.type. \\ 
  \midrule
\texttt{t()} & Matrix transpose & $\boldsymbol{A}^T$ & \texttt{t(A)} \\ 
  \texttt{\%*\%} & Matrix multiply & $\boldsymbol{A} \cdot \boldsymbol{A}$ & \texttt{A\%*\%A} \\ 
  \texttt{det()} & Determinant & $Det(\boldsymbol{A})$ & \texttt{det(a)} \\ 
  \texttt{solve()} & Matrix inverse & $\boldsymbol{A}^{-1}$ & \texttt{solve(A)} \\ 
   \bottomrule
\end{tabular}
\end{table}

\index[R]{\texttt{t()}}
\index[R]{*f@\texttt{\%*\%}}
\index[R]{\texttt{det()}}
\index[R]{\texttt{solve()}}

\begin{example}
\protect\hypertarget{exm:exm-3M}{}\label{exm:exm-3M}

\(\text{}\)\\
The matrix \texttt{A}, defined above, has the form:\\
\[\boldsymbol{A} = \begin{bmatrix}
1 & 3\\
2 & 2
\end{bmatrix}.\]
Consider the operations:

\vspace{0.1in}

\begin{Shaded}
\begin{Highlighting}[]
\FunctionTok{t}\NormalTok{(A)}
\end{Highlighting}
\end{Shaded}

\begin{verbatim}
     [,1] [,2]
[1,]    1    2
[2,]    3    2
\end{verbatim}

\vspace{0.1in}

\begin{Shaded}
\begin{Highlighting}[]
\NormalTok{A }\SpecialCharTok{\%*\%}\NormalTok{ A}
\end{Highlighting}
\end{Shaded}

\begin{verbatim}
     [,1] [,2]
[1,]    7    9
[2,]    6   10
\end{verbatim}

\vspace{0.1in}

\begin{Shaded}
\begin{Highlighting}[]
\FunctionTok{det}\NormalTok{(A)}
\end{Highlighting}
\end{Shaded}

\begin{verbatim}
[1] -4
\end{verbatim}

\vspace{0.1in}

\begin{Shaded}
\begin{Highlighting}[]
\FunctionTok{solve}\NormalTok{(A)}
\end{Highlighting}
\end{Shaded}

\begin{verbatim}
     [,1]  [,2]
[1,] -0.5  0.75
[2,]  0.5 -0.25
\end{verbatim}

\end{example}

\(\blacksquare\)

We can use the function \texttt{cbind()}\index[R]{\texttt{cbind()}} to combine vectors into matrix columns,

\vspace{0.1in}

\begin{Shaded}
\begin{Highlighting}[]
\NormalTok{a }\OtherTok{\textless{}{-}} \FunctionTok{c}\NormalTok{(}\DecValTok{1}\NormalTok{, }\DecValTok{2}\NormalTok{, }\DecValTok{3}\NormalTok{); b }\OtherTok{\textless{}{-}} \FunctionTok{c}\NormalTok{(}\DecValTok{2}\NormalTok{, }\DecValTok{3}\NormalTok{, }\DecValTok{4}\NormalTok{)}
\FunctionTok{cbind}\NormalTok{(a, b)}
\end{Highlighting}
\end{Shaded}

\begin{verbatim}
     a b
[1,] 1 2
[2,] 2 3
[3,] 3 4
\end{verbatim}

and use the function \texttt{rbind()}\index[R]{\texttt{rbind()}} to combine vectors into matrix rows.

\vspace{0.1in}

\begin{Shaded}
\begin{Highlighting}[]
\FunctionTok{rbind}\NormalTok{(a,b)}
\end{Highlighting}
\end{Shaded}

\begin{verbatim}
  [,1] [,2] [,3]
a    1    2    3
b    2    3    4
\end{verbatim}

\subsection{Arrays}\label{arrays}

\emph{Arrays} are one, two dimensional (matrix), or three or more dimensional data structures whose elements contain a single type of data.\index[terms]{R!classes!array} Thus, while all matrices are arrays, not all arrays are matrices.

\vspace{0.1in}

\begin{Shaded}
\begin{Highlighting}[]
\FunctionTok{class}\NormalTok{(A)}
\end{Highlighting}
\end{Shaded}

\begin{verbatim}
[1] "matrix" "array" 
\end{verbatim}

As with matrices, elements in arrays can have only one data storage mode.

\vspace{0.1in}

\begin{Shaded}
\begin{Highlighting}[]
\FunctionTok{typeof}\NormalTok{(A) }\CommentTok{\# base type (data storage mode)}
\end{Highlighting}
\end{Shaded}

\begin{verbatim}
[1] "double"
\end{verbatim}

\index[terms]{R!base type!double}
The function \texttt{array()}\index[R]{\texttt{array()}} can be used to create arrays. The first argument in \texttt{array()} defines the data. The second argument is a vector that defines both the number of dimensions (this will be the length of the vector), and the number of levels in each dimension (numbers in dimension elements).

\begin{example}
\protect\hypertarget{exm:exm-3A}{}\label{exm:exm-3A}

\(\text{}\)\\
Here is a \(2 \times 2 \times 2\) array:

\vspace{0.1in}

\begin{Shaded}
\begin{Highlighting}[]
\NormalTok{some.data }\OtherTok{\textless{}{-}} \FunctionTok{c}\NormalTok{(}\DecValTok{1}\NormalTok{, }\DecValTok{2}\NormalTok{, }\DecValTok{3}\NormalTok{, }\DecValTok{4}\NormalTok{, }\DecValTok{5}\NormalTok{, }\DecValTok{6}\NormalTok{, }\DecValTok{7}\NormalTok{, }\DecValTok{8}\NormalTok{)}
\NormalTok{B }\OtherTok{\textless{}{-}} \FunctionTok{array}\NormalTok{(some.data, }\FunctionTok{c}\NormalTok{(}\DecValTok{2}\NormalTok{, }\DecValTok{2}\NormalTok{, }\DecValTok{2}\NormalTok{))}
\NormalTok{B}
\end{Highlighting}
\end{Shaded}

\begin{verbatim}
, , 1

     [,1] [,2]
[1,]    1    3
[2,]    2    4

, , 2

     [,1] [,2]
[1,]    5    7
[2,]    6    8
\end{verbatim}

\vspace{0.1in}

\begin{Shaded}
\begin{Highlighting}[]
\FunctionTok{class}\NormalTok{(B)}
\end{Highlighting}
\end{Shaded}

\begin{verbatim}
[1] "array"
\end{verbatim}

\end{example}

\(\blacksquare\)

\subsection{Dataframes}\label{dataframes}

\emph{Dataframes} are two-dimensional structures whose columns can have different data storage modes (e.g., quantitative \emph{and} categorical). The function \texttt{data.frame()}\index[R]{\texttt{data.frame()}} can be used to create dataframes.
\index[terms]{R!classes!data.frame}

\vspace{0.1in}

\begin{Shaded}
\begin{Highlighting}[]
\NormalTok{df }\OtherTok{\textless{}{-}} \FunctionTok{data.frame}\NormalTok{(}\AttributeTok{numeric =} \FunctionTok{c}\NormalTok{(}\DecValTok{1}\NormalTok{, }\DecValTok{2}\NormalTok{, }\DecValTok{3}\NormalTok{), }\AttributeTok{non.numeric =} \FunctionTok{c}\NormalTok{(}\StringTok{"a"}\NormalTok{, }\StringTok{"b"}\NormalTok{, }\StringTok{"c"}\NormalTok{))}
\NormalTok{df}
\end{Highlighting}
\end{Shaded}

\begin{verbatim}
  numeric non.numeric
1       1           a
2       2           b
3       3           c
\end{verbatim}

\vspace{0.1in}

\begin{Shaded}
\begin{Highlighting}[]
\FunctionTok{class}\NormalTok{(df)}
\end{Highlighting}
\end{Shaded}

\begin{verbatim}
[1] "data.frame"
\end{verbatim}

Because of the possibility of different data storage modes for distinct columns, the data storage mode of a dataframe is \texttt{"list"}. Specifically, a dataframe is a list, whose storage \emph{elements} are columns.

\vspace{0.1in}

\begin{Shaded}
\begin{Highlighting}[]
\FunctionTok{typeof}\NormalTok{(df)}
\end{Highlighting}
\end{Shaded}

\begin{verbatim}
[1] "list"
\end{verbatim}

A \texttt{names} attribute will exist for each dataframe column\footnote{Matrices and arrays which, optimally, will both be numeric storage structures, cannot have a \texttt{names} attribute. Instead, row names and column names can be applied using the functions \texttt{row.names()} and \texttt{col.names()}. These, however, cannot be made visible to search paths with \texttt{attach()} or called with \texttt{\$}.}.
\index[R]{\texttt{row.names()}}
\index[R]{\texttt{col.names()}}

\vspace{0.1in}

\begin{Shaded}
\begin{Highlighting}[]
\FunctionTok{names}\NormalTok{(df)}
\end{Highlighting}
\end{Shaded}

\begin{verbatim}
[1] "numeric"     "non.numeric"
\end{verbatim}

The \texttt{\$} operator \index[R]{*e@\texttt{\$}} allows access to dataframe columns.

\vspace{0.1in}

\begin{Shaded}
\begin{Highlighting}[]
\NormalTok{df}\SpecialCharTok{$}\NormalTok{non.numeric}
\end{Highlighting}
\end{Shaded}

\begin{verbatim}
[1] "a" "b" "c"
\end{verbatim}

The function \texttt{attach()}\index[R]{\texttt{attach()}} allows \textbf{R} to recognize column names of a dataframe as global variables.

\vspace{0.1in}

\begin{Shaded}
\begin{Highlighting}[]
\FunctionTok{attach}\NormalTok{(df)}
\NormalTok{non.numeric}
\end{Highlighting}
\end{Shaded}

\begin{verbatim}
[1] "a" "b" "c"
\end{verbatim}

The function \texttt{detach()}\index[R]{\texttt{detach()}} is the programming inverse of \texttt{attach()}.

\vspace{0.1in}

\begin{Shaded}
\begin{Highlighting}[]
\FunctionTok{detach}\NormalTok{(df)}
\NormalTok{non.numeric}
\end{Highlighting}
\end{Shaded}

\begin{verbatim}
Error: object 'non.numeric' not found
\end{verbatim}

The functions \texttt{rm()}\index[R]{\texttt{rm()}} and \texttt{remove()}\index[R]{\texttt{remove()}} will entirely remove any \textbf{R}-object, including a vector, matrix, or dataframe from a session. To remove \emph{all} objects from the workspace one can use \texttt{rm(list=ls())} or (in RStudio) the ``broom'' button in the environments and history panel\footnote{All objects from a specific class can also be removed from a workspace. For example, to remove all dataframes, from a work session one could use: \texttt{rm(list=ls(all=TRUE){[}sapply(mget(ls(all=TRUE)),\ class)\ ==\ "data.frame"{]})}}.

A safer alternative to \texttt{attach()} is the function \texttt{with()}.\index[R]{\texttt{with()}} Using \texttt{with()} eliminates concerns about multiple variables with the same name becoming mixed up in functions. This is because the variable names for a dataframe specified in \texttt{with()} will not be permanently attached in an \textbf{R}-session.

\vspace{0.1in}

\begin{Shaded}
\begin{Highlighting}[]
\FunctionTok{with}\NormalTok{(df, non.numeric)}
\end{Highlighting}
\end{Shaded}

\begin{verbatim}
[1] "a" "b" "c"
\end{verbatim}

\subsection{Lists}\label{lists}

\emph{Lists} are often used to contain miscellaneous associated objects. Like dataframes, lists need not use a single data storage mode. Unlike dataframes, however, lists can include objects that are not two-dimensional or with different data classes including character strings\index[terms]{Character string} (i.e., units of character objects), multiple matrices and dataframes with varying dimensionality, and even other lists. The function \texttt{list()}\index[R]{\texttt{list()}} can be used to create lists.

\vspace{0.1in}

\begin{Shaded}
\begin{Highlighting}[]
\NormalTok{ldata1 }\OtherTok{\textless{}{-}} \FunctionTok{list}\NormalTok{(}\AttributeTok{a =} \FunctionTok{c}\NormalTok{(}\DecValTok{1}\NormalTok{, }\DecValTok{2}\NormalTok{, }\DecValTok{3}\NormalTok{), }\AttributeTok{b =} \StringTok{"this.is.a.list"}\NormalTok{)}
\NormalTok{ldata1}
\end{Highlighting}
\end{Shaded}

\begin{verbatim}
$a
[1] 1 2 3

$b
[1] "this.is.a.list"
\end{verbatim}

\vspace{0.1in}

\begin{Shaded}
\begin{Highlighting}[]
\FunctionTok{class}\NormalTok{(ldata1)}
\end{Highlighting}
\end{Shaded}

\begin{verbatim}
[1] "list"
\end{verbatim}

\vspace{0.1in}

\begin{Shaded}
\begin{Highlighting}[]
\FunctionTok{typeof}\NormalTok{(ldata1)}
\end{Highlighting}
\end{Shaded}

\begin{verbatim}
[1] "list"
\end{verbatim}

\index[terms]{R!base type!list}
Objects in lists can be called using the \texttt{\$} operator. Here is the character string \texttt{b} from \texttt{ldata}.

\vspace{0.1in}

\begin{Shaded}
\begin{Highlighting}[]
\NormalTok{ldata1}\SpecialCharTok{$}\NormalTok{b}
\end{Highlighting}
\end{Shaded}

\begin{verbatim}
[1] "this.is.a.list"
\end{verbatim}

We note that to create an \textbf{R}-object containing character strings, we need to place quotation
marks around entries.

\vspace{0.1in}

\begin{Shaded}
\begin{Highlighting}[]
\NormalTok{x }\OtherTok{\textless{}{-}} \FunctionTok{c}\NormalTok{(}\StringTok{"low"}\NormalTok{, }\StringTok{"med"}\NormalTok{, }\StringTok{"high"}\NormalTok{)}
\NormalTok{x}
\end{Highlighting}
\end{Shaded}

\begin{verbatim}
[1] "low"  "med"  "high"
\end{verbatim}

The function \texttt{str} attempts to display the \emph{internal structure} of an \textbf{R} object. It is extremely useful for succinctly displaying the contents of complex objects like lists.

\vspace{0.1in}

\begin{Shaded}
\begin{Highlighting}[]
\FunctionTok{str}\NormalTok{(ldata1)}
\end{Highlighting}
\end{Shaded}

\begin{verbatim}
List of 2
 $ a: num [1:3] 1 2 3
 $ b: chr "this.is.a.list"
\end{verbatim}

We are told that \texttt{ldata1} is a list containing two objects: a sequence of numbers from 1 to 3, and a character string.
\index[R]{\texttt{str()}}

The function \texttt{do.call()} is useful for large scale manipulations of data storage objects. For example, what if you had a list containing multiple dataframes with the same column names that you wanted to bind together?

\vspace{0.1in}

\begin{Shaded}
\begin{Highlighting}[]
\NormalTok{ldata2 }\OtherTok{\textless{}{-}} \FunctionTok{list}\NormalTok{(}\AttributeTok{df1 =} \FunctionTok{data.frame}\NormalTok{(}\AttributeTok{lo.temp =} \FunctionTok{c}\NormalTok{(}\SpecialCharTok{{-}}\DecValTok{1}\NormalTok{,}\DecValTok{3}\NormalTok{,}\DecValTok{5}\NormalTok{), }
                                \AttributeTok{high.temp =} \FunctionTok{c}\NormalTok{(}\DecValTok{78}\NormalTok{, }\DecValTok{67}\NormalTok{, }\DecValTok{90}\NormalTok{)),}
              \AttributeTok{df2 =} \FunctionTok{data.frame}\NormalTok{(}\AttributeTok{lo.temp =} \FunctionTok{c}\NormalTok{(}\SpecialCharTok{{-}}\DecValTok{4}\NormalTok{,}\DecValTok{3}\NormalTok{,}\DecValTok{7}\NormalTok{), }
                               \AttributeTok{high.temp =} \FunctionTok{c}\NormalTok{(}\DecValTok{75}\NormalTok{, }\DecValTok{87}\NormalTok{, }\DecValTok{80}\NormalTok{)),}
              \AttributeTok{df3 =} \FunctionTok{data.frame}\NormalTok{(}\AttributeTok{lo.temp =} \FunctionTok{c}\NormalTok{(}\SpecialCharTok{{-}}\DecValTok{0}\NormalTok{,}\DecValTok{2}\NormalTok{), }
                               \AttributeTok{high.temp =} \FunctionTok{c}\NormalTok{(}\DecValTok{70}\NormalTok{, }\DecValTok{80}\NormalTok{)))}
\end{Highlighting}
\end{Shaded}

You could do something like:

\vspace{0.1in}

\begin{Shaded}
\begin{Highlighting}[]
\FunctionTok{do.call}\NormalTok{(}\StringTok{"rbind"}\NormalTok{,ldata2)}
\end{Highlighting}
\end{Shaded}

\begin{verbatim}
      lo.temp high.temp
df1.1      -1        78
df1.2       3        67
df1.3       5        90
df2.1      -4        75
df2.2       3        87
df2.3       7        80
df3.1       0        70
df3.2       2        80
\end{verbatim}

Or what if I wanted to replicate the \texttt{df3} dataframe from \texttt{ldata} above, by binding it onto the bottom of itself three times? I could do something like:

\vspace{0.1in}

\begin{Shaded}
\begin{Highlighting}[]
\FunctionTok{do.call}\NormalTok{(}\StringTok{"rbind"}\NormalTok{, }\FunctionTok{replicate}\NormalTok{(}\DecValTok{3}\NormalTok{, ldata2}\SpecialCharTok{$}\NormalTok{df3, }\AttributeTok{simplify =} \ConstantTok{FALSE}\NormalTok{))}
\end{Highlighting}
\end{Shaded}

\begin{verbatim}
  lo.temp high.temp
1       0        70
2       2        80
3       0        70
4       2        80
5       0        70
6       2        80
\end{verbatim}

Note the use of the function \texttt{replicate()}.
\index[R]{\texttt{do.call()}}
\index[R]{\texttt{replicate()}}

\section{Boolean Operations}\label{boolean-operations}

Computer operations that dichotomously classify true and false statements are called \emph{logical} or \emph{Boolean}. In \textbf{R}, a Boolean operation will always return one of the values \texttt{TRUE} or \texttt{FALSE}. \textbf{R} logical operators are listed in Table \ref{tab:boo}.
\index[terms]{Boolean (logical)}

\begin{table}[ht]
\centering
\caption{Logical (Boolean) operators in \textbf{R}; \texttt{x}, \texttt{y}, and \texttt{z} in columns three and four are \textbf{R} objects.} 
\label{tab:boo}
\begin{tabular}{llll}
  \toprule
Operator & Operation & To ask: & We type: \\ 
  \midrule
\texttt{>} & $>$ & Is \texttt{x} greater than \texttt{y}? & \texttt{x > y} \\ 
  \texttt{>=} & $\geq$ & Is \texttt{x} greater than or equal to \texttt{y}?  & \texttt{x >= y} \\ 
  \texttt{<} & $<$ & Is \texttt{x} less than \texttt{y}? & \texttt{x < y} \\ 
  \texttt{<=} & $\leq$ & Is \texttt{x} less than or equal to \texttt{y}  & \texttt{x <= y} \\ 
  \texttt{==} & $=$ & Is \texttt{x} equal to \texttt{y}? & \texttt{x == y} \\ 
  \texttt{!=} & $\neq$ & Is \texttt{x} not equal to \texttt{y}? & \texttt{x != y} \\ 
  \texttt{\&} & \textit{and} & Do \texttt{x} and \texttt{y} equal \texttt{z}? & \texttt{x \&  y == z} \\ 
  \texttt{\&\&} &  \textit{and} (control flow) & Do \texttt{x} and \texttt{y} equal \texttt{z}? & \texttt{x \&\& y == z} \\ 
  \texttt{|} & \textit{or} & Do \texttt{x} or \texttt{y} equal \texttt{z}? & \texttt{x | y == z} \\ 
  \texttt{||} & \textit{or} (control flow) &  Do \texttt{x} or \texttt{y} equal \texttt{z}? & \texttt{x || y == z} \\ 
   \bottomrule
\end{tabular}
\end{table}

\index[R]{*less@\texttt{\textless}}
\index[R]{*lesse@\texttt{\textless=}}
\index[R]{*lessg@\texttt{\textgreater}}
\index[R]{*lessh@\texttt{\textgreater=}}
\index[R]{*equal@\texttt{==}}
\index[R]{*equb@\texttt{^^21=}}
\index[R]{*and@\texttt{\&}}
\index[R]{*and1@\texttt{\&\&}}
\index[R]{*or@\texttt{\textbar}}
\index[R]{*or1@\texttt{\textbar\textbar}}

Note that there are two ways to specify ``and'' (\texttt{\&} and \texttt{\&\&}), and two ways to specify ``or'' (\texttt{\textbar{}} and \texttt{\textbar{}\textbar{}}). The longer forms of ``and'' and ``or'' evaluate queries from left to right, stopping when a result is determined. Thus, this form is more appropriate for programming control-flow operations.

\begin{example}
\protect\hypertarget{exm:exm-1}{}\label{exm:exm-1}

\hfill\break
For demonstration purposes, here is a simple dataframe:

\vspace{0.1in}

\begin{Shaded}
\begin{Highlighting}[]
\NormalTok{dframe }\OtherTok{\textless{}{-}} \FunctionTok{data.frame}\NormalTok{(}
\AttributeTok{Age =} \FunctionTok{c}\NormalTok{(}\DecValTok{18}\NormalTok{,}\DecValTok{22}\NormalTok{,}\DecValTok{23}\NormalTok{,}\DecValTok{21}\NormalTok{,}\DecValTok{22}\NormalTok{,}\DecValTok{19}\NormalTok{,}\DecValTok{18}\NormalTok{,}\DecValTok{18}\NormalTok{,}\DecValTok{19}\NormalTok{,}\DecValTok{21}\NormalTok{),}
\AttributeTok{Sex =} \FunctionTok{c}\NormalTok{(}\StringTok{"M"}\NormalTok{,}\StringTok{"M"}\NormalTok{,}\StringTok{"M"}\NormalTok{,}\StringTok{"M"}\NormalTok{,}\StringTok{"M"}\NormalTok{,}\StringTok{"F"}\NormalTok{,}\StringTok{"F"}\NormalTok{,}\StringTok{"F"}\NormalTok{,}\StringTok{"F"}\NormalTok{,}\StringTok{"F"}\NormalTok{),}
\AttributeTok{Weight\_kg =} \FunctionTok{c}\NormalTok{(}\FloatTok{63.5}\NormalTok{,}\FloatTok{77.1}\NormalTok{,}\FloatTok{86.1}\NormalTok{,}\FloatTok{81.6}\NormalTok{,}\FloatTok{70.3}\NormalTok{,}\FloatTok{49.8}\NormalTok{,}\FloatTok{54.4}\NormalTok{,}\FloatTok{59.0}\NormalTok{,}\DecValTok{65}\NormalTok{,}\DecValTok{69}\NormalTok{)}
\NormalTok{)}

\NormalTok{dframe}
\end{Highlighting}
\end{Shaded}

\begin{verbatim}
   Age Sex Weight_kg
1   18   M      63.5
2   22   M      77.1
3   23   M      86.1
4   21   M      81.6
5   22   M      70.3
6   19   F      49.8
7   18   F      54.4
8   18   F      59.0
9   19   F      65.0
10  21   F      69.0
\end{verbatim}

The \textbf{R} logical operator for equals is \texttt{==} (Table \ref{tab:boo}). Thus, to identify \texttt{Age} outcomes equal to 21 we type:

\vspace{0.1in}

\begin{Shaded}
\begin{Highlighting}[]
\FunctionTok{with}\NormalTok{(dframe, Age }\SpecialCharTok{==} \DecValTok{21}\NormalTok{)}
\end{Highlighting}
\end{Shaded}

\begin{verbatim}
 [1] FALSE FALSE FALSE  TRUE FALSE FALSE FALSE FALSE FALSE  TRUE
\end{verbatim}

The argument \texttt{Age\ ==\ 21} has base type \texttt{logical}.

\begin{Shaded}
\begin{Highlighting}[]
\FunctionTok{typeof}\NormalTok{(dframe}\SpecialCharTok{$}\NormalTok{Age }\SpecialCharTok{==} \DecValTok{21}\NormalTok{)}
\end{Highlighting}
\end{Shaded}

\begin{verbatim}
[1] "logical"
\end{verbatim}

\index[terms]{R!base type!logical}
The \emph{unary operator} for ``not'' is \texttt{!} (Table \ref{tab:boo}). Thus, to identify \texttt{Age} outcomes not equal to 21 we could type:

\vspace{0.1in}

\begin{Shaded}
\begin{Highlighting}[]
\FunctionTok{with}\NormalTok{(dframe, Age }\SpecialCharTok{!=} \DecValTok{21}\NormalTok{)}
\end{Highlighting}
\end{Shaded}

\begin{verbatim}
 [1]  TRUE  TRUE  TRUE FALSE  TRUE  TRUE  TRUE  TRUE  TRUE FALSE
\end{verbatim}

Multiple Boolean queries can be made. Here we identify \texttt{Age} data less than 19, or equal to 21.

\vspace{0.1in}

\begin{Shaded}
\begin{Highlighting}[]
\FunctionTok{with}\NormalTok{(dframe, Age }\SpecialCharTok{\textless{}} \DecValTok{19} \SpecialCharTok{|}\NormalTok{ Age }\SpecialCharTok{==} \DecValTok{21}\NormalTok{)}
\end{Highlighting}
\end{Shaded}

\begin{verbatim}
 [1]  TRUE FALSE FALSE  TRUE FALSE FALSE  TRUE  TRUE FALSE  TRUE
\end{verbatim}

Queries can involve multiple variables. For instance, here we identify males less than or equal to 21 years old that weigh less than 80 kg.

\vspace{0.1in}

\begin{Shaded}
\begin{Highlighting}[]
\FunctionTok{with}\NormalTok{(dframe, Age }\SpecialCharTok{\textless{}=} \DecValTok{21} \SpecialCharTok{\&}\NormalTok{ Sex }\SpecialCharTok{==} \StringTok{"M"}\NormalTok{, weight }\SpecialCharTok{\textless{}} \DecValTok{80}\NormalTok{)}
\end{Highlighting}
\end{Shaded}

\begin{verbatim}
 [1]  TRUE FALSE FALSE  TRUE FALSE FALSE FALSE FALSE FALSE FALSE
\end{verbatim}

\end{example}

\(\blacksquare\)

\section{Testing and Coercing Classes}\label{sctc}

We have already considered functions for identifying the class of an arbitrary \textbf{R} object, \emph{foo}. These include \texttt{class(foo)} and \texttt{attr(foo,\ "class")} (Section \ref{oc}). We have also considered approaches for identifying the base type of an object, including \texttt{typeof(foo)} (Section \ref{bt}). This section considers methods for identifying object membership in \emph{particular} classes and base types, and coercing an object's class and base type membership.
\index[R]{\texttt{class()}}
\index[R]{\texttt{attr()}}
\index[R]{\texttt{typeof()}}

\subsection{Testing}\label{testing}

Functions exist to logically \emph{test} for object membership to major \textbf{R} classes. These functions generally begin with an \texttt{.is} prefix and include: \texttt{is.matrix()}, \texttt{is.array()}, \texttt{is.list()}, \texttt{is.factor()}, \texttt{is.double()}, \texttt{is.integer()} \texttt{is.numeric()}, \texttt{is.character()}, and many others.
\index[R]{\texttt{is.matrix()}}
\index[R]{\texttt{is.array()}}
\index[R]{\texttt{is.list()}}
\index[R]{\texttt{factor()}}
\index[R]{\texttt{is.factor()}}
\index[R]{\texttt{is.double()}}
\index[R]{\texttt{is.integer()}}
\index[R]{\texttt{is.numeric()}}
\index[R]{\texttt{is.character()}}
The Boolean function \texttt{is.numeric()} can be used to test if an object or an object's components behave like numbers\footnote{The \texttt{numeric} class is often used as an alias for class \texttt{double}. In fact, \texttt{as.numeric()} is identical to \texttt{as.double()}, and \texttt{numeric()} is identical to \texttt{double()} \citep{wickham2019advanced}.}.

\vspace{0.1in}

\begin{Shaded}
\begin{Highlighting}[]
\NormalTok{x }\OtherTok{\textless{}{-}} \FunctionTok{c}\NormalTok{(}\DecValTok{23}\NormalTok{, }\DecValTok{34}\NormalTok{, }\DecValTok{10}\NormalTok{)}
\FunctionTok{is.numeric}\NormalTok{(x)}
\end{Highlighting}
\end{Shaded}

\begin{verbatim}
[1] TRUE
\end{verbatim}

\vspace{0.1in}

\begin{Shaded}
\begin{Highlighting}[]
\FunctionTok{is.double}\NormalTok{(x)}
\end{Highlighting}
\end{Shaded}

\begin{verbatim}
[1] TRUE
\end{verbatim}

Thus, \texttt{x} contains numbers stored with double precision. However,

\vspace{0.1in}

\begin{Shaded}
\begin{Highlighting}[]
\NormalTok{x }\OtherTok{\textless{}{-}} \FunctionTok{c}\NormalTok{(}\StringTok{"a"}\NormalTok{, }\StringTok{"b"}\NormalTok{, }\DecValTok{10}\NormalTok{)}
\FunctionTok{is.numeric}\NormalTok{(x)}
\end{Highlighting}
\end{Shaded}

\begin{verbatim}
[1] FALSE
\end{verbatim}

Data objects with categorical entries can be created using the function \texttt{factor()}. In statistics the term ``factor'' refers to a categorical variable whose categories (factor levels) are likely replicated as treatments in an experimental design.

\vspace{0.1in}

\begin{Shaded}
\begin{Highlighting}[]
\NormalTok{x }\OtherTok{\textless{}{-}} \FunctionTok{factor}\NormalTok{(}\FunctionTok{c}\NormalTok{(}\DecValTok{1}\NormalTok{,}\DecValTok{2}\NormalTok{,}\DecValTok{3}\NormalTok{,}\DecValTok{4}\NormalTok{))}
\NormalTok{x}
\end{Highlighting}
\end{Shaded}

\begin{verbatim}
[1] 1 2 3 4
Levels: 1 2 3 4
\end{verbatim}

\vspace{0.1in}

\begin{Shaded}
\begin{Highlighting}[]
\FunctionTok{is.factor}\NormalTok{(x)}
\end{Highlighting}
\end{Shaded}

\begin{verbatim}
[1] TRUE
\end{verbatim}

The \textbf{R} class \texttt{factor} streamlines many analytical processes, including summarization of a quantitative variable with respect to a factor and specifying interactions of two or more factors. Here we see the interaction of levels in \texttt{x} with levels in another factor, \texttt{y}.

\vspace{0.1in}

\begin{Shaded}
\begin{Highlighting}[]
\NormalTok{y }\OtherTok{\textless{}{-}} \FunctionTok{factor}\NormalTok{(}\FunctionTok{c}\NormalTok{(}\StringTok{"a"}\NormalTok{,}\StringTok{"b"}\NormalTok{,}\StringTok{"c"}\NormalTok{,}\StringTok{"d"}\NormalTok{))}
\FunctionTok{interaction}\NormalTok{(x, y)}
\end{Highlighting}
\end{Shaded}

\begin{verbatim}
[1] 1.a 2.b 3.c 4.d
16 Levels: 1.a 2.a 3.a 4.a 1.b 2.b 3.b 4.b 1.c 2.c 3.c 4.c 1.d 2.d ... 4.d
\end{verbatim}

\index[R]{\texttt{interaction()}}

Sixteen interactions are possible, although only four actually occur when simultaneously considering \texttt{x} and \texttt{y}.

To decrease memory usage\footnote{All \texttt{numeric} objects in \textbf{R} are stored with double-precision, and will require two adjacent locations in computer memory (see Ch 12). Numeric objects coerced to be integers (with \texttt{as.intger()}) will be stored with double precision, although one of the storage locations will not be used. As a result, integers are not conventional double precision data.}, objects of class \texttt{factor} have an unexpected base type:

\vspace{0.1in}

\begin{Shaded}
\begin{Highlighting}[]
\FunctionTok{typeof}\NormalTok{(x)}
\end{Highlighting}
\end{Shaded}

\begin{verbatim}
[1] "integer"
\end{verbatim}

\index[terms]{R!base type!integer}
Despite this designation, and the fact that categories in \texttt{x} are distinguished using numbers, the entries in \texttt{x} do not have a numerical meaning and cannot be evaluated mathematically.

\vspace{0.1in}

\begin{Shaded}
\begin{Highlighting}[]
\FunctionTok{is.numeric}\NormalTok{(x)}
\end{Highlighting}
\end{Shaded}

\begin{verbatim}
[1] FALSE
\end{verbatim}

\vspace{0.1in}

\begin{Shaded}
\begin{Highlighting}[]
\NormalTok{x }\SpecialCharTok{+} \DecValTok{5}
\end{Highlighting}
\end{Shaded}

\begin{verbatim}
Warning in Ops.factor(x, 5): '+' not meaningful for factors
\end{verbatim}

\begin{verbatim}
[1] NA NA NA NA
\end{verbatim}

Occasionally an ordering of categorical levels is desirable. For instance, assume that we wish to apply three different imprecise temperature treatments \texttt{"low"}, \texttt{"med"} and \texttt{"high"} in an experiment with six experimental units. While we do not know the exact temperatures of these levels, we know that \texttt{"med"} is hotter than \texttt{"low"} and \texttt{"high"} is hotter than \texttt{"med"}. To provide this categorical ordering we can use \texttt{factor(data,\ ordered\ =\ TRUE)} or the function \texttt{ordered()}.
\index[R]{\texttt{ordered()}}

\vspace{0.1in}

\begin{Shaded}
\begin{Highlighting}[]
\NormalTok{x }\OtherTok{\textless{}{-}} \FunctionTok{factor}\NormalTok{(}\FunctionTok{c}\NormalTok{(}\StringTok{"med"}\NormalTok{,}\StringTok{"low"}\NormalTok{,}\StringTok{"high"}\NormalTok{,}\StringTok{"high"}\NormalTok{,}\StringTok{"med"}\NormalTok{,}\StringTok{"low"}\NormalTok{),}
            \AttributeTok{levels =} \FunctionTok{c}\NormalTok{(}\StringTok{"low"}\NormalTok{,}\StringTok{"med"}\NormalTok{,}\StringTok{"high"}\NormalTok{),}
            \AttributeTok{ordered =} \ConstantTok{TRUE}\NormalTok{)}
\NormalTok{x}
\end{Highlighting}
\end{Shaded}

\begin{verbatim}
[1] med  low  high high med  low 
Levels: low < med < high
\end{verbatim}

\vspace{0.1in}

\begin{Shaded}
\begin{Highlighting}[]
\FunctionTok{is.factor}\NormalTok{(x)}
\end{Highlighting}
\end{Shaded}

\begin{verbatim}
[1] TRUE
\end{verbatim}

\vspace{0.1in}

\begin{Shaded}
\begin{Highlighting}[]
\FunctionTok{is.ordered}\NormalTok{(x)}
\end{Highlighting}
\end{Shaded}

\begin{verbatim}
[1] TRUE
\end{verbatim}

The \texttt{levels} argument in \texttt{factor()} specifies the correct ordering of levels.

\subsubsection{\texttt{ifelse()}}

\index[R]{\texttt{ifelse()}}
The function \texttt{ifelse()} can be applied to atomic vectors or one dimensional arrays (e.g., rows or columns) to evaluate a logical argument and provide particular outcomes if the argument is \texttt{TRUE} or \texttt{FALSE}. The function requires three arguments.

\begin{itemize}
\tightlist
\item
  The first argument, \texttt{test}, gives the logical test to be evaluated.
\item
  The second argument, \texttt{yes}, provides the output if the test is true.
\item
  The third argument, \texttt{no}, provides the output if the test is false.
\end{itemize}

For instance:

\vspace{0.1in}

\begin{Shaded}
\begin{Highlighting}[]
\FunctionTok{ifelse}\NormalTok{(dframe}\SpecialCharTok{$}\NormalTok{Age }\SpecialCharTok{\textless{}} \DecValTok{20}\NormalTok{, }\StringTok{"Young"}\NormalTok{, }\StringTok{"Not so young"}\NormalTok{)}
\end{Highlighting}
\end{Shaded}

\begin{verbatim}
 [1] "Young"        "Not so young" "Not so young" "Not so young"
 [5] "Not so young" "Young"        "Young"        "Young"       
 [9] "Young"        "Not so young"
\end{verbatim}

\subsubsection{\texttt{if}, \texttt{else}, \texttt{any}, and \texttt{all}}

A more generalized approach to providing a condition and then defining the consequences (often used in functions) uses the commands \texttt{if} and \texttt{else}, potentially in combination with the functions \texttt{any()} and \texttt{all()}. For instance:

\vspace{0.1in}

\begin{Shaded}
\begin{Highlighting}[]
\ControlFlowTok{if}\NormalTok{(}\FunctionTok{any}\NormalTok{(dframe}\SpecialCharTok{$}\NormalTok{Age }\SpecialCharTok{\textless{}} \DecValTok{20}\NormalTok{))}\StringTok{"Young"} \ControlFlowTok{else} \StringTok{"Not so Young"}
\end{Highlighting}
\end{Shaded}

\begin{verbatim}
[1] "Young"
\end{verbatim}

and

\vspace{0.1in}

\begin{Shaded}
\begin{Highlighting}[]
\ControlFlowTok{if}\NormalTok{(}\FunctionTok{all}\NormalTok{(dframe}\SpecialCharTok{$}\NormalTok{Age }\SpecialCharTok{\textless{}} \DecValTok{20}\NormalTok{))}\StringTok{"Young"} \ControlFlowTok{else} \StringTok{"Not so Young"}
\end{Highlighting}
\end{Shaded}

\begin{verbatim}
[1] "Not so Young"
\end{verbatim}

\index[R]{\texttt{if()}}
\index[R]{\texttt{else()}}
\index[R]{\texttt{any()}}
\index[R]{\texttt{all()}}.

\subsection{Coercion}\label{s3coer}

Objects can be switched from one class to another using \emph{coercion} functions that begin with an \texttt{as.} prefix\footnote{Coercion can also be implemented using class generating functions described earlier. For instance, \texttt{data.frame(matrix(nrow\ =\ 2,\ data\ =\ rnorm(4)))} converts a \(2 \times 2\) matrix into an equivalent dataframe.}. Analogues to the testing (\texttt{.is}) functions listed above are:
\texttt{as.matrix()}, \texttt{as.array()}, \texttt{as.list()}, \texttt{as.factor()}, \texttt{as.double()}, \texttt{as.integer()} \texttt{as.numeric()}, and \texttt{as.character()}.
\index[R]{\texttt{as.matrix()}}
\index[R]{\texttt{as.array()}}
\index[R]{\texttt{as.list()}}
\index[R]{\texttt{as.factor()}}
\index[R]{\texttt{as.double()}}
\index[R]{\texttt{as.integer()}}
\index[R]{\texttt{as.numeric()}}
\index[R]{\texttt{as.character()}}

For instance, a non-factor object can be coerced to have class \texttt{factor} with the function \texttt{as.factor()}.

\vspace{0.1in}

\begin{Shaded}
\begin{Highlighting}[]
\NormalTok{x }\OtherTok{\textless{}{-}} \FunctionTok{c}\NormalTok{(}\DecValTok{23}\NormalTok{, }\DecValTok{34}\NormalTok{, }\DecValTok{10}\NormalTok{)}
\FunctionTok{is.factor}\NormalTok{(x)}
\end{Highlighting}
\end{Shaded}

\begin{verbatim}
[1] FALSE
\end{verbatim}

\vspace{0.1in}

\begin{Shaded}
\begin{Highlighting}[]
\NormalTok{y }\OtherTok{\textless{}{-}} \FunctionTok{as.factor}\NormalTok{(x)}
\FunctionTok{is.factor}\NormalTok{(y)}
\end{Highlighting}
\end{Shaded}

\begin{verbatim}
[1] TRUE
\end{verbatim}

Coercion may result in removal and addition of attributes. For instance conversion from atomic vector to matrix below results in the loss of the names attribute.

\vspace{0.1in}

\begin{Shaded}
\begin{Highlighting}[]
\NormalTok{x }\OtherTok{\textless{}{-}} \FunctionTok{c}\NormalTok{(}\AttributeTok{eulers\_num =} \FunctionTok{exp}\NormalTok{(}\DecValTok{1}\NormalTok{), }\AttributeTok{log\_exp =} \FunctionTok{log}\NormalTok{(}\FunctionTok{exp}\NormalTok{(}\DecValTok{1}\NormalTok{)), }\AttributeTok{pi =}\NormalTok{ pi)}
\NormalTok{x}
\end{Highlighting}
\end{Shaded}

\begin{verbatim}
eulers_num    log_exp         pi 
    2.7183     1.0000     3.1416 
\end{verbatim}

\vspace{0.1in}

\begin{Shaded}
\begin{Highlighting}[]
\FunctionTok{names}\NormalTok{(x)}
\end{Highlighting}
\end{Shaded}

\begin{verbatim}
[1] "eulers_num" "log_exp"    "pi"        
\end{verbatim}

\vspace{0.1in}

\begin{Shaded}
\begin{Highlighting}[]
\NormalTok{y }\OtherTok{\textless{}{-}} \FunctionTok{as.matrix}\NormalTok{(x)}
\FunctionTok{names}\NormalTok{(y)}
\end{Highlighting}
\end{Shaded}

\begin{verbatim}
NULL
\end{verbatim}

Coercion may also have unexpected results. Here \texttt{NA}s result when attempting to coerce a object with apparent mixed storage modes to class numeric.

\begin{Shaded}
\begin{Highlighting}[]
\NormalTok{x }\OtherTok{\textless{}{-}} \FunctionTok{c}\NormalTok{(}\StringTok{"a"}\NormalTok{, }\StringTok{"b"}\NormalTok{, }\DecValTok{10}\NormalTok{)}
\FunctionTok{as.numeric}\NormalTok{(x)}
\end{Highlighting}
\end{Shaded}

\begin{verbatim}
Warning: NAs introduced by coercion
\end{verbatim}

\begin{verbatim}
[1] NA NA 10
\end{verbatim}

\subsection{\texorpdfstring{\texttt{NA}}{NA}}\label{na}

\textbf{R} identifies \emph{missing values} (empty cells) as \texttt{NA}, which means ``not available.'' Hence, the \textbf{R} function to identify missing values is \texttt{is.na()}.
\index[R]{\texttt{is.na()}}
\index[terms]{R!\texttt{NA}}
For example:

\vspace{0.1in}

\begin{Shaded}
\begin{Highlighting}[]
\NormalTok{x }\OtherTok{\textless{}{-}} \FunctionTok{c}\NormalTok{(}\DecValTok{2}\NormalTok{, }\DecValTok{3}\NormalTok{, }\DecValTok{1}\NormalTok{, }\DecValTok{2}\NormalTok{, }\ConstantTok{NA}\NormalTok{, }\DecValTok{3}\NormalTok{, }\DecValTok{2}\NormalTok{)}
\FunctionTok{is.na}\NormalTok{(x)}
\end{Highlighting}
\end{Shaded}

\begin{verbatim}
[1] FALSE FALSE FALSE FALSE  TRUE FALSE FALSE
\end{verbatim}

Conversely, to identify outcomes that are not missing, I would use the ``not'' operator to specify \texttt{!is.na()}.

\vspace{0.1in}

\begin{Shaded}
\begin{Highlighting}[]
\SpecialCharTok{!}\FunctionTok{is.na}\NormalTok{(x)}
\end{Highlighting}
\end{Shaded}

\begin{verbatim}
[1]  TRUE  TRUE  TRUE  TRUE FALSE  TRUE  TRUE
\end{verbatim}

There are a number of \textbf{R} functions to get rid of missing values. These include \texttt{na.omit()}.\index[R]{\texttt{is.na()}}

\vspace{0.1in}

\begin{Shaded}
\begin{Highlighting}[]
\FunctionTok{na.omit}\NormalTok{(x)}
\end{Highlighting}
\end{Shaded}

\begin{verbatim}
[1] 2 3 1 2 3 2
attr(,"na.action")
[1] 5
attr(,"class")
[1] "omit"
\end{verbatim}

We see that \textbf{R} dropped the missing observation and then told us which observation was omitted (observation number 5).

Functions in \textbf{R} often, but not always, have built in capacities to handle missing data, for instance, by calling \texttt{na.omit()}. Consider the following dataframe which provides plant percent cover data for four plant species at two sites. Plant species are identified with four letter codes, consisting of the first two letters of the Linnaean genus and species names.

\vspace{0.1in}

\begin{Shaded}
\begin{Highlighting}[]
\NormalTok{field.data }\OtherTok{\textless{}{-}} \FunctionTok{data.frame}\NormalTok{(}\AttributeTok{ACMI =} \FunctionTok{c}\NormalTok{(}\DecValTok{12}\NormalTok{, }\DecValTok{13}\NormalTok{), }\AttributeTok{ELSC =} \FunctionTok{c}\NormalTok{(}\DecValTok{0}\NormalTok{, }\DecValTok{4}\NormalTok{), }\AttributeTok{CAEL =} \FunctionTok{c}\NormalTok{(}\ConstantTok{NA}\NormalTok{, }\DecValTok{2}\NormalTok{),}
                         \AttributeTok{CAPA =} \FunctionTok{c}\NormalTok{(}\DecValTok{20}\NormalTok{, }\DecValTok{30}\NormalTok{), }\AttributeTok{TACE =} \FunctionTok{c}\NormalTok{(}\DecValTok{0}\NormalTok{, }\DecValTok{2}\NormalTok{))}
\FunctionTok{row.names}\NormalTok{(field.data) }\OtherTok{\textless{}{-}} \FunctionTok{c}\NormalTok{(}\StringTok{"site1"}\NormalTok{, }\StringTok{"site2"}\NormalTok{)}

\NormalTok{field.data}
\end{Highlighting}
\end{Shaded}

\begin{verbatim}
      ACMI ELSC CAEL CAPA TACE
site1   12    0   NA   20    0
site2   13    4    2   30    2
\end{verbatim}

\index[terms]{R!missing data}

The function \texttt{complete.cases()}\index[R]{\texttt{complete.cases()}} checks for completeness of the data in rows of a data array.

\vspace{0.1in}

\begin{Shaded}
\begin{Highlighting}[]
\FunctionTok{complete.cases}\NormalTok{(field.data)}
\end{Highlighting}
\end{Shaded}

\begin{verbatim}
[1] FALSE  TRUE
\end{verbatim}

If \texttt{na.omit()} is applied in this context, the entire row containing the missing observation will be dropped.

\vspace{0.1in}

\begin{Shaded}
\begin{Highlighting}[]
\FunctionTok{na.omit}\NormalTok{(field.data)}
\end{Highlighting}
\end{Shaded}

\begin{verbatim}
      ACMI ELSC CAEL CAPA TACE
site2   13    4    2   30    2
\end{verbatim}

Unfortunately, this means that information about the other four species at site one will lost. Thus, it is generally more rational to remove \texttt{NA} values while retaining non-missing values. For instance, many statistical functions have to capacity to base summaries on non-\texttt{NA} data.

\vspace{0.1in}

\begin{Shaded}
\begin{Highlighting}[]
\FunctionTok{mean}\NormalTok{(field.data[}\DecValTok{1}\NormalTok{,], }\AttributeTok{na.rm =}\NormalTok{ T)}
\end{Highlighting}
\end{Shaded}

\begin{verbatim}
Warning in mean.default(field.data[1, ], na.rm = T): argument is not
numeric or logical: returning NA
\end{verbatim}

\begin{verbatim}
[1] NA
\end{verbatim}

\subsection{\texorpdfstring{\texttt{NaN}}{NaN}}\label{nan}

The designation \texttt{NaN} is associated with the current conventions of the IEEE 754-2008 \index[terms]{IEEE (Institute of Electrical and Electronics Engineers)} arithmetic used by \textbf{R}. It means ``not a number.'' Mathematical operations which produce \texttt{NaN} include:
\index[terms]{R! \texttt{NaN}}

\vspace{0.1in}

\begin{Shaded}
\begin{Highlighting}[]
\DecValTok{0}\SpecialCharTok{/}\DecValTok{0}
\end{Highlighting}
\end{Shaded}

\begin{verbatim}
[1] NaN
\end{verbatim}

\vspace{0.1in}

\begin{Shaded}
\begin{Highlighting}[]
\ConstantTok{Inf}\SpecialCharTok{{-}}\ConstantTok{Inf}
\end{Highlighting}
\end{Shaded}

\begin{verbatim}
[1] NaN
\end{verbatim}

\vspace{0.1in}

\begin{Shaded}
\begin{Highlighting}[]
\FunctionTok{sin}\NormalTok{(}\ConstantTok{Inf}\NormalTok{)}
\end{Highlighting}
\end{Shaded}

\begin{verbatim}
Warning in sin(Inf): NaNs produced
\end{verbatim}

\begin{verbatim}
[1] NaN
\end{verbatim}

\subsection{\texorpdfstring{\texttt{NULL}}{NULL}}\label{null}

In object oriented programming, a \emph{null object} has no referenced value or has a defined neutral behavior \citep{wikinull}. Occasionally one may wish to specify that an \textbf{R} object is \texttt{NULL}. For example, a \texttt{NULL} object can be included as an argument in a function without requiring that it has a particular value or meaning. As with \texttt{NA} and \texttt{NaN}, the \texttt{NULL} specification is easy.

\vspace{0.1in}

\begin{Shaded}
\begin{Highlighting}[]
\NormalTok{x }\OtherTok{\textless{}{-}} \ConstantTok{NULL}
\end{Highlighting}
\end{Shaded}

The class and base type of \texttt{x} are \texttt{NULL}:

\begin{Shaded}
\begin{Highlighting}[]
\FunctionTok{class}\NormalTok{(x)}
\end{Highlighting}
\end{Shaded}

\begin{verbatim}
[1] "NULL"
\end{verbatim}

\begin{Shaded}
\begin{Highlighting}[]
\FunctionTok{typeof}\NormalTok{(x)}
\end{Highlighting}
\end{Shaded}

\begin{verbatim}
[1] "NULL"
\end{verbatim}

\index[terms]{R!base type!NULL}
\index[terms]{R!classes!environment}

It should be emphasized that \textbf{R}-objects or elements within objects that are \texttt{NA}, \texttt{NaN} or \texttt{NULL} cannot be identified with the Boolean operators \texttt{==} or \texttt{!=}. For instance:

\vspace{0.1in}

\begin{Shaded}
\begin{Highlighting}[]
\NormalTok{x }\SpecialCharTok{==} \ConstantTok{NULL}
\end{Highlighting}
\end{Shaded}

\begin{verbatim}
logical(0)
\end{verbatim}

\vspace{0.1in}

\begin{Shaded}
\begin{Highlighting}[]
\NormalTok{y }\OtherTok{\textless{}{-}} \ConstantTok{NA}
\end{Highlighting}
\end{Shaded}

\vspace{0.1in}

\begin{Shaded}
\begin{Highlighting}[]
\NormalTok{y }\SpecialCharTok{==} \ConstantTok{NA}
\end{Highlighting}
\end{Shaded}

\begin{verbatim}
[1] NA
\end{verbatim}

Instead one should use \texttt{is.na()}, \texttt{is.nan()} or \texttt{is.null()} to identify \texttt{NA}, \texttt{NaN} or \texttt{NULL} components, respectively. \index[R]{\texttt{is.null()}}

\vspace{0.1in}

\begin{Shaded}
\begin{Highlighting}[]
\FunctionTok{is.null}\NormalTok{(x)}
\end{Highlighting}
\end{Shaded}

\begin{verbatim}
[1] TRUE
\end{verbatim}

\vspace{0.1in}

\begin{Shaded}
\begin{Highlighting}[]
\SpecialCharTok{!}\FunctionTok{is.null}\NormalTok{(x)}
\end{Highlighting}
\end{Shaded}

\begin{verbatim}
[1] FALSE
\end{verbatim}

\vspace{0.1in}

\begin{Shaded}
\begin{Highlighting}[]
\FunctionTok{is.na}\NormalTok{(y)}
\end{Highlighting}
\end{Shaded}

\begin{verbatim}
[1] TRUE
\end{verbatim}

\vspace{0.1in}

\begin{Shaded}
\begin{Highlighting}[]
\SpecialCharTok{!}\FunctionTok{is.na}\NormalTok{(y)}
\end{Highlighting}
\end{Shaded}

\begin{verbatim}
[1] FALSE
\end{verbatim}

\index[terms]{R!\texttt{NULL}}
\index[R]{\texttt{is.nan(x)}}

\section{\texorpdfstring{Accessing and Subsetting Data With \texttt{{[}{]}}}{Accessing and Subsetting Data With {[}{]}}}\label{accessing-and-subsetting-data-with}

One can subset data storage objects using \emph{square bracket operators}, i.e., \texttt{{[}{]}}, along with a variety of functions\footnote{For instance, \texttt{subset()}, \texttt{split()}, and \texttt{dplyr::filter()}.}. Because of their simplicity, I focus on square brackets for subsetting here. Gaining skills with square brackets will greatly enhance your ability to manipulate datasets in \textbf{R}. As toy datasets here are an atomic vector (with a names attribute), a matrix, a three dimensional array, a dataframe, and a list:

\index[R]{\texttt{split()}}
\index[R]{\texttt{subset()}}
\index[R]{\texttt{dplyr::filter()}}

\vspace{0.1in}

\begin{Shaded}
\begin{Highlighting}[]
\NormalTok{vdat }\OtherTok{\textless{}{-}} \FunctionTok{c}\NormalTok{(}\AttributeTok{a =} \DecValTok{1}\NormalTok{, }\AttributeTok{b =} \DecValTok{2}\NormalTok{, }\AttributeTok{c =} \DecValTok{3}\NormalTok{)}
\NormalTok{vdat}
\end{Highlighting}
\end{Shaded}

\begin{verbatim}
a b c 
1 2 3 
\end{verbatim}

\vspace{0.1in}

\begin{Shaded}
\begin{Highlighting}[]
\NormalTok{mdat }\OtherTok{\textless{}{-}} \FunctionTok{matrix}\NormalTok{(}\AttributeTok{ncol =} \DecValTok{2}\NormalTok{, }\AttributeTok{nrow =} \DecValTok{2}\NormalTok{, }\AttributeTok{data =} \FunctionTok{c}\NormalTok{(}\DecValTok{1}\NormalTok{, }\DecValTok{2}\NormalTok{, }\DecValTok{3}\NormalTok{, }\DecValTok{4}\NormalTok{))}
\NormalTok{mdat}
\end{Highlighting}
\end{Shaded}

\begin{verbatim}
     [,1] [,2]
[1,]    1    3
[2,]    2    4
\end{verbatim}

\vspace{0.1in}

\begin{Shaded}
\begin{Highlighting}[]
\NormalTok{adat }\OtherTok{\textless{}{-}} \FunctionTok{array}\NormalTok{(}\AttributeTok{dim =} \FunctionTok{c}\NormalTok{(}\DecValTok{2}\NormalTok{, }\DecValTok{2}\NormalTok{, }\DecValTok{2}\NormalTok{), }\AttributeTok{data =} \FunctionTok{c}\NormalTok{(}\DecValTok{1}\NormalTok{, }\DecValTok{2}\NormalTok{, }\DecValTok{3}\NormalTok{, }\DecValTok{4}\NormalTok{, }\DecValTok{5}\NormalTok{, }\DecValTok{6}\NormalTok{, }\DecValTok{7}\NormalTok{, }\DecValTok{8}\NormalTok{))}
\NormalTok{adat}
\end{Highlighting}
\end{Shaded}

\begin{verbatim}
, , 1

     [,1] [,2]
[1,]    1    3
[2,]    2    4

, , 2

     [,1] [,2]
[1,]    5    7
[2,]    6    8
\end{verbatim}

\vspace{0.1in}

\begin{Shaded}
\begin{Highlighting}[]
\NormalTok{ddat }\OtherTok{\textless{}{-}} \FunctionTok{data.frame}\NormalTok{(}\AttributeTok{numeric =} \FunctionTok{c}\NormalTok{(}\DecValTok{1}\NormalTok{, }\DecValTok{2}\NormalTok{, }\DecValTok{3}\NormalTok{), }\AttributeTok{non.numeric =} \FunctionTok{c}\NormalTok{(}\StringTok{"a"}\NormalTok{, }\StringTok{"b"}\NormalTok{, }\StringTok{"c"}\NormalTok{))}
\NormalTok{ddat}
\end{Highlighting}
\end{Shaded}

\begin{verbatim}
  numeric non.numeric
1       1           a
2       2           b
3       3           c
\end{verbatim}

\vspace{0.1in}

\begin{Shaded}
\begin{Highlighting}[]
\NormalTok{ldat }\OtherTok{\textless{}{-}} \FunctionTok{list}\NormalTok{(}\AttributeTok{element1 =} \FunctionTok{c}\NormalTok{(}\DecValTok{1}\NormalTok{, }\DecValTok{2}\NormalTok{, }\DecValTok{3}\NormalTok{), }\AttributeTok{element2 =} \StringTok{"this.is.a.list"}\NormalTok{)}
\NormalTok{ldat}
\end{Highlighting}
\end{Shaded}

\begin{verbatim}
$element1
[1] 1 2 3

$element2
[1] "this.is.a.list"
\end{verbatim}

\index[R]{\texttt{subset()}}
\index[R]{*sqbraca@\texttt{[]}}

To obtain the \(i\)th component from an atomic vector, matrix, array, dataframe or list named \texttt{foo} we would specify \texttt{foo{[}i{]}}. For instance, here is the first component of our toy data objects:
\vspace{0.1in}

\begin{Shaded}
\begin{Highlighting}[]
\NormalTok{vdat[}\DecValTok{1}\NormalTok{]}
\end{Highlighting}
\end{Shaded}

\begin{verbatim}
a 
1 
\end{verbatim}

\vspace{0.1in}

\begin{Shaded}
\begin{Highlighting}[]
\NormalTok{mdat[}\DecValTok{1}\NormalTok{]}
\end{Highlighting}
\end{Shaded}

\begin{verbatim}
[1] 1
\end{verbatim}

\vspace{0.1in}

\begin{Shaded}
\begin{Highlighting}[]
\NormalTok{adat[}\DecValTok{1}\NormalTok{]}
\end{Highlighting}
\end{Shaded}

\begin{verbatim}
[1] 1
\end{verbatim}

\vspace{0.1in}

\begin{Shaded}
\begin{Highlighting}[]
\NormalTok{ddat[}\DecValTok{1}\NormalTok{]}
\end{Highlighting}
\end{Shaded}

\begin{verbatim}
  numeric
1       1
2       2
3       3
\end{verbatim}

\vspace{0.1in}

\begin{Shaded}
\begin{Highlighting}[]
\NormalTok{ldat[}\DecValTok{1}\NormalTok{]}
\end{Highlighting}
\end{Shaded}

\begin{verbatim}
$element1
[1] 1 2 3
\end{verbatim}

Importantly, dataframes and lists view their \(i\)th element as the \(i\)th column and the \(i\)th list element, respectively.\\

We can also apply double square brackets, i.e., \texttt{{[}{[}{]}{]}} to list-type objects, i.e., atomic vectors and explicit lists, with similar results. Note, however, that the data subsets are now missing their name attributes.
\index[R]{*sqbracb@\texttt{[[]]}}
\vspace{0.1in}

\begin{Shaded}
\begin{Highlighting}[]
\NormalTok{vdat[[}\DecValTok{1}\NormalTok{]]}
\end{Highlighting}
\end{Shaded}

\begin{verbatim}
[1] 1
\end{verbatim}

\vspace{0.1in}

\begin{Shaded}
\begin{Highlighting}[]
\NormalTok{ldat[[}\DecValTok{1}\NormalTok{]]}
\end{Highlighting}
\end{Shaded}

\begin{verbatim}
[1] 1 2 3
\end{verbatim}

If a data storage object has a names attribute, then a name can be placed in square brackets to obtain corresponding data.\\
\vspace{0.1in}

\begin{Shaded}
\begin{Highlighting}[]
\NormalTok{ddat[}\StringTok{"numeric"}\NormalTok{]}
\end{Highlighting}
\end{Shaded}

\begin{verbatim}
  numeric
1       1
2       2
3       3
\end{verbatim}

The advantage of square brackets over \texttt{\$} in this an application is that several components can be specified simultaneously using the former approach:
\vspace{0.1in}

\begin{Shaded}
\begin{Highlighting}[]
\NormalTok{ddat[}\FunctionTok{c}\NormalTok{(}\StringTok{"non.numeric"}\NormalTok{,}\StringTok{"numeric"}\NormalTok{)]}
\end{Highlighting}
\end{Shaded}

\begin{verbatim}
  non.numeric numeric
1           a       1
2           b       2
3           c       3
\end{verbatim}

If \emph{foo} has a row \(\times\) column structure, i.e., a matrix, array, or dataframe, we could obtain the \(i\)th column from \emph{foo} using \texttt{foo{[},i{]}} (or \texttt{foo{[}{[}i{]}{]}}) and the \(j\)th row from \emph{foo} using \texttt{foo{[}j,{]}}. For instance, here is the second column from \texttt{mdat}, and the first row from \texttt{ddat}.
\vspace{0.1in}

\begin{Shaded}
\begin{Highlighting}[]
\NormalTok{mdat[,}\DecValTok{2}\NormalTok{]}
\end{Highlighting}
\end{Shaded}

\begin{verbatim}
[1] 3 4
\end{verbatim}

\vspace{0.1in}

\begin{Shaded}
\begin{Highlighting}[]
\NormalTok{ddat[}\DecValTok{1}\NormalTok{,]}
\end{Highlighting}
\end{Shaded}

\begin{verbatim}
  numeric non.numeric
1       1           a
\end{verbatim}

The element from \texttt{foo} corresponding to row \emph{j} and column \emph{i} can be accessed using: \texttt{foo{[}j,\ i{]}}, or \texttt{foo{[},i{]}{[}j{]}}, or \texttt{foo{[}j,{]}{[}i{]}}.
\vspace{0.1in}

\begin{Shaded}
\begin{Highlighting}[]
\NormalTok{mdat[}\DecValTok{1}\NormalTok{,}\DecValTok{2}\NormalTok{]; mdat[,}\DecValTok{2}\NormalTok{][}\DecValTok{1}\NormalTok{]; mdat[}\DecValTok{1}\NormalTok{,][}\DecValTok{2}\NormalTok{] }\CommentTok{\# 1st element from 2nd column}
\end{Highlighting}
\end{Shaded}

\begin{verbatim}
[1] 3
\end{verbatim}

\begin{verbatim}
[1] 3
\end{verbatim}

\begin{verbatim}
[1] 3
\end{verbatim}

Arrays may require more than two indices. For instance, for a three dimensional array, \texttt{foo}, the specification \texttt{foo{[},j,i{]}} will return the entirety of the \(j\)th column in the \(i\)th component of the outermost dimension of \texttt{foo}, whereas \texttt{foo{[}k,j,i{]}} will return the \(k\)th element from the \(j\)th column in the \(i\)th component of the outermost dimension of \texttt{foo}.
\vspace{0.1in}

\begin{Shaded}
\begin{Highlighting}[]
\NormalTok{adat[,}\DecValTok{2}\NormalTok{,}\DecValTok{1}\NormalTok{]}
\end{Highlighting}
\end{Shaded}

\begin{verbatim}
[1] 3 4
\end{verbatim}

\vspace{0.1in}

\begin{Shaded}
\begin{Highlighting}[]
\NormalTok{adat[}\DecValTok{1}\NormalTok{,}\DecValTok{2}\NormalTok{,}\DecValTok{1}\NormalTok{]}
\end{Highlighting}
\end{Shaded}

\begin{verbatim}
[1] 3
\end{verbatim}

\vspace{0.1in}

\begin{Shaded}
\begin{Highlighting}[]
\NormalTok{adat[}\DecValTok{2}\NormalTok{,}\DecValTok{2}\NormalTok{,}\DecValTok{1}\NormalTok{]}
\end{Highlighting}
\end{Shaded}

\begin{verbatim}
[1] 4
\end{verbatim}

Ranges or particular subsets of elements from a data storage object can also be selected. For instance, here I access rows two and three of \texttt{ddat}:
\vspace{0.1in}

\begin{Shaded}
\begin{Highlighting}[]
\NormalTok{ddat[}\DecValTok{2}\SpecialCharTok{:}\DecValTok{3}\NormalTok{,] }\CommentTok{\# note the position of the comma}
\end{Highlighting}
\end{Shaded}

\begin{verbatim}
  numeric non.numeric
2       2           b
3       3           c
\end{verbatim}

I can \texttt{drop} data object components by using negative integers in square brackets. Here I obtain an identical result to the example above by dropping row one from \texttt{ddat}:
\vspace{0.1in}

\begin{Shaded}
\begin{Highlighting}[]
\NormalTok{ddat[}\SpecialCharTok{{-}}\DecValTok{1}\NormalTok{,] }\CommentTok{\# drop row one}
\end{Highlighting}
\end{Shaded}

\begin{verbatim}
  numeric non.numeric
2       2           b
3       3           c
\end{verbatim}

Here I obtain \texttt{ddat} rows one and three in two different ways:
\vspace{0.1in}

\begin{Shaded}
\begin{Highlighting}[]
\NormalTok{ddat[}\FunctionTok{c}\NormalTok{(}\DecValTok{1}\NormalTok{,}\DecValTok{3}\NormalTok{),]}
\end{Highlighting}
\end{Shaded}

\begin{verbatim}
  numeric non.numeric
1       1           a
3       3           c
\end{verbatim}

\vspace{0.1in}

\begin{Shaded}
\begin{Highlighting}[]
\NormalTok{ddat[}\SpecialCharTok{{-}}\DecValTok{2}\NormalTok{,]}
\end{Highlighting}
\end{Shaded}

\begin{verbatim}
  numeric non.numeric
1       1           a
3       3           c
\end{verbatim}

Square braces can also be used to rearrange data components:
\vspace{0.1in}

\begin{Shaded}
\begin{Highlighting}[]
\NormalTok{ddat[}\FunctionTok{c}\NormalTok{(}\DecValTok{3}\NormalTok{,}\DecValTok{1}\NormalTok{,}\DecValTok{2}\NormalTok{),]}
\end{Highlighting}
\end{Shaded}

\begin{verbatim}
  numeric non.numeric
3       3           c
1       1           a
2       2           b
\end{verbatim}

Duplicate components:
\vspace{0.1in}

\begin{Shaded}
\begin{Highlighting}[]
\NormalTok{ldat[}\FunctionTok{c}\NormalTok{(}\DecValTok{2}\NormalTok{,}\DecValTok{2}\NormalTok{)]}
\end{Highlighting}
\end{Shaded}

\begin{verbatim}
$element2
[1] "this.is.a.list"

$element2
[1] "this.is.a.list"
\end{verbatim}

Or even replace data components:
\vspace{0.1in}

\begin{Shaded}
\begin{Highlighting}[]
\NormalTok{ddat[,}\DecValTok{2}\NormalTok{] }\OtherTok{\textless{}{-}} \FunctionTok{c}\NormalTok{(}\StringTok{"d"}\NormalTok{,}\StringTok{"e"}\NormalTok{,}\StringTok{"f"}\NormalTok{)}
\NormalTok{ddat}
\end{Highlighting}
\end{Shaded}

\begin{verbatim}
  numeric non.numeric
1       1           d
2       2           e
3       3           f
\end{verbatim}

\subsection{Subsetting a Factor}\label{subsetting-a-factor}

Importantly, the factor level structure of a \texttt{factor} will remain intact even if one or more of the levels are entirely removed.
\vspace{0.1in}

\begin{Shaded}
\begin{Highlighting}[]
\NormalTok{fdat }\OtherTok{\textless{}{-}} \FunctionTok{as.factor}\NormalTok{(ddat[,}\DecValTok{2}\NormalTok{])}
\NormalTok{fdat}
\end{Highlighting}
\end{Shaded}

\begin{verbatim}
[1] d e f
Levels: d e f
\end{verbatim}

\vspace{0.1in}

\begin{Shaded}
\begin{Highlighting}[]
\NormalTok{fdat[}\SpecialCharTok{{-}}\DecValTok{1}\NormalTok{]}
\end{Highlighting}
\end{Shaded}

\begin{verbatim}
[1] e f
Levels: d e f
\end{verbatim}

Note that the level \texttt{a} remains a characteristic of \texttt{fdat}, even though the cell containing the lone observation of \texttt{a} was removed from the dataset. This outcome is allowed because it is desirable for certain analytical situations (for instance, summarizations that should acknowledge missing data for some levels). To remove levels that no longer occur in a \texttt{factor}, we can use the function \texttt{droplevels()}.
\vspace{0.1in}

\begin{Shaded}
\begin{Highlighting}[]
\FunctionTok{droplevels}\NormalTok{(fdat[}\SpecialCharTok{{-}}\DecValTok{1}\NormalTok{])}
\end{Highlighting}
\end{Shaded}

\begin{verbatim}
[1] e f
Levels: e f
\end{verbatim}

\index[R]{\texttt{droplevels()}}

\subsection{Subsetting with Boolean Operators}\label{subsetting-with-boolean-operators}

Boolean (\texttt{TRUE} or \texttt{FALSE}) outcomes can be used in combination with square brackets to subset data. Consider the dataframe used earlier (Exercise \ref{exm:exm-1}) to demonstrate logical commands.
\vspace{0.1in}

\begin{Shaded}
\begin{Highlighting}[]
\NormalTok{dframe }\OtherTok{\textless{}{-}} \FunctionTok{data.frame}\NormalTok{(}
\AttributeTok{Age =} \FunctionTok{c}\NormalTok{(}\DecValTok{18}\NormalTok{,}\DecValTok{22}\NormalTok{,}\DecValTok{23}\NormalTok{,}\DecValTok{21}\NormalTok{,}\DecValTok{22}\NormalTok{,}\DecValTok{19}\NormalTok{,}\DecValTok{18}\NormalTok{,}\DecValTok{18}\NormalTok{,}\DecValTok{19}\NormalTok{,}\DecValTok{21}\NormalTok{),}
\AttributeTok{Sex =} \FunctionTok{c}\NormalTok{(}\StringTok{"M"}\NormalTok{,}\StringTok{"M"}\NormalTok{,}\StringTok{"M"}\NormalTok{,}\StringTok{"M"}\NormalTok{,}\StringTok{"M"}\NormalTok{,}\StringTok{"F"}\NormalTok{,}\StringTok{"F"}\NormalTok{,}\StringTok{"F"}\NormalTok{,}\StringTok{"F"}\NormalTok{,}\StringTok{"F"}\NormalTok{),}
\AttributeTok{Weight\_kg =} \FunctionTok{c}\NormalTok{(}\FloatTok{63.5}\NormalTok{,}\FloatTok{77.1}\NormalTok{,}\FloatTok{86.1}\NormalTok{,}\FloatTok{81.6}\NormalTok{,}\FloatTok{70.3}\NormalTok{,}\FloatTok{49.8}\NormalTok{,}\FloatTok{54.4}\NormalTok{,}\FloatTok{59.0}\NormalTok{,}\DecValTok{65}\NormalTok{,}\DecValTok{69}\NormalTok{)}
\NormalTok{)}
\end{Highlighting}
\end{Shaded}

Here we extract \texttt{Age} outcomes less than or equal to 21.
\vspace{0.1in}

\begin{Shaded}
\begin{Highlighting}[]
\NormalTok{ageTF }\OtherTok{\textless{}{-}}\NormalTok{ dframe}\SpecialCharTok{$}\NormalTok{Age }\SpecialCharTok{\textless{}=} \DecValTok{21}
\NormalTok{dframe}\SpecialCharTok{$}\NormalTok{Age[ageTF]}
\end{Highlighting}
\end{Shaded}

\begin{verbatim}
[1] 18 21 19 18 18 19 21
\end{verbatim}

We could also use this information to obtain entire rows of the dataframe.
\vspace{0.1in}

\begin{Shaded}
\begin{Highlighting}[]
\NormalTok{dframe[ageTF,]}
\end{Highlighting}
\end{Shaded}

\begin{verbatim}
   Age Sex Weight_kg
1   18   M      63.5
4   21   M      81.6
6   19   F      49.8
7   18   F      54.4
8   18   F      59.0
9   19   F      65.0
10  21   F      69.0
\end{verbatim}

\subsection{When Subset Is Larger Than Underlying Data}\label{when-subset-is-larger-than-underlying-data}

\textbf{R} allows one to make a data subset larger than underlying data itself, although this results in the generation of filler \texttt{NA}s. Consider the following example:
\vspace{0.1in}

\begin{Shaded}
\begin{Highlighting}[]
\NormalTok{x }\OtherTok{\textless{}{-}} \FunctionTok{c}\NormalTok{(}\SpecialCharTok{{-}}\DecValTok{2}\NormalTok{, }\DecValTok{3}\NormalTok{, }\DecValTok{4}\NormalTok{, }\DecValTok{6}\NormalTok{, }\DecValTok{45}\NormalTok{)}
\end{Highlighting}
\end{Shaded}

The atomic vector \texttt{x} has length five. If I ask for a subset of length seven, I get:
\vspace{0.1in}

\begin{Shaded}
\begin{Highlighting}[]
\NormalTok{x[}\DecValTok{1}\SpecialCharTok{:}\DecValTok{7}\NormalTok{]}
\end{Highlighting}
\end{Shaded}

\begin{verbatim}
[1] -2  3  4  6 45 NA NA
\end{verbatim}

\subsection{\texorpdfstring{Subsetting with \texttt{upper.tri()}, \texttt{lower.tri()}, and \texttt{diag()}}{Subsetting with upper.tri(), lower.tri(), and diag()}}\label{utlt}

We can use square brackets alongside the functions \texttt{upper.tri()}, \texttt{lower.tri()}, and \texttt{diag()} to examine the upper triangle, lower triangle, and diagonal parts of a matrix, respectively.
\index[R]{\texttt{upper.tri()}}\index[R]{\texttt{lower.tri()}}\index[R]{\texttt{diag()}}
\vspace{0.1in}

\begin{Shaded}
\begin{Highlighting}[]
\NormalTok{mat }\OtherTok{\textless{}{-}} \FunctionTok{matrix}\NormalTok{(}\AttributeTok{ncol =} \DecValTok{3}\NormalTok{, }\AttributeTok{nrow =} \DecValTok{3}\NormalTok{, }\AttributeTok{data =} \FunctionTok{c}\NormalTok{(}\DecValTok{1}\NormalTok{, }\DecValTok{2}\NormalTok{, }\DecValTok{3}\NormalTok{, }\DecValTok{2}\NormalTok{, }\DecValTok{4}\NormalTok{, }\DecValTok{3}\NormalTok{, }\DecValTok{5}\NormalTok{, }\DecValTok{1}\NormalTok{, }\DecValTok{4}\NormalTok{))}
\NormalTok{mat}
\end{Highlighting}
\end{Shaded}

\begin{verbatim}
     [,1] [,2] [,3]
[1,]    1    2    5
[2,]    2    4    1
[3,]    3    3    4
\end{verbatim}

\vspace{0.1in}

\begin{Shaded}
\begin{Highlighting}[]
\NormalTok{mat[}\FunctionTok{upper.tri}\NormalTok{(mat)]}
\end{Highlighting}
\end{Shaded}

\begin{verbatim}
[1] 2 5 1
\end{verbatim}

\vspace{0.1in}

\begin{Shaded}
\begin{Highlighting}[]
\NormalTok{mat[}\FunctionTok{lower.tri}\NormalTok{(mat)]}
\end{Highlighting}
\end{Shaded}

\begin{verbatim}
[1] 2 3 3
\end{verbatim}

\vspace{0.1in}

\begin{Shaded}
\begin{Highlighting}[]
\FunctionTok{diag}\NormalTok{(mat)}
\end{Highlighting}
\end{Shaded}

\begin{verbatim}
[1] 1 4 4
\end{verbatim}

Note that \texttt{upper.tri()} and \texttt{lower.tri()} are used identify the appropriate triangle in the object \texttt{mat}. Subsetting is then accomplished using square brackets.

\section{Packages}\label{package}

An \textbf{R} \emph{package} contains a set of related functions, documentation, and (often) data files that have been bundled together.\index[terms]{R!package} The so-called \textbf{R}-\emph{distribution packages} are included with a conventional download of \textbf{R} (Table \ref{tab:Rdist}). These packages are directly controlled by the \textbf{R} core development team and are extremely well-vetted and trustworthy.

Packages in Table \ref{tab:Rrecc} constitute the \textbf{R}-\emph{recommended packages}. These are not necessarily controlled by the \textbf{R} core development team, but are also extremely useful, well-tested, and stable, and like the \textbf{R}-distribution packages, are included in conventional downloads of \textbf{R}.

Aside from distribution and recommended packages, there are a large number of \emph{contributed packages} that have been created by \textbf{R}-users (\(> 20000\) as of 9/12/2023). Table \ref{tab:Rcon} lists a few.

\subsection{Package Installation}\label{packagei}

Contributed packages can be \emph{installed} from CRAN (the Comprehensive \textbf{R} Archive Network)\index[terms]{R!CRAN (archive network)}. To do this, one can go to \textbf{Packages\(>\)Install package(s)} on the \textbf{R}-GUI toolbar, and choose a nearby CRAN mirror site to minimize download time (non-Unix only). Once a mirror site is selected, the packages available at the site will appear. One can simply click on the desired packages to install them. Packages can also be downloaded directly from the command line using \texttt{install.packages("package\ name")}\index[R]{\texttt{install.packages()}}. Thus, to install the package \emph{vegan} (see Table \ref{tab:Rcon}), I would simply type:
\vspace{0.1in}

\begin{Shaded}
\begin{Highlighting}[]
\FunctionTok{install.packages}\NormalTok{(}\StringTok{"vegan"}\NormalTok{)}
\end{Highlighting}
\end{Shaded}

If local web access is not available, packages can be installed as compressed (.zip, .tar) files which can then be placed manually on a workstation by inserting the package files into the \textbf{library} folder within the top level \textbf{R} directory, or into a path-defined \textbf{R} library folder in a user directory.

The installation pathway for contributed packages can be identified using \texttt{.libPath()}.
\index[R]{\texttt{.libPaths()}}
\vspace{0.1in}

\begin{Shaded}
\begin{Highlighting}[]
\FunctionTok{.libPaths}\NormalTok{()}
\end{Highlighting}
\end{Shaded}

\begin{verbatim}
[1] "C:/Users/ahoken/AppData/Local/R/win-library/4.4"
[2] "C:/Program Files/R/R-4.4.2/library"             
\end{verbatim}

This process can be facilitated in RStudio via the plots and files (see Section \ref{RStudio}).

Several functions exist for updating packages and for comparing currently installed versions packages with their latest versions on CRAN or other repositories. The function \texttt{old.packages()} indicates which currently installed packages which have a (suitable) later version. Here are a few of the packages I have installed that have later versions.
\vspace{0.1in}

\begin{Shaded}
\begin{Highlighting}[]
\FunctionTok{head}\NormalTok{(}\FunctionTok{old.packages}\NormalTok{(}\AttributeTok{repos =} \StringTok{"https://cloud.r{-}project.org"}\NormalTok{))[,}\FunctionTok{c}\NormalTok{(}\DecValTok{1}\NormalTok{,}\DecValTok{3}\NormalTok{,}\DecValTok{4}\NormalTok{,}\DecValTok{5}\NormalTok{)]}
\end{Highlighting}
\end{Shaded}

\begin{verbatim}
         Package    Installed Built   ReposVer 
ape      "ape"      "5.8"     "4.4.1" "5.8-1"  
askpass  "askpass"  "1.2.0"   "4.4.1" "1.2.1"  
bit      "bit"      "4.0.5"   "4.4.1" "4.5.0.1"
bit64    "bit64"    "4.0.5"   "4.4.1" "4.6.0-1"
bookdown "bookdown" "0.39"    "4.4.0" "0.42"   
broom    "broom"    "1.0.6"   "4.4.1" "1.0.7"  
\end{verbatim}

The function \texttt{update.packages()} will identify, and offer to download and install later versions of installed packages.

\index[R]{\texttt{old.packages()}}
\index[R]{\texttt{update.packages()}}

\subsection{Loading Packages}\label{packagel}

Once a contributed package is installed on a computer it never needs to be re-installed. However, for use in an \textbf{R} session, recommended packages, and installed contributed packages will need to be \emph{loaded}. This can be done using the \texttt{library()} function, or point and click tools if one is using RStudio. For example, to load the installed contributed \emph{vegan} package, I would type:

\index[R]{\texttt{library()}}
\vspace{0.1in}

\begin{Shaded}
\begin{Highlighting}[]
\FunctionTok{library}\NormalTok{(vegan)}
\end{Highlighting}
\end{Shaded}

\begin{verbatim}
Loading required package: permute
\end{verbatim}

\begin{verbatim}
Loading required package: lattice
\end{verbatim}

\begin{verbatim}
This is vegan 2.6-8
\end{verbatim}

We see that two other packages are loaded when we load \emph{vegan}: \emph{permute} and \emph{lattice}.

To detach \emph{vegan} from the global environment, I would type:
\vspace{0.1in}

\begin{Shaded}
\begin{Highlighting}[]
\FunctionTok{detach}\NormalTok{(package}\SpecialCharTok{:}\NormalTok{vegan)}
\end{Highlighting}
\end{Shaded}

\index[R]{\texttt{detach()}}

We can check if a specific package is loaded using the function \texttt{.packages()}. Most of the \textbf{R} distribution packages are loaded (by default) upon opening a session. Exceptions include \texttt{compiler}, \texttt{grid}, \texttt{parallel}, \texttt{splines}, \texttt{stats4}, and \texttt{tools}.

\vspace{0.1in}

\begin{Shaded}
\begin{Highlighting}[]
\NormalTok{bpack }\OtherTok{\textless{}{-}} \FunctionTok{c}\NormalTok{(}\StringTok{"base"}\NormalTok{, }\StringTok{"compiler"}\NormalTok{, }\StringTok{"datasets"}\NormalTok{, }\StringTok{"grDevices"}\NormalTok{, }\StringTok{"graphics"}\NormalTok{,}
           \StringTok{"grid"}\NormalTok{, }\StringTok{"methods"}\NormalTok{, }\StringTok{"parallel"}\NormalTok{, }\StringTok{"splines"}\NormalTok{, }\StringTok{"stats"}\NormalTok{, }\StringTok{"stats4"}\NormalTok{, }
           \StringTok{"tcltk"}\NormalTok{, }\StringTok{"tools"}\NormalTok{, }\StringTok{"translations"}\NormalTok{, }\StringTok{"utils"}\NormalTok{)}
\FunctionTok{sapply}\NormalTok{(bpack, }\ControlFlowTok{function}\NormalTok{(x) (x }\SpecialCharTok{\%in\%} \FunctionTok{.packages}\NormalTok{()))}
\end{Highlighting}
\end{Shaded}

\begin{verbatim}
        base     compiler     datasets    grDevices     graphics 
        TRUE        FALSE         TRUE         TRUE         TRUE 
        grid      methods     parallel      splines        stats 
       FALSE         TRUE        FALSE        FALSE         TRUE 
      stats4        tcltk        tools translations        utils 
       FALSE         TRUE        FALSE        FALSE         TRUE 
\end{verbatim}

The function \texttt{sapply()}, which allows application of a function to each element in a vector or list, is formally introduced in Section \ref{apply}.

The package \emph{vegan} is no longer loaded because of the application of \texttt{detach(package:vegan)}.
\vspace{0.1in}

\begin{Shaded}
\begin{Highlighting}[]
\StringTok{"vegan"} \SpecialCharTok{\%in\%} \FunctionTok{.packages}\NormalTok{()}
\end{Highlighting}
\end{Shaded}

\begin{verbatim}
[1] FALSE
\end{verbatim}

\index[R]{\texttt{sapply()}}
\index[R]{\texttt{packages()}}

We can get a summary of information about a session, including details about the version of \textbf{R} being used, the underlying computer platform, and the loaded packages with the function \texttt{sessionInfo()}.
\vspace{0.1in}

\begin{Shaded}
\begin{Highlighting}[]
\NormalTok{si }\OtherTok{\textless{}{-}} \FunctionTok{sessionInfo}\NormalTok{()}
\NormalTok{si}\SpecialCharTok{$}\NormalTok{R.version}\SpecialCharTok{$}\NormalTok{version.string}
\end{Highlighting}
\end{Shaded}

\begin{verbatim}
[1] "R version 4.4.2 (2024-10-31 ucrt)"
\end{verbatim}

\vspace{0.1in}

\begin{Shaded}
\begin{Highlighting}[]
\NormalTok{si}\SpecialCharTok{$}\NormalTok{running}
\end{Highlighting}
\end{Shaded}

\begin{verbatim}
[1] "Windows 10 x64 (build 17134)"
\end{verbatim}

\vspace{0.1in}

\begin{Shaded}
\begin{Highlighting}[]
\FunctionTok{head}\NormalTok{(}\FunctionTok{names}\NormalTok{(si}\SpecialCharTok{$}\NormalTok{loadedOnly))}
\end{Highlighting}
\end{Shaded}

\begin{verbatim}
[1] "nlme"         "cli"          "rlang"        "xfun"        
[5] "deSolve"      "multcompView"
\end{verbatim}

This information is important to include when reporting issues to package maintainers.
\index[R]{\texttt{sessionInfo()}}

Once a package is installed its functions can generally be accessed using the \emph{double colon} metacharacter, \texttt{::}, even if the package is not actually loaded. For instance, the function \texttt{vegan::diversity()} will allow access to the function \texttt{diversity()} from \emph{vegan}, even when \emph{vegan} is not loaded.
\vspace{0.1in}

\begin{Shaded}
\begin{Highlighting}[]
\FunctionTok{head}\NormalTok{(vegan}\SpecialCharTok{::}\NormalTok{diversity)[}\DecValTok{1}\SpecialCharTok{:}\DecValTok{2}\NormalTok{]}
\end{Highlighting}
\end{Shaded}

\begin{verbatim}
[1] function (x, index = "shannon", groups, equalize.groups = FALSE, 
[2]     MARGIN = 1, base = exp(1))                                   
\end{verbatim}

The \emph{triple colon} metacharacter, \texttt{:::}, can be used to access internal package functions. These functions, however, are generally kept internal for good reason, and probably shouldn't be used outside of the context of the rest of the package.
\index[R]{*a2c@\texttt{::}}
\index[R]{*a3c@\texttt{:::}}
\index[R]{\texttt{vegan::diversity()}}

\subsection{Other Package Repositories}\label{other-package-repositories}

Aside from CRAN, there are currently three other extensive repositories of \textbf{R} packages. First, the Bioconductor project (\url{http://www.bioconductor.org/packages/release/Software/html}) contains a large number of packages for the analysis of data from current and emerging biological assays.\index[terms]{bioconductor@Bioconductor (package repository)} Bioconductor packages are generally not stored at CRAN. Packages can be downloaded from Bioconductor using an \texttt{R} script called \texttt{biocLite}. To access the script and download the package \emph{RCytoscape} from Biocondctor, I could type:
\vspace{0.1in}

\begin{Shaded}
\begin{Highlighting}[]
\FunctionTok{source}\NormalTok{(}\StringTok{"http://bioconductor.org/biocLite.R"}\NormalTok{)}
\FunctionTok{biocLite}\NormalTok{(}\StringTok{"RCytoscape"}\NormalTok{)}
\end{Highlighting}
\end{Shaded}

Second, the \emph{Posit Package Manager} (formerly the RStudio Package Manager) provides a repository interface for \textbf{R} packages from CRAN, Bioconductor, and packages for the Python system (see Section \ref{python}). Third, \textbf{R}-forge (\url{http://r-forge.r-project.org/}) contains releases of packages that have not yet been implemented into CRAN, and other miscellaneous code. Bioconductor, Posit, and \textbf{R}-forge can be specified as repositories from \textbf{Packages\(>\)Select Repositories} in the \textbf{R}-GUI (non-Unix only).\index[terms]{R-forge (package repository)} Other informal \textbf{R} package and code repositories currently include \href{https://github.com/}{GitHub} and \href{https://zenodo.org/}{Zenodo}.
\index[terms]{Github (package and code repository)}
\index[terms]{Zenodo (package and code repository)}
\index[terms]{Posit package manager (package repository)}

\begin{landscape}
\begin{table}[ht]
\centering
\caption{The \textbf{R}-distribution packages.} 
\label{tab:Rdist}
\begin{tabular}{llll}
  \toprule
Package & Maintainer & Topic(s) addressed by package & Author/Citation \\ 
  \midrule
\textit{base} & \textbf{R} Core Team & Base \textbf{R} functions & \citet{R} \\ 
  \textit{compiler} & \textbf{R} Core Team & \textbf{R} byte code compiler & \citet{R} \\ 
  \textit{datasets} & \textbf{R} Core Team & Base \textbf{R} datasets & \citet{R} \\ 
  \textit{grDevices} & \textbf{R} Core Team & Devices for base and grid graphics & \citet{R} \\ 
  \textit{graphics} & \textbf{R} Core Team & \textbf{R} functions for base graphics & \citet{R} \\ 
  \textit{grid} & \textbf{R} Core Team & Grid graphics layout capabilities & \citet{R} \\ 
  \textit{methods} & \textbf{R} Core Team & Formal methods and classes for \textbf{R} objects & \citet{R} \\ 
  \textit{parallel} & \textbf{R} Core Team & Support for parallel computation & \citet{R} \\ 
  \textit{splines} & \textbf{R} Core Team & Regression spline functions and classes & \citet{R} \\ 
  \textit{stats} & \textbf{R} Core Team & \textbf{R} statistical functions & \citet{R} \\ 
  \textit{stats4} & \textbf{R} Core Team & Statistical functions with S4 classes & \citet{R} \\ 
  \textit{tcltk} & \textbf{R} Core Team & Language bindings to Tcl/Tk & \citet{R} \\ 
  \textit{tools} & \textbf{R} Core Team & Tools forpackage development/administration & \citet{R} \\ 
  \textit{utils} & \textbf{R} Core Team & \textbf{R} utility functions & \citet{R} \\ 
   \bottomrule
\end{tabular}
\end{table}

\index[terms]{R!package!\textit{base}}
\index[terms]{R!package!\textit{compiler}}
\index[terms]{R!package!\textit{datasets}}
\index[terms]{R!package!\textit{grDevices}}
\index[terms]{R!package!\textit{graphics}}
\index[terms]{R!package!\textit{grid}}
\index[terms]{R!package!\textit{methods}}
\index[terms]{R!package!\textit{parallel}}
\index[terms]{R!package!\textit{splines}}
\index[terms]{R!package!\textit{stats}}
\index[terms]{R!package!\textit{stats4}}
\index[terms]{R!package!\textit{tcltk}}
\index[terms]{R!package!\textit{tools}}
\index[terms]{R!package!\textit{utils}}

\begin{table}[ht]
\centering
\caption{The \textbf{R}-recommended packages.} 
\label{tab:Rrecc}
\begin{tabular}{llll}
  \toprule
Package & Maintainer & Topic(s) addressed by package & Author/Citation \\ 
  \midrule
\textit{KernSmooth} & B. Ripley & Kernel smoothing & \citet{wand} \\ 
  \textit{MASS} & B. Ripley & Important statistical methods & \citet{venables2002} \\ 
  \textit{Matrix} & M. Maechler & Classes and methods for matrices & \citet{Matrix} \\ 
  \textit{boot} & B. Ripley & Bootstrapping & \citet{boot} \\ 
  \textit{class} & B. Ripley & Classification & \citet{venables2002} \\ 
  \textit{cluster} & M. Maechler & Cluster analysis & \citet{cluster} \\ 
  \textit{codetools} & S. Wood & Code analysis tools & \citet{codetools} \\ 
  \textit{foreign} & \textbf{R} core team & Data stored by non-\textbf{R} software & \citet{R} \\ 
  \textit{lattice} & D. Sarkar & Lattice graphics & \citet{lattice} \\ 
  \textit{mgcv} & S. Wood & Generalized Additive Models & \citet{wood2011, wood2017} \\ 
  \textit{nlme} & \textbf{R} core team & Linear and non-linear mixed effect models & \citet{pinheiro2000} \\ 
  \textit{nnet} & B. Ripley & Feed-forward neural networks & \citet{venables2002} \\ 
  \textit{rpart} & B. Ripley & Partitioning and regression trees & \citet{venables2002} \\ 
  \textit{spatial} & B. Ripley & Kriging and point pattern analysis & \citet{venables2002} \\ 
   \bottomrule
\end{tabular}
\end{table}

\index[terms]{R!package!\textit{KernSmooth}}
\index[terms]{R!package!\textit{MASS}}
\index[terms]{R!package!\textit{Matrix}}
\index[terms]{R!package!\textit{grDevices}}
\index[terms]{R!package!\textit{boot}}
\index[terms]{R!package!\textit{class}}
\index[terms]{R!package!\textit{cluster}}
\index[terms]{R!package!\textit{codetools}}
\index[terms]{R!package!\textit{foreign}}
\index[terms]{R!package!\textit{lattice}}
\index[terms]{R!package!\textit{mgcv}}
\index[terms]{R!package!\textit{nlme}}
\index[terms]{R!package!\textit{tabular}}
\index[terms]{R!package!\textit{nnet}}
\index[terms]{R!package!\textit{rpart}}
\index[terms]{R!package!\textit{tabular}}
\index[terms]{R!package!\textit{spatial}}
\index[terms]{R!package!\textit{survival}}

\begin{table}[ht]
\centering
\caption{Useful contributed \textbf{R} packages.} 
\label{tab:Rcon}
\begin{tabular}{llll}
  \toprule
Package & Maintainer & Topic(s) addressed by package & Author/Citation \\ 
  \midrule
\textit{asbio} & K. Aho & Stats pedagogy and applied stats & \citet{aho2023} \\ 
  \textit{car} & J. Fox & General linear models & \citet{fox2019} \\ 
  \textit{coin} & T. Hothorn & Non-parametric analysis & \citet{hothorn2006, hothorn2008} \\ 
  \textit{ggplot2} & H. Wickham & \textit{Tidyverse} grid graphics & \citet{wickham2016} \\ 
  \textit{lme4} & B. Bolker & Linear mixed-effects models & \citet{lme4} \\ 
  \textit{plotrix} & J. Lemonetal. & Helpful graphical ideas & \citet{plotrix} \\ 
  \textit{spdep} & R. Bivand & Spatial analysis & \citet{bivand2013, pebesma} \\ 
  \textit{tidyverse} & H. Wickham & Data science under the \textit{tidyverse} & \citet{tidyverse} \\ 
  \textit{vegan} & J. Oksanen & Multivariate and ecological analysis & \citet{vegan} \\ 
   \bottomrule
\end{tabular}
\end{table}
\end{landscape}

\index[terms]{R!package!\textit{asbio}}
\index[terms]{R!package!\textit{car}}
\index[terms]{R!package!\textit{coin}}
\index[terms]{R!package!\textit{ggplot2}}
\index[terms]{R!package!\textit{lme4}}
\index[terms]{R!package!\textit{plotrix}}
\index[terms]{R!package!\textit{spdep}}
\index[terms]{R!package!\textit{tidyverse}}
\index[terms]{R!package!\textit{vegan}}

\subsection{Accessing Package Information}\label{accessing-package-information}

Important information concerning a package can be obtained from the \texttt{packageDescription()} family of functions. Here is the version of the \textbf{R} contributed package \emph{asbio} on my work station:
\vspace{0.1in}

\begin{Shaded}
\begin{Highlighting}[]
\FunctionTok{packageVersion}\NormalTok{(}\StringTok{"asbio"}\NormalTok{)}
\end{Highlighting}
\end{Shaded}

\begin{verbatim}
[1] '1.11'
\end{verbatim}

Here is the version of \textbf{R} used to build the installed version of \emph{asbio}, and the package's build date:
\vspace{0.1in}

\begin{Shaded}
\begin{Highlighting}[]
\FunctionTok{packageDescription}\NormalTok{(}\StringTok{"asbio"}\NormalTok{, }\AttributeTok{fields=}\StringTok{"Built"}\NormalTok{)}
\end{Highlighting}
\end{Shaded}

\begin{verbatim}
[1] "R 4.4.2; ; 2025-01-21 02:43:26 UTC; windows"
\end{verbatim}

\index[R]{\texttt{packageVersion()}}
\index[R]{\texttt{packageDescription()}}

\subsection{Accessing Datasets in R-packages}\label{accessing-datasets-in-r-packages}

The command:
\vspace{0.1in}

\begin{Shaded}
\begin{Highlighting}[]
\FunctionTok{data}\NormalTok{()}
\end{Highlighting}
\end{Shaded}

results in a listing of a datasets available in a session from within \textbf{R} packages loaded in a particular \textbf{R} session. Whereas the code:
\vspace{0.1in}

\begin{Shaded}
\begin{Highlighting}[]
\FunctionTok{data}\NormalTok{(}\AttributeTok{package =} \FunctionTok{.packages}\NormalTok{(}\AttributeTok{all.available =} \ConstantTok{TRUE}\NormalTok{))}
\end{Highlighting}
\end{Shaded}

results in a listing of a datasets available in a session from within \emph{installed} \textbf{R} packages.

If one is interested in datasets from a particular package, for instance the package \emph{datasets}, one could type:
\vspace{0.1in}

\begin{Shaded}
\begin{Highlighting}[]
\FunctionTok{data}\NormalTok{(}\AttributeTok{package =} \StringTok{"datasets"}\NormalTok{)}
\end{Highlighting}
\end{Shaded}

The dataset \texttt{Loblolly} in the \emph{datasets} package contains height, age, and seed type records for a sample of loblolly pine trees (\emph{Pinus taeda}). To access the data we can type:\footnote{This actually isn't necessary since datasets in the package \emph{dataset} default to being read into an \textbf{R}-session automatically. This step will be necessary, however, in order to obtain datasets from packages that are not lazy loaded (Ch 10).}
\vspace{0.1in}

\begin{Shaded}
\begin{Highlighting}[]
\FunctionTok{data}\NormalTok{(Loblolly)}
\end{Highlighting}
\end{Shaded}

The data are now contained in a dataframe (called \texttt{Loblolly}) that we can manipulate and analyze.
\vspace{0.1in}

\begin{Shaded}
\begin{Highlighting}[]
\FunctionTok{class}\NormalTok{(Loblolly)}
\end{Highlighting}
\end{Shaded}

\begin{verbatim}
[1] "nfnGroupedData" "nfGroupedData"  "groupedData"    "data.frame"    
\end{verbatim}

Note that there are three classes (\texttt{nfnGroupedData,\ nfGroupedData,\ groupedData}) in addition to \texttt{dataframe}. These classes allow recognition of the nested structure of the \texttt{age} and \texttt{Seed} variables (defined to \texttt{height} is a function of \texttt{age} \emph{in} \texttt{Seed}), and facilitates the analysis of the data using mixed effect model algorithms in the package \emph{nlme}. We can get a general feel for the \texttt{Loblolly} dataset by accessing the first few rows using the function \texttt{head()}\index[R]{\texttt{head()}}:
\vspace{0.1in}

\begin{Shaded}
\begin{Highlighting}[]
\FunctionTok{head}\NormalTok{(Loblolly, }\DecValTok{5}\NormalTok{)}
\end{Highlighting}
\end{Shaded}

\begin{verbatim}
Grouped Data: height ~ age | Seed
   height age Seed
1    4.51   3  301
15  10.89   5  301
29  28.72  10  301
43  41.74  15  301
57  52.70  20  301
\end{verbatim}

The function \texttt{summary()}\index[R]{\texttt{summary()}} provides the mean and a conventional \emph{five number summary} (minimum, 1st quartile, median, 3rd quartile, maximum) of both quantitative variables (\texttt{height} and \texttt{age}) and a count of the number of observations (six) in each level of the categorical variable \texttt{Seed}.
\vspace{0.1in}

\begin{Shaded}
\begin{Highlighting}[]
\FunctionTok{summary}\NormalTok{(Loblolly)}
\end{Highlighting}
\end{Shaded}

\begin{verbatim}
     height           age            Seed   
 Min.   : 3.46   Min.   : 3.0   329    : 6  
 1st Qu.:10.47   1st Qu.: 5.0   327    : 6  
 Median :34.00   Median :12.5   325    : 6  
 Mean   :32.36   Mean   :13.0   307    : 6  
 3rd Qu.:51.36   3rd Qu.:20.0   331    : 6  
 Max.   :64.10   Max.   :25.0   311    : 6  
                                (Other):48  
\end{verbatim}

\textbf{R} provides a spreadsheet-style data editor if one types \texttt{fix(x)}, when \texttt{x} is a dataframe or a two dimensional array. For instance, the command \texttt{fix(loblolly)} will open the \texttt{Loblolly} pine dataframe in the data editor (Figure \ref{fig:fix}). When \texttt{x} is a function or character string, then a script editor is opened containing \texttt{x}. The data editor has limited flexibility compared to software whose main interface is a spreadsheet, and whose primary purpose is data entry and manipulation, e.g., Microsoft Excel\(^{\circledR}\). Changes made to an object using \texttt{fix()} will only be maintained for the current work session. They will not permanently alter objects brought in remotely to a session. The function \texttt{View(x)} (RStudio only) will provide a non-editable spreadsheet representation of a dataframe or numeric array.
\index[R]{\texttt{fix()}}\index[R]{view@\texttt{View()}}

\begin{figure}[H]

{\centering \includegraphics[width=0.4\linewidth,]{figs3/Fix} 

}

\caption{The default \textbf{R} spreadsheet editor.}\label{fig:fix}
\end{figure}

\section{Facilitating Command Line Data Entry}\label{facilitating-command-line-data-entry}

Command line data entry is made easier with with several \textbf{R} functions. The function \texttt{scan()}\index[R]{\texttt{scan()}} can speed up data entry because a prompt is given for each data point\footnote{If one uses the default \texttt{scan()} argument: \texttt{file\ =\ ""}. See Section \ref{rt}.}, and separators are created by the function itself. Data entries can be separated using the space bar or line breaks. The \texttt{scan()} function will be terminated by a additional blank line or an end of file (EOF) signal.\index[terms]{End of file (EOF) signal} This will be Ctrl + D in Unix-alike operating systems and Windows.

Below I enter the numbers 1, 2, and 3 as datapoints, separated by spaces, and end data entry using an additional line break. The data are saved as the object \texttt{a}.
\vspace{0.1in}

\begin{Shaded}
\begin{Highlighting}[]
\NormalTok{a }\OtherTok{\textless{}{-}} \FunctionTok{scan}\NormalTok{()}
\DecValTok{1}\SpecialCharTok{:} \DecValTok{1} \DecValTok{2} \DecValTok{3}
\DecValTok{4}\SpecialCharTok{:}
\NormalTok{Read }\DecValTok{3}\NormalTok{ items}
\end{Highlighting}
\end{Shaded}

Sequences can be generated quickly in \textbf{R} using the \texttt{:}\index[R]{*a1c@\texttt{:}} operator
\vspace{0.1in}

\begin{Shaded}
\begin{Highlighting}[]
\DecValTok{1}\SpecialCharTok{:}\DecValTok{10}
\end{Highlighting}
\end{Shaded}

\begin{verbatim}
 [1]  1  2  3  4  5  6  7  8  9 10
\end{verbatim}

or the function \texttt{seq()}\index[R]{\texttt{seq()}}, which allows additional options:
\vspace{0.1in}

\begin{Shaded}
\begin{Highlighting}[]
\FunctionTok{seq}\NormalTok{(}\DecValTok{1}\NormalTok{, }\DecValTok{10}\NormalTok{)}
\end{Highlighting}
\end{Shaded}

\begin{verbatim}
 [1]  1  2  3  4  5  6  7  8  9 10
\end{verbatim}

\vspace{0.1in}

\begin{Shaded}
\begin{Highlighting}[]
\FunctionTok{seq}\NormalTok{(}\DecValTok{1}\NormalTok{, }\DecValTok{10}\NormalTok{, }\AttributeTok{by =} \DecValTok{2}\NormalTok{) }\CommentTok{\# 1 to 10 by two}
\end{Highlighting}
\end{Shaded}

\begin{verbatim}
[1] 1 3 5 7 9
\end{verbatim}

\vspace{0.1in}

\begin{Shaded}
\begin{Highlighting}[]
\FunctionTok{seq}\NormalTok{(}\DecValTok{1}\NormalTok{, }\DecValTok{10}\NormalTok{, }\AttributeTok{length =} \DecValTok{4}\NormalTok{) }\CommentTok{\# 1 to 10 in four evenly spaced points}
\end{Highlighting}
\end{Shaded}

\begin{verbatim}
[1]  1  4  7 10
\end{verbatim}

Entries can be repeated with the function \texttt{rep()}\index[R]{\texttt{rep()}}. For example, to repeat the sequence 1 through 5, five times, I could type:
\vspace{0.1in}

\begin{Shaded}
\begin{Highlighting}[]
\FunctionTok{rep}\NormalTok{(}\FunctionTok{c}\NormalTok{(}\DecValTok{1}\SpecialCharTok{:}\DecValTok{5}\NormalTok{), }\DecValTok{5}\NormalTok{)}
\end{Highlighting}
\end{Shaded}

\begin{verbatim}
 [1] 1 2 3 4 5 1 2 3 4 5 1 2 3 4 5 1 2 3 4 5 1 2 3 4 5
\end{verbatim}

Note that the first argument in \texttt{rep()}, defines the thing we want to repeat and the second argument, 5, specifies the number of repetitions. I can use the argument \texttt{each} to repeat individual elements a particular number of times.
\vspace{0.1in}

\begin{Shaded}
\begin{Highlighting}[]
\FunctionTok{rep}\NormalTok{(}\FunctionTok{c}\NormalTok{(}\DecValTok{1}\SpecialCharTok{:}\DecValTok{5}\NormalTok{), }\AttributeTok{each =} \DecValTok{2}\NormalTok{)}
\end{Highlighting}
\end{Shaded}

\begin{verbatim}
 [1] 1 1 2 2 3 3 4 4 5 5
\end{verbatim}

We can use \texttt{seq()} and \texttt{rep()} simultaneously to create complex sequences. For instance, to repeat the sequence 1,3,5,7,9,11,13,15,17,19, three times, we could type:
\vspace{0.1in}

\begin{Shaded}
\begin{Highlighting}[]
\FunctionTok{rep}\NormalTok{(}\FunctionTok{seq}\NormalTok{(}\DecValTok{1}\NormalTok{, }\DecValTok{20}\NormalTok{, }\AttributeTok{by =} \DecValTok{2}\NormalTok{), }\DecValTok{3}\NormalTok{)}
\end{Highlighting}
\end{Shaded}

\begin{verbatim}
 [1]  1  3  5  7  9 11 13 15 17 19  1  3  5  7  9 11 13 15 17 19  1  3  5
[24]  7  9 11 13 15 17 19
\end{verbatim}

\section{Importing Data Into R}\label{importing-data-into-r}

While it is possible to enter data into \textbf{R} at the command line, this will normally be inadvisable except for small datasets. In general it will be much easier to import data. \textbf{R} can read data from many different kinds of formats including .txt, and .csv (comma separator) files, and files with space, tab, and carriage return datum separators. I generally organize my datasets using Excel\(^{\circledR}\) or some other spreadsheet program (although \textbf{R} can handle much larger datasets than these platforms), then save them as .csv files. I then import the .csv files into \textbf{R} using the \texttt{read.table()}, \texttt{read.csv()}, or \texttt{scan()} functions. The function \texttt{load()}\index[R]{\texttt{load()}} can be used to import data files in .rda\index[terms]{R!.rda file} data formats, or other \textbf{R} objects. Datasets read into \textbf{R} will generally be of class \texttt{dataframe} and data storage mode \texttt{list}.

\subsection{\texorpdfstring{Import Using \texttt{read.table()}, \texttt{read.csv()}, and \texttt{scan()}}{Import Using read.table(), read.csv(), and scan()}}\label{rt}

The \texttt{read.table()}\index[R]{\texttt{read.table()}} function can import data organized under a wide range of formats. It's first three arguments are very important.

\begin{itemize}
\tightlist
\item
  \texttt{file} defines the name of the file and directory hierarchy which the data are to be read from.
\item
  \texttt{header} is a a logical (\texttt{TRUE} or \texttt{FALSE}) value indicating whether \texttt{file} contains column names as its first line.
\item
  \texttt{sep} refers to the type of data separator used for columns. Comma separated files use commas to separate data entries. Thus, in this case \texttt{sep\ =\ ","}. Tab separators are specified as \texttt{"\textbackslash{}t"}. Space separators are specified as spaces, specified as simply \texttt{"\ "}.
\end{itemize}

Other useful \texttt{read.table()} arguments include \texttt{row.names}, \texttt{header}, and \texttt{na.strings}. The specification \texttt{row.names\ =\ 1} indicates that the first column in the imported dataset contains row names. The specification \texttt{header\ =\ TRUE}, the default setting, indicates that the first row of data contains column names. The argument \texttt{na.strings\ =\ "."} indicates that missing values in the imported dataset are designated with periods. By default \texttt{na.strings\ =\ NA}.

As an example of \texttt{read.table()} usage, assume that I want to import a .csv file called \texttt{veg.csv} located in folder called veg\_data, in my working directory. The first row of \texttt{veg.csv} contains column names, while the first column contains row names. Missing data in the file are indicated with periods. I would type:
\vspace{0.1in}

\begin{Shaded}
\begin{Highlighting}[]
\FunctionTok{read.table}\NormalTok{(}\StringTok{"veg\_data/veg.csv"}\NormalTok{, }\AttributeTok{sep =} \StringTok{","}\NormalTok{, }\AttributeTok{header =} \ConstantTok{TRUE}\NormalTok{, row.names}
\OtherTok{=} \DecValTok{1}\NormalTok{, }\AttributeTok{na.strings =} \StringTok{"."}\NormalTok{)}
\end{Highlighting}
\end{Shaded}

As before, note that as a legacy of its development under Unix, \textbf{R} locates files in directories using forward slashes (or doubled backslashes) rather than single Windows backslashes.

The \texttt{read.csv()} function assumes data are in a .csv format. Because the argument \texttt{sep} is unnecessary, this results in a simpler code statement.
\vspace{0.1in}

\begin{Shaded}
\begin{Highlighting}[]
\FunctionTok{read.csv}\NormalTok{(}\StringTok{"veg\_data}\SpecialCharTok{\textbackslash{}\textbackslash{}}\StringTok{veg.csv"}\NormalTok{, }\AttributeTok{header =} \ConstantTok{TRUE}\NormalTok{, row.names}
\OtherTok{=} \DecValTok{1}\NormalTok{, }\AttributeTok{na.strings =} \StringTok{"."}\NormalTok{)}
\end{Highlighting}
\end{Shaded}

The function \texttt{scan()}\index[R]{\texttt{scan()}} can read in data from an essentially unlimited number of formats, and is extremely flexible with respect to character fields and storage modes of numeric data.
In addition to arguments used by \texttt{read.table()}, \texttt{scan()} has the arguments

\begin{itemize}
\tightlist
\item
  \texttt{what} which describes the storage mode of data e.g., \texttt{"logical",\ "integer"}, etc., or if \texttt{what} is a list, components of variables including column names (see below), and
\item
  \texttt{dec} which describes the decimal point character (European scientists and journals often use commas).
\end{itemize}

Assume that \texttt{veg\_data/veg.csv} has a column of species names, called \texttt{species}, that will serve as the dataframe's row names, and 3 columns of numeric data, named \texttt{site1,\ site2}, and \texttt{site3}. We would read the data in with scan using:
\vspace{0.1in}

\begin{Shaded}
\begin{Highlighting}[]
\FunctionTok{scan}\NormalTok{(}\StringTok{"veg.csv"}\NormalTok{, }\AttributeTok{what =} \FunctionTok{list}\NormalTok{(}\AttributeTok{species =} \StringTok{""}\NormalTok{, }\AttributeTok{site1 =} \DecValTok{0}\NormalTok{, }\AttributeTok{site2 =} \DecValTok{0}\NormalTok{, }\AttributeTok{site3 =} \DecValTok{0}\NormalTok{),}
\AttributeTok{na.strings =} \StringTok{"."}\NormalTok{)}
\end{Highlighting}
\end{Shaded}

The empty string \texttt{species\ =\ ""} in the list comprising the argument \texttt{what}, indicates that \texttt{species} contains character data. Stating that the remaining variables equal 0, or any other number, indicates that they contain numeric data.

The easiest way to import data, if the directory structure is unknown or complex, is to use \texttt{read.csv()} or \texttt{read.table()}, with the \texttt{file.choose()}\index[R]{\texttt{file.choose()}} function as the \texttt{file} argument. For instance, by typing:
\vspace{0.1in}

\begin{Shaded}
\begin{Highlighting}[]
\NormalTok{df }\OtherTok{\textless{}{-}} \FunctionTok{read.csv}\NormalTok{(}\FunctionTok{file.choose}\NormalTok{())}
\end{Highlighting}
\end{Shaded}

We can now browse for a .csv files to open that will, following import, be a dataframe with the name \texttt{df}. Other arguments (e.g., \texttt{header}, \texttt{row.names}) will need to be used, when appropriate, to import the file correctly.

Occasionally strange characters, e.g., \texttt{ï..}, may appear in front of the first header name when reading in files created in Excel\(^{\circledR}\) or other Microsoft applications. This is due to the addition of \emph{Byte Order Mark} (BOM) characters which indicate, among other things, the Unicode character encoding of the file. These characters can generally be eliminated by using the argument \texttt{fileEncoding="UTF-8-BOM"} in \texttt{read.table()}, \texttt{read.csv()}, or \texttt{scan()}.
\index[terms]{Byte Order Mark (BOM)}

\subsection{Import Using RStudio}\label{import-using-rstudio}

RStudio allows direct menu-driven import of file types from a number of spreadsheet and statistical packages including Excel\(^{\circledR}\), SPSS\(^{\circledR}\), SAS\(^{\circledR}\), and Stata\(^{\circledR}\) by going to \textbf{File\(>\)Import Dataset}. We note, however, that restrictions may exist, which may not be present for \texttt{read.table()} and \texttt{read.csv()}. These are summarized in Table \ref{tab:RStudioImport}.

\begin{table}[ht]
\centering
\caption{Data import options in RStudio by data storage file type.} 
\label{tab:RStudioImport}
\begin{tabular}{rlll}
  \toprule
 & CSV or Text & Excel$^{\circledR}$ & SAS$^{\circledR}$, SPSS$^{\circledR}$, Stata$^{\circledR}$ \\ 
  \midrule
Import from file system or URL & X & X & X \\ 
  Change column data types & X & X &  \\ 
  Skip or include columns & X & X & X \\ 
  Rename dataset & X & X &  \\ 
  Skip the first \textit{n} rows & X & X &  \\ 
  Use header row for column names & X &  &  \\ 
  Trim spaces in names & X &  &  \\ 
  Change column delimiter & X &  &  \\ 
  Encodingselection & X &  &  \\ 
  Select quote identifiers & X &  &  \\ 
  Select escape identifiers & X &  &  \\ 
  Select comment identifiers & X &  &  \\ 
  Select \texttt{NA} identifiers & X & X &  \\ 
  Specify model file &  &  & X \\ 
   \bottomrule
\end{tabular}
\end{table}

\subsection{Final Considerations}\label{final-considerations}

It is generally recommended that datasets imported and used by \textbf{R} be smaller than 25\% of the physical memory of the computer. For instance, they should use less than 8 GB on a computer with 32 GB of RAM. \textbf{R} can handle extremely large datasets, i.e.~\(> 10\) GB, and \(> 1.2 \times 10^{10}\) rows. In this case, however, specific \textbf{R} packages can be used to aid in efficient data handling. Parallel computing and workstation modifications may allow even greater efficiency. The actual upper physical limit for an \textbf{R} dataframe is \(2 \times 10^{31}-1\) elements. Note that this exceeds Excel\(^{\circledR}\) by 31 orders of magnitude (Excel 2019 worksheets can handle approximately \(1.7 \times 10^{10}\) cell elements). \textbf{R} also allows interfacing with a number relational database storage platforms. These include open source entities that express queries in SQL (Structured Query Language). For more information see \citet[][pg 178]{chambers2008} and \citet[][pg 157]{adler2010}.\index[terms]{R!memory limits on datasets}

\section*{Exercises}\label{exercises-2}
\addcontentsline{toc}{section}{Exercises}

\begin{enumerate}
\def\labelenumi{\arabic{enumi}.}
\tightlist
\item
  Create the following data structures:

  \begin{enumerate}
  \def\labelenumii{(\alph{enumii})}
  \tightlist
  \item
    An atomic vector object with the numeric entries \texttt{1,2,3,4}.
  \item
    A matrix object with two rows and two columns with the numeric entries \texttt{1,2,3,4}.
  \item
    A dataframe object with two columns; one column containing the numeric entries \texttt{1,2,3,4}, and one column containing the character entries \texttt{"a","b","c","d"}.
  \item
    A list containing the objects created in (b) and (c).
  \item
    Using \texttt{class()}, identify the class and the data storage mode for the objects created in problems a-d.~Discuss the characteristics of the identified classes.
    \vspace{0.1in}
  \end{enumerate}
\item
  Assume that you have developed an \textbf{R} algorithm that saves hourly stream temperature sensor outputs greater than \(20^\text{o}\) from each day as separate dataframes and places them into a list container, because some days may have several points exceeding the threshold and some days may have none. Complete the following based on the list \texttt{hi.temps} given below:

  \begin{enumerate}
  \def\labelenumii{(\alph{enumii})}
  \item
    Combine the dataframes in \texttt{hi.temps} into a single dataframe using \texttt{do.call()}.
  \item
    Create a dataframe consisting of 10 sets of repeated measures from the dataframe \texttt{hi.temps\$day2} using \texttt{do.call()}.
    \vspace{0.1in}

\begin{Shaded}
\begin{Highlighting}[]
\NormalTok{hi.temps }\OtherTok{\textless{}{-}} \FunctionTok{list}\NormalTok{(}\AttributeTok{day1 =} \FunctionTok{data.frame}\NormalTok{(}\AttributeTok{time =} \FunctionTok{c}\NormalTok{(), }\AttributeTok{temp =} \FunctionTok{c}\NormalTok{()),}
                 \AttributeTok{day2 =} \FunctionTok{data.frame}\NormalTok{(}\AttributeTok{time =} \FunctionTok{c}\NormalTok{(}\DecValTok{15}\NormalTok{,}\DecValTok{16}\NormalTok{), }
                                   \AttributeTok{temp =} \FunctionTok{c}\NormalTok{(}\FloatTok{21.1}\NormalTok{,}\FloatTok{22.2}\NormalTok{)),}
                 \AttributeTok{day3 =} \FunctionTok{data.frame}\NormalTok{(}\AttributeTok{time =} \FunctionTok{c}\NormalTok{(}\DecValTok{14}\NormalTok{,}\DecValTok{15}\NormalTok{,}\DecValTok{16}\NormalTok{),}
                                   \AttributeTok{temp =} \FunctionTok{c}\NormalTok{(}\FloatTok{21.3}\NormalTok{,}\FloatTok{20.2}\NormalTok{,}\FloatTok{21.5}\NormalTok{)))}
\end{Highlighting}
\end{Shaded}

    \vspace{0.1in}
  \end{enumerate}
\item
  Given the dataframe \texttt{boo} below, provide solutions to the following questions:

  \begin{enumerate}
  \def\labelenumii{(\alph{enumii})}
  \item
    Identify heights that are less than or equal to 80 inches.
  \item
    Identify heights that are more than 80 inches.
  \item
    Identify females (i.e.~\texttt{F}) greater than or equal to 59 inches but less 63 inches.
  \item
    Subset rows of \texttt{boo} to only contain only data for males (i.e.~\texttt{M}) greater than or equal to 75 inches tall.
  \item
    Find the mean weight of males who are 75 or 76 inches tall.
  \item
    Use \texttt{ifelse()} or \texttt{if()} to classify heights equal to 60 inches as \texttt{"small"}, and heights greater than or equal to 60 inches as \texttt{"tall"}.
    \vspace{0.1in}

\begin{Shaded}
\begin{Highlighting}[]
\NormalTok{boo }\OtherTok{\textless{}{-}} \FunctionTok{data.frame}\NormalTok{(}\AttributeTok{height.in =} \FunctionTok{c}\NormalTok{(}\DecValTok{70}\NormalTok{, }\DecValTok{76}\NormalTok{, }\DecValTok{72}\NormalTok{, }\DecValTok{73}\NormalTok{, }\DecValTok{81}\NormalTok{, }\DecValTok{66}\NormalTok{, }\DecValTok{69}\NormalTok{, }\DecValTok{75}\NormalTok{, }
                                \DecValTok{80}\NormalTok{, }\DecValTok{81}\NormalTok{, }\DecValTok{60}\NormalTok{, }\DecValTok{64}\NormalTok{, }\DecValTok{59}\NormalTok{, }\DecValTok{61}\NormalTok{, }\DecValTok{66}\NormalTok{, }\DecValTok{63}\NormalTok{, }
                                \DecValTok{59}\NormalTok{, }\DecValTok{58}\NormalTok{, }\DecValTok{67}\NormalTok{, }\DecValTok{59}\NormalTok{),}
                  \AttributeTok{weight.lbs =} \FunctionTok{c}\NormalTok{(}\DecValTok{160}\NormalTok{, }\DecValTok{185}\NormalTok{, }\DecValTok{180}\NormalTok{, }\DecValTok{186}\NormalTok{, }\DecValTok{200}\NormalTok{, }\DecValTok{156}\NormalTok{, }
                                 \DecValTok{163}\NormalTok{, }\DecValTok{178}\NormalTok{, }\DecValTok{186}\NormalTok{, }\DecValTok{189}\NormalTok{, }\DecValTok{140}\NormalTok{, }\DecValTok{156}\NormalTok{, }
                                 \DecValTok{136}\NormalTok{, }\DecValTok{141}\NormalTok{, }\DecValTok{158}\NormalTok{, }\DecValTok{154}\NormalTok{, }\DecValTok{135}\NormalTok{, }\DecValTok{120}\NormalTok{, }
                                 \DecValTok{145}\NormalTok{, }\DecValTok{117}\NormalTok{),}
                  \AttributeTok{sex =} \FunctionTok{c}\NormalTok{(}\FunctionTok{rep}\NormalTok{(}\StringTok{"M"}\NormalTok{, }\DecValTok{10}\NormalTok{), }\FunctionTok{rep}\NormalTok{(}\StringTok{"F"}\NormalTok{, }\DecValTok{10}\NormalTok{)))}
\end{Highlighting}
\end{Shaded}
  \end{enumerate}
\item
  Create \texttt{x\ \textless{}-\ NA}, \texttt{y\ \textless{}-\ NaN}, and \texttt{z\ \textless{}-\ NULL}.

  \begin{enumerate}
  \def\labelenumii{(\alph{enumii})}
  \tightlist
  \item
    Test for the class of \texttt{x} using \texttt{x\ ==\ NA} and \texttt{is.na(x)} and discuss the results.
  \item
    Test for the class of \texttt{y} using \texttt{y\ ==\ NaN} and \texttt{is.nan(y)} and discuss the results.
  \item
    Test for the class of \texttt{z} using \texttt{z\ ==\ NULL} and \texttt{is.null(z)} and discuss the results.
  \item
    Discuss \texttt{NA}, \texttt{NaN}, and \texttt{NULL} designations what are these classes used for and what do they represent?
    \vspace{0.1in}
  \end{enumerate}
\item
  For the following questions, use data from Table \ref{tab:Q5} below.

  \begin{enumerate}
  \def\labelenumii{(\alph{enumii})}
  \tightlist
  \item
    Write the data into an \textbf{R} dataframe called \texttt{plant}. Use the functions \texttt{seq()} and \texttt{rep()} to help.
  \item
    Use \texttt{names()} to find the names of the variables.
  \item
    Access the first row of data using square brackets.
  \item
    Access the third column of data using square brackets.
  \item
    Access rows three through five using square brackets.
  \item
    Access all rows \emph{except} rows three, five and seven using square brackets.
  \item
    Access the fourth element from the third column using square brackets.
  \item
    Apply \texttt{na.omit()} to the dataframe and discuss the consequences.
  \item
    Create a copy of \texttt{plant} called \texttt{plant2}. Using square brackets, replace the 7th item in the 2nd column in \texttt{plant2}, an \texttt{NA} value, with the value \texttt{12.1}.
  \item
    Switch the locations of columns two and three in \texttt{plant2} using square brackets.
  \item
    Export the \texttt{plant2} dataframe to your working directory.
  \item
    Convert the \texttt{plant2} dataframe into a matrix using the function \texttt{as.matrix}. Discuss the consequences.

    \begin{table}[ht]
    \centering
    \caption{Data for Question 5.} 
    \label{tab:Q5}
    \begin{tabular}{llll}
      \toprule
    Plant height (dm) & Soil N (\%) & Water index (1-10) & Management type \\ 
      \midrule
    22.3 & 12 & 1 & A \\ 
      21 & 12.5 & 2 & A \\ 
      24.7 & 14.3 & 3 & B \\ 
      25 & 14.2 & 4 & B \\ 
      26.3 & 15 & 5 & C \\ 
      22 & 14 & 6 & C \\ 
      31 &  & 7 & D \\ 
      32 & 15 & 8 & D \\ 
      34 & 13.3 & 9 & E \\ 
      42 & 15.2 & 10 & E \\ 
      28.9 & 13.6 & 1 & A \\ 
      33.3 & 14.7 & 2 & A \\ 
      35.2 & 14.3 & 3 & B \\ 
      36.7 & 16.1 & 4 & B \\ 
      34.4 & 15.8 & 5 & C \\ 
      33.2 & 15.3 & 6 & C \\ 
      35 & 14 & 7 & D \\ 
      41 & 14.1 & 8 & D \\ 
      43 & 16.3 & 9 & E \\ 
      44 & 16.5 & 10 & E \\ 
    \bottomrule
    \end{tabular}
    \end{table}
    \vspace{0.1in}
  \end{enumerate}
\item
  Let:
  \[\boldsymbol{A} = \begin{bmatrix}
  2 & -3\\
  1 & 0
  \end{bmatrix}
  \text{and  } \boldsymbol{b} = \begin{bmatrix}
  1\\
  5
  \end{bmatrix} \]
  Perform the following operations using \textbf{R}:

  \begin{enumerate}
  \def\labelenumii{(\alph{enumii})}
  \tightlist
  \item
    \(\boldsymbol{A}\boldsymbol{b}\)
  \item
    \(\boldsymbol{b}\boldsymbol{A}\)
  \item
    \(det(\boldsymbol{A})\)
  \item
    \(\boldsymbol{A}^{-1}\)
  \item
    \(\boldsymbol{A}'\)
  \end{enumerate}
\end{enumerate}

\vspace{0.1in}

\begin{enumerate}
\def\labelenumi{\arabic{enumi}.}
\setcounter{enumi}{6}
\tightlist
\item
  We can solve systems of linear equations using matrix algebra under the framework \(\boldsymbol{A}\boldsymbol{x} = \boldsymbol{b}\), and (thus) \(\boldsymbol{A}^{-1}\boldsymbol{b} = \boldsymbol{x}\). In this notation \(\boldsymbol{A}\) contains the coefficients from a series of linear equations (by row), \(\boldsymbol{b}\) is a vector of solutions given in the individuals equations, and \(\boldsymbol{x}\) is a vector of solutions sought in the system of models. Thus, for the linear equations:
\end{enumerate}

\[\begin{aligned}
x + y &= 2\\
-x + 3y &= 4
\end{aligned}\]

~~~~~~~~~~~~we have:

\[\boldsymbol{A} = \begin{bmatrix}
1 & 1\\
-1 & 3
\end{bmatrix}, \boldsymbol{ x} = \begin{bmatrix}
x\\
y
\end{bmatrix}, \text{ and  } \boldsymbol{b} = \begin{bmatrix}
2\\
4
\end{bmatrix}.\]

~~~~~~~~~~~Thus, we have

\[\boldsymbol{A}^{-1}\boldsymbol{b} = \boldsymbol{x} = \begin{bmatrix}
1/2\\
3/2
\end{bmatrix}.\]

~~~~~~~~~~~Given this framework, solve the system of equations below with linear algebra using \textbf{R}.

\[
\begin{aligned}
3x + 2y - z &= 1\\
2x - 2y + 4z &= -2\\
-x + 0.5y -z &= 0
\end{aligned}
\]

\vspace{0.1in}

\begin{enumerate}
\def\labelenumi{\arabic{enumi}.}
\setcounter{enumi}{7}
\tightlist
\item
  Complete the following exercises concerning the \textbf{R} contributed package \emph{asbio}:

  \begin{enumerate}
  \def\labelenumii{(\alph{enumii})}
  \tightlist
  \item
    Install\footnote{Installation of packages while knitting of \textbf{R} Markdown or Sweave \textbf{R} code chunks is not allowed. Instead, one should install packages from the console. Required packages can (and should) be \emph{loaded} while knitting once they are installed.} and load the package \emph{asbio} for the current work session.
  \item
    Access the help file for \texttt{bplot()} (a function in \emph{asbio}).
  \item
    Load the dataset \texttt{fly.sex} from \emph{asbio}.
  \item
    Obtain documentation for the dataset \texttt{fly.sex} and describe the dataset variables.
  \item
    Access the column \texttt{longevity} in \emph{fly.sex} using the function \texttt{with()}.
    \vspace{0.1in}
  \end{enumerate}
\item
  Create .csv and .txt datasets, place them in your working directory, and read them into \textbf{R}.
\end{enumerate}

\chapter{Basic Data Management}\label{ch4}

\begin{quote}
\emph{``I think, therefore I \textbf{R}."}
\end{quote}

- \textbf{William B. King}, \emph{Psychologist and \textbf{R} enthusiast}

\(\text{}\)\\
An important characteristic of \textbf{R} is its capacity to efficiently manage and analyze large, complex, datasets. In this chapter I list a few functions and approaches useful for data management in base \textbf{R}. Data management considerations for the \emph{tidyverse} are given in Chapter \ref{ch5}.

\section{Operations on Arrays, Lists and Vectors}\label{operations-on-arrays-lists-and-vectors}

Operators can be applied individually to every row or column of an array, or every component of a list or atomic vector using a number of time saving methods.

\subsection{\texorpdfstring{The \texttt{apply} Family of Functions}{The apply Family of Functions}}\label{apply}

\subsubsection{\texorpdfstring{\texttt{apply()}}{apply()}}\label{apply-1}

Operations can be performed quickly on rows and columns of two dimensional arrays with the function \texttt{apply()}. The function requires three
arguments.\index[R]{\texttt{apply()}}

\begin{itemize}
\tightlist
\item
  The first argument, \texttt{X}, specifies an array to be analyzed.
\item
  The second argument, \texttt{MARGIN}, connotes whether rows or columns are to be analyzed. \texttt{MARGIN\ =\ 1} indicates rows, \texttt{MARGIN\ =\ 2} indicates columns, whereas \texttt{MARGIN\ =\ c(1,\ 2)} indicates rows and columns.
\item
  The third argument, \texttt{FUN}, defines a function to be applied to the margins of the object in the first argument.
\end{itemize}

\begin{example}
\protect\hypertarget{exm:exm-41}{}\label{exm:exm-41}

\hfill\break
Consider the \texttt{asbio::bats} dataset which contains forearm length data, in millimeters, for northern myotis bats (\emph{Myotis septentrionalis}), along with corresponding bat ages in in days.

\vspace{0.1in}

\begin{Shaded}
\begin{Highlighting}[]
\FunctionTok{library}\NormalTok{(asbio)}
\FunctionTok{data}\NormalTok{(bats)}
\FunctionTok{head}\NormalTok{(bats)}
\end{Highlighting}
\end{Shaded}

\begin{verbatim}
  days forearm.length
1    1           10.5
2    1           11.0
3    1           12.3
4    1           13.7
5    1           14.2
6    1           14.8
\end{verbatim}

Here we obtain minimum values for the \texttt{days} and \texttt{forearm.length} columns.

\vspace{0.1in}

\begin{Shaded}
\begin{Highlighting}[]
\FunctionTok{apply}\NormalTok{(bats, }\DecValTok{2}\NormalTok{, min)}
\end{Highlighting}
\end{Shaded}

\begin{verbatim}
          days forearm.length 
           1.0           10.5 
\end{verbatim}

It is straightforward to change the third argument in \texttt{apply()} to obtain different summaries, like the mean.

\vspace{0.1in}

\begin{Shaded}
\begin{Highlighting}[]
\FunctionTok{apply}\NormalTok{(bats, }\DecValTok{2}\NormalTok{, mean)}
\end{Highlighting}
\end{Shaded}

\begin{verbatim}
          days forearm.length 
        13.579         23.603 
\end{verbatim}

or the standard deviation

\vspace{0.1in}

\begin{Shaded}
\begin{Highlighting}[]
\FunctionTok{apply}\NormalTok{(bats, }\DecValTok{2}\NormalTok{, sd)}
\end{Highlighting}
\end{Shaded}

\begin{verbatim}
          days forearm.length 
       12.4610         8.4347 
\end{verbatim}

Several summary statistical functions exist for numerical arrays that can be used in some instances in the place of \texttt{apply()}. These include \texttt{rowMeans()} and \texttt{colMeans()} which give the sample means of specified rows and columns, respectively, and \texttt{rowSums()} and \texttt{colSums()} which give the sums of specified rows and columns, respectively. For instance:

\vspace{0.1in}

\begin{Shaded}
\begin{Highlighting}[]
\FunctionTok{colMeans}\NormalTok{(bats)}
\end{Highlighting}
\end{Shaded}

\begin{verbatim}
          days forearm.length 
        13.579         23.603 
\end{verbatim}

\end{example}

\index[R]{\texttt{rowMeans()}}
\index[R]{\texttt{colMeans()}}
\index[R]{\texttt{rowSums()}}
\index[R]{\texttt{colSums()}}
\(\blacksquare\)

\subsubsection{\texorpdfstring{\texttt{lapply()}}{lapply()}}\label{lapply}

The function \texttt{lapply()} allows one to sweep functions through list components. It has two main arguments:

\begin{itemize}
\tightlist
\item
  The first argument, \texttt{X}, specifies a list to be analyzed.
\item
  The second argument, \texttt{FUN}, defines a function to be applied to each element in \texttt{X}.
\end{itemize}

\begin{example}
\protect\hypertarget{exm:exm-42}{}\label{exm:exm-42}

\hfill\break

Consider the following simple list, whose three components have different lengths.

\vspace{0.1in}

\begin{Shaded}
\begin{Highlighting}[]
\NormalTok{x }\OtherTok{\textless{}{-}} \FunctionTok{list}\NormalTok{(}\AttributeTok{a =} \DecValTok{1}\SpecialCharTok{:}\DecValTok{8}\NormalTok{, }\AttributeTok{norm.obs =} \FunctionTok{rnorm}\NormalTok{(}\DecValTok{10}\NormalTok{), }
          \AttributeTok{logic =} \FunctionTok{c}\NormalTok{(}\ConstantTok{TRUE}\NormalTok{, }\ConstantTok{TRUE}\NormalTok{, }\ConstantTok{FALSE}\NormalTok{, }\ConstantTok{FALSE}\NormalTok{))}
\NormalTok{x}
\end{Highlighting}
\end{Shaded}

\begin{verbatim}
$a
[1] 1 2 3 4 5 6 7 8

$norm.obs
 [1] -0.066028  0.113701 -0.021340  0.107835 -1.104468 -0.276380 -0.750744
 [8]  0.797818  0.117907  0.941843

$logic
[1]  TRUE  TRUE FALSE FALSE
\end{verbatim}

Here we sweep the function \texttt{mean()} through the list:

\vspace{0.1in}

\begin{Shaded}
\begin{Highlighting}[]
\FunctionTok{lapply}\NormalTok{(x, mean)}
\end{Highlighting}
\end{Shaded}

\begin{verbatim}
$a
[1] 4.5

$norm.obs
[1] -0.013986

$logic
[1] 0.5
\end{verbatim}

Note the Boolean outcomes in \texttt{logic} have been coerced to numeric outcomes. Specifically, \texttt{TRUE\ =\ 1} and \texttt{FALSE\ =\ 0}. Here are the 1st, 2nd (median), and 3rd quartiles of \texttt{x}:

\vspace{0.1in}

\begin{Shaded}
\begin{Highlighting}[]
\FunctionTok{lapply}\NormalTok{(x, quantile, }\AttributeTok{probs =} \DecValTok{1}\SpecialCharTok{:}\DecValTok{3}\SpecialCharTok{/}\DecValTok{4}\NormalTok{)}
\end{Highlighting}
\end{Shaded}

\begin{verbatim}
$a
 25%  50%  75% 
2.75 4.50 6.25 

$norm.obs
      25%       50%       75% 
-0.223792  0.043247  0.116856 

$logic
25% 50% 75% 
0.0 0.5 1.0 
\end{verbatim}

\end{example}

\(\blacksquare\)

\index[R]{\texttt{lapply()}}

\subsubsection{\texorpdfstring{\texttt{sapply()}}{sapply()}}\label{sapply}

The function \texttt{sapply()} is a user friendly wrapper for \texttt{lapply()} that can return a vector or array instead of a list.\index[R]{\texttt{sapply()}}

\vspace{0.1in}

\begin{Shaded}
\begin{Highlighting}[]
\FunctionTok{sapply}\NormalTok{(x, quantile, }\AttributeTok{probs =} \DecValTok{1}\SpecialCharTok{:}\DecValTok{3}\SpecialCharTok{/}\DecValTok{4}\NormalTok{)}
\end{Highlighting}
\end{Shaded}

\begin{verbatim}
       a  norm.obs logic
25% 2.75 -0.223792   0.0
50% 4.50  0.043247   0.5
75% 6.25  0.116856   1.0
\end{verbatim}

\subsubsection{\texorpdfstring{\texttt{tapply()}}{tapply()}}\label{tapply}

The \texttt{tapply()} function allows summarization of a one dimensional array (e.g., a column or row from a matrix) with respect to levels in a categorical variable. The function requires three arguments.\index[R]{\texttt{tapply()}}

\begin{itemize}
\tightlist
\item
  The first argument, \texttt{X}, defines a one dimensional array to be analyzed.
\item
  The second argument, \texttt{INDEX} should provide a list of one or more factors (see example below) with the same length as \texttt{X}.
\item
  The third argument, \texttt{FUN}, is used to specify a function to be applied to \texttt{X} for each level in \texttt{INDEX}.
\end{itemize}

\begin{example}
\protect\hypertarget{exm:exm-43}{}\label{exm:exm-43}

\(\text{}\)\\
Consider the dataset \texttt{asbio::heart}, which documents pulse rates for twenty four subjects at four time periods following administration of a experimental treatment. These were two active heart medications and a control. Here are average heart rates for the treatments.

\vspace{0.1in}

\begin{Shaded}
\begin{Highlighting}[]
\FunctionTok{data}\NormalTok{(heart)}
\FunctionTok{with}\NormalTok{(heart, }\FunctionTok{tapply}\NormalTok{(rate, drug, mean))}
\end{Highlighting}
\end{Shaded}

\begin{verbatim}
  AX23   BWW9   Ctrl 
76.281 81.031 71.906 
\end{verbatim}

Here are the mean heart rates for treatments, for each time frame. Note that the second argument is defined as a list with two components, each of which can be coerced to be a factor.

\vspace{0.1in}

\begin{Shaded}
\begin{Highlighting}[]
\FunctionTok{with}\NormalTok{(heart, }\FunctionTok{tapply}\NormalTok{(rate, }\FunctionTok{list}\NormalTok{(}\AttributeTok{drug =}\NormalTok{ drug, }\AttributeTok{time =}\NormalTok{ time), mean))}
\end{Highlighting}
\end{Shaded}

\begin{verbatim}
      time
drug      t1     t2     t3     t4
  AX23 70.50 80.500 81.000 73.125
  BWW9 81.75 84.000 78.625 79.750
  Ctrl 72.75 72.375 71.500 71.000
\end{verbatim}

\end{example}

\(\blacksquare\)

\index[R]{\texttt{lapply()}}

The function \texttt{aggregate()} can be considered a more sophisticated extension of \texttt{tapply()}. It allows objects under consideration to be expressed as functions of explanatory factors, and contains additional arguments for data specification and time series analyses.
\index[R]{\texttt{aggregate()}}

\begin{example}
\protect\hypertarget{exm:exm-43b}{}\label{exm:exm-43b}\(\text{}\)\\
Here we use \texttt{aggregate()} to get identical (but reformatted) results to the prior example.

\vspace{0.1in}

\begin{Shaded}
\begin{Highlighting}[]
\FunctionTok{aggregate}\NormalTok{(rate }\SpecialCharTok{\textasciitilde{}}\NormalTok{ drug }\SpecialCharTok{+}\NormalTok{ time, mean, }\AttributeTok{data =}\NormalTok{ heart)}
\end{Highlighting}
\end{Shaded}

\begin{verbatim}
   drug time   rate
1  AX23   t1 70.500
2  BWW9   t1 81.750
3  Ctrl   t1 72.750
4  AX23   t2 80.500
5  BWW9   t2 84.000
6  Ctrl   t2 72.375
7  AX23   t3 81.000
8  BWW9   t3 78.625
9  Ctrl   t3 71.500
10 AX23   t4 73.125
11 BWW9   t4 79.750
12 Ctrl   t4 71.000
\end{verbatim}

Importantly, the first argument, \texttt{rate\ \textasciitilde{}\ drug\ +\ time} is in the form of a formula:

\vspace{0.1in}

\begin{Shaded}
\begin{Highlighting}[]
\NormalTok{f.rate }\OtherTok{\textless{}{-}} \FunctionTok{with}\NormalTok{(heart, rate }\SpecialCharTok{\textasciitilde{}}\NormalTok{ drug }\SpecialCharTok{+}\NormalTok{ time)}
\FunctionTok{class}\NormalTok{(f.rate)}
\end{Highlighting}
\end{Shaded}

\begin{verbatim}
[1] "formula"
\end{verbatim}

Here the tilde operator, \texttt{\textasciitilde{}}, allows expression of the formulaic framework: \texttt{y\ \textasciitilde{}\ model}, where \texttt{y} is a response variable and \texttt{model} specifies a system of (generally) one or more predictor variables.
\end{example}

\(\blacksquare\)
\index[terms]{R!classes!formula}\index[R]{\texttt{$\sim$}}

\subsection{\texorpdfstring{\texttt{outer()}}{outer()}}\label{outer}

Another important function for matrix operations is \texttt{outer()}\index[R]{\texttt{outer()}}. This algorithm allows creation of an array that contains all possible combinations of two atomic vectors or arrays with respect to a user-specified function. The \texttt{outer()} function has three required arguments.

\begin{itemize}
\tightlist
\item
  The first two arguments, \texttt{X} and \texttt{Y}, define arrays or atomic vectors. \texttt{X} and \texttt{Y} can be identical if one wishes to examine pairwise operations of the array elements (see example below).
\item
  The third argument, \texttt{FUN}, specifies a function to be used in operations.
\end{itemize}

\begin{example}
\protect\hypertarget{exm:exm-44}{}\label{exm:exm-44}

\(\text{}\)\\
Suppose I wish to find the means of all possible pairs of observations from an atomic vector. I could use the following commands:

\vspace{0.1in}

\begin{Shaded}
\begin{Highlighting}[]
\NormalTok{x }\OtherTok{\textless{}{-}} \FunctionTok{c}\NormalTok{(}\DecValTok{1}\NormalTok{, }\DecValTok{2}\NormalTok{, }\DecValTok{3}\NormalTok{, }\DecValTok{5}\NormalTok{, }\DecValTok{4}\NormalTok{)}
\FunctionTok{outer}\NormalTok{(x, x, }\StringTok{"+"}\NormalTok{)}\SpecialCharTok{/}\DecValTok{2}
\end{Highlighting}
\end{Shaded}

\begin{verbatim}
     [,1] [,2] [,3] [,4] [,5]
[1,]  1.0  1.5  2.0  3.0  2.5
[2,]  1.5  2.0  2.5  3.5  3.0
[3,]  2.0  2.5  3.0  4.0  3.5
[4,]  3.0  3.5  4.0  5.0  4.5
[5,]  2.5  3.0  3.5  4.5  4.0
\end{verbatim}

The argument \texttt{FUN\ =\ "+"} indicates that we wish to add elements to each other. We divide these sums by two to obtain means. Note that the diagonal of the output matrix contains the original elements of \texttt{x}, because the mean of a number and itself is the original number. The upper and lower triangles are identical because the mean of elements \emph{a} and \emph{b} will be the same as the mean of the elements \emph{b} and \emph{a}. Note that the result \texttt{outer(x,\ x,\ "*")} can also be obtained using \texttt{x\ \%o\%\ x} because \texttt{\%o\%} is the matrix algebra outer product operator in \textbf{R}.

\vspace{0.1in}

\begin{Shaded}
\begin{Highlighting}[]
\FunctionTok{outer}\NormalTok{(x, x, }\StringTok{"*"}\NormalTok{)}
\end{Highlighting}
\end{Shaded}

\begin{verbatim}
     [,1] [,2] [,3] [,4] [,5]
[1,]    1    2    3    5    4
[2,]    2    4    6   10    8
[3,]    3    6    9   15   12
[4,]    5   10   15   25   20
[5,]    4    8   12   20   16
\end{verbatim}

\begin{Shaded}
\begin{Highlighting}[]
\NormalTok{x }\SpecialCharTok{\%o\%}\NormalTok{ x}
\end{Highlighting}
\end{Shaded}

\begin{verbatim}
     [,1] [,2] [,3] [,4] [,5]
[1,]    1    2    3    5    4
[2,]    2    4    6   10    8
[3,]    3    6    9   15   12
[4,]    5   10   15   25   20
[5,]    4    8   12   20   16
\end{verbatim}

\end{example}

\(\blacksquare\)

\index[R]{*ino@\texttt{\%o\%}}
\index[terms]{Matrix algebra}

\subsection{\texorpdfstring{\texttt{stack()}, \texttt{unstack()} and \texttt{reshape()}}{stack(), unstack() and reshape()}}\label{stack-unstack-and-reshape}

When manipulating lists and dataframes it is often useful to move between so-called ``long'' and ``wide'' data table formats. These operations can be handled with the functions \texttt{stack()} and \texttt{unstack()}. Specifically, \texttt{stack()} concatenates multiple vectors into a single vector along with a factor indicating where each observation originated, whereas \texttt{unstack()} reverses this process.
\index[R]{\texttt{stack()}}\index[R]{\texttt{unstack()}}
\index[R]{\texttt{reshape()}}

\begin{example}
\protect\hypertarget{exm:exm-45}{}\label{exm:exm-45}

\(\text{}\)\\
Consider the 4 x 4 dataframe below.

\vspace{0.1in}

\begin{Shaded}
\begin{Highlighting}[]
\NormalTok{dataf }\OtherTok{\textless{}{-}} \FunctionTok{data.frame}\NormalTok{(}\FunctionTok{matrix}\NormalTok{(}\AttributeTok{nrow =} \DecValTok{4}\NormalTok{, }\AttributeTok{data =} \FunctionTok{rnorm}\NormalTok{(}\DecValTok{16}\NormalTok{)))}
\FunctionTok{names}\NormalTok{(dataf) }\OtherTok{\textless{}{-}} \FunctionTok{c}\NormalTok{(}\StringTok{"col1"}\NormalTok{, }\StringTok{"col2"}\NormalTok{, }\StringTok{"col3"}\NormalTok{, }\StringTok{"col4"}\NormalTok{)}
\NormalTok{dataf}
\end{Highlighting}
\end{Shaded}

\begin{verbatim}
      col1     col2     col3      col4
1  0.69691  1.33109 -0.32276 -0.629645
2 -0.58314  0.41149 -1.20581  0.103334
3  0.46143 -0.55242  0.14266 -0.786863
4 -1.83067  0.54949 -0.19166 -0.066147
\end{verbatim}

Here I stack \texttt{dataf} into a long table format.

\vspace{0.1in}

\begin{Shaded}
\begin{Highlighting}[]
\NormalTok{sdataf }\OtherTok{\textless{}{-}} \FunctionTok{stack}\NormalTok{(dataf)}
\NormalTok{sdataf}
\end{Highlighting}
\end{Shaded}

\begin{verbatim}
      values  ind
1   0.696907 col1
2  -0.583140 col1
3   0.461429 col1
4  -1.830672 col1
5   1.331090 col2
6   0.411492 col2
7  -0.552423 col2
8   0.549494 col2
9  -0.322765 col3
10 -1.205808 col3
11  0.142656 col3
12 -0.191663 col3
13 -0.629645 col4
14  0.103334 col4
15 -0.786863 col4
16 -0.066147 col4
\end{verbatim}

Here I unstack \texttt{sdataf}.

\vspace{0.1in}

\begin{Shaded}
\begin{Highlighting}[]
\FunctionTok{unstack}\NormalTok{(sdataf)}
\end{Highlighting}
\end{Shaded}

\begin{verbatim}
      col1     col2     col3      col4
1  0.69691  1.33109 -0.32276 -0.629645
2 -0.58314  0.41149 -1.20581  0.103334
3  0.46143 -0.55242  0.14266 -0.786863
4 -1.83067  0.54949 -0.19166 -0.066147
\end{verbatim}

The function \texttt{reshape()} can handle both stacking and unstacking operations. Here I stack \texttt{dataf}. The arguments \texttt{timevar}, \texttt{idvar}, and \texttt{v.names} are used to provide recognizable identifiers for the columns in the wide table format, observations within those columns, and responses for those combinations.

\vspace{0.1in}

\begin{Shaded}
\begin{Highlighting}[]
\FunctionTok{reshape}\NormalTok{(dataf, }\AttributeTok{direction =} \StringTok{"long"}\NormalTok{,}
        \AttributeTok{varying =} \FunctionTok{list}\NormalTok{(}\FunctionTok{names}\NormalTok{(dataf)),}
        \AttributeTok{timevar =} \StringTok{"Column"}\NormalTok{,}
        \AttributeTok{idvar =} \StringTok{"Column obs."}\NormalTok{,}
        \AttributeTok{v.names =} \StringTok{"Response"}\NormalTok{)}
\end{Highlighting}
\end{Shaded}

\begin{verbatim}
    Column  Response Column obs.
1.1      1  0.696907           1
2.1      1 -0.583140           2
3.1      1  0.461429           3
4.1      1 -1.830672           4
1.2      2  1.331090           1
2.2      2  0.411492           2
3.2      2 -0.552423           3
4.2      2  0.549494           4
1.3      3 -0.322765           1
2.3      3 -1.205808           2
3.3      3  0.142656           3
4.3      3 -0.191663           4
1.4      4 -0.629645           1
2.4      4  0.103334           2
3.4      4 -0.786863           3
4.4      4 -0.066147           4
\end{verbatim}

\end{example}

\(\blacksquare\)

\section{Other Simple Data Management Functions}\label{other-simple-data-management-functions}

\subsection{\texorpdfstring{\texttt{replace()}}{replace()}}\label{replace}

One use the function \texttt{replace()} to replace elements in an atomic vector based, potentially, on Boolean logic. The function requires three arguments.

\begin{itemize}
\tightlist
\item
  The first argument, \texttt{x}, specifies the vector to be analyzed.
\item
  The second argument, \texttt{list}, connotes which elements need to be replaced. A logical argument can be used here as a replacement index.
\item
  The third argument, \texttt{values}, defines the replacement value(s).
  \index[R]{\texttt{replace()}}
\end{itemize}

\begin{example}
\protect\hypertarget{exm:exm-46}{}\label{exm:exm-46}

\(\text{}\)\\
For instance:

\vspace{0.1in}

\begin{Shaded}
\begin{Highlighting}[]
\NormalTok{Age }\OtherTok{\textless{}{-}} \FunctionTok{c}\NormalTok{(}\DecValTok{21}\NormalTok{, }\DecValTok{19}\NormalTok{, }\DecValTok{25}\NormalTok{, }\DecValTok{26}\NormalTok{, }\DecValTok{18}\NormalTok{, }\DecValTok{19}\NormalTok{)}
\FunctionTok{replace}\NormalTok{(Age, Age }\SpecialCharTok{\textless{}} \DecValTok{25}\NormalTok{, }\StringTok{"R is Cool"}\NormalTok{)}
\end{Highlighting}
\end{Shaded}

\begin{verbatim}
[1] "R is Cool" "R is Cool" "25"        "26"        "R is Cool" "R is Cool"
\end{verbatim}

Of course, one can also use square brackets for this operation.

\vspace{0.1in}

\begin{Shaded}
\begin{Highlighting}[]
\NormalTok{Age[Age }\SpecialCharTok{\textless{}} \DecValTok{25}\NormalTok{] }\OtherTok{\textless{}{-}} \StringTok{"R is Cool"}
\NormalTok{Age}
\end{Highlighting}
\end{Shaded}

\begin{verbatim}
[1] "R is Cool" "R is Cool" "25"        "26"        "R is Cool" "R is Cool"
\end{verbatim}

\end{example}

\(\blacksquare\)

\subsection{\texorpdfstring{\texttt{which()}}{which()}}\label{which}

The function which can be used with logical commands to obtain address indices for data storage object.
\index[R]{\texttt{which()}}

\begin{example}
\protect\hypertarget{exm:exm-47}{}\label{exm:exm-47}

\hfill\break
For instance:

\vspace{0.1in}

\begin{Shaded}
\begin{Highlighting}[]
\NormalTok{Age }\OtherTok{\textless{}{-}} \FunctionTok{c}\NormalTok{(}\DecValTok{21}\NormalTok{, }\DecValTok{19}\NormalTok{, }\DecValTok{25}\NormalTok{, }\DecValTok{26}\NormalTok{, }\DecValTok{18}\NormalTok{, }\DecValTok{19}\NormalTok{)}
\NormalTok{w }\OtherTok{\textless{}{-}} \FunctionTok{which}\NormalTok{(Age }\SpecialCharTok{\textless{}=} \DecValTok{21}\NormalTok{)}
\NormalTok{w}
\end{Highlighting}
\end{Shaded}

\begin{verbatim}
[1] 1 2 5 6
\end{verbatim}

Elements one, two, and five meet this criterion. We can now subset based on the index \texttt{w}.

\vspace{0.1in}

\begin{Shaded}
\begin{Highlighting}[]
\NormalTok{Age[w]}
\end{Highlighting}
\end{Shaded}

\begin{verbatim}
[1] 21 19 18 19
\end{verbatim}

To find which element in \texttt{Age} is closest to 24 I could do something like:

\vspace{0.1in}

\begin{Shaded}
\begin{Highlighting}[]
\FunctionTok{which}\NormalTok{(}\FunctionTok{abs}\NormalTok{(Age }\SpecialCharTok{{-}} \DecValTok{24}\NormalTok{) }\SpecialCharTok{==} \FunctionTok{min}\NormalTok{(}\FunctionTok{abs}\NormalTok{(Age }\SpecialCharTok{{-}} \DecValTok{24}\NormalTok{)))}
\end{Highlighting}
\end{Shaded}

\begin{verbatim}
[1] 3
\end{verbatim}

\end{example}

\(\blacksquare\)

\subsection{\texorpdfstring{\texttt{sort()}}{sort()}}\label{sort}

By default, The function \texttt{sort()} sorts data from an atomic vector into an alphanumeric ascending order.\index[R]{\texttt{sort()}}

\vspace{0.1in}

\begin{Shaded}
\begin{Highlighting}[]
\FunctionTok{sort}\NormalTok{(Age)}
\end{Highlighting}
\end{Shaded}

\begin{verbatim}
[1] 18 19 19 21 25 26
\end{verbatim}

Data can be sorted in a descending order by specifying \texttt{decreasing\ =\ TRUE}.

\vspace{0.1in}

\begin{Shaded}
\begin{Highlighting}[]
\FunctionTok{sort}\NormalTok{(Age, }\AttributeTok{decreasing =}\NormalTok{ T)}
\end{Highlighting}
\end{Shaded}

\begin{verbatim}
[1] 26 25 21 19 19 18
\end{verbatim}

\subsection{\texorpdfstring{\texttt{rank()}}{rank()}}\label{rank}

The function rank gives the ascending alphanumeric rank of elements in a vector. Ties are given the average of their ranks. This operation is important to rank-based permutation analyses \citep[][Ch 6]{aho2014}.\index[R]{\texttt{rank()}}

\vspace{0.1in}

\begin{Shaded}
\begin{Highlighting}[]
\FunctionTok{rank}\NormalTok{(Age)}
\end{Highlighting}
\end{Shaded}

\begin{verbatim}
[1] 4.0 2.5 5.0 6.0 1.0 2.5
\end{verbatim}

The second and last observations were the second smallest in \texttt{Age}. Thus, their average rank is 2.5.

\subsection{\texorpdfstring{\texttt{order()}}{order()}}\label{order}

The function \texttt{order()} is similar to \texttt{which()} in that it provides element indices that accord with an alphanumeric ordering. This allows one to sort a vector, matrix or dataframe into an ascending or descending order, based on one or several ordered vectors.
\index[R]{\texttt{order()}}

\begin{example}
\protect\hypertarget{exm:exm-48}{}\label{exm:exm-48}

\(\text{}\)\\
Consider the dataframe below which lists plant percent cover data for four plant species at three sites. In accordance with the \texttt{field.data} example from Ch 3, plant species are identified with four letter codes, corresponding to the first two letters of the Linnaean genus and species names.

\vspace{0.1in}

\begin{Shaded}
\begin{Highlighting}[]
\NormalTok{field.data }\OtherTok{\textless{}{-}} \FunctionTok{data.frame}\NormalTok{(}\AttributeTok{code =} \FunctionTok{c}\NormalTok{(}\StringTok{"ACMI"}\NormalTok{, }\StringTok{"ELSC"}\NormalTok{, }\StringTok{"CAEL"}\NormalTok{, }\StringTok{"TACE"}\NormalTok{),}
                         \AttributeTok{site1 =} \FunctionTok{c}\NormalTok{(}\DecValTok{12}\NormalTok{, }\DecValTok{13}\NormalTok{, }\DecValTok{14}\NormalTok{, }\DecValTok{11}\NormalTok{),}
                         \AttributeTok{site2 =} \FunctionTok{c}\NormalTok{(}\DecValTok{0}\NormalTok{, }\DecValTok{20}\NormalTok{, }\DecValTok{4}\NormalTok{, }\DecValTok{5}\NormalTok{),}
                         \AttributeTok{site3 =} \FunctionTok{c}\NormalTok{(}\DecValTok{20}\NormalTok{, }\DecValTok{10}\NormalTok{, }\DecValTok{30}\NormalTok{, }\DecValTok{0}\NormalTok{))}
\NormalTok{field.data}
\end{Highlighting}
\end{Shaded}

\begin{verbatim}
  code site1 site2 site3
1 ACMI    12     0    20
2 ELSC    13    20    10
3 CAEL    14     4    30
4 TACE    11     5     0
\end{verbatim}

Assume that we wish to sort the data with respect to an alphanumeric ordering of species codes. Here we obtain the ordering of the codes

\vspace{0.1in}

\begin{Shaded}
\begin{Highlighting}[]
\NormalTok{o }\OtherTok{\textless{}{-}} \FunctionTok{order}\NormalTok{(field.data}\SpecialCharTok{$}\NormalTok{code)}
\NormalTok{o}
\end{Highlighting}
\end{Shaded}

\begin{verbatim}
[1] 1 3 2 4
\end{verbatim}

Now we can sort the rows of \texttt{field.data} based on this ordering.

\vspace{0.1in}

\begin{Shaded}
\begin{Highlighting}[]
\NormalTok{field.data[o,]}
\end{Highlighting}
\end{Shaded}

\begin{verbatim}
  code site1 site2 site3
1 ACMI    12     0    20
3 CAEL    14     4    30
2 ELSC    13    20    10
4 TACE    11     5     0
\end{verbatim}

\end{example}

\(\blacksquare\)

\subsection{\texorpdfstring{\texttt{unique()}}{unique()}}\label{unique}

To identify unique values in dataset we can use the function \texttt{unique()}.
\index[R]{\texttt{unique()}}

\begin{example}
\protect\hypertarget{exm:exm-49}{}\label{exm:exm-49}

\hfill\break
Below is an atomic vector listing species from a bird survey on islands in southeast Alaska. Species ciphers follow the same coding method used in Example \ref{exm:exm-48}. Note that there are a large number of repeats.

\vspace{0.1in}

\begin{Shaded}
\begin{Highlighting}[]
\NormalTok{AK.bird }\OtherTok{\textless{}{-}} \FunctionTok{c}\NormalTok{(}\StringTok{"GLGU"}\NormalTok{, }\StringTok{"MEGU"}\NormalTok{, }\StringTok{"DOCO"}\NormalTok{, }\StringTok{"PAJA"}\NormalTok{, }\StringTok{"COLO"}\NormalTok{, }\StringTok{"BUFF"}\NormalTok{, }\StringTok{"COGO"}\NormalTok{, }
             \StringTok{"WHSC"}\NormalTok{, }\StringTok{"TUSW"}\NormalTok{, }\StringTok{"GRSC"}\NormalTok{, }\StringTok{"GRTE"}\NormalTok{, }\StringTok{"REME"}\NormalTok{, }\StringTok{"BLOY"}\NormalTok{, }\StringTok{"REPH"}\NormalTok{, }
             \StringTok{"SEPL"}\NormalTok{, }\StringTok{"LESA"}\NormalTok{, }\StringTok{"ROSA"}\NormalTok{, }\StringTok{"WESA"}\NormalTok{, }\StringTok{"WISN"}\NormalTok{, }\StringTok{"BAEA"}\NormalTok{, }\StringTok{"SHOW"}\NormalTok{, }
             \StringTok{"GLGU"}\NormalTok{, }\StringTok{"MEGU"}\NormalTok{, }\StringTok{"PAJA"}\NormalTok{, }\StringTok{"DOCO"}\NormalTok{, }\StringTok{"GRSC"}\NormalTok{, }\StringTok{"GRTE"}\NormalTok{, }\StringTok{"BUFF"}\NormalTok{, }
             \StringTok{"MADU"}\NormalTok{, }\StringTok{"TUSW"}\NormalTok{, }\StringTok{"REME"}\NormalTok{, }\StringTok{"SEPL"}\NormalTok{, }\StringTok{"REPH"}\NormalTok{, }\StringTok{"ROSA"}\NormalTok{, }\StringTok{"LESA"}\NormalTok{, }
             \StringTok{"COSN"}\NormalTok{, }\StringTok{"BAEA"}\NormalTok{, }\StringTok{"ROHA"}\NormalTok{)}

\FunctionTok{length}\NormalTok{(AK.bird)}
\end{Highlighting}
\end{Shaded}

\begin{verbatim}
[1] 38
\end{verbatim}

Applying \texttt{unique()} we obtain a listing of the 24 unique bird species.

\vspace{0.1in}

\begin{Shaded}
\begin{Highlighting}[]
\FunctionTok{unique}\NormalTok{(AK.bird)}
\end{Highlighting}
\end{Shaded}

\begin{verbatim}
 [1] "GLGU" "MEGU" "DOCO" "PAJA" "COLO" "BUFF" "COGO" "WHSC" "TUSW" "GRSC"
[11] "GRTE" "REME" "BLOY" "REPH" "SEPL" "LESA" "ROSA" "WESA" "WISN" "BAEA"
[21] "SHOW" "MADU" "COSN" "ROHA"
\end{verbatim}

\end{example}

\(\blacksquare\)

\subsection{\texorpdfstring{\texttt{match()}}{match()}}\label{match}

Given two vectors, the function \texttt{match()} indexes where objects in the second vector appear in the elements of the first vector. For instance:
\index[R]{\texttt{match()}}

\vspace{0.1in}

\begin{Shaded}
\begin{Highlighting}[]
\NormalTok{x }\OtherTok{\textless{}{-}} \FunctionTok{c}\NormalTok{(}\DecValTok{6}\NormalTok{, }\DecValTok{5}\NormalTok{, }\DecValTok{4}\NormalTok{, }\DecValTok{3}\NormalTok{, }\DecValTok{2}\NormalTok{, }\DecValTok{7}\NormalTok{)}
\NormalTok{y }\OtherTok{\textless{}{-}} \FunctionTok{c}\NormalTok{(}\DecValTok{2}\NormalTok{, }\DecValTok{1}\NormalTok{, }\DecValTok{4}\NormalTok{, }\DecValTok{3}\NormalTok{, }\DecValTok{5}\NormalTok{, }\DecValTok{6}\NormalTok{)}
\NormalTok{m }\OtherTok{\textless{}{-}} \FunctionTok{match}\NormalTok{(y, x)}
\NormalTok{m}
\end{Highlighting}
\end{Shaded}

\begin{verbatim}
[1]  5 NA  3  4  2  1
\end{verbatim}

The number 2 (the 1st element in \texttt{y}) is the 5th element of \texttt{x}, thus the number 5 is put 1st in the vector \texttt{m} created by match. The number 1 (the 2nd element of \texttt{y}) does not occur in \texttt{x} (it is \texttt{NA}). The number 4 is the 3rd element
of \texttt{y} and \texttt{x}. Thus, the number 3 is given as the third element of \texttt{m}, and so on.

\begin{example}
\protect\hypertarget{exm:exm-410}{}\label{exm:exm-410}

\(\text{}\)\\
The usefulness of \texttt{match()} may seem unclear at first, but consider a scenario in which I want to convert species code identifiers in field data into actual species names. The following dataframe is a species list that matches four letter species codes to scientific names. Note that the list contains more species than than the \texttt{field.data} dataset used in Example \ref{exm:exm-48}.

\vspace{0.1in}

\begin{Shaded}
\begin{Highlighting}[]
\NormalTok{species.list }\OtherTok{\textless{}{-}} \FunctionTok{data.frame}\NormalTok{(}\AttributeTok{code =} \FunctionTok{c}\NormalTok{(}\StringTok{"ACMI"}\NormalTok{, }\StringTok{"ASFO"}\NormalTok{, }\StringTok{"ELSC"}\NormalTok{, }\StringTok{"ERRY"}\NormalTok{, }\StringTok{"CAEL"}\NormalTok{,}
\StringTok{"CAPA"}\NormalTok{, }\StringTok{"TACE"}\NormalTok{), }\AttributeTok{names =} \FunctionTok{c}\NormalTok{(}\StringTok{"Achillea millefolium"}\NormalTok{, }\StringTok{"Aster foliaceus"}\NormalTok{,}
                           \StringTok{"Elymus scribneri"}\NormalTok{, }\StringTok{"Erigeron rydbergii"}\NormalTok{,}
                           \StringTok{"Carex elynoides"}\NormalTok{, }\StringTok{"Carex paysonis"}\NormalTok{,}
                           \StringTok{"Taraxacum ceratophorum"}\NormalTok{))}

\NormalTok{species.list}
\end{Highlighting}
\end{Shaded}

\begin{verbatim}
  code                  names
1 ACMI   Achillea millefolium
2 ASFO        Aster foliaceus
3 ELSC       Elymus scribneri
4 ERRY     Erigeron rydbergii
5 CAEL        Carex elynoides
6 CAPA         Carex paysonis
7 TACE Taraxacum ceratophorum
\end{verbatim}

Here I add a column in the \texttt{field.data} of the actual species names using \texttt{match()}.

\vspace{0.1in}

\begin{Shaded}
\begin{Highlighting}[]
\NormalTok{m }\OtherTok{\textless{}{-}} \FunctionTok{match}\NormalTok{(field.data}\SpecialCharTok{$}\NormalTok{code, species.list}\SpecialCharTok{$}\NormalTok{code)}
\NormalTok{field.data.new }\OtherTok{\textless{}{-}}\NormalTok{ field.data }\CommentTok{\# make a copy of field data}
\NormalTok{field.data.new}\SpecialCharTok{$}\NormalTok{species.name }\OtherTok{\textless{}{-}}\NormalTok{ species.list}\SpecialCharTok{$}\NormalTok{names[m]}
\NormalTok{field.data.new}
\end{Highlighting}
\end{Shaded}

\begin{verbatim}
  code site1 site2 site3           species.name
1 ACMI    12     0    20   Achillea millefolium
2 ELSC    13    20    10       Elymus scribneri
3 CAEL    14     4    30        Carex elynoides
4 TACE    11     5     0 Taraxacum ceratophorum
\end{verbatim}

\end{example}

\(\blacksquare\)

\subsection{\texorpdfstring{\texttt{which()} and \texttt{\%in\%}}{which() and \%in\%}}\label{whichin}

We can use the operator \texttt{\%in\%} in conjunction with the function \texttt{which()} to achieve the same results as \texttt{match()}. \index[R]{*in@\texttt{\%in\%}}

\vspace{0.1in}

\begin{Shaded}
\begin{Highlighting}[]
\NormalTok{m }\OtherTok{\textless{}{-}} \FunctionTok{which}\NormalTok{(species.list}\SpecialCharTok{$}\NormalTok{code }\SpecialCharTok{\%in\%}\NormalTok{ field.data}\SpecialCharTok{$}\NormalTok{code)}
\NormalTok{field.data.new}\SpecialCharTok{$}\NormalTok{species.name }\OtherTok{\textless{}{-}}\NormalTok{ species.list}\SpecialCharTok{$}\NormalTok{names[m]}
\NormalTok{field.data.new}
\end{Highlighting}
\end{Shaded}

\begin{verbatim}
  code site1 site2 site3           species.name
1 ACMI    12     0    20   Achillea millefolium
2 ELSC    13    20    10       Elymus scribneri
3 CAEL    14     4    30        Carex elynoides
4 TACE    11     5     0 Taraxacum ceratophorum
\end{verbatim}

Note that the arrangement of arguments are reversed in \texttt{match()} and \texttt{which()}. In the former we have: \texttt{match(field.data\$code, species.list\$code)}. In the latter we have: \texttt{which(species.list\$code\ \%in\%\ field.data\$code)}.

\section{Matching, Querying and Substituting in Strings}\label{regex}

\textbf{R} contains a number of useful methods for handling \emph{character string}\footnote{In computer programming, a string is generally a (non-numeric) sequence of characters \citep{wikichar}. \textbf{R} frequently uses \emph{character vectors}, i.e.,
  \texttt{vec\ \textless{}-\ c("a",\ "b",\ "c")}. Each entry in \texttt{vec} would be conventionally considered to be a character string.} data. Strings will have class and base type \texttt{character}.
\index[terms]{R!base type!character}
\index[terms]{R!classes!character}
\index[terms]{R!character vector}
\index[terms]{R!character string}

\subsection{\texorpdfstring{\texttt{strtrim()} and \texttt{substr()}}{strtrim() and substr()}}\label{strtrim-and-substr}

The functions \texttt{strtrim()} and \texttt{substr()} are useful for extracting subsets from strings or string vectors.
\index[R]{\texttt{strsplit()}}
\index[R]{\texttt{substr()}}

\begin{example}
\protect\hypertarget{exm:exm-411}{}\label{exm:exm-411}

\(\text{}\)\\
For the taxonomic codes in the character vector below, the first capital letter indicates whether a species is a flowering plant (anthophyte) or moss (bryophyte) while the last four letters give species codes (see Example \ref{exm:exm-48}).

\vspace{0.1in}

\begin{Shaded}
\begin{Highlighting}[]
\NormalTok{plant }\OtherTok{\textless{}{-}} \FunctionTok{c}\NormalTok{(}\StringTok{"A\_CAAT"}\NormalTok{, }\StringTok{"B\_CASP"}\NormalTok{, }\StringTok{"A\_SARI"}\NormalTok{)}
\end{Highlighting}
\end{Shaded}

Assume that I want to distinguish anthophytes from bryophytes by extracting the first letter. This can be done by specifying \texttt{1} in the second \texttt{strtrim} argument, \texttt{width}.

\vspace{0.1in}

\begin{Shaded}
\begin{Highlighting}[]
\NormalTok{phylum }\OtherTok{\textless{}{-}} \FunctionTok{strtrim}\NormalTok{(plant, }\DecValTok{1}\NormalTok{)}
\NormalTok{phylum}
\end{Highlighting}
\end{Shaded}

\begin{verbatim}
[1] "A" "B" "A"
\end{verbatim}

\begin{Shaded}
\begin{Highlighting}[]
\NormalTok{plant[phylum }\SpecialCharTok{==} \StringTok{"A"}\NormalTok{]}
\end{Highlighting}
\end{Shaded}

\begin{verbatim}
[1] "A_CAAT" "A_SARI"
\end{verbatim}

The function \texttt{substr()} is useful for imposing the start and end of strings to be subset. Here I extract string components 3-4 (the first two letters of the genus name).
\vspace{0.1in}

\begin{Shaded}
\begin{Highlighting}[]
\FunctionTok{substr}\NormalTok{(plant, }\DecValTok{3}\NormalTok{, }\DecValTok{4}\NormalTok{)}
\end{Highlighting}
\end{Shaded}

\begin{verbatim}
[1] "CA" "CA" "SA"
\end{verbatim}

\end{example}

\(\blacksquare\)

\subsection{\texorpdfstring{\texttt{strsplit()}}{strsplit()}}\label{strsplit}

The function \texttt{strsplit()} splits a character string into substrings based on user defined criteria. It contains two important arguments.
\index[R]{\texttt{strsplit}}

\begin{itemize}
\tightlist
\item
  The first argument, \texttt{x}, specifies the character string to be analyzed.
\item
  The second argument, \texttt{split}, is a character criterion that is used for splitting.
\end{itemize}

\begin{example}
\protect\hypertarget{exm:exm-412}{}\label{exm:exm-412}\(\text{}\)\\
Below I split the character string \texttt{ACMI} in two, based on the space between the words \texttt{Achillea} and \texttt{millefolium}.

\vspace{0.1in}

\begin{Shaded}
\begin{Highlighting}[]
\NormalTok{ACMI }\OtherTok{\textless{}{-}} \StringTok{"Achillea millefolium"}
\FunctionTok{strsplit}\NormalTok{(ACMI, }\StringTok{" "}\NormalTok{)}
\end{Highlighting}
\end{Shaded}

\begin{verbatim}
[[1]]
[1] "Achillea"    "millefolium"
\end{verbatim}

Note that the result is a list. To get back to a vector (now with two components), I can use the function \texttt{unlist()}.

\vspace{0.1in}

\begin{Shaded}
\begin{Highlighting}[]
\FunctionTok{unlist}\NormalTok{(}\FunctionTok{strsplit}\NormalTok{(ACMI, }\StringTok{" "}\NormalTok{))}
\end{Highlighting}
\end{Shaded}

\begin{verbatim}
[1] "Achillea"    "millefolium"
\end{verbatim}

Here I split based on the letter \texttt{"l"}.

\vspace{0.1in}

\begin{Shaded}
\begin{Highlighting}[]
\FunctionTok{strsplit}\NormalTok{(ACMI, }\StringTok{"l"}\NormalTok{)}
\end{Highlighting}
\end{Shaded}

\begin{verbatim}
[[1]]
[1] "Achi"  ""      "ea mi" ""      "efo"   "ium"  
\end{verbatim}

Interestingly, letting the \texttt{split} criterion equal \texttt{NULL} results in spaces being placed between every character in a string.

\vspace{0.1in}

\begin{Shaded}
\begin{Highlighting}[]
\FunctionTok{strsplit}\NormalTok{(ACMI, }\ConstantTok{NULL}\NormalTok{)}
\end{Highlighting}
\end{Shaded}

\begin{verbatim}
[[1]]
 [1] "A" "c" "h" "i" "l" "l" "e" "a" " " "m" "i" "l" "l" "e" "f" "o" "l"
[18] "i" "u" "m"
\end{verbatim}

We can use this outcome to reverse the order of characters in a string.

\vspace{0.1in}

\begin{Shaded}
\begin{Highlighting}[]
\FunctionTok{sapply}\NormalTok{(}\FunctionTok{lapply}\NormalTok{(}\FunctionTok{strsplit}\NormalTok{(ACMI, }\ConstantTok{NULL}\NormalTok{), rev), paste, }\AttributeTok{collapse =} \StringTok{""}\NormalTok{)}
\end{Highlighting}
\end{Shaded}

\begin{verbatim}
[1] "muilofellim aellihcA"
\end{verbatim}

The function \texttt{rev()} provides a reversed version of its first argument, in this case a result from \texttt{strsplit()}.
The function \texttt{paste()} can be use to paste together character strings.
\end{example}

\(\blacksquare\)

\index[R]{\texttt{unlist()}}
\index[R]{\texttt{rev()}}
\index[R]{\texttt{paste()}}

Criteria for querying strings can include multiple characters in a particular order, and a particular case:

\vspace{0.1in}

\begin{Shaded}
\begin{Highlighting}[]
\NormalTok{x }\OtherTok{\textless{}{-}} \StringTok{"R is free software and comes with ABSOLUTELY NO WARRANTY"}
\FunctionTok{strsplit}\NormalTok{(x, }\StringTok{"so"}\NormalTok{)}
\end{Highlighting}
\end{Shaded}

\begin{verbatim}
[[1]]
[1] "R is free "                                  
[2] "ftware and comes with ABSOLUTELY NO WARRANTY"
\end{verbatim}

Note that the \texttt{"SO"} in \texttt{"ABSOLUTELY"} is ignored because it is upper case.

\subsection{\texorpdfstring{\texttt{grep()} and \texttt{grepl()}}{grep() and grepl()}}\label{grep-and-grepl}

The functions \texttt{grep()} and \texttt{grepl()} can be used to identify which elements in a \emph{character vector}\index[terms]{R!character vector} have a specified pattern. They have the same first two arguments.

\begin{itemize}
\tightlist
\item
  The first argument, \texttt{pattern} specifies a patterns to be matched. This can be a character string, or object coercible to a character string, or a regular expression (see below).
\item
  The second argument, \texttt{x}, is a character vector where matches are sought.
\end{itemize}

\begin{example}
\protect\hypertarget{exm:exm-413}{}\label{exm:exm-413}

\(\text{}\)\\
The function \texttt{grep()} returns indices identifying which entries in a vector contain a queried pattern. In the character vector below, we see that entries five and six have the same genus, \emph{Carex}.

\vspace{0.1in}

\begin{Shaded}
\begin{Highlighting}[]
\NormalTok{names }\OtherTok{=} \FunctionTok{c}\NormalTok{(}\StringTok{"Achillea millefolium"}\NormalTok{, }\StringTok{"Aster foliaceus"}\NormalTok{,}
                           \StringTok{"Elymus scribneri"}\NormalTok{, }\StringTok{"Erigeron rydbergii"}\NormalTok{,}
                           \StringTok{"Carex elynoides"}\NormalTok{, }\StringTok{"Carex paysonis"}\NormalTok{,}
                           \StringTok{"Taraxacum ceratophorum"}\NormalTok{)}

\FunctionTok{grep}\NormalTok{(}\StringTok{"Carex"}\NormalTok{, names)}
\end{Highlighting}
\end{Shaded}

\begin{verbatim}
[1] 5 6
\end{verbatim}

The function \texttt{grepl()} does the same thing with Boolean outcomes.

\vspace{0.1in}

\begin{Shaded}
\begin{Highlighting}[]
\FunctionTok{grepl}\NormalTok{(}\StringTok{"Carex"}\NormalTok{, names)}
\end{Highlighting}
\end{Shaded}

\begin{verbatim}
[1] FALSE FALSE FALSE FALSE  TRUE  TRUE FALSE
\end{verbatim}

Of course, we could use this information to subset \texttt{names}.

\vspace{0.1in}

\begin{Shaded}
\begin{Highlighting}[]
\NormalTok{names[}\FunctionTok{grep}\NormalTok{(}\StringTok{"Carex"}\NormalTok{, names)]}
\end{Highlighting}
\end{Shaded}

\begin{verbatim}
[1] "Carex elynoides" "Carex paysonis" 
\end{verbatim}

We can also get \texttt{grep} to return the values directly by specifying \texttt{value\ =\ TRUE}.

\vspace{0.1in}

\begin{Shaded}
\begin{Highlighting}[]
\FunctionTok{grep}\NormalTok{(}\StringTok{"Carex"}\NormalTok{, names, }\AttributeTok{value =} \ConstantTok{TRUE}\NormalTok{)}
\end{Highlighting}
\end{Shaded}

\begin{verbatim}
[1] "Carex elynoides" "Carex paysonis" 
\end{verbatim}

\end{example}

\(\blacksquare\)
\index[R]{\texttt{grep()}}\index[R]{\texttt{grepl()}}

\subsection{\texorpdfstring{\texttt{gsub()}}{gsub()}}\label{gsub}

The function \texttt{gsub()} can be used to substitute text that has a specified pattern. Several of its arguments are identical to \texttt{grep()} and \texttt{grepl()}:
\index[R]{\texttt{gsub()}}

\begin{itemize}
\tightlist
\item
  As before, the first argument, \texttt{pattern}, specifies a pattern to be matched.
\item
  The second argument, \texttt{replacement}, specifies a replacement for the matched pattern.
\item
  The third argument, \texttt{x}, is a character vector wherein matches are sought and substitutions are made.
\end{itemize}

\begin{example}
\protect\hypertarget{exm:exm-414}{}\label{exm:exm-414}

\(\text{}\)\\
Here we substitute \texttt{"C."} for occurrences of \texttt{"Carex"} in \texttt{names}.

\vspace{0.1in}

\begin{Shaded}
\begin{Highlighting}[]
\FunctionTok{gsub}\NormalTok{(}\StringTok{"Carex"}\NormalTok{, }\StringTok{"C."}\NormalTok{, names)}
\end{Highlighting}
\end{Shaded}

\begin{verbatim}
[1] "Achillea millefolium"   "Aster foliaceus"       
[3] "Elymus scribneri"       "Erigeron rydbergii"    
[5] "C. elynoides"           "C. paysonis"           
[7] "Taraxacum ceratophorum"
\end{verbatim}

\end{example}

\(\blacksquare\)

\subsection{\texorpdfstring{\texttt{gregexpr()}}{gregexpr()}}\label{gregexpr}

The function \texttt{gregexpr()} identifies the start and end of matching sections in a character vector.

\begin{example}
\protect\hypertarget{exm:exm-415}{}\label{exm:exm-415}\(\text{}\)\\
Here we examine the first two entries in \texttt{names}, looking for the genus \emph{Aster}.

\vspace{0.1in}

\begin{Shaded}
\begin{Highlighting}[]
\FunctionTok{gregexpr}\NormalTok{(}\StringTok{"Aster"}\NormalTok{, names[}\FunctionTok{c}\NormalTok{(}\DecValTok{1}\SpecialCharTok{:}\DecValTok{2}\NormalTok{)])}
\end{Highlighting}
\end{Shaded}

\begin{verbatim}
[[1]]
[1] -1
attr(,"match.length")
[1] -1
attr(,"index.type")
[1] "chars"
attr(,"useBytes")
[1] TRUE

[[2]]
[1] 1
attr(,"match.length")
[1] 5
attr(,"index.type")
[1] "chars"
attr(,"useBytes")
[1] TRUE
\end{verbatim}

The output list is cryptic at best and requires some explanation. The first two elements in each of the two list components indicate the character number of the start and end of the matched string. For the first list component, these elements are given the identifier \texttt{-1} because \texttt{"Achillea\ millefolium"} does not contain the pattern \texttt{"Aster"}. For the second list component, these elements are \texttt{1} and \texttt{5} because \texttt{"Aster"} makes up the first five letters of \texttt{"Aster\ foliaceus"}.
\end{example}

\(\blacksquare\)
\index[R]{\texttt{gregexpr()}}

\subsection{Regular Expressions}\label{rgx}

A number of \textbf{R} functions for managing character strings, including \texttt{grep()}, \texttt{grepl()}, \texttt{gregexpr()}, \texttt{gsub()}, and \texttt{strsplit()}, can can incorporate \emph{regular expressions}. In computer programming, a regular expression (often abbreviated as \emph{regex}) is a sequence of characters that allow pattern matching in text. Regular expressions have developed within a number of programming frameworks including the \emph{POSIX standard} (the Portable Operating System Interface standard), developed by the IEEE, and particularly the language Perl\footnote{The Perl programming language was introduced by Larry Wall in 1987 as a Unix scripting tool to facilitate report processing \citep{wikiperl}. Despite criticisms as an awkward language, Perl remains widely used for its regular expression framework and string parsing capabilities.}\index[terms]{Perl (programming language)}\index[terms]{Regular expression}.\index[terms]{IEEE (Institute of Electrical and Electronics Engineers)} Regular expressions in \textbf{R} include \emph{extended} regular expressions (this is the default for most pattern matching and replacement \textbf{R} functions), and \emph{Perl-like} regular expressions.

\subsubsection{Extended Regular Expressions}\label{ergx}

Default \emph{extended regular expressions} in \textbf{R} use a POSIX framework for commands\footnote{Specifically, they use a version of the \href{https://pubs.opengroup.org/onlinepubs/9699919799/basedefs/V1_chap09.html.}{POSIX 1003.2 standard}.}, which includes the use of particular metacharacters. These are: \texttt{\textbackslash{}}, \texttt{\textbar{}}, \texttt{(\ )}, \texttt{{[}\ {]}}, \texttt{\^{}}, \texttt{\$}, \texttt{.}, \texttt{\{\ \}}, \texttt{*}, \texttt{+}, and \texttt{?}. The metacharacters will vary in meaning depending if they occur outside of square brackets, \texttt{{[}} and \texttt{{]}}, or inside of square brackets. The former usage means that they are part of a character class (see below). In non-bracketed usage, the metacharacters in the subset below have the following applications (see \url{https://www.pcre.org/original/pcre.txt}):

\begin{itemize}
\tightlist
\item
  \texttt{\^{}} start of string or line.
\item
  \texttt{\$} end of string or line.
\item
  \texttt{.} match any character except newline.
\item
  \texttt{\textbar{}} start of alternative branch.
\item
  \texttt{(\ )} start and end subpattern.
\item
  \texttt{\{\ \}} start and end min/max repetition specification.
\end{itemize}

Several regular expression metacharacters can be placed at the end of the end of a regular expression to specify repetition. For instance, \texttt{"*"} indicates the preceding pattern should be matched zero or more times, \texttt{"\{+\}"} indicates the preceding pattern should be matched one or more times, \texttt{"\{n\}"} indicates the preceding pattern should be matched exactly \texttt{n} more times, and \texttt{"\{n,\}"} indicates the preceding pattern should be matched \texttt{n} or more times. \index[R]{\texttt{regmatches()}}

\begin{example}
\protect\hypertarget{exm:exm-416}{}\label{exm:exm-416}

\(\text{}\)\\
We will use the function \texttt{regmatches()}, which extracts or replaces matched substrings from \texttt{gregexpr()} summaries, to illustrate.

\vspace{0.1in}

\begin{Shaded}
\begin{Highlighting}[]
\NormalTok{string }\OtherTok{\textless{}{-}} \StringTok{"\%aaabaaab"}
\NormalTok{ID }\OtherTok{\textless{}{-}} \FunctionTok{gregexpr}\NormalTok{(}\StringTok{"a\{1\}"}\NormalTok{, string)}
\FunctionTok{regmatches}\NormalTok{(string, ID)}
\end{Highlighting}
\end{Shaded}

\begin{verbatim}
[[1]]
[1] "a" "a" "a" "a" "a" "a"
\end{verbatim}

\vspace{0.1in}

\begin{Shaded}
\begin{Highlighting}[]
\NormalTok{ID }\OtherTok{\textless{}{-}} \FunctionTok{gregexpr}\NormalTok{(}\StringTok{"a\{2\}"}\NormalTok{, string)}
\FunctionTok{regmatches}\NormalTok{(string, ID)}
\end{Highlighting}
\end{Shaded}

\begin{verbatim}
[[1]]
[1] "aa" "aa"
\end{verbatim}

\vspace{0.1in}

\begin{Shaded}
\begin{Highlighting}[]
\NormalTok{ID }\OtherTok{\textless{}{-}} \FunctionTok{gregexpr}\NormalTok{(}\StringTok{"a\{2,\}"}\NormalTok{, string)}
\FunctionTok{regmatches}\NormalTok{(string, ID)}
\end{Highlighting}
\end{Shaded}

\begin{verbatim}
[[1]]
[1] "aaa" "aaa"
\end{verbatim}

\end{example}

\(\blacksquare\)

\begin{example}
\protect\hypertarget{exm:exm-417}{}\label{exm:exm-417}

\(\text{}\)\\
Metacharacters can be used together. For instance, the code below demonstrates how one might get rid of one or more extra spaces at the end of character strings.

\vspace{0.1in}

\begin{Shaded}
\begin{Highlighting}[]
\NormalTok{string }\OtherTok{\textless{}{-}} \FunctionTok{c}\NormalTok{(}\StringTok{"\#\#\#Now is the time      "}\NormalTok{,}
            \StringTok{"\# for all  "}\NormalTok{,}
            \StringTok{"\#"}\NormalTok{,}
            \StringTok{" good men"}\NormalTok{,}
            \StringTok{"\#\#\# to come to the aid of their country.       "}\NormalTok{)}

\NormalTok{out }\OtherTok{\textless{}{-}} \FunctionTok{gsub}\NormalTok{(}\StringTok{" +$"}\NormalTok{, }\StringTok{""}\NormalTok{, string) }\CommentTok{\# drop extra space(s) at end of strings}
\NormalTok{out }\OtherTok{\textless{}{-}} \FunctionTok{gsub}\NormalTok{(}\StringTok{"\^{}\#*"}\NormalTok{,}\StringTok{""}\NormalTok{, out) }\CommentTok{\# drop pound sign(s)}

\FunctionTok{paste}\NormalTok{(out, }\AttributeTok{collapse =} \StringTok{""}\NormalTok{)}
\end{Highlighting}
\end{Shaded}

\begin{verbatim}
[1] "Now is the time for all good men to come to the aid of their country."
\end{verbatim}

\end{example}

\(\blacksquare\)

\begin{example}
\protect\hypertarget{exm:exm-417b}{}\label{exm:exm-417b}\(\text{}\)\\
As a biological example, microbial ``taxa'' identifiers can include cryptic Amplicon Sequence Variant (ASV) codes, followed by a general taxonomic assignment. For example, here is an ASV identifier for a bacterium within the family Comamonadaceae.
\vspace{0.1in}

\begin{Shaded}
\begin{Highlighting}[]
\NormalTok{asv }\OtherTok{\textless{}{-}} \StringTok{"6abc517aa40e9e7b9c652902fe04bb1a:f\_\_Comamonadaceae"}
\end{Highlighting}
\end{Shaded}

We can delete the ASV code, which ends in a colon, with:
\vspace{0.1in}

\begin{Shaded}
\begin{Highlighting}[]
\FunctionTok{gsub}\NormalTok{(}\StringTok{".*:"}\NormalTok{, }\StringTok{""}\NormalTok{, asv)}
\end{Highlighting}
\end{Shaded}

\begin{verbatim}
[1] "f__Comamonadaceae"
\end{verbatim}

The regex script in the first argument means: ``match any character string occurring zero or more times that ends in \texttt{:}''.
\end{example}

\(\blacksquare\)

\begin{example}
\protect\hypertarget{exm:exm-418}{}\label{exm:exm-418}

\(\text{}\)\\
As another example, \textbf{R} Markdown delimits monospace font using accent grave metacharacters, \texttt{\textasciigrave{}} \texttt{\textasciigrave{}}, while LaTeX applies this font between the expression \texttt{\textbackslash{}texttt\{} and \texttt{\}}. Below I convert a \textbf{R} Markdown-style character vector containing some monospace strings to a LaTeX-style character vector.

\vspace{0.1in}

\begin{Shaded}
\begin{Highlighting}[]
\NormalTok{char.vec }\OtherTok{\textless{}{-}} \FunctionTok{c}\NormalTok{(}\StringTok{"\textasciigrave{}+\textasciigrave{}"}\NormalTok{, }\StringTok{"addition"}\NormalTok{, }\StringTok{"$2 + 2$"}\NormalTok{, }\StringTok{"\textasciigrave{}2 + 2\textasciigrave{}"}\NormalTok{)}
\FunctionTok{gsub}\NormalTok{(}\StringTok{"(\textasciigrave{})(.*)(\textasciigrave{})"}\NormalTok{,}\StringTok{"}\SpecialCharTok{\textbackslash{}\textbackslash{}\textbackslash{}t}\StringTok{exttt\{}\SpecialCharTok{\textbackslash{}\textbackslash{}}\StringTok{2\}"}\NormalTok{, char.vec)}
\end{Highlighting}
\end{Shaded}

\begin{verbatim}
[1] "\texttt{+}"     "addition"      "$2 + 2$"       "\texttt{2 + 2}"
\end{verbatim}

With the code

\vspace{0.1in}

\begin{Shaded}
\begin{Highlighting}[]
\StringTok{"(\textasciigrave{})(.*)(\textasciigrave{})"}
\end{Highlighting}
\end{Shaded}

I subset \textbf{R} Markdown strings in \texttt{char.vec} into three potential components: 1) the \texttt{\textasciigrave{}} metacharacter beginning the string, 2) the text content between \texttt{\textasciigrave{}} metacharacters, and 3) the closing \texttt{\textasciigrave{}} metacharacter itself. I insert the content in item 2 (indicated as \texttt{\textbackslash{}\textbackslash{}2}) between \texttt{\textbackslash{}texttt\{} and \texttt{\}} using:

\vspace{0.1in}

\begin{Shaded}
\begin{Highlighting}[]
\StringTok{"}\SpecialCharTok{\textbackslash{}\textbackslash{}\textbackslash{}t}\StringTok{exttt\{}\SpecialCharTok{\textbackslash{}\textbackslash{}}\StringTok{2\}"}
\end{Highlighting}
\end{Shaded}

\end{example}

\(\blacksquare\)
\index[terms]{\LaTeX}
\index[terms]{RStudio!R Markdown}

Importantly, Example \ref{exm:exm-418} illustrates the procedure to use if a queried character is itself a general expression metacharacter. For instance, the backslash in \texttt{\textbackslash{}texttt}. In this case, the metacharacter must be escaped using single or double backslashes. That is, \texttt{\textbackslash{}texttt} must be specified as \texttt{\textbackslash{}\textbackslash{}\textbackslash{}texttt} in \texttt{gsub()}.

\begin{example}
\protect\hypertarget{exm:exm-420}{}\label{exm:exm-420}

\(\text{}\)\\
Here I ask for a string split based on the appearance of \texttt{?} (which is a regex metacharacter) and \texttt{\%} (which is not).

\vspace{0.1in}

\begin{Shaded}
\begin{Highlighting}[]
\NormalTok{string }\OtherTok{\textless{}{-}} \StringTok{"m?2\%b"}
\FunctionTok{strsplit}\NormalTok{(string, }\StringTok{"[}\SpecialCharTok{\textbackslash{}\textbackslash{}}\StringTok{?\%]"}\NormalTok{)}
\end{Highlighting}
\end{Shaded}

\begin{verbatim}
[[1]]
[1] "m" "2" "b"
\end{verbatim}

\end{example}

\(\blacksquare\)

\paragraph*{Character class}\label{character-class}
\addcontentsline{toc}{paragraph}{Character class}

A regular expression \emph{character class} is comprised of a collection of characters, specifying some query or pattern, situated between quotes (single or double) and square brace metacharacters, e.g., \texttt{"{[}"} and \texttt{"{]}"}. Thus, the code \texttt{"{[}\textbackslash{}\textbackslash{}?\%{]}"} in the previous example defines a character class. Character class pattern matches will be evaluated for any single character in the specified text. The reverse will occur if the first character of the pattern is the regular expression caret metacharacter, \texttt{\^{}}. For example, the expression \texttt{"{[}0-9{]}"} matches any single numeric character in a string, (the regular expression metacharacter \texttt{-} can be used to specify a range) and \texttt{{[}\^{}abc{]}} matches anything \emph{except} the characters \texttt{"a"}, \texttt{"b"} or \texttt{"c"}.

\begin{example}
\protect\hypertarget{exm:exm-419}{}\label{exm:exm-419}

\(\text{}\)\\
Consider the following examples:

\vspace{0.1in}

\begin{Shaded}
\begin{Highlighting}[]
\NormalTok{string }\OtherTok{\textless{}{-}} \StringTok{"a1c\&m2\%b"}
\FunctionTok{strsplit}\NormalTok{(string, }\StringTok{"[0{-}9]"}\NormalTok{) }
\end{Highlighting}
\end{Shaded}

\begin{verbatim}
[[1]]
[1] "a"   "c&m" "%b" 
\end{verbatim}

\vspace{0.1in}

\begin{Shaded}
\begin{Highlighting}[]
\FunctionTok{strsplit}\NormalTok{(string, }\StringTok{"[\^{}abc]"}\NormalTok{) }
\end{Highlighting}
\end{Shaded}

\begin{verbatim}
[[1]]
[1] "a" "c" ""  ""  ""  "b"
\end{verbatim}

\end{example}

\(\blacksquare\)

\begin{example}
\protect\hypertarget{exm:exm-421}{}\label{exm:exm-421}

\(\text{}\)\\
This regular expression will match most email addresses:

\vspace{0.1in}

\begin{Shaded}
\begin{Highlighting}[]
\NormalTok{pattern }\OtherTok{\textless{}{-}} \StringTok{"[{-}a{-}z0{-}9\_.\%]+}\SpecialCharTok{\textbackslash{}\textbackslash{}}\StringTok{@[{-}a{-}z0{-}9\_.\%]+}\SpecialCharTok{\textbackslash{}\textbackslash{}}\StringTok{.[a{-}z]+"}
\end{Highlighting}
\end{Shaded}

The expression literally reads: ``1) find one or more occurrences of characters in a-z or A-Z or 0-9 or dashes or periods, followed by 2) the ampersand symbol (literally), followed by 3) one or more occurrences of characters in a-z or A-Z or 0-9 or dashes or periods, followed by 4) a literal period, followed by one or more occurrences of the letters a-z or A-Z.'' Here is a string we wish to query:

\vspace{0.1in}

\begin{Shaded}
\begin{Highlighting}[]
\NormalTok{string }\OtherTok{\textless{}{-}} \FunctionTok{c}\NormalTok{(}\StringTok{"abc\_noboby@isu.edu"}\NormalTok{,}
            \StringTok{"text with no email"}\NormalTok{,}
            \StringTok{"me@mything.com"}\NormalTok{,}
            \StringTok{"also"}\NormalTok{,}
            \StringTok{"you@yourspace.com"}\NormalTok{,}
            \StringTok{"@you"}
\NormalTok{            )}
\end{Highlighting}
\end{Shaded}

We confirm that elements 1, 3, and 5 from \texttt{string} are email addresses.

\vspace{0.1in}

\begin{Shaded}
\begin{Highlighting}[]
\FunctionTok{grep}\NormalTok{(pattern, string, }\AttributeTok{ignore.case =} \ConstantTok{TRUE}\NormalTok{, }\AttributeTok{value =} \ConstantTok{TRUE}\NormalTok{)}
\end{Highlighting}
\end{Shaded}

\begin{verbatim}
[1] "abc_noboby@isu.edu" "me@mything.com"     "you@yourspace.com" 
\end{verbatim}

\end{example}

\(\blacksquare\)

Certain character classes are predefined. These classes have names that are bounded by two square brackets and colons, and include \texttt{"{[}{[}:lower:{]}{]}"} and \texttt{"{[}{[}:upper:{]}{]}"} which identify lower and upper case letters, \texttt{"{[}:punct:{]}"} which identifies punctuation, \texttt{{[}{[}:alnum:{]}{]}}, which identifies all alphanumeric characters, and \texttt{"{[}{[}:space:{]}{]}"}, which identifies space characters, e.g., tab and newline.

\vspace{0.1in}

\begin{Shaded}
\begin{Highlighting}[]
\NormalTok{string }\OtherTok{\textless{}{-}} \FunctionTok{c}\NormalTok{(}\StringTok{"M2Ab"}\NormalTok{, }\StringTok{"def"}\NormalTok{, }\StringTok{"?"}\NormalTok{, }\StringTok{"\%"}\NormalTok{, }\StringTok{"}\SpecialCharTok{\textbackslash{}n}\StringTok{"}\NormalTok{)}
\FunctionTok{grepl}\NormalTok{(}\StringTok{"[[:lower:]]"}\NormalTok{, string)}
\end{Highlighting}
\end{Shaded}

\begin{verbatim}
[1]  TRUE  TRUE FALSE FALSE FALSE
\end{verbatim}

\vspace{0.1in}

\begin{Shaded}
\begin{Highlighting}[]
\FunctionTok{grepl}\NormalTok{(}\StringTok{"[[:upper:]]"}\NormalTok{, string)}
\end{Highlighting}
\end{Shaded}

\begin{verbatim}
[1]  TRUE FALSE FALSE FALSE FALSE
\end{verbatim}

\vspace{0.1in}

\begin{Shaded}
\begin{Highlighting}[]
\FunctionTok{grepl}\NormalTok{(}\StringTok{"[[:punct:]]"}\NormalTok{, string)}
\end{Highlighting}
\end{Shaded}

\begin{verbatim}
[1] FALSE FALSE  TRUE  TRUE FALSE
\end{verbatim}

\vspace{0.1in}

\begin{Shaded}
\begin{Highlighting}[]
\FunctionTok{grepl}\NormalTok{(}\StringTok{"[[:space:]]"}\NormalTok{, string) }\CommentTok{\# item five is a newline request}
\end{Highlighting}
\end{Shaded}

\begin{verbatim}
[1] FALSE FALSE FALSE FALSE  TRUE
\end{verbatim}

Here I ask \textbf{R} to return elements from \texttt{string} that are three or more characters long.

\vspace{0.1in}

\begin{Shaded}
\begin{Highlighting}[]
\FunctionTok{grep}\NormalTok{(}\StringTok{"[[:alnum:]]\{3\}"}\NormalTok{, string, }\AttributeTok{value =} \ConstantTok{TRUE}\NormalTok{)}
\end{Highlighting}
\end{Shaded}

\begin{verbatim}
[1] "M2Ab" "def" 
\end{verbatim}

\paragraph{Turning off regular expressions}

For some pattern matching and replacement jobs it may be best turn off the default extended regular expressions and use exact matching by specifying \texttt{fixed\ =\ TRUE}. For example, \textbf{R} may place periods in the place of spaces in character strings and column names in dataframes and arrays.

\begin{example}
\protect\hypertarget{exm:exm-422}{}\label{exm:exm-422}

\(\text{}\)\\
Consider the following example:

\vspace{0.1in}

\begin{Shaded}
\begin{Highlighting}[]
\NormalTok{countries }\OtherTok{\textless{}{-}} \FunctionTok{c}\NormalTok{(}\StringTok{"United.States"}\NormalTok{, }\StringTok{"United.Arab.Emirates"}\NormalTok{, }\StringTok{"China"}\NormalTok{, }\StringTok{"Germany"}\NormalTok{)}
\FunctionTok{gsub}\NormalTok{(}\StringTok{"."}\NormalTok{, }\StringTok{" "}\NormalTok{, countries)}
\end{Highlighting}
\end{Shaded}

\begin{verbatim}
[1] "             "        "                    " "     "               
[4] "       "             
\end{verbatim}

Note that using \texttt{gsub(".",\ "\ ",\ countries)} results in the replacement of all text with spaces because of the meaning of the period metacharacter. To get the desired result we could use:

\vspace{0.1in}

\begin{Shaded}
\begin{Highlighting}[]
\FunctionTok{gsub}\NormalTok{(}\StringTok{"."}\NormalTok{, }\StringTok{" "}\NormalTok{, countries, }\AttributeTok{fixed =} \ConstantTok{TRUE}\NormalTok{)}
\end{Highlighting}
\end{Shaded}

\begin{verbatim}
[1] "United States"        "United Arab Emirates" "China"               
[4] "Germany"             
\end{verbatim}

Of course we could also double escape the period.

\vspace{0.1in}

\begin{Shaded}
\begin{Highlighting}[]
\FunctionTok{gsub}\NormalTok{(}\StringTok{"}\SpecialCharTok{\textbackslash{}\textbackslash{}}\StringTok{."}\NormalTok{, }\StringTok{" "}\NormalTok{, countries)}
\end{Highlighting}
\end{Shaded}

\begin{verbatim}
[1] "United States"        "United Arab Emirates" "China"               
[4] "Germany"             
\end{verbatim}

\end{example}

\(\blacksquare\)

\subsubsection{Perl-like Regular Expressions}\label{perl-like-regular-expressions}

The \textbf{R} character string functions \texttt{grep()}, \texttt{grepl()}, \texttt{regexpr()}, \texttt{gregexpr()}, \texttt{sub()}, \texttt{gsub()}, and \texttt{strsplit()} allow \emph{Perl-like regular expression} pattern matching. This is done by specifying \texttt{perl\ =\ TRUE}, which switches regular expression handling to the \emph{PRCE} package. Perl allows handling of the POSIX predefined character classes, e.g., \texttt{"{[}{[}:lower:{]}{]}"}, along with a wide variety of other calls which are generally implemented using metacharacters and double backslash commands. Here are some examples. \index[terms]{R!package!\textit{PRCE}}

\begin{itemize}
\tightlist
\item
  \texttt{\textbackslash{}\textbackslash{}d} any decimal digit.
\item
  \texttt{\textbackslash{}\textbackslash{}D} any character that is not a decimal digit.
\item
  \texttt{\textbackslash{}\textbackslash{}h} any horizontal white space character (e.g., tab, space).
\item
  \texttt{\textbackslash{}\textbackslash{}H} any character that is not a horizontal white space character.
\item
  \texttt{\textbackslash{}\textbackslash{}s} any white space character.
\item
  \texttt{\textbackslash{}\textbackslash{}S} any character that is not a white space character.
\item
  \texttt{\textbackslash{}\textbackslash{}v} any vertical white space character (e.g., newline).
\item
  \texttt{\textbackslash{}\textbackslash{}V} any character that is not a vertical white space character.
\item
  \texttt{\textbackslash{}\textbackslash{}w} any word, i.e., letter or character components separated by white space.
\item
  \texttt{\textbackslash{}\textbackslash{}W} any non word.
\item
  \texttt{\textbackslash{}\textbackslash{}b} a word boundary.
\item
  \texttt{\textbackslash{}\textbackslash{}U} upper case character (dependent on context).
\item
  \texttt{\textbackslash{}\textbackslash{}L} lower case character (dependent on context).
\end{itemize}

Note that reversals in meaning occur for capitalized and uncapitalized commands.

\begin{example}
\protect\hypertarget{exm:exm-42-5}{}\label{exm:exm-42-5}

\(\text{}\)\\

Here we identify string entries containing numbers.

\vspace{0.1in}

\begin{Shaded}
\begin{Highlighting}[]
\NormalTok{string }\OtherTok{\textless{}{-}} \FunctionTok{c}\NormalTok{(}\StringTok{"Acidobacteria"}\NormalTok{, }\StringTok{"Actinobacteria"}\NormalTok{, }\StringTok{"TM7.1"}\NormalTok{, }\StringTok{"Gitt{-}GS{-}136"}\NormalTok{, }
            \StringTok{"Chloroflexia"}\NormalTok{, }\StringTok{"Bacili"}\NormalTok{)}

\FunctionTok{grep}\NormalTok{(}\StringTok{"}\SpecialCharTok{\textbackslash{}\textbackslash{}}\StringTok{d"}\NormalTok{, string, }\AttributeTok{perl =} \ConstantTok{TRUE}\NormalTok{)}
\end{Highlighting}
\end{Shaded}

\begin{verbatim}
[1] 3 4
\end{verbatim}

And those containing non-numeric characters (i.e., all of the entries).

\vspace{0.1in}

\begin{Shaded}
\begin{Highlighting}[]
\FunctionTok{grep}\NormalTok{(}\StringTok{"}\SpecialCharTok{\textbackslash{}\textbackslash{}}\StringTok{D"}\NormalTok{, string, }\AttributeTok{perl =} \ConstantTok{TRUE}\NormalTok{)}
\end{Highlighting}
\end{Shaded}

\begin{verbatim}
[1] 1 2 3 4 5 6
\end{verbatim}

To subset non-numeric entries, one could do something like:

\vspace{0.1in}

\begin{Shaded}
\begin{Highlighting}[]
\NormalTok{string[}\SpecialCharTok{{-}}\FunctionTok{grep}\NormalTok{(}\StringTok{"}\SpecialCharTok{\textbackslash{}\textbackslash{}}\StringTok{d"}\NormalTok{, string, }\AttributeTok{perl =} \ConstantTok{TRUE}\NormalTok{)]}
\end{Highlighting}
\end{Shaded}

\begin{verbatim}
[1] "Acidobacteria"  "Actinobacteria" "Chloroflexia"   "Bacili"        
\end{verbatim}

\end{example}

\(\blacksquare\)

\begin{example}
\protect\hypertarget{exm:exm-423}{}\label{exm:exm-423}\(\text{}\)\\
As a slightly extended example we will count the number of words in the description of the GNU public licences in \textbf{R} (obtained via \texttt{RShowDoc("COPYING")}). Ideas here largely follow from the function \texttt{DescTools::StrCountW()} \citep{signor}.

Text can be read from a connection using the function \texttt{readLines()}.

\vspace{0.1in}

\begin{Shaded}
\begin{Highlighting}[]
\NormalTok{GNU }\OtherTok{\textless{}{-}} \FunctionTok{readLines}\NormalTok{(}\FunctionTok{RShowDoc}\NormalTok{(}\StringTok{"COPYING"}\NormalTok{))}
\FunctionTok{head}\NormalTok{(GNU)}
\end{Highlighting}
\end{Shaded}

\begin{verbatim}
[1] "\t\t    GNU GENERAL PUBLIC LICENSE"                                               
[2] "\t\t       Version 2, June 1991"                                                  
[3] ""                                                                               
[4] " Copyright (C) 1989, 1991 Free Software Foundation, Inc."                       
[5] "                       51 Franklin St, Fifth Floor, Boston, MA  02110-1301  USA"
[6] " Everyone is permitted to copy and distribute verbatim copies"                  
\end{verbatim}

Note that the escaped command \texttt{\textbackslash{}t} represent the ASCII (American character encoding standard) control character for tab return. \index[terms]{Command character}\index[terms]{Command character!\texttt{\textbackslash t}} Other useful escaped control characters include \texttt{\textbackslash{}n}\index[terms]{Command character!\texttt{\textbackslash n}}, indicating new line or carriage return.

To search for words, we will actually identify string components that are not words, identified with the Perl regex command \texttt{\textbackslash{}\textbackslash{}W} and word boundaries, i.e., \texttt{\textbackslash{}\textbackslash{}b}. We can combine these summarily as: \texttt{\textbackslash{}\textbackslash{}b\textbackslash{}\textbackslash{}W+\textbackslash{}\textbackslash{}b}. The call \texttt{\textbackslash{}\textbackslash{}W+} indicates a non-word match occurring one or more times. Here we apply this regular expression to the first element of \texttt{GNU}.

\vspace{0.1in}

\begin{Shaded}
\begin{Highlighting}[]
\NormalTok{GNU[}\DecValTok{1}\NormalTok{]}
\end{Highlighting}
\end{Shaded}

\begin{verbatim}
[1] "\t\t    GNU GENERAL PUBLIC LICENSE"
\end{verbatim}

\begin{Shaded}
\begin{Highlighting}[]
\FunctionTok{gregexpr}\NormalTok{(}\StringTok{"}\SpecialCharTok{\textbackslash{}\textbackslash{}}\StringTok{b}\SpecialCharTok{\textbackslash{}\textbackslash{}}\StringTok{W+}\SpecialCharTok{\textbackslash{}\textbackslash{}}\StringTok{b"}\NormalTok{, GNU[}\DecValTok{1}\NormalTok{], }\AttributeTok{perl =} \ConstantTok{TRUE}\NormalTok{)}
\end{Highlighting}
\end{Shaded}

\begin{verbatim}
[[1]]
[1] 10 18 25
attr(,"match.length")
[1] 1 1 1
attr(,"index.type")
[1] "chars"
attr(,"useBytes")
[1] TRUE
\end{verbatim}

Matches occur at three locations, \texttt{10}, \texttt{18}, and \texttt{25}, which separate the four words \texttt{GNU\ GENERAL\ PUBLIC\ LICENSE}. Thus, to analyze the entire document we could use:

\vspace{0.1in}

\begin{Shaded}
\begin{Highlighting}[]
\FunctionTok{sum}\NormalTok{(}\FunctionTok{sapply}\NormalTok{(}\FunctionTok{gregexpr}\NormalTok{(}\StringTok{"}\SpecialCharTok{\textbackslash{}\textbackslash{}}\StringTok{b}\SpecialCharTok{\textbackslash{}\textbackslash{}}\StringTok{W+}\SpecialCharTok{\textbackslash{}\textbackslash{}}\StringTok{b"}\NormalTok{, GNU, }\AttributeTok{perl =} \ConstantTok{TRUE}\NormalTok{),}
           \ControlFlowTok{function}\NormalTok{(x) }\FunctionTok{sum}\NormalTok{(x }\SpecialCharTok{\textgreater{}} \DecValTok{0}\NormalTok{)) }\SpecialCharTok{+} \DecValTok{1}\NormalTok{)}
\end{Highlighting}
\end{Shaded}

\begin{verbatim}
[1] 3048
\end{verbatim}

There are 3048 total words in the license description.
\end{example}

\(\blacksquare\)

\index[R]{\texttt{DescTools::StrCountW()}}
\index[R]{\texttt{readLines()}}

One can identify substrings by number using Perl.

\begin{example}
\protect\hypertarget{exm:exm-424}{}\label{exm:exm-424}

\(\text{}\)\\
In this example, I subdivide a string into two components, the first character, i.e., \texttt{"(\textbackslash{}\textbackslash{}w)"}, and the remaining zero or more characters: \texttt{"(\textbackslash{}\textbackslash{}w*)"}. These are referred to in the \texttt{substitute} argument of \texttt{gsub} as items \texttt{\textbackslash{}\textbackslash{}1} and \texttt{\textbackslash{}\textbackslash{}2}, respectively. Capitalization for these substrings are handled in different ways below.

\vspace{0.1in}

\begin{Shaded}
\begin{Highlighting}[]
\NormalTok{string }\OtherTok{\textless{}{-}} \StringTok{"achillea"}
\FunctionTok{gsub}\NormalTok{(}\StringTok{"(}\SpecialCharTok{\textbackslash{}\textbackslash{}}\StringTok{w)(}\SpecialCharTok{\textbackslash{}\textbackslash{}}\StringTok{w*)"}\NormalTok{, }\StringTok{"}\SpecialCharTok{\textbackslash{}\textbackslash{}}\StringTok{U}\SpecialCharTok{\textbackslash{}\textbackslash{}}\StringTok{1}\SpecialCharTok{\textbackslash{}\textbackslash{}}\StringTok{U}\SpecialCharTok{\textbackslash{}\textbackslash{}}\StringTok{2"}\NormalTok{, string, }\AttributeTok{perl=}\ConstantTok{TRUE}\NormalTok{) }\CommentTok{\# all caps}
\end{Highlighting}
\end{Shaded}

\begin{verbatim}
[1] "ACHILLEA"
\end{verbatim}

\begin{Shaded}
\begin{Highlighting}[]
\FunctionTok{gsub}\NormalTok{(}\StringTok{"(}\SpecialCharTok{\textbackslash{}\textbackslash{}}\StringTok{w)(}\SpecialCharTok{\textbackslash{}\textbackslash{}}\StringTok{w*)"}\NormalTok{, }\StringTok{"}\SpecialCharTok{\textbackslash{}\textbackslash{}}\StringTok{L}\SpecialCharTok{\textbackslash{}\textbackslash{}}\StringTok{1}\SpecialCharTok{\textbackslash{}\textbackslash{}}\StringTok{U}\SpecialCharTok{\textbackslash{}\textbackslash{}}\StringTok{2"}\NormalTok{, string, }\AttributeTok{perl=}\ConstantTok{TRUE}\NormalTok{) }\CommentTok{\# low, then upper case}
\end{Highlighting}
\end{Shaded}

\begin{verbatim}
[1] "aCHILLEA"
\end{verbatim}

\begin{Shaded}
\begin{Highlighting}[]
\FunctionTok{gsub}\NormalTok{(}\StringTok{"(}\SpecialCharTok{\textbackslash{}\textbackslash{}}\StringTok{w)(}\SpecialCharTok{\textbackslash{}\textbackslash{}}\StringTok{w*)"}\NormalTok{, }\StringTok{"}\SpecialCharTok{\textbackslash{}\textbackslash{}}\StringTok{U}\SpecialCharTok{\textbackslash{}\textbackslash{}}\StringTok{1}\SpecialCharTok{\textbackslash{}\textbackslash{}}\StringTok{L}\SpecialCharTok{\textbackslash{}\textbackslash{}}\StringTok{2"}\NormalTok{, string, }\AttributeTok{perl=}\ConstantTok{TRUE}\NormalTok{) }\CommentTok{\# up, then lower case}
\end{Highlighting}
\end{Shaded}

\begin{verbatim}
[1] "Achillea"
\end{verbatim}

The functions \texttt{tolower()} and \texttt{toupper()} provide simpler approaches to convert letters to lower and upper case, respectively.

\vspace{0.1in}

\begin{Shaded}
\begin{Highlighting}[]
\FunctionTok{toupper}\NormalTok{(string)}
\end{Highlighting}
\end{Shaded}

\begin{verbatim}
[1] "ACHILLEA"
\end{verbatim}

\end{example}

\(\blacksquare\)
\index[R]{\texttt{tolower()}}
\index[R]{\texttt{toupper()}}

\section{Date-Time Classes}\label{datetime}

There are two basic \textbf{R} date-time classes, \emph{POSIXlt} and \emph{POSIXct}\footnote{Again, the POSIX prefix refers to the IEEE standard Portable Operating System Interface}. Class POSIXct represents the (signed) number of seconds since the beginning of 1970 (in the UTC time zone) as a numeric vector. An object of class POSIXlt will be comprised of a list of vectors with the names \texttt{sec}, \texttt{min}, \texttt{hour}, \texttt{mday} (day of month), \texttt{mon} (month), \texttt{year}, \texttt{wday} (day of week), and \texttt{yday} (day of year).
\index[terms]{R!classes!POSIXlt}
\index[terms]{R!classes!POSIXct}

POSIX naming conventions include:
\index[R]{\texttt{strptime()}}

\begin{itemize}
\tightlist
\item
  \texttt{\%m} = Month as a decimal number (01--12).
\item
  \texttt{\%d} = Day of the month as a decimal number (01--31).
\item
  \texttt{\%Y} = Year. Designations in \texttt{0:9999} are accepted.
\item
  \texttt{\%H} = Hour as a decimal number (00--23).
\item
  \texttt{\%M} = Minute as a decimal number (00--59
\end{itemize}

\begin{example}
\protect\hypertarget{exm:exm-425}{}\label{exm:exm-425}

\(\text{}\)\\
As an example, below are twenty dates and corresponding binary water presence measures (0 = water absent, 1 = water present) recorded at 2.5 hour intervals for an intermittent stream site in southwest Idaho \citep{aho2023communication}.

\vspace{0.1in}

\begin{Shaded}
\begin{Highlighting}[]
\NormalTok{dates }\OtherTok{\textless{}{-}} \FunctionTok{c}\NormalTok{(}\StringTok{"08/13/2019 04:00"}\NormalTok{, }\StringTok{"08/13/2019 06:30"}\NormalTok{, }\StringTok{"08/13/2019 09:00"}\NormalTok{,}
           \StringTok{"08/13/2019 11:30"}\NormalTok{, }\StringTok{"08/13/2019 14:00"}\NormalTok{, }\StringTok{"08/13/2019 16:30"}\NormalTok{,}
           \StringTok{"08/13/2019 19:00"}\NormalTok{, }\StringTok{"08/13/2019 21:30"}\NormalTok{, }\StringTok{"08/14/2019 00:00"}\NormalTok{,}
           \StringTok{"08/14/2019 02:30"}\NormalTok{, }\StringTok{"08/14/2019 05:00"}\NormalTok{, }\StringTok{"08/14/2019 07:30"}\NormalTok{,}
           \StringTok{"08/14/2019 10:00"}\NormalTok{, }\StringTok{"08/14/2019 12:30"}\NormalTok{, }\StringTok{"08/14/2019 15:00"}\NormalTok{,}
           \StringTok{"08/14/2019 17:30"}\NormalTok{, }\StringTok{"08/14/2019 20:00"}\NormalTok{, }\StringTok{"08/14/2019 22:30"}\NormalTok{,}
           \StringTok{"08/15/2019 01:00"}\NormalTok{, }\StringTok{"08/15/2019 03:30"}\NormalTok{)}

\NormalTok{pres.abs }\OtherTok{\textless{}{-}} \FunctionTok{c}\NormalTok{(}\DecValTok{1}\NormalTok{, }\DecValTok{1}\NormalTok{, }\DecValTok{1}\NormalTok{, }\DecValTok{1}\NormalTok{, }\DecValTok{1}\NormalTok{, }\DecValTok{0}\NormalTok{, }\DecValTok{0}\NormalTok{, }\DecValTok{1}\NormalTok{, }\DecValTok{1}\NormalTok{, }\DecValTok{1}\NormalTok{, }\DecValTok{1}\NormalTok{, }\DecValTok{1}\NormalTok{, }\DecValTok{1}\NormalTok{, }\DecValTok{1}\NormalTok{, }\DecValTok{0}\NormalTok{, }\DecValTok{0}\NormalTok{, }\DecValTok{0}\NormalTok{, }\DecValTok{1}\NormalTok{, }\DecValTok{1}\NormalTok{, }\DecValTok{1}\NormalTok{)}
\end{Highlighting}
\end{Shaded}

To convert the character string \texttt{dates} to a date-time object we can use the function \texttt{strptime()}. We have:

\vspace{0.1in}

\begin{Shaded}
\begin{Highlighting}[]
\NormalTok{dates.ts }\OtherTok{\textless{}{-}} \FunctionTok{strptime}\NormalTok{(dates, }\AttributeTok{format =} \StringTok{"\%m/\%d/\%Y \%H:\%M"}\NormalTok{)}
\FunctionTok{class}\NormalTok{(dates.ts)}
\end{Highlighting}
\end{Shaded}

\begin{verbatim}
[1] "POSIXlt" "POSIXt" 
\end{verbatim}

Note that the dates can now be evaluated numerically.

\vspace{0.1in}

\begin{Shaded}
\begin{Highlighting}[]
\NormalTok{dates.df }\OtherTok{\textless{}{-}} \FunctionTok{data.frame}\NormalTok{(}\AttributeTok{dates =}\NormalTok{ dates.ts, }\AttributeTok{pres.abs =}\NormalTok{ pres.abs)}
\FunctionTok{summary}\NormalTok{(dates.df)}
\end{Highlighting}
\end{Shaded}

\begin{verbatim}
     dates                        pres.abs   
 Min.   :2019-08-13 04:00:00   Min.   :0.00  
 1st Qu.:2019-08-13 15:52:30   1st Qu.:0.75  
 Median :2019-08-14 03:45:00   Median :1.00  
 Mean   :2019-08-14 03:45:00   Mean   :0.75  
 3rd Qu.:2019-08-14 15:37:30   3rd Qu.:1.00  
 Max.   :2019-08-15 03:30:00   Max.   :1.00  
\end{verbatim}

I can also easily extract time series components.

\vspace{0.1in}

\begin{Shaded}
\begin{Highlighting}[]
\NormalTok{dates.ts}\SpecialCharTok{$}\NormalTok{mday }\CommentTok{\# day of month}
\end{Highlighting}
\end{Shaded}

\begin{verbatim}
 [1] 13 13 13 13 13 13 13 13 14 14 14 14 14 14 14 14 14 14 15 15
\end{verbatim}

\begin{Shaded}
\begin{Highlighting}[]
\NormalTok{dates.ts}\SpecialCharTok{$}\NormalTok{wday }\CommentTok{\# day of week}
\end{Highlighting}
\end{Shaded}

\begin{verbatim}
 [1] 2 2 2 2 2 2 2 2 3 3 3 3 3 3 3 3 3 3 4 4
\end{verbatim}

\begin{Shaded}
\begin{Highlighting}[]
\NormalTok{dates.ts}\SpecialCharTok{$}\NormalTok{hour }\CommentTok{\# hour}
\end{Highlighting}
\end{Shaded}

\begin{verbatim}
 [1]  4  6  9 11 14 16 19 21  0  2  5  7 10 12 15 17 20 22  1  3
\end{verbatim}

\end{example}

\(\blacksquare\)

\section*{Exercises}\label{exercises-3}
\addcontentsline{toc}{section}{Exercises}

\begin{enumerate}
\def\labelenumi{\arabic{enumi}.}
\item
  Using the \texttt{plant} dataset from Question 5 in the Exercises at the end of Chapter \ref{ch3}, perform the following operations.

  \begin{enumerate}
  \def\labelenumii{(\alph{enumii})}
  \tightlist
  \item
    Attempt to simultaneously calculate the column means for plant height and soil \% N using \texttt{FUN\ =\ mean} in \texttt{apply()}. Was there an issue? Why?
  \item
    Eliminate missing rows in \texttt{plant} using \texttt{na.omit()} and repeat (a). Did this change the mean for plant height? Why?
  \item
    Modify the \texttt{FUN} argument in \texttt{apply()} to be: \texttt{FUN\ =\ function(x)\ mean(x,\ na.rm\ =\ TRUE)}. This will eliminate \texttt{NA}s on a column by column basis.
  \item
    Compare the results in (a), (b), (c). Which is the best approach? Why?
  \item
    Find the mean and variance of plant heights for each Management Type in \texttt{plant} using \texttt{tapply()}. Use the best practice approach for \texttt{FUN}, as deduced in (d).
    \vspace{0.1in}
  \end{enumerate}
\item
  For the questions below, use the list object \texttt{list.data}.

  \begin{enumerate}
  \def\labelenumii{(\alph{enumii})}
  \item
    Use \texttt{sapply(list.data,\ FUN\ =\ length)} to get the number of components in each element of \texttt{list.data}.
  \item
    Repeat (a) using \texttt{lapply()}. How is the output in (b) different from (a)?
    \vspace{0.1in}

\begin{Shaded}
\begin{Highlighting}[]
\NormalTok{list.data }\OtherTok{\textless{}{-}} \FunctionTok{list}\NormalTok{(}\AttributeTok{a =} \DecValTok{1}\SpecialCharTok{:}\DecValTok{9}\NormalTok{, }\AttributeTok{height =} \FunctionTok{rnorm}\NormalTok{(}\DecValTok{50}\NormalTok{),}
                  \AttributeTok{greet =} \FunctionTok{c}\NormalTok{(}\StringTok{"hello"}\NormalTok{, }\StringTok{"goodbye"}\NormalTok{, }\StringTok{"hello"}\NormalTok{))}
\end{Highlighting}
\end{Shaded}

    \vspace{0.1in}
  \end{enumerate}
\item
  A frequently used statistical application is the calculation of all possible mean differences. Assume that we have arithmetic means for the treatments \texttt{trt1}, \texttt{trt2}, \texttt{trt3}, \texttt{trt4} and \texttt{trt5}, given in the object \texttt{means} below.

  \begin{enumerate}
  \def\labelenumii{(\alph{enumii})}
  \item
    Calculate all possible mean differences using \texttt{means} as the first two arguments in \texttt{outer()}, and letting \texttt{FUN\ =\ "-"}.
  \item
    Extract meaningful and non-redundant differences by using \texttt{upper.tri()} or \texttt{lower.tri()} (Section \ref{utlt}). There should be \({5 \choose 2} = 10\) meaningful (not simply a mean subtracted from itself) and non-redundant differences.
    \vspace{0.1in}

\begin{Shaded}
\begin{Highlighting}[]
\NormalTok{means }\OtherTok{\textless{}{-}} \FunctionTok{c}\NormalTok{(}\AttributeTok{trt1 =} \FloatTok{20.5}\NormalTok{, }\AttributeTok{trt2 =} \FloatTok{15.3}\NormalTok{, }\AttributeTok{trt3 =} \FloatTok{22.1}\NormalTok{, }\AttributeTok{trt4 =} \FloatTok{30.4}\NormalTok{, }
           \AttributeTok{trt5 =} \DecValTok{28}\NormalTok{)}
\end{Highlighting}
\end{Shaded}

    \vspace{0.1in}
  \end{enumerate}
\item
  Using the \texttt{plant} dataset from Question 5 in the Exercises for Chapter \ref{ch3}, perform the following operations.

  \begin{enumerate}
  \def\labelenumii{(\alph{enumii})}
  \tightlist
  \item
    Use the function \texttt{replace()} to identify samples with soil N less than 13.5\% by identifying them as \texttt{"Npoor"}.
  \item
    Use the function \texttt{which()} to identify which plant heights are greater than or equal to 33.2 dm.
  \item
    Sort plant heights using the function \texttt{sort()}.
  \item
    Sort the \texttt{plant} dataset with respect to ascending values of plant height using the function \texttt{order()}.
    \vspace{0.1in}
  \end{enumerate}
\item
  Using \texttt{match()} or \texttt{which} and \texttt{\%in\%}, replace the code column names in the dataset \texttt{cliff.sp} from the package \emph{asbio}, with the correct scientific names (genus and specific epithet) from the dataframe \texttt{sp.list} below.
  \vspace{0.1in}

\begin{Shaded}
\begin{Highlighting}[]
\NormalTok{sp.list }\OtherTok{\textless{}{-}} \FunctionTok{data.frame}\NormalTok{(}\AttributeTok{code =} \FunctionTok{c}\NormalTok{(}\StringTok{"L\_ASCA"}\NormalTok{,}\StringTok{"L\_CLCI"}\NormalTok{,}\StringTok{"L\_COSPP"}\NormalTok{,}\StringTok{"L\_COUN"}\NormalTok{,}
\StringTok{"L\_DEIN"}\NormalTok{,}\StringTok{"L\_LCAT"}\NormalTok{, }\StringTok{"L\_LCST"}\NormalTok{,}\StringTok{"L\_LEDI"}\NormalTok{,}\StringTok{"M\_POSP"}\NormalTok{,}\StringTok{"L\_STDR"}\NormalTok{,}\StringTok{"L\_THSP"}\NormalTok{,}
\StringTok{"L\_TOCA"}\NormalTok{,}\StringTok{"L\_XAEL"}\NormalTok{,}\StringTok{"M\_AMSE"}\NormalTok{, }\StringTok{"M\_CRFI"}\NormalTok{,}\StringTok{"M\_DISP"}\NormalTok{,}\StringTok{"M\_WECO"}\NormalTok{,}\StringTok{"P\_MIGU"}\NormalTok{,}
\StringTok{"P\_POAR"}\NormalTok{,}\StringTok{"P\_SAOD"}\NormalTok{), }
\AttributeTok{sci.name =} \FunctionTok{c}\NormalTok{(}\StringTok{"Aspicilia caesiocineria"}\NormalTok{,}\StringTok{"Caloplaca citrina"}\NormalTok{,}
\StringTok{"Collema spp."}\NormalTok{, }\StringTok{"Collema undulatum"}\NormalTok{, }\StringTok{"Dermatocarpon intestiniforme"}\NormalTok{,}
\StringTok{"Lecidea atrobrunnea"}\NormalTok{, }\StringTok{"Lecidella stigmatea"}\NormalTok{, }\StringTok{"Lecanora dispersa"}\NormalTok{,}
\StringTok{"Pohlia sp."}\NormalTok{, }\StringTok{"Staurothele drummondii"}\NormalTok{, }\StringTok{"Thelidium species"}\NormalTok{,}
\StringTok{"Toninia candida"}\NormalTok{, }\StringTok{"Xanthoria elegans"}\NormalTok{, }\StringTok{"Amblystegium serpens"}\NormalTok{,}
\StringTok{"Cratoneuron filicinum"}\NormalTok{, }\StringTok{"Dicranella species"}\NormalTok{, }\StringTok{"Weissia controversa"}\NormalTok{,}
\StringTok{"Mimulus guttatus"}\NormalTok{, }\StringTok{"Poa pattersonii"}\NormalTok{, }\StringTok{"Saxifraga odontoloma"}\NormalTok{))}
\end{Highlighting}
\end{Shaded}

  \vspace{0.1in}
\item
  Using the \texttt{sp.list} dataframe from the previous question, perform the following operations:

  \begin{enumerate}
  \def\labelenumii{(\alph{enumii})}
  \tightlist
  \item
    Apply \texttt{strsplit()} to the the column \texttt{sp.list\$sci.name} to create a two column dataframe with genus and corresponding species names.
  \item
    A two character prefix in the column \texttt{sp.list\$code} indicates whether a taxon is a lichen (prefix = \texttt{"L\_"}), a marchantiophyte (prefix = \texttt{"M\_"}), or a vascular plant (prefix = \texttt{"P\_"}). Use \texttt{grep()} to identify marchantiophytes.
    \vspace{0.1in}
  \end{enumerate}
\item
  Use the string vector \texttt{string} below to answer the following questions:

  \begin{enumerate}
  \def\labelenumii{(\alph{enumii})}
  \item
    Use regular expressions in the \texttt{pattern} argument of \texttt{gsub()} to get rid of extra spaces at the start of string elements while preserving spaces between words.
  \item
    Use the predefined character class \texttt{{[}{[}:alnum:{]}{]}} and an accompanying quantifier in the \texttt{pattern} argument from \texttt{grep()} to count the number of words whose length is greater than or equal to four characters.
    \vspace{0.1in}

\begin{Shaded}
\begin{Highlighting}[]
\NormalTok{string }\OtherTok{\textless{}{-}} \FunctionTok{c}\NormalTok{(}\StringTok{"   Statistics is "}\NormalTok{, }\StringTok{"      a "}\NormalTok{, }\StringTok{" great topic."}\NormalTok{)}
\end{Highlighting}
\end{Shaded}

    \vspace{0.1in}
  \end{enumerate}
\item
  Remove the numbers from the character vector below using \texttt{gsub()} and an appropriate Perl-like regular expression.\\
  \vspace{0.1in}

\begin{Shaded}
\begin{Highlighting}[]
\NormalTok{x }\OtherTok{\textless{}{-}} \FunctionTok{c}\NormalTok{(}\StringTok{"enzyme1"}\NormalTok{,}\StringTok{"enzyme12"}\NormalTok{,}\StringTok{"enzyme3"}\NormalTok{,}\StringTok{"tRNA1"}\NormalTok{,}\StringTok{"tRNA205"}\NormalTok{,}
       \StringTok{"mRNA6"}\NormalTok{,}\StringTok{"mRNA17"}\NormalTok{,}\StringTok{"mRNA8"}\NormalTok{,}\StringTok{"mRNA100"}\NormalTok{) }
\end{Highlighting}
\end{Shaded}

  \vspace{0.1in}
\item
  Consider the character vector \texttt{times} below, which has the format: \texttt{day-month-year\ hour:minute:second}.

  \begin{enumerate}
  \def\labelenumii{(\alph{enumii})}
  \item
    Convert \texttt{times} into an object of class \texttt{POSIXlt} called \texttt{time.pos} using the function \texttt{strptime()}.
  \item
    Extract the day of the week from \texttt{time.pos}.
  \item
    Sort \texttt{time.pos} using \texttt{sort()} to verify that \texttt{time.pos} is quantitative.
    \vspace{0.1in}

\begin{Shaded}
\begin{Highlighting}[]
\NormalTok{times }\OtherTok{\textless{}{-}} \FunctionTok{c}\NormalTok{(}\StringTok{"12{-}12{-}2023 12:12:20"}\NormalTok{,}
           \StringTok{"12{-}01{-}2021 01:12:40"}\NormalTok{,}
           \StringTok{"15{-}10{-}2021 23:10:15"}\NormalTok{,}
           \StringTok{"25{-}07{-}2022 13:09:45"}\NormalTok{)}
\end{Highlighting}
\end{Shaded}
  \end{enumerate}
\end{enumerate}

\chapter{Welcome to the Tidyverse}\label{ch5}

\begin{quote}
\emph{``Data is like garbage. You'd better know what you are going to do with it before you collect it.''}
\end{quote}

- \textbf{Mark Twain}, \emph{1835 - 1910}

\section{The Tidyverse}\label{the-tidyverse}

This chapter demonstrates the data management capabilities of the \emph{tidyverse} \citep{tidyverse}. Thus, Chapter 5 can be considered a \emph{tidyverse} reconsideration of Ch \ref{ch4}. The \emph{tidyverse} is currently a collection of eight core packages (Fig \ref{fig:tidyvpackages}). These are:

\begin{itemize}
\tightlist
\item
  \href{https://dplyr.tidyverse.org/}{\emph{dplyr}} Grammar and functions for data manipulation.
\item
  \href{https://forcats.tidyverse.org/}{\emph{forcats}} Tools for solving common problems with factors.
\item
  \href{https://ggplot2.tidyverse.org/}{\emph{ggplot2}} A system for ``declaratively creating graphics'', based on the book \emph{The Grammar of Graphics} \citep{wilkinson2012}.
\item
  \href{https://purrr.tidyverse.org/}{\emph{purrr}} An enhancement of \textbf{R}'s functional programming (FP) toolkit.
\item
  \href{https://readr.tidyverse.org/}{\emph{readr}} Methods for reading rectangular data.
\item
  \href{https://www.tidyverse.org/packages/}{\emph{stringr}} Functions to facilitate working with strings.
\item
  \href{https://tibble.tidyverse.org/}{\emph{tibble}} A ``modern re-imagining of the data frame.''
\item
  \href{https://tidyr.tidyverse.org/}{\emph{tidyr}} A set of functions for ``tidying data.''
\end{itemize}

\begin{figure}[H]

{\centering \includegraphics[width=0.5\linewidth,]{figs5/tidyv} 

}

\caption{The main packages of the \textit{tidyverse}.}\label{fig:tidyvpackages}
\end{figure}

The \emph{tidyverse} library also contains several useful ancillary packages, including \emph{lubridate}, \emph{reshape2}, \emph{hms}, \emph{blob}, \emph{margrittr}, and \emph{glue}. While installing \emph{tidyverse} will result in the installation of both main and ancillary packages, loading the \emph{tidyverse} will result only in the complete loading of the eight main \emph{tidyverse} packages.

Importantly, this chapter is not meant to be an authoritative summary of the \emph{tidyverse}. Coverage here is mostly limited to the core data management packages \emph{magrittr}, \emph{tibble}, \emph{dplyr}, \emph{stringer}, and the ancillary packages \emph{lubridate} and \emph{reshape2}. The \emph{tidyverse} \emph{ggplot2} package is the major focus of Chapter \ref{ch7}. \citet{wickham2023} provides a succinct but thorough introduction to the \emph{tidyverse} in the open source book \href{https://r4ds.hadley.nz/}{\emph{R for Data Science}}. Useful \emph{tidyverse} ``cheatsheets'' can be found \href{https://posit.co/resources/cheatsheets/}{here}.\\

The \emph{tidyverse} packages can be downloaded using:
\vspace{0.1in}

\begin{Shaded}
\begin{Highlighting}[]
\FunctionTok{install.packages}\NormalTok{(}\StringTok{"tidyverse"}\NormalTok{)}
\end{Highlighting}
\end{Shaded}

\index[terms]{R!package!\textit{tidyverse}}
\index[terms]{R!package!\textit{glue}}
\index[terms]{R!package!\textit{ggplot2}}
\index[terms]{R!package!\textit{forcats}}
\index[terms]{R!package!\textit{stringr}}
\index[terms]{R!package!\textit{dplyr}}
\index[terms]{R!package!\textit{tibble}}
\index[terms]{R!package!\textit{tidyr}}
\index[terms]{R!package!\textit{purrr}}
\index[terms]{R!package!\textit{readr}}
\index[terms]{R!package!\textit{lubridate}}
\index[terms]{R!package!\textit{glue}}
\index[terms]{R!package!\textit{blob}}
\index[terms]{R!package!\textit{margrittr}}

\section{Pipes}\label{pipes}

An important convention of the \emph{tidyverse} is the widespread use of the forward \emph{pipe operator}: \texttt{\textbar{}\textgreater{}} \index[R]{\texttt{margrittr::\%>\%} (pipe operator)}\index[R]{\texttt{\|\textgreater} (pipe operator)}. In programming, a pipe\index[terms]{Pipe} is a set of commands connected in series, where the output of one command is the input of the next\footnote{Pipe programming dates back to early developments in Unix operating systems \citep{ritchie1984, bellunix}, wherein pipes are codified as vertical bars \texttt{"\textbar{}"}. Along with Unix/Linux, pipes are widely used in the languages F\#, Julia, and JavaScript, among others.
  \index[terms]{F\# (programming language)}
  \index[terms]{Java (programming language)}
  \index[terms]{Julia (programming language)}}.
In many cases, use of pipes allows clearer representations of coding processes\footnote{In particular, when you see \texttt{\textbar{}\textgreater{}} it is helpful to think ``and then''.}. Incidentally, the \texttt{\textbar{}\textgreater{}} pipe, from the \emph{base} package, is motivated by an older forward pipe operator from the \emph{tidyverse} package \emph{magrittr}, \texttt{\%\textgreater{}\%}. As of \textbf{R} 4.1, the native pipe operator for the \emph{tidyverse} is \texttt{\textbar{}\textgreater{}} (although \texttt{\%\textgreater{}\%} will still work if \emph{magrittr} is loaded)\footnote{The RStudio shortcut for \texttt{\%\textgreater{}\%} is \textbf{Ctrl\(+\)Shift\(+\)m}. To force RStudio to default to \texttt{\textbar{}\textgreater{}} when using \textbf{Ctrl\(+\)Shift\(+\)m} (or some other keyboard shortcut), one can modify appropriate settings in \textbf{Tools\(>\)Global Options\(>\)Code}.}. Notably, while \texttt{\textbar{}\textgreater{}} is more syntactically (and algorithmically) streamlined than \texttt{\%\textgreater{}\%}, there are several features available to \texttt{\%\textgreater{}\%} that do not exist for \texttt{\textbar{}\textgreater{}}, including the potential for a placeholder operator\footnote{In general, the dot placeholder operator, \texttt{.}, from \emph{magrittr} allows operations like \(f(x,y)\) by specifying \texttt{x\ \textbar{}\textgreater{}\ f(.,y)}.\index[R]{\texttt{tidyverse::.} (dot operator)} For example: \texttt{2\ \%\textgreater{}\%\ log(10,\ base\ =\ .)}. In this script the number 2 will be piped into the \texttt{base} argument in the function \texttt{log()}.}. Nonetheless, I focus on \texttt{\textbar{}\textgreater{}}, not \texttt{\%\textgreater{}\%}, here.
\index[terms]{R!package!\textit{margrittr}}

\begin{example}
\protect\hypertarget{exm:exm-51}{}\label{exm:exm-51}

\(\text{}\)\\
Consider the circular operation: \(\log_e(\exp(1))\). We could write this as,
\vspace{0.1in}

\begin{Shaded}
\begin{Highlighting}[]
\DecValTok{1} \SpecialCharTok{|\textgreater{}} \FunctionTok{exp}\NormalTok{() }\SpecialCharTok{|\textgreater{}} \FunctionTok{log}\NormalTok{()}
\end{Highlighting}
\end{Shaded}

\begin{verbatim}
[1] 1
\end{verbatim}

Here the number \(1\) is piped into the function \texttt{exp()}, with the result: \(\exp(1) = e^1 = e\), and this outcome is piped into the function \texttt{log()}, with the result: \(\log_e e = 1\). Because the first arguments of \texttt{exp()} and \texttt{log()} are simply calls to numeric data, and these are provided by the previous pipe segment, we do not have to include information about \texttt{x} for \texttt{f(x)} operations. Thus, when functions require only the previous pipe segment result as an argument, then \texttt{x\ \textbar{}\textgreater{}\ f()} is equivalent to \(f(x)\)\footnote{The \texttt{\%\textgreater{}\%} forward pipe does not even require the \texttt{()} \emph{no argument} designation. That is, \texttt{x\ \%\textgreater{}\%\ f} is equivalent to \(f(x)\).}. In the case that multiple arguments need to be specified, the script \texttt{x\ \textbar{}\textgreater{}\ f(y)} is equivalent to \(f(x, y)\), and \texttt{x\ \textbar{}\textgreater{}\ f(y)\ \textbar{}\textgreater{}\ g(z)} describes \(g(f(x, y), z)\). For instance,

\vspace{0.1in}

\begin{Shaded}
\begin{Highlighting}[]
\DecValTok{10} \SpecialCharTok{|\textgreater{}} \FunctionTok{log}\NormalTok{(}\AttributeTok{base =} \DecValTok{2}\NormalTok{)}
\end{Highlighting}
\end{Shaded}

\begin{verbatim}
[1] 3.3219
\end{verbatim}

\end{example}

\(\blacksquare\)
\index[R]{\texttt{log()}}
\index[R]{\texttt{exp()}}

\begin{example}
\protect\hypertarget{exm:exm-52}{}\label{exm:exm-52}

\(\text{}\)\\
This example illustrates that the forward pipe works recursively from the result of the previous pipe segment.

\vspace{0.1in}

\begin{Shaded}
\begin{Highlighting}[]
\FunctionTok{head}\NormalTok{(Loblolly) }\CommentTok{\# First 6 rows of data}
\end{Highlighting}
\end{Shaded}

\begin{verbatim}
Grouped Data: height ~ age | Seed
   height age Seed
1    4.51   3  301
15  10.89   5  301
29  28.72  10  301
43  41.74  15  301
57  52.70  20  301
71  60.92  25  301
\end{verbatim}

\begin{Shaded}
\begin{Highlighting}[]
\NormalTok{Loblolly }\SpecialCharTok{|\textgreater{}}
\FunctionTok{head}\NormalTok{() }\SpecialCharTok{|\textgreater{}}
\FunctionTok{tail}\NormalTok{(}\DecValTok{2}\NormalTok{) }\CommentTok{\# Last 2 rows from first 6 rows}
\end{Highlighting}
\end{Shaded}

\begin{verbatim}
Grouped Data: height ~ age | Seed
   height age Seed
57  52.70  20  301
71  60.92  25  301
\end{verbatim}

\end{example}

\(\blacksquare\)

\index[R]{\texttt{head()}}
\index[R]{\texttt{tail()}}

\begin{example}
\protect\hypertarget{exm:exm-53}{}\label{exm:exm-53}

\(\text{}\)\\
We can define the result of a pipe to be a global variable. Consider the script below (Fig \ref{fig:sinx}).

\vspace{0.1in}

\begin{Shaded}
\begin{Highlighting}[]
\NormalTok{x }\OtherTok{\textless{}{-}} \FunctionTok{seq}\NormalTok{(}\DecValTok{1}\NormalTok{,}\DecValTok{10}\NormalTok{,}\AttributeTok{length=}\DecValTok{100}\NormalTok{)}
\NormalTok{y }\OtherTok{\textless{}{-}}\NormalTok{ x }\SpecialCharTok{|\textgreater{}} \FunctionTok{sin}\NormalTok{()}
\FunctionTok{plot}\NormalTok{(x, y, }\AttributeTok{type =} \StringTok{"l"}\NormalTok{, }\AttributeTok{ylab =} \StringTok{"sin(x)"}\NormalTok{, }\AttributeTok{xlab =} \StringTok{"x (radians)"}\NormalTok{)}
\end{Highlighting}
\end{Shaded}

\begin{figure}[H]

{\centering \includegraphics{figure/minimal-sinx-1} 

}

\caption{Creating a global variable (object) resulting from a pipeline.}\label{fig:sinx}
\end{figure}

\end{example}

\(\blacksquare\)
\index[R]{\texttt{sin()}}

\subsection{Other Pipes}\label{other-pipes}

It is worth noting that, in addition to \texttt{\%\textgreater{}\%}, \emph{magrittr} contains several other potentially useful pipe operators. These include the \emph{assignment pipe} and the \emph{tee pipe}. The assignment pipe operator, \texttt{\%\textless{}\textgreater{}\%}, will pipe \texttt{x} into one or more \texttt{f(x)} expressions, and then assign the result to the name \texttt{x}\footnote{For instance, \texttt{library(magrittr);\ x\ \textless{}-\ -4:4;\ x\ \ \%\textless{}\textgreater{}\%\ abs\ \%\textgreater{}\%\ sort;\ x} would print the pipe-modified version of \texttt{x}.}. The tee pipe operator \texttt{\%T\textgreater{}\%} works like \texttt{\%\textgreater{}\%}, except the return value in \texttt{x\ \%T\textgreater{}\%\ f(x)} is \texttt{x} itself. This is useful when a pipeline requires a side-effect like plotting or printing\footnote{For instance, \texttt{rnorm(20)\ \textbar{}\textgreater{}\ matrix(ncol\ =\ 2)\ \%T\textgreater{}\%\ plot\ \textbar{}\textgreater{}\ colSums}. In this case a plot and the sums of columns will both be printed (see Example \ref{exm:exm-59}).}.

\index[R]{\texttt{magrittr::\%<>\%} (assignment pipe)}
\index[R]{\texttt{magrittr::\%T>\%} (tee pipe)}

\section{\texorpdfstring{\emph{tibble}}{tibble}}\label{tibble}

\index[terms]{R!package!\textit{tibble}}
The \emph{tidyverse} package \emph{tibble} provides an alternative to the \texttt{data.frame} format of data storage, called a tibble\index[terms]{R!package!*tibble*!tibble}.\index[terms]{R!package!*tibble*!tibble} Tibbles have classes \texttt{dataframe} and \texttt{tbl\_df}, allowing them to posses additional characteristics including enhanced printing (see Example \ref{exm:exm-54} below).
Additional distinguishing characteristics of tibbles include: 1) a character vector is not automatically coerced to have class \texttt{factor}, 2) recycling (see Section \ref{vectors}) only occurs for an input of length one, and 3) there is no partial matching when \texttt{\$} is used to index by tibble columns by name\footnote{According to package \emph{tibble}: ``\(\ldots\)tibbles are lazy and surly: they do less and complain more than base dataframes. This forces problems to be tackled earlier and more explicitly, typically leading to code that is more expressive and robust.''}. The functions \texttt{tibble()} generates tibbles. The function \texttt{as\_tibble()} coerces a dataframe to be a tibble.
\index[R]{\texttt{tibble::tibble()}}
\index[R]{\texttt{tibble::as\_tibble()}}

\begin{example}
\protect\hypertarget{exm:exm-54}{}\label{exm:exm-54}

\(\text{}\)\\
Here we compare dataframe and tibble output of the same data.

\vspace{0.1in}

\begin{Shaded}
\begin{Highlighting}[]
\NormalTok{data }\OtherTok{\textless{}{-}} \FunctionTok{data.frame}\NormalTok{(}\AttributeTok{numbers =} \DecValTok{1}\SpecialCharTok{:}\DecValTok{3}\NormalTok{, }\AttributeTok{letters =} \FunctionTok{c}\NormalTok{(}\StringTok{"a"}\NormalTok{,}\StringTok{"b"}\NormalTok{,}\StringTok{"c"}\NormalTok{),}
                   \AttributeTok{date =} \FunctionTok{as.Date}\NormalTok{(}\FunctionTok{c}\NormalTok{(}\StringTok{"2021{-}12{-}1"}\NormalTok{, }\StringTok{"2021{-}12{-}2"}\NormalTok{,}
                                    \StringTok{"2021{-}12{-}2"}\NormalTok{),}
                           \AttributeTok{format =} \StringTok{"\%Y{-}\%m{-}\%d"}\NormalTok{))}
\NormalTok{data}
\end{Highlighting}
\end{Shaded}

\begin{verbatim}
  numbers letters       date
1       1       a 2021-12-01
2       2       b 2021-12-02
3       3       c 2021-12-02
\end{verbatim}

\vspace{0.1in}

\begin{Shaded}
\begin{Highlighting}[]
\FunctionTok{library}\NormalTok{(tidyverse)}
\NormalTok{datat }\OtherTok{\textless{}{-}} \FunctionTok{as\_tibble}\NormalTok{(data)}
\NormalTok{datat}
\end{Highlighting}
\end{Shaded}

\begin{verbatim}
# A tibble: 3 x 3
  numbers letters date      
    <int> <chr>   <date>    
1       1 a       2021-12-01
2       2 b       2021-12-02
3       3 c       2021-12-02
\end{verbatim}

\end{example}

\(\blacksquare\)

\section{\texorpdfstring{\emph{dplyr}}{dplyr}}\label{dplyr}

\index[terms]{R!package!\textit{dplyr}}
The \emph{dplyr} package contains a collection of core \emph{tidyverse} algorithms for data manipulation\footnote{\emph{dplyr} has largely replaced the now retired \emph{plyr} package.}. Table \ref{tab:dplyr} lists some useful \emph{dplyr} functions.

\index[terms]{R!package!\textit{dplyr}}

\vspace{0.1in}

\begin{table}[ht]
\centering
\caption{Important \textit{dplyr} data management functions.} 
\label{tab:dplyr}
\begin{tabular}{ll}
  \toprule
Function & Usage \\ 
  \midrule
\texttt{summarise()}  & Numerical summaries of variables.   \\ 
  \texttt{group\_by}  & Group a dataframe by a categorical variable.   \\ 
  \texttt{filter()}  & Subset variables based on outcomes.   \\ 
  \texttt{arrange()}  & Reorder rows in a dataframe or tibble.   \\ 
  \texttt{mutate()}   & Creates new variables from functions of existing variables.  \\ 
  \texttt{select()}  & Selects variables from tibbles or dataframes.   \\ 
   \bottomrule
\end{tabular}
\end{table}

\index[R]{\texttt{dplyr::summarise()}}
\index[R]{\texttt{dplyr::mutate()}}
\index[R]{\texttt{dplyr::filter()}}
\index[R]{\texttt{dpylr::select()}}
\index[R]{\texttt{dplyr::arrange()}}
\index[R]{\texttt{dplyr::group\_by()}}

\subsection{\texorpdfstring{\texttt{summarize()}}{summarize()}}\label{summarize}

\index[R]{\texttt{dplyr::summarise()}}
The function \texttt{summarize()}, or equivalently \texttt{summarise()}, creates a new data frame with one row for each combination of specified grouping variables. If no groups are given (for instance, in the case that \texttt{group\_by} is \emph{not} used to group data), dataframe rows will be summaries of all observations in the required input \texttt{.data} argument.

\begin{example}
\protect\hypertarget{exm:exm-55}{}\label{exm:exm-55}

\(\text{}\)\\
Here we use \texttt{summarize()} to obtain means for loblolly pine height (in feet) and age (in years).

\vspace{0.1in}

\begin{Shaded}
\begin{Highlighting}[]
\NormalTok{Loblolly }\SpecialCharTok{|\textgreater{}}
  \FunctionTok{summarise}\NormalTok{(}\AttributeTok{mean.height.ft =} \FunctionTok{mean}\NormalTok{(height), }\AttributeTok{mean.age.yrs =} \FunctionTok{mean}\NormalTok{(age))}
\end{Highlighting}
\end{Shaded}

\begin{verbatim}
  mean.height.ft mean.age.yrs
1         32.364           13
\end{verbatim}

\end{example}

\(\blacksquare\)

\subsection{\texorpdfstring{\texttt{group\_by()}}{group\_by()}}\label{group_by}

\index[R]{\texttt{dplyr::group\_by()}}
The \texttt{group\_by()} function is often used in conjunction with other \emph{dplyr} functions, including \texttt{summarize()}, to provide an underlying grouping framework for data summaries.

\begin{example}
\protect\hypertarget{exm:exm-55a}{}\label{exm:exm-55a}

\(\text{}\)\\
Here we use \texttt{group\_by()} with \texttt{summarize()} to describe the \texttt{Loblolly} height data. Specifically, we will take the mean and the variance of \texttt{Loblolly\$height} with respect to categories specified in \texttt{group\_by()}.

\vspace{0.1in}

\begin{Shaded}
\begin{Highlighting}[]
\NormalTok{Loblolly }\SpecialCharTok{|\textgreater{}}
  \FunctionTok{group\_by}\NormalTok{(Seed) }\SpecialCharTok{|\textgreater{}}
  \FunctionTok{summarise}\NormalTok{(}\AttributeTok{mean.height.ft =} \FunctionTok{mean}\NormalTok{(height),}
            \AttributeTok{var.height.ft2 =} \FunctionTok{var}\NormalTok{(height)}
\NormalTok{            ) }\SpecialCharTok{|\textgreater{}}
  \FunctionTok{head}\NormalTok{(}\DecValTok{5}\NormalTok{)}
\end{Highlighting}
\end{Shaded}

\begin{verbatim}
# A tibble: 5 x 3
  Seed  mean.height.ft var.height.ft2
  <ord>          <dbl>          <dbl>
1 329             30.3           443.
2 327             30.6           440.
3 325             31.9           468.
4 307             31.3           494.
5 331             31.0           495.
\end{verbatim}

More than only grouping variable can be specified in \texttt{group\_by()}:

\begin{Shaded}
\begin{Highlighting}[]
\NormalTok{Loblolly }\SpecialCharTok{|\textgreater{}}
  \FunctionTok{group\_by}\NormalTok{(Seed, age) }\SpecialCharTok{|\textgreater{}}
  \FunctionTok{summarise}\NormalTok{(}\AttributeTok{mean.height.ft =} \FunctionTok{mean}\NormalTok{(height),}
            \AttributeTok{var.height.ft2 =} \FunctionTok{var}\NormalTok{(height)}
\NormalTok{            ) }\SpecialCharTok{|\textgreater{}}
  \FunctionTok{head}\NormalTok{(}\DecValTok{5}\NormalTok{)}
\end{Highlighting}
\end{Shaded}

\begin{verbatim}
`summarise()` has grouped output by 'Seed'. You can override using the
`.groups` argument.
\end{verbatim}

\begin{verbatim}
# A tibble: 5 x 4
# Groups:   Seed [1]
  Seed    age mean.height.ft var.height.ft2
  <ord> <dbl>          <dbl>          <dbl>
1 329       3           3.93             NA
2 329       5           9.34             NA
3 329      10          26.1              NA
4 329      15          37.8              NA
5 329      20          48.3              NA
\end{verbatim}

Clearly, \texttt{group\_by()} and \texttt{summarise()} allow more options than the base function \texttt{tapply()} (Section \ref{tapply}). The latter function only provides summaries of groups within a single categorical \texttt{INDEX}, with respect to a single quantitative vector, and a single user-defined function.

Starting with \emph{dplyr} 1.1.0, we can use the \texttt{.by} argument in \texttt{summarize} to bypass \texttt{group\_by()}, although this argument is experimental, and may be deprecated in the future (see \texttt{?summarise}).

\vspace{0.1in}

\begin{Shaded}
\begin{Highlighting}[]
\NormalTok{Loblolly }\SpecialCharTok{|\textgreater{}}
  \FunctionTok{summarise}\NormalTok{(}\AttributeTok{mean.height.ft =} \FunctionTok{mean}\NormalTok{(height),}
            \AttributeTok{var.height.ft2 =} \FunctionTok{var}\NormalTok{(height),}
            \AttributeTok{.by =}\NormalTok{ Seed) }\SpecialCharTok{|\textgreater{}}
  \FunctionTok{head}\NormalTok{(}\DecValTok{5}\NormalTok{)}
\end{Highlighting}
\end{Shaded}

\begin{verbatim}
  Seed mean.height.ft var.height.ft2
1  301         33.247         512.50
2  303         34.107         552.24
3  305         35.115         572.51
4  307         31.328         493.83
5  309         33.782         535.12
\end{verbatim}

\end{example}

\(\blacksquare\)

\subsection{\texorpdfstring{\texttt{filter()}}{filter()}}\label{filter}

\index[R]{\texttt{dplyr::filter()}}
The function \texttt{filter()} provides a straightforward way to extract dataframe rows based on Boolean operators.

\begin{example}
\protect\hypertarget{exm:exm-56}{}\label{exm:exm-56}

\(\text{}\)\\
Here we obtain rows in \texttt{Loblolly} associated with seed type 301.

\vspace{0.1in}

\begin{Shaded}
\begin{Highlighting}[]
\NormalTok{Loblolly }\SpecialCharTok{|\textgreater{}}
  \FunctionTok{filter}\NormalTok{(Seed }\SpecialCharTok{==} \StringTok{"301"}\NormalTok{)}
\end{Highlighting}
\end{Shaded}

\begin{verbatim}
Grouped Data: height ~ age | Seed
   height age Seed
1    4.51   3  301
15  10.89   5  301
29  28.72  10  301
43  41.74  15  301
57  52.70  20  301
71  60.92  25  301
\end{verbatim}

Here are \texttt{Loblolly} rows associated with \texttt{height} responses greater than 60 feet.

\vspace{0.1in}

\begin{Shaded}
\begin{Highlighting}[]
\NormalTok{Loblolly }\SpecialCharTok{|\textgreater{}}
  \FunctionTok{filter}\NormalTok{(height }\SpecialCharTok{\textgreater{}} \DecValTok{60}\NormalTok{)}
\end{Highlighting}
\end{Shaded}

\begin{verbatim}
Grouped Data: height ~ age | Seed
   height age Seed
71  60.92  25  301
72  63.39  25  303
73  64.10  25  305
75  63.05  25  309
77  60.07  25  315
78  60.69  25  319
79  60.28  25  321
80  61.62  25  323
\end{verbatim}

\end{example}

\(\blacksquare\)

\subsection{\texorpdfstring{\texttt{arrange()}}{arrange()}}\label{arrange}

\index[R]{\texttt{dplyr::arrange()}}
The function \texttt{arrange()} orders the rows of a data frame based on the alphanumeric ordering of specified data.

\begin{example}
\protect\hypertarget{exm:exm-57}{}\label{exm:exm-57}

\(\text{}\)\\
Here we use \texttt{arrange()} to sort the result from the previous chunk from smallest to largest loblolly pine heights.

\vspace{0.1in}

\begin{Shaded}
\begin{Highlighting}[]
\NormalTok{Loblolly }\SpecialCharTok{|\textgreater{}}
  \FunctionTok{filter}\NormalTok{(height }\SpecialCharTok{\textgreater{}} \DecValTok{60}\NormalTok{) }\SpecialCharTok{|\textgreater{}}
  \FunctionTok{arrange}\NormalTok{(height)}
\end{Highlighting}
\end{Shaded}

\begin{verbatim}
Grouped Data: height ~ age | Seed
   height age Seed
77  60.07  25  315
79  60.28  25  321
78  60.69  25  319
71  60.92  25  301
80  61.62  25  323
75  63.05  25  309
72  63.39  25  303
73  64.10  25  305
\end{verbatim}

One can use \texttt{arrange(desc())} to sort a dataframe in descending (largest-to-smallest) order.

\vspace{0.1in}

\begin{Shaded}
\begin{Highlighting}[]
\NormalTok{Loblolly }\SpecialCharTok{|\textgreater{}}
  \FunctionTok{filter}\NormalTok{(height }\SpecialCharTok{\textgreater{}} \DecValTok{60}\NormalTok{) }\SpecialCharTok{|\textgreater{}}
  \FunctionTok{arrange}\NormalTok{(}\FunctionTok{desc}\NormalTok{(height))}
\end{Highlighting}
\end{Shaded}

\begin{verbatim}
Grouped Data: height ~ age | Seed
   height age Seed
73  64.10  25  305
72  63.39  25  303
75  63.05  25  309
80  61.62  25  323
71  60.92  25  301
78  60.69  25  319
79  60.28  25  321
77  60.07  25  315
\end{verbatim}

\end{example}

\(\blacksquare\)
\index[R]{\texttt{dplyr::desc()}}

\subsection{\texorpdfstring{\texttt{slice\_min()} and \texttt{slice\_max()}}{slice\_min() and slice\_max()}}\label{slice_min-and-slice_max}

\index[R]{\texttt{dplyr::slice\_min()}}
\index[R]{\texttt{dplyr::slice\_max()}}
The helpful \emph{dplyr} functions \texttt{slice\_min()} and \texttt{slice\_max()} allow subsetting of dataframe rows by minimum and maximum values in some column, respectively.

\begin{example}
\protect\hypertarget{exm:exm-58}{}\label{exm:exm-58}

\(\text{}\)\\

\vspace{0.1in}

\begin{Shaded}
\begin{Highlighting}[]
\NormalTok{Loblolly }\SpecialCharTok{|\textgreater{}}
  \FunctionTok{slice\_max}\NormalTok{(height, }\AttributeTok{n =} \DecValTok{5}\NormalTok{)}
\end{Highlighting}
\end{Shaded}

\begin{verbatim}
Grouped Data: height ~ age | Seed
   height age Seed
73  64.10  25  305
72  63.39  25  303
75  63.05  25  309
80  61.62  25  323
71  60.92  25  301
\end{verbatim}

\end{example}

\(\blacksquare\)

\subsection{\texorpdfstring{\texttt{select()}}{select()}}\label{select}

\index[R]{\texttt{dplyr::select()}}
The \texttt{select()} function allows one to select particular variables in a data frame.

\begin{example}
\protect\hypertarget{exm:exm-58a}{}\label{exm:exm-58a}

\(\text{}\)\\
For instance, here I select \texttt{height} from \texttt{Loblolly}.

\vspace{0.1in}

\begin{Shaded}
\begin{Highlighting}[]
\NormalTok{Loblolly }\SpecialCharTok{|\textgreater{}} 
  \FunctionTok{select}\NormalTok{(height) }\SpecialCharTok{|\textgreater{}}
  \FunctionTok{head}\NormalTok{()}
\end{Highlighting}
\end{Shaded}

\begin{verbatim}
   height
1    4.51
15  10.89
29  28.72
43  41.74
57  52.70
71  60.92
\end{verbatim}

\end{example}

\(\blacksquare\)

The \texttt{select()} function can be used in more sophisticated ways by combining it with other \emph{dplyr} functions like \texttt{starts\_with()} and \texttt{ends\_with()}, or other Boolean operators. \index[R]{\texttt{dplyr::starts\_with()}}
\index[R]{\texttt{dplyr::ends\_with()}}

\begin{example}
\protect\hypertarget{exm:exm-58b}{}\label{exm:exm-58b}

\(\text{}\)\\
He we select the \texttt{height} and \texttt{age} columns by calling for variable names that start with \texttt{"h"} or end with \texttt{"e"}.

\vspace{0.1in}

\begin{Shaded}
\begin{Highlighting}[]
\NormalTok{Loblolly }\SpecialCharTok{|\textgreater{}}
  \FunctionTok{select}\NormalTok{(}\FunctionTok{starts\_with}\NormalTok{(}\StringTok{"h"}\NormalTok{), }\FunctionTok{ends\_with}\NormalTok{(}\StringTok{"e"}\NormalTok{)) }\SpecialCharTok{|\textgreater{}}
  \FunctionTok{head}\NormalTok{(}\DecValTok{3}\NormalTok{)}
\end{Highlighting}
\end{Shaded}

\begin{verbatim}
   height age
1    4.51   3
15  10.89   5
29  28.72  10
\end{verbatim}

\end{example}

\(\blacksquare\)

\subsection{\texorpdfstring{\texttt{mutate()}}{mutate()}}\label{mutate}

\index[R]{\texttt{dplyr::mutate()}}
The function \texttt{mutate()} creates new dataframe columns that are functions of existing variables.

\begin{example}
\protect\hypertarget{exm:exm-59}{}\label{exm:exm-59}

\(\text{}\)\\
Below we select the \texttt{age} and \texttt{height} columns using \texttt{select()}, convert height in feet to height in meters using \texttt{mutate()}, plot the result as a side-task using the tee pipe, \texttt{\%T\textgreater{}\%} (note the use of the \texttt{.} placeholder operator) (Fig \ref{fig:mutate}), and then take the column means of \texttt{age} and \texttt{height}. Note that by default, all columns from the previous pipe segment will be in the \texttt{mutate()} output although all columns need not be explicitly mutated. Output columns can be specified using the \texttt{mutate()} argument \texttt{.keep}.

\vspace{0.1in}

\begin{Shaded}
\begin{Highlighting}[]
\FunctionTok{library}\NormalTok{(magrittr) }\CommentTok{\# to access tee pipe}

\NormalTok{Loblolly }\SpecialCharTok{|\textgreater{}}
  \FunctionTok{select}\NormalTok{(}\FunctionTok{c}\NormalTok{(age, height)) }\SpecialCharTok{|\textgreater{}}
  \FunctionTok{mutate}\NormalTok{(}\AttributeTok{height =}\NormalTok{ height }\SpecialCharTok{*} \FloatTok{0.3048}\NormalTok{) }\SpecialCharTok{\%T\textgreater{}\%}
  \FunctionTok{plot}\NormalTok{(.,}\AttributeTok{ylab =} \StringTok{"Height (m)"}\NormalTok{, }\AttributeTok{xlab =} \StringTok{"Age (yrs)"}\NormalTok{) }\SpecialCharTok{|\textgreater{}}
  \FunctionTok{colMeans}\NormalTok{()}
\end{Highlighting}
\end{Shaded}

\begin{verbatim}
    age  height 
13.0000  9.8647 
\end{verbatim}

\begin{figure}[H]

{\centering \includegraphics{figure/minimal-mutate-1} 

}

\caption{Plot of loblolly pine height as a function of age, after converting height to meters. In base \textbf{R} dialect we could use: ``\texttt{with(Loblolly, plot(age, height * 0.3048, ylab = "Height (m)", xlab = "Age (yrs)"))}. This is quite a bit harder to decipher.}\label{fig:mutate}
\end{figure}

\end{example}

\(\blacksquare\)
\index[R]{\texttt{magrittr::\%T>\%} (tee pipe)}

\subsection{\texorpdfstring{\texttt{across()}}{across()}}\label{across}

The \emph{dplyr} function \texttt{across()} allows extensions similar to those in \texttt{apply()}\index[R]{\texttt{apply()}} wherein the same function can be applied to all columns in the first argument of \texttt{across()}. Specifying the first argument in \texttt{across()} as \texttt{everything()} would allow application of a function to all columns in a dataframe.

\begin{example}
\protect\hypertarget{exm:exm-510}{}\label{exm:exm-510}

\(\text{}\)\\
Here we take the medians of the quantitative columns in \texttt{Loblolly} using \texttt{across()} and \texttt{summarize()}.

\vspace{0.1in}

\begin{Shaded}
\begin{Highlighting}[]
\NormalTok{Loblolly }\SpecialCharTok{|\textgreater{}}
   \FunctionTok{summarize}\NormalTok{(}\FunctionTok{across}\NormalTok{(}\FunctionTok{c}\NormalTok{(age, height), median))}
\end{Highlighting}
\end{Shaded}

\begin{verbatim}
   age height
1 12.5     34
\end{verbatim}

\end{example}

\(\blacksquare\)

\section{\texorpdfstring{\emph{stringr}}{stringr}}\label{stringr}

\index[terms]{R!character vector}
\index[terms]{R!character string}
As evident in Section \ref{regex}, use of regular expressions for matching, querying and substituting strings can be confusing. The \emph{stringr} package attempts to simplify some of these difficulties. The \emph{stringr} package uses processing tools from the package \emph{stringi} \citep{stringi} for pattern searching under a wide array of potential approaches. All \emph{stringr} functions have the prefix \texttt{str\_} and take a character string or string vector as the first argument.

Consider the vector of plant scientific names used to demonstrate string management in Section \ref{regex}.

\vspace{0.1in}

\begin{Shaded}
\begin{Highlighting}[]
\NormalTok{names }\OtherTok{=} \FunctionTok{c}\NormalTok{(}\StringTok{"Achillea millefolium"}\NormalTok{, }\StringTok{"Aster foliaceus"}\NormalTok{,}
                           \StringTok{"Elymus scribneri"}\NormalTok{, }\StringTok{"Erigeron rydbergii"}\NormalTok{,}
                           \StringTok{"Carex elynoides"}\NormalTok{, }\StringTok{"Carex paysonis"}\NormalTok{,}
                           \StringTok{"Taraxacum ceratophorum"}\NormalTok{)}
\end{Highlighting}
\end{Shaded}

\begin{example}
\protect\hypertarget{exm:exm-5str1}{}\label{exm:exm-5str1}

\(\text{}\)\\
The function \texttt{str\_length()} can be used to count the number of characters in a string.

\vspace{0.1in}

\begin{Shaded}
\begin{Highlighting}[]
\FunctionTok{str\_length}\NormalTok{(names)}
\end{Highlighting}
\end{Shaded}

\begin{verbatim}
[1] 20 15 16 18 15 14 22
\end{verbatim}

\end{example}

\(\blacksquare\)
\index[R]{\texttt{stringr::str\_length()}}

\begin{example}
\protect\hypertarget{exm:exm-5str0}{}\label{exm:exm-5str0}

\(\text{}\)\\
The function \texttt{str\_detect()} tests for the presence or absence of a pattern in a string. Here I test for presence of the genus \emph{Aster}.

\vspace{0.1in}

\begin{Shaded}
\begin{Highlighting}[]
\FunctionTok{str\_detect}\NormalTok{(names, }\StringTok{"Aster"}\NormalTok{)}
\end{Highlighting}
\end{Shaded}

\begin{verbatim}
[1] FALSE  TRUE FALSE FALSE FALSE FALSE FALSE
\end{verbatim}

Here are entries \emph{not} containing \emph{Aster}.

\vspace{0.1in}

\begin{Shaded}
\begin{Highlighting}[]
\FunctionTok{str\_detect}\NormalTok{(names, }\StringTok{"Aster"}\NormalTok{, }\AttributeTok{negate =} \ConstantTok{TRUE}\NormalTok{)}
\end{Highlighting}
\end{Shaded}

\begin{verbatim}
[1]  TRUE FALSE  TRUE  TRUE  TRUE  TRUE  TRUE
\end{verbatim}

\end{example}

\(\blacksquare\)
\index[R]{\texttt{stringr::str\_replace()}}

\begin{example}
\protect\hypertarget{exm:exm-5str2}{}\label{exm:exm-5str2}

\(\text{}\)\\
Here we subset \texttt{names} using the function \texttt{stringr::str\_subset()} to obtain species within the genus \emph{Carex}.

\vspace{0.1in}

\begin{Shaded}
\begin{Highlighting}[]
\FunctionTok{str\_subset}\NormalTok{(names, }\StringTok{"Carex"}\NormalTok{)}
\end{Highlighting}
\end{Shaded}

\begin{verbatim}
[1] "Carex elynoides" "Carex paysonis" 
\end{verbatim}

\end{example}

\(\blacksquare\)

\begin{example}
\protect\hypertarget{exm:exm-5str05}{}\label{exm:exm-5str05}

\(\text{}\)\\
The function \texttt{str\_replace()} is analogous to the base \textbf{R} function \texttt{gsub()}. It can be used to replace text based on a pattern.

\vspace{0.1in}

\begin{Shaded}
\begin{Highlighting}[]
\FunctionTok{str\_replace}\NormalTok{(names, }\StringTok{"Carex"}\NormalTok{, }\StringTok{"C."}\NormalTok{)}
\end{Highlighting}
\end{Shaded}

\begin{verbatim}
[1] "Achillea millefolium"   "Aster foliaceus"       
[3] "Elymus scribneri"       "Erigeron rydbergii"    
[5] "C. elynoides"           "C. paysonis"           
[7] "Taraxacum ceratophorum"
\end{verbatim}

\end{example}

\(\blacksquare\)
\index[R]{\texttt{stringr::str\_replace()}}

\index[R]{\texttt{stringr::str\_subset()}}
Most \emph{stringr} functions work with regular expressions (Section \ref{rgx}).

\begin{example}
\protect\hypertarget{exm:exm-5str3}{}\label{exm:exm-5str3}

\(\text{}\)\\
Here we count upper and lower case vowels with the function \texttt{stringr::str\_count()} using a pattern defined by the regex character class \texttt{{[}AEIOUaeiou{]}}.

\vspace{0.1in}

\begin{Shaded}
\begin{Highlighting}[]
\FunctionTok{str\_count}\NormalTok{(names, }\StringTok{"[AEIOUaeiou]"}\NormalTok{)}
\end{Highlighting}
\end{Shaded}

\begin{verbatim}
[1] 9 7 5 7 6 5 9
\end{verbatim}

and use \texttt{stringr::str\_extract()} to extract strings nine alphanumeric characters long, and then sort the strings with a pipe.

\vspace{0.1in}

\begin{Shaded}
\begin{Highlighting}[]
\FunctionTok{str\_extract}\NormalTok{(names, }\StringTok{"[[:alnum:]]\{9\}"}\NormalTok{) }\SpecialCharTok{|\textgreater{}}
  \FunctionTok{sort}\NormalTok{()}
\end{Highlighting}
\end{Shaded}

\begin{verbatim}
[1] "elynoides" "foliaceus" "millefoli" "rydbergii" "scribneri" "Taraxacum"
\end{verbatim}

\end{example}

\(\blacksquare\)
\index[R]{\texttt{stringr::str\_extract()}}

\section{\texorpdfstring{\emph{lubridate}}{lubridate}}\label{lubridate}

\index[terms]{R!package!\textit{lubridate}}
Base \textbf{R} approaches for handling date-time data are described in Section \ref{datetime}. The package \emph{lubridate} (\url{https://lubridate.tidyverse.org/}) contains functions for simplifying and extending some of these operations.

\begin{example}
\protect\hypertarget{exm:exm-5lub1}{}\label{exm:exm-5lub1}

\(\text{}\)\\
As an example dataset, I will use the time series used to illustrate date-time classes in Section \ref{datetime}.

\vspace{0.1in}

\begin{Shaded}
\begin{Highlighting}[]
\NormalTok{dates }\OtherTok{\textless{}{-}} \FunctionTok{c}\NormalTok{(}\StringTok{"08/13/2019 04:00"}\NormalTok{, }\StringTok{"08/13/2019 06:30"}\NormalTok{, }\StringTok{"08/13/2019 09:00"}\NormalTok{,}
           \StringTok{"08/13/2019 11:30"}\NormalTok{, }\StringTok{"08/13/2019 14:00"}\NormalTok{, }\StringTok{"08/13/2019 16:30"}\NormalTok{,}
           \StringTok{"08/13/2019 19:00"}\NormalTok{, }\StringTok{"08/13/2019 21:30"}\NormalTok{, }\StringTok{"08/14/2019 00:00"}\NormalTok{,}
           \StringTok{"08/14/2019 02:30"}\NormalTok{, }\StringTok{"08/14/2019 05:00"}\NormalTok{, }\StringTok{"08/14/2019 07:30"}\NormalTok{,}
           \StringTok{"08/14/2019 10:00"}\NormalTok{, }\StringTok{"08/14/2019 12:30"}\NormalTok{, }\StringTok{"08/14/2019 15:00"}\NormalTok{,}
           \StringTok{"08/14/2019 17:30"}\NormalTok{, }\StringTok{"08/14/2019 20:00"}\NormalTok{, }\StringTok{"08/14/2019 22:30"}\NormalTok{,}
           \StringTok{"08/15/2019 01:00"}\NormalTok{, }\StringTok{"08/15/2019 03:30"}\NormalTok{)}

\FunctionTok{library}\NormalTok{(lubridate)}
\end{Highlighting}
\end{Shaded}

We will define the timezone to be timezone of our computer workstation.

\vspace{0.1in}

\begin{Shaded}
\begin{Highlighting}[]
\NormalTok{tz }\OtherTok{\textless{}{-}} \FunctionTok{Sys.timezone}\NormalTok{(}\AttributeTok{location =} \ConstantTok{TRUE}\NormalTok{)}
\end{Highlighting}
\end{Shaded}

The package \emph{lubridate} contains data-time parsers that may be easier to use than the base functions \texttt{strptime} and \texttt{as.Date}.
For the current example, we note that the data are in a month/day/year hour:minute format. So we can create a time series using the function \texttt{lubridate::mdy\_hm}.

\vspace{0.1in}

\begin{Shaded}
\begin{Highlighting}[]
\NormalTok{date\_lub }\OtherTok{\textless{}{-}} \FunctionTok{mdy\_hm}\NormalTok{(dates, }\AttributeTok{tz =}\NormalTok{ tz)}
\NormalTok{date\_lub}
\end{Highlighting}
\end{Shaded}

\begin{verbatim}
 [1] "2019-08-13 04:00:00 MDT" "2019-08-13 06:30:00 MDT"
 [3] "2019-08-13 09:00:00 MDT" "2019-08-13 11:30:00 MDT"
 [5] "2019-08-13 14:00:00 MDT" "2019-08-13 16:30:00 MDT"
 [7] "2019-08-13 19:00:00 MDT" "2019-08-13 21:30:00 MDT"
 [9] "2019-08-14 00:00:00 MDT" "2019-08-14 02:30:00 MDT"
[11] "2019-08-14 05:00:00 MDT" "2019-08-14 07:30:00 MDT"
[13] "2019-08-14 10:00:00 MDT" "2019-08-14 12:30:00 MDT"
[15] "2019-08-14 15:00:00 MDT" "2019-08-14 17:30:00 MDT"
[17] "2019-08-14 20:00:00 MDT" "2019-08-14 22:30:00 MDT"
[19] "2019-08-15 01:00:00 MDT" "2019-08-15 03:30:00 MDT"
\end{verbatim}

Other \emph{lubridate} parsers include \texttt{ymd()}, \texttt{ymd\_hms()}, \texttt{dmy()}, \texttt{dmy\_hms()}, and \texttt{mdy()}. The \emph{lubridate} parsers can often handle mixed methods of data entry. From the \texttt{ymd()} documentation we have the following example:

\vspace{0.1in}

\begin{Shaded}
\begin{Highlighting}[]
\NormalTok{x }\OtherTok{\textless{}{-}} \FunctionTok{c}\NormalTok{(}\DecValTok{20090101}\NormalTok{, }\StringTok{"2009{-}01{-}02"}\NormalTok{, }\StringTok{"2009 01 03"}\NormalTok{, }\StringTok{"2009{-}1{-}4"}\NormalTok{,}
       \StringTok{"2009{-}1, 5"}\NormalTok{, }\StringTok{"Created on 2009 1 6"}\NormalTok{, }\StringTok{"200901 !!! 07"}\NormalTok{)}
\FunctionTok{ymd}\NormalTok{(x)}
\end{Highlighting}
\end{Shaded}

\begin{verbatim}
[1] "2009-01-01" "2009-01-02" "2009-01-03" "2009-01-04" "2009-01-05"
[6] "2009-01-06" "2009-01-07"
\end{verbatim}

\end{example}

\(\blacksquare\)
\index[R]{\texttt{as.Date()}}
\index[R]{\texttt{Sys.timezone()}}
\index[R]{\texttt{lubridate::as\_datetime()}}
\index[R]{\texttt{lubridate::ymd()}}
\index[R]{\texttt{lubridate::ymd\_hms()}}
\index[R]{\texttt{lubridate::dmy()}}
\index[R]{\texttt{lubridate::dmy\_hms()}}
\index[R]{\texttt{lubridate::mdy()}}

\emph{Lubridate} also allows extended mathematical operations for its date-time objects with the functions \texttt{duration()}, \texttt{period()}, and \texttt{interval()}.

\begin{example}
\protect\hypertarget{exm:exm-5lub2}{}\label{exm:exm-5lub2}

\(\text{}\)\\
Duration functions include \texttt{dseconds()}, \texttt{dminutes()}, \texttt{ddays()}, and \texttt{dmonths()}.

\vspace{0.1in}

\begin{Shaded}
\begin{Highlighting}[]
\FunctionTok{duration}\NormalTok{(}\StringTok{"12m"}\NormalTok{, }\AttributeTok{units =} \StringTok{"seconds"}\NormalTok{) }\CommentTok{\# seconds in 1 year}
\end{Highlighting}
\end{Shaded}

\begin{verbatim}
[1] "31557600s (~1 years)"
\end{verbatim}

\vspace{0.1in}

\begin{Shaded}
\begin{Highlighting}[]
\FunctionTok{dmonths}\NormalTok{(}\DecValTok{12}\NormalTok{)}
\end{Highlighting}
\end{Shaded}

\begin{verbatim}
[1] "31557600s (~1 years)"
\end{verbatim}

\vspace{0.1in}

\begin{Shaded}
\begin{Highlighting}[]
\NormalTok{date\_lub[}\DecValTok{1}\NormalTok{]}
\end{Highlighting}
\end{Shaded}

\begin{verbatim}
[1] "2019-08-13 04:00:00 MDT"
\end{verbatim}

\vspace{0.1in}

\begin{Shaded}
\begin{Highlighting}[]
\NormalTok{date\_lub[}\DecValTok{1}\NormalTok{] }\SpecialCharTok{+} \FunctionTok{ddays}\NormalTok{(}\DecValTok{1}\NormalTok{)}
\end{Highlighting}
\end{Shaded}

\begin{verbatim}
[1] "2019-08-14 04:00:00 MDT"
\end{verbatim}

\end{example}

\(\blacksquare\)

\index[R]{\texttt{lubridate::dseconds()}}
\index[R]{\texttt{lubridate::ddays()}}
\index[R]{\texttt{lubridate::dminutes()}}
\index[R]{\texttt{lubridate::dmonths()}}

\begin{example}
\protect\hypertarget{exm:exm-5lub3}{}\label{exm:exm-5lub3}

\(\text{}\)\\
Periodic functions include \texttt{seconds()}, \texttt{minutes()}, \texttt{hours()}, and \texttt{days()}.

\vspace{0.1in}

\begin{Shaded}
\begin{Highlighting}[]
\FunctionTok{days}\NormalTok{(}\DecValTok{12}\NormalTok{) }\SpecialCharTok{+} \FunctionTok{minutes}\NormalTok{(}\DecValTok{2}\NormalTok{) }\SpecialCharTok{+} \FunctionTok{seconds}\NormalTok{(}\DecValTok{3}\NormalTok{)}
\end{Highlighting}
\end{Shaded}

\begin{verbatim}
[1] "12d 0H 2M 3S"
\end{verbatim}

\begin{Shaded}
\begin{Highlighting}[]
\NormalTok{date\_lub[}\DecValTok{1}\NormalTok{]}
\end{Highlighting}
\end{Shaded}

\begin{verbatim}
[1] "2019-08-13 04:00:00 MDT"
\end{verbatim}

\begin{Shaded}
\begin{Highlighting}[]
\NormalTok{date\_lub[}\DecValTok{1}\NormalTok{] }\SpecialCharTok{{-}} \FunctionTok{days}\NormalTok{(}\DecValTok{12}\NormalTok{)}
\end{Highlighting}
\end{Shaded}

\begin{verbatim}
[1] "2019-08-01 04:00:00 MDT"
\end{verbatim}

\end{example}

\(\blacksquare\)
\index[R]{\texttt{lubridate::seconds()}}
\index[R]{\texttt{lubridate::days()}}
\index[R]{\texttt{lubridate::minutes()}}
\index[R]{\texttt{lubridate::months()}}

\begin{example}
\protect\hypertarget{exm:exm-5lub4}{}\label{exm:exm-5lub4}

\(\text{}\)\\
Interval functions include \texttt{int\_length()}, \texttt{int\_start()}, and \texttt{int\_end()}.

\vspace{0.1in}

\begin{Shaded}
\begin{Highlighting}[]
\NormalTok{int }\OtherTok{\textless{}{-}} \FunctionTok{interval}\NormalTok{(}\AttributeTok{start =} \FunctionTok{first}\NormalTok{(date\_lub), }\AttributeTok{end =} \FunctionTok{last}\NormalTok{(date\_lub))}

\FunctionTok{int\_length}\NormalTok{(int) }\SpecialCharTok{|\textgreater{}}
  \FunctionTok{duration}\NormalTok{()}
\end{Highlighting}
\end{Shaded}

\begin{verbatim}
[1] "171000s (~1.98 days)"
\end{verbatim}

\end{example}

\(\blacksquare\)
\index[R]{\texttt{lubridate::int\_length()}}
\index[R]{\texttt{lubridate::int\_start()}}
\index[R]{\texttt{lubridate::int\_end()}}
\index[R]{\texttt{lubridate::dmonths()}}

\section{\texorpdfstring{\emph{reshape2}}{reshape2}}\label{reshape2}

\index[terms]{R!package!\textit{reshape2}}
Tidyverse functions generally require that data are in a \emph{long table} format. That is, data are stored with columns containing all the values for a particular variable of interest. Unfortunately, this format is not conventional for many scientific applications, particularly longitudinal studies that follow experimental units over time. These will often have a \emph{wide table} format. The \emph{tidyverse} \emph{reshape2} package contains several functions for converting dataframes from wide to a long table formats, including the functions \texttt{reshape2::melt()} and \texttt{tidyr::gather()}. The \texttt{reshape2::melt.data.frame()} function generates a \texttt{value} column based on data commonalities of outcomes given in a variable or variables defined in the \texttt{id} argument. A remaining column, if any, that captures these commonalities will be given the name \texttt{variable}. The names of the \texttt{value} and \texttt{variable} output columns can be changed with the arguments \texttt{value.name} and \texttt{variable.name}, respectively.
\index[terms]{R!package!\textit{reshape2}}
\index[R]{\texttt{reshape2::melt()}}
\index[R]{\texttt{reshape2::melt.data.frame()}}
\index[R]{\texttt{tidyr::gather()}}

\begin{example}
\protect\hypertarget{exm:exm-511}{}\label{exm:exm-511}\(\text{}\)\\
Consider the \texttt{asbio::asthma} dataset, which has a wide table format.
The dataset documents the effect of three respiratory treatments (measured as Forced Expiratory Volume in one second (FEV1))
for 24 asthmatic patients over time (\texttt{11H} - \texttt{18H}, i.e, hour 11 to hour 18). A baseline measure of FEV1 (\texttt{BASEFEV1}) was also taken 11 hours before application of the treatment.

\vspace{0.1in}

\begin{Shaded}
\begin{Highlighting}[]
\FunctionTok{data}\NormalTok{(asthma)}
\FunctionTok{head}\NormalTok{(asthma)}
\end{Highlighting}
\end{Shaded}

\begin{verbatim}
  PATIENT BASEFEV1 FEV11H FEV12H FEV13H FEV14H FEV15H FEV16H FEV17H FEV18H
1     201     2.46   2.68   2.76   2.50   2.30   2.14   2.40   2.33   2.20
2     202     3.50   3.95   3.65   2.93   2.53   3.04   3.37   3.14   2.62
3     203     1.96   2.28   2.34   2.29   2.43   2.06   2.18   2.28   2.29
4     204     3.44   4.08   3.87   3.79   3.30   3.80   3.24   2.98   2.91
5     205     2.80   4.09   3.90   3.54   3.35   3.15   3.23   3.46   3.27
6     206     2.36   3.79   3.97   3.78   3.69   3.31   2.83   2.72   3.00
  DRUG
1    a
2    a
3    a
4    a
5    a
6    a
\end{verbatim}

\begin{Shaded}
\begin{Highlighting}[]
\FunctionTok{library}\NormalTok{(reshape2)}
\NormalTok{asthma.long }\OtherTok{\textless{}{-}}\NormalTok{ asthma }\SpecialCharTok{|\textgreater{}} \FunctionTok{melt}\NormalTok{(}\AttributeTok{id =} \FunctionTok{c}\NormalTok{(}\StringTok{"DRUG"}\NormalTok{, }\StringTok{"PATIENT"}\NormalTok{),}
                               \AttributeTok{value.name =} \StringTok{"FEV1"}\NormalTok{,}
                               \AttributeTok{variable.name =} \StringTok{"TIME"}\NormalTok{)}

\CommentTok{\# here I simplify the names in the TIME variable}
\NormalTok{asthma.long}\SpecialCharTok{$}\NormalTok{TIME }\OtherTok{\textless{}{-}} \FunctionTok{factor}\NormalTok{(asthma.long}\SpecialCharTok{$}\NormalTok{TIME,}
                           \AttributeTok{labels =} \FunctionTok{c}\NormalTok{(}\StringTok{"BASE"}\NormalTok{,}
                                      \FunctionTok{paste}\NormalTok{(}\StringTok{"H"}\NormalTok{, }\DecValTok{11}\SpecialCharTok{:}\DecValTok{18}\NormalTok{, }\AttributeTok{sep =} \StringTok{""}\NormalTok{)))}
\FunctionTok{head}\NormalTok{(asthma.long)}
\end{Highlighting}
\end{Shaded}

\begin{verbatim}
  DRUG PATIENT TIME FEV1
1    a     201 BASE 2.46
2    a     202 BASE 3.50
3    a     203 BASE 1.96
4    a     204 BASE 3.44
5    a     205 BASE 2.80
6    a     206 BASE 2.36
\end{verbatim}

In the code above, the function \texttt{reshape2::melt()} is used to convert to a long table format, and time designations are simplified using the \emph{base} function \texttt{factor()}. The \texttt{factor()} function can be used to create a categorical variable with particular levels (Section \ref{sctc}), or to change the names of levels. The latter application is used here.
\end{example}

\(\blacksquare\)
\index[R]{\texttt{factor()}}

\section*{Exercises}\label{exercises-4}
\addcontentsline{toc}{section}{Exercises}

\begin{enumerate}
\def\labelenumi{\arabic{enumi}.}
\tightlist
\item
  Create a tibble from the \texttt{Downs} dataframe shown below. The data comprise part of a report summarizing Down's syndrome cases in British Columbia, compiled by the British Columbia Health Surveillance Registry \citep{geyer1991}.

  \begin{enumerate}
  \def\labelenumii{(\alph{enumii})}
  \item
    Examine both the original \texttt{Downs} dataframe and the tibble representation of \texttt{Downs} by printing them. Do we gain additional information from the tibble?
  \item
    Find the mean and variance of the \texttt{Age} column from the \texttt{Downs} dataset using pipes and \emph{dpylr} functions.
    \vspace{0.1in}

\begin{Shaded}
\begin{Highlighting}[]
\NormalTok{Downs }\OtherTok{\textless{}{-}} \FunctionTok{data.frame}\NormalTok{(}\AttributeTok{Age =} \FunctionTok{c}\NormalTok{(}\DecValTok{17}\NormalTok{, }\FloatTok{20.5}\NormalTok{, }\FloatTok{21.5}\NormalTok{, }\FloatTok{29.5}\NormalTok{, }\FloatTok{30.5}\NormalTok{, }\FloatTok{38.5}\NormalTok{, }\FloatTok{39.5}\NormalTok{,}
                            \FloatTok{40.5}\NormalTok{, }\FloatTok{44.5}\NormalTok{, }\FloatTok{45.5}\NormalTok{, }\DecValTok{47}\NormalTok{),}
                    \AttributeTok{Births =} \FunctionTok{c}\NormalTok{(}\DecValTok{13555}\NormalTok{, }\DecValTok{22005}\NormalTok{, }\DecValTok{23896}\NormalTok{, }\DecValTok{15685}\NormalTok{, }\DecValTok{13954}\NormalTok{,}
                               \DecValTok{4834}\NormalTok{, }\DecValTok{3961}\NormalTok{, }\DecValTok{2952}\NormalTok{, }\DecValTok{596}\NormalTok{, }\DecValTok{327}\NormalTok{, }\DecValTok{249}\NormalTok{),}
                    \AttributeTok{Cases =} \FunctionTok{c}\NormalTok{(}\DecValTok{16}\NormalTok{, }\DecValTok{22}\NormalTok{, }\DecValTok{16}\NormalTok{, }\DecValTok{9}\NormalTok{, }\DecValTok{12}\NormalTok{, }\DecValTok{15}\NormalTok{, }\DecValTok{30}\NormalTok{, }\DecValTok{31}\NormalTok{, }\DecValTok{22}\NormalTok{, }\DecValTok{11}\NormalTok{,}
                              \DecValTok{7}\NormalTok{)}
\NormalTok{                    )}
\end{Highlighting}
\end{Shaded}

    \vspace{0.1in}
  \end{enumerate}
\item
  Bring in the \texttt{world.emissions} dataset from package \emph{asbio}.

  \begin{enumerate}
  \def\labelenumii{(\alph{enumii})}
  \tightlist
  \item
    Using the forward pipe operator, \texttt{\textbar{}\textgreater{}}, and \texttt{filter()} from \emph{dplyr}, create a dataframe of just US data.
  \item
    Using \texttt{\textbar{}\textgreater{}}, \texttt{filter()}, and \texttt{summarise()}, find the first and last year of emissions data for the US.
  \item
    Using \texttt{\textbar{}\textgreater{}}, \texttt{\%T\textgreater{}\%}, \texttt{filter()}, \texttt{mutate()}, and \texttt{plot()}, plot per capita CO\(_2\) emissions for the US by year (as an intermediate pipeline step) and find the maximum CO\(_2\) emission level. Hint: see Exercise \ref{exm:exm-59}.
  \item
    Using \texttt{\textbar{}\textgreater{}} and \texttt{filter()} create a new dataframe called \texttt{no.repeats} that eliminates rows with the entry \texttt{"redundant"} in the \texttt{world.emissions\$continent} column.
  \item
    With the \texttt{no.repeats} dataframe and the functions \texttt{group\_by()}, and \texttt{summarise()}, get mean CO\(_2\) levels for each country over time.
  \item
    Using \texttt{\textbar{}\textgreater{}}, \texttt{group\_by()}, \texttt{summarise()} and \texttt{slice\_max()}, identify the 10 countries with the highest recorded cumulative CO\(_2\) emissions.
    \vspace{0.1in}
  \end{enumerate}
\item
  Consider the character vector \texttt{omics} below \citep{bonn}.

  \begin{enumerate}
  \def\labelenumii{(\alph{enumii})}
  \item
    Use \texttt{stringr::str\_detect()} to test for strings with the pattern \texttt{"genom"}.
  \item
    Using \texttt{str\_detect()}, test for strings \emph{starting} with the pattern \texttt{"genom"} by using an extended regular expression: \texttt{\^{}genom} in the \texttt{str\_detect()} argument \texttt{pattern} (see Section \ref{ergx}).
  \item
    Using \texttt{str\_detect()}, test for strings \emph{ending} with the pattern \texttt{"omics"} by using an extended regular expression (see Section \ref{ergx}).
  \item
    Using \texttt{str\_subset()}, subset the string vector \texttt{omics} to string entries containing the pattern \texttt{"genom"}.
  \item
    Using \texttt{str\_replace()}, replace the text \texttt{"omics"} with \texttt{"ome"}.
    \vspace{0.1in}

\begin{Shaded}
\begin{Highlighting}[]
\NormalTok{omics }\OtherTok{\textless{}{-}} \FunctionTok{c}\NormalTok{(}\StringTok{"genomics"}\NormalTok{, }\StringTok{"proteomics"}\NormalTok{, }\StringTok{"proteome"}\NormalTok{,}
           \StringTok{"transcriptomics"}\NormalTok{, }\StringTok{"metagenomics"}\NormalTok{, }\StringTok{"metabolomics"}\NormalTok{)}
\end{Highlighting}
\end{Shaded}

    \vspace{0.1in}
  \end{enumerate}
\item
  Consider the character vector \texttt{times} below, which has the format: \texttt{day-month-year\ hour:minute:second}.

  \begin{enumerate}
  \def\labelenumii{(\alph{enumii})}
  \item
    Convert \texttt{times} into a \emph{lubridate} date-time object using an appropriate \emph{lubridate} function.
  \item
    Add two days and seven seconds to each entry in \texttt{time} using \texttt{lubridate::days}.
  \item
    Using \emph{lubridate} functions, find the difference, in seconds, between the beginning and the end of the time series.
    \vspace{0.1in}

\begin{Shaded}
\begin{Highlighting}[]
\NormalTok{times }\OtherTok{\textless{}{-}} \FunctionTok{c}\NormalTok{(}\StringTok{"12{-}12{-}2023 12:12:20"}\NormalTok{,}
           \StringTok{"12{-}01{-}2021 01:12:40"}\NormalTok{,}
           \StringTok{"15{-}10{-}2021 23:10:15"}\NormalTok{,}
           \StringTok{"25{-}07{-}2022 13:09:45"}\NormalTok{)}
\end{Highlighting}
\end{Shaded}
  \end{enumerate}
\end{enumerate}

\chapter{Base Graphics}\label{ch6}

\begin{quote}
\emph{"Mankind invented a system to cope with the fact that we are so intrinsically lousy at manipulating numbers. It's called the graph."}
\end{quote}

- \textbf{Charlie Munger}, \emph{businessman and philanthropist}

\section{Introduction}\label{introduction}

\index[terms]{R!graphics}
An important feature of \textbf{R} is its capacity to create publication-quality graphics with tremendous user flexibility. Generally speaking, \textbf{R} graphics are non-interactive and changes to plots require the creation of entirely new static plots. This may feel like a major departure for those used to point-and-click graphics, characteristic of software like Excel\(^{\circledR}\) and SigmaPlot\(^{\circledR}\).\index[R]{\texttt{identify()}}
\index[R]{\texttt{locator()}}

There are two general graphics approaches in \textbf{R}: \emph{base} graphics and \emph{grid} graphics \citep{murrell2019}. Base graphics are applied using the \textbf{R} distribution package \emph{graphics},\index[terms]{R!package!\textit{graphics}} whereas the grid graphics system relies on low level facilities in the \textbf{R} distributed \emph{grid} package\index[terms]{R!package!\textit{grid}}, which are generally implemented via high level functions in other packages. The base and grid graphics systems generally do not interact well, although both rely on the distributed \emph{grDevices} package\index[terms]{R!package!\textit{grDevices}} which provides the fundamental infrastructure for \textbf{R} graphics, including graphical devices. Both base and grid systems follow the \emph{painters model} in which later output obscures earlier overlapping output.

The base graphics system is the focus of this chapter. The grid system, and its most popular adherent, the package \emph{ggplot2}, is described in Chapter \ref{ch7}.
\index[terms]{R!package!\textit{ggplot2}}

\section{Simple Base Graphics Examples}\label{simple-base-graphics-examples}

The base graphics system allows creation of a wide variety of plots for single variables and multiple variables (see Figs \ref{fig:d1} and \ref{fig:d2}, respectively). Approaches for making many of these example plots are elaborated later in this chapter.

\begin{figure}[H]

{\centering \includegraphics{figure/minimal-d1-1} 

}

\caption{Base graphics approaches for single variables.  Figure follows \citet{murrell2019}. Classes of plotted objects are distinguished by name and color in main headings: \textcolor{cyan}{num} = numeric, \textcolor{brown}{tab} = table, \textcolor{green}{fac} = factor. By row, from left to right, graphics are: 1) a \textit{scatterplot} created by applying the function \texttt{plot()} to a vector of class numeric, 2) the \texttt{plot()} function applied to a one-dimensional object of class table, resulting in a distributional plot, 3) a \textit{barplot}, useful for comparing categorical outcomes, 4) a *pie chart*, 5) a     extit{dotchart}, which provides a dot variant of a bar plot), 6) a  extit{boxplot}, i.e., the interquartile range (hinges) and whiskers delimiting outliers, 7) a \textit{histogram} (the most common graphical distributional summary), 8) a \textit{stripchart}, i.e., a one dimensional scatter plot that provides a horizontal view of distributional outcomes, and 9) a \textit{stem chart}.}\label{fig:d1}
\end{figure}

\index[R]{\texttt{plot()}}
\index[R]{\texttt{hist()}}
\index[R]{\texttt{stripchart()}}
\index[R]{\texttt{barplot()}}
\index[R]{\texttt{boxplot()}}
\index[R]{\texttt{dotchart()}}
\index[R]{\texttt{pie()}}
\index[R]{\texttt{stem()}}

\index[terms]{R!graphics!strip charts}
\index[terms]{R!graphics!dot plots}
\index[terms]{R!graphics!pie charts}
\index[terms]{R!graphics!stem plots}

\begin{figure}[H]

{\centering \includegraphics{figure/minimal-d2-1} 

}

\caption{Base graphics approaches for considering multiple variables.  Figure follows \citet{murrell2019}. Classes of plotted objects are distinguished by name and color in main headings: \textcolor{cyan}{num} = numeric, \textcolor{blue}{mat} = matrix, \textcolor{green}{fac} = factor. By row, from left to right, graphics are: 1) a \textit{scatterplot} based on two quantitative variables, 2) a \textit{scatterplot} with smoothed densities, 3) a \textit{sunflower plot}, which uses special symbols to indicate overplotting of points, 4) a \textit{boxplot} based on a factor (with two levels) and a numeric variable, 5) and 6) \textit{stacked and beside barplots} of matrices, 7) a dotchart, 8) a stripchart, based on two numeric variables, and 9) a \textit{spineplot}, a special cases of a  extit{mosaic plot} (obtained using \texttt{mosaicplot()}), representing a generalization of a stacked (or highlighted) bar plot.}\label{fig:d2}
\end{figure}

\index[R]{\texttt{plot()}}
\index[R]{\texttt{smoothScatter()}}
\index[R]{\texttt{sunflowerplot()}}
\index[R]{\texttt{spineplot()}}

\index[terms]{R!graphics!sunflower plots}
\index[terms]{R!graphics!smooth scatter plots}
\index[terms]{R!graphics!spine plots}
\index[terms]{R!graphics!mosaic plots}

\subsection{\texorpdfstring{\texttt{plot()}}{plot()}}\label{plot}

\index[R]{\texttt{plot()}}
The workhorse of base graphics is the function \texttt{plot()}. From Figs \ref{fig:d1} and \ref{fig:d2} it is evident that \texttt{plot()} can be used in a number of different ways, depending on the characteristics of data being plotted. For example, if data are two numeric vectors, then a conventional scatterplot is created. However, if the first two arguments in \texttt{plot()} call a numerical vector and a factor vector (in that order), then a boxplot is created, and if the first two arguments in \texttt{plot()} call a factor vector and a numeric vector (in that order), then a stripchart is created. Further, plotting methods for particular classes of objects can be designed that can be implemented by calling \texttt{plot()}. For instance, the dendrogram in Fig \ref{fig:den} was created using a plotting method called \texttt{plot.agnes()}, designed for objects of class \texttt{agnes}\footnote{An object class resulting from hierarchical agglomerative cluster analyses produced by the function \texttt{cluster::agnes}.}. However, the function can be run using a generic call to \texttt{plot()}. See Ch \ref{ch8} for additional details on plotting methods for \textbf{R} classes.
\index[R]{\texttt{cluster::agnes()}}
\index[R]{\texttt{cluster::plot.agnes()}}

\vspace{0.1in}

\begin{Shaded}
\begin{Highlighting}[]
\FunctionTok{library}\NormalTok{(cluster)}
\NormalTok{aa.ga }\OtherTok{\textless{}{-}} \FunctionTok{agnes}\NormalTok{(animals, }\AttributeTok{method =} \StringTok{"average"}\NormalTok{)}
\FunctionTok{plot}\NormalTok{(aa.ga, }\AttributeTok{sub =} \StringTok{""}\NormalTok{, }\AttributeTok{main =} \StringTok{""}\NormalTok{, }\AttributeTok{which.plots =} \DecValTok{2}\NormalTok{, }\AttributeTok{xlab =} \StringTok{""}\NormalTok{)}
\end{Highlighting}
\end{Shaded}

\begin{figure}[H]

{\centering \includegraphics{figure/minimal-den-1} 

}

\caption{Dendrogram of an average linkage classification of animals based on six variables: warm vs. cold blooded, ability to fly, vertebrate or invertebrate, whether or not the animal is endangered, whether or not the animal lives in groups, and whether or not the animal has hair. The plotting function used, \texttt{plot.agnes()}, is called using the generic name \texttt{plot()}.}\label{fig:den}
\end{figure}

By default, the function \texttt{plot()} creates a projection at user defined Cartesian coordinates. Under this usage \texttt{plot()} has only
two required arguments.

\begin{itemize}
\tightlist
\item
  \texttt{x} defines the x-coordinate values.
\item
  \texttt{y} defines the y-coordinate values.
\end{itemize}

If coordinates for only one dimension, \texttt{x} are supplied, then \texttt{x} is plotted on the vertical axis against the series \(1:n\), where \(n\) is the number of points in \texttt{x}. A coordinate system can also be supplied to the argument \texttt{x} in the form of a formula, list, matrix, or dataframe.

Important optional arguments include the following:

\begin{itemize}
\tightlist
\item
  \texttt{pch} specifies the symbol type(s), i.e., the plotting character(s) to be used.
\item
  \texttt{col} defines the color(s) to be used with the symbols.
\item
  \texttt{cex} defines the size (character expansion) of the plot symbols and text.
\item
  \texttt{xlab} and \texttt{ylab} allow the user to specify the x and y-axis labels.
\item
  \texttt{type} allows the user to define the type of graph to be drawn. Possible types are \texttt{"p"} for scatterplot points (the default), \texttt{"l"} for a line plot, \texttt{"b"} for both, \texttt{"c"} for the line component of \texttt{"b"}, \texttt{"o"} for overplotted, \texttt{"h"} for `histogram' like vertical lines (see middle plot in top row of Fig \ref{fig:d1}), \texttt{"s"} for stair steps, and \texttt{n"} for no plotting.
\end{itemize}

\begin{example}
\protect\hypertarget{exm:exm-plotf1}{}\label{exm:exm-plotf1}\(\text{}\)\\
We can see some symbol and color alternatives by calling them in \texttt{plot()} (Fig \ref{fig:sp1}).
\vspace{0.1in}

\begin{Shaded}
\begin{Highlighting}[numbers=left,,]
\FunctionTok{plot}\NormalTok{(}\DecValTok{1}\SpecialCharTok{:}\DecValTok{20}\NormalTok{, }\DecValTok{1}\SpecialCharTok{:}\DecValTok{20}\NormalTok{, }\AttributeTok{pch =} \DecValTok{1}\SpecialCharTok{:}\DecValTok{20}\NormalTok{, }\AttributeTok{col =} \DecValTok{1}\SpecialCharTok{:}\DecValTok{20}\NormalTok{,}
     \AttributeTok{ylab =}\StringTok{"Symbol number"}\NormalTok{,}
     \AttributeTok{xlab =} \StringTok{"Color number"}\NormalTok{,}
     \AttributeTok{cex =} \FloatTok{1.6}\NormalTok{, }\AttributeTok{cex.lab =} \FloatTok{1.1}\NormalTok{, }\AttributeTok{cex.axis =} \FloatTok{1.1}\NormalTok{)}
\end{Highlighting}
\end{Shaded}

\begin{figure}[H]

{\centering \includegraphics{figure/minimal-sp1-1} 

}

\caption{Some symbol and color plotting possibilities in \texttt{plot()}.}\label{fig:sp1}
\end{figure}

In line one from the code above, the \emph{x} and \emph{y} coordinates are both sequences of numbers from 1 to 20 obtained from the command \texttt{1:20}. I varied symbol colors and plotting characters (\texttt{col} and \texttt{pch}, respectively) using \texttt{1:20} as well. Thus, the combination \texttt{col\ =\ 1} and \texttt{pch\ =\ 1} results in a black open circle, whereas the combination \texttt{col\ =\ 20}, \texttt{pch\ =\ 20} results in a blue filled circle. Note that we need to enclose the axis names in quotations for \texttt{R} to recognize
them as text. Symbol numbers \texttt{21-26} allow background color specification using the argument \texttt{bg}. Many other symbol types are also possible.
\end{example}

\(\blacksquare\)

\section{Graphical Devices}\label{graphical-devices}

Graphics in \textbf{R} are created within \emph{graphics devices}\index[terms]{R!graphics devices}, encoded in the package \emph{grDevices}. These vary with respect to storage modes, display modes, available typefaces, and other characteristics. In a basic \textbf{R} download, six graphics devices will be available:

\begin{itemize}
\tightlist
\item
  \texttt{windows()} is available for Windows releases of \textbf{R}. Provides on-screen rendering of graphics (outside of RStudio)\footnote{RStudio has its own native on-screen graphics device. A non RStudio graphics device can be opened (within RStudio) using \texttt{dev.new(RStudioGD\ =\ FALSE)}.}, and creates Windows metafile graphics files.
\item
  \texttt{pdf()} renders graphics into .pdf files.
\item
  \texttt{postscript()} renders graphics into PostScript, .ps, graphics files.
\item
  \texttt{xfig()} renders graphics files using the Xfig graphics file format.
\item
  \texttt{bitmap()} renders graphics into bitmap graphics files It requires the open source software \href{https://www.ghostscript.com/}{ghostscript}.
\item
  \texttt{piktex()} Writes TeX/PicTeX graphics commands to a file and is of historical interest only.
\end{itemize}

\index[R]{\texttt{windows()}}
\index[R]{\texttt{pdf()}}\index[R]{\texttt{postscript()}}
\index[R]{\texttt{xfig()}}\index[R]{\texttt{xfig()}}\index[R]{\texttt{bitmap()}}

A number of other graphics devices also exist, although they may return a warning if \textbf{R} was not compiled to use them upon installation.

\begin{itemize}
\item
  \texttt{cairo\_pdf()}, \texttt{cairo\_ps()} and \texttt{svg()} are PDF, PostScript SVG (Scalar Vector Graphics) devices based on the open source \href{https://www.cairographics.org/}{Cairo} graphics.
\item
  \texttt{bmp()}, \texttt{jpeg()}, \texttt{png()}, and \texttt{tiff()} render graphics as .bmp, .jpg, .png, and .tif bitmap files, respectively.
\item
  \texttt{X11()} is the graphics device for the X11 windowing system, and is commonly used in Unix-alike operating systems, including MacOS.
\item
  \texttt{quartz()} is only functional on MacOS and supports plotting to the screen (default) and to various graphics file formats. The device requires the open source software \href{https://www.xquartz.org/}{XQuartz} for rendering some \textbf{R} graphical user interfaces (see Ch \ref{ch11}).
\end{itemize}

\index[R]{\texttt{cairo\_pdf()}}
\index[R]{\texttt{cairo\_ps()}}\index[R]{\texttt{svg()}}
\index[R]{\texttt{bmp()}}\index[R]{\texttt{jpeg()}}
\index[R]{\texttt{png()}}\index[R]{\texttt{tiff()}}
\index[R]{\texttt{X11()}}\index[R]{\texttt{quartz()}}

Multiple devices (currently up to 63) may exist simultaneously in an \textbf{R} work session, although there will only be one \emph{active} device. To find the current (active) graphics device can type \texttt{dev.cur()}. I get:

\vspace{0.1in}

\begin{Shaded}
\begin{Highlighting}[]
\FunctionTok{dev.cur}\NormalTok{()}
\end{Highlighting}
\end{Shaded}

\begin{verbatim}
pdf 
  2 
\end{verbatim}

\textbf{R} tells me there are two devices open. The current device is a Windows device. The second device is the so-called ``null device.'' The null device is always open but only serves as a placeholder. Any attempt to use it will open a new device in \textbf{R}. Occasionally, on purpose or by accident, all graphics devices (except the null device) may become turned off. A new active graphics device can be created at any time by typing:

\vspace{0.1in}

\begin{Shaded}
\begin{Highlighting}[]
\FunctionTok{dev.new}\NormalTok{()}
\end{Highlighting}
\end{Shaded}

One can close the current (active) device using:

\vspace{0.1in}

\begin{Shaded}
\begin{Highlighting}[]
\FunctionTok{dev.off}\NormalTok{()}
\end{Highlighting}
\end{Shaded}

The active device can be changed using the function \texttt{dev.set()}. For instance, if there were three or more accessible devices, and one wished to define device three as the active device, one could type:

\vspace{0.1in}

\begin{Shaded}
\begin{Highlighting}[]
\FunctionTok{dev.set}\NormalTok{(}\DecValTok{3}\NormalTok{)}
\end{Highlighting}
\end{Shaded}

It is possible to scroll through graphics devices using keyboard shortcuts. Specifically, let \(n\) be the current device number, then the combination Ctrl + Alt + F11 (Windows or Linux) or Cmd + Alt + F11 (Mac) shows device \(n-1\), whereas Ctrl + Alt + F12 (Windows or Linux) or Cmd + Alt + F12 (Mac) shows device \(n+1\).

\index[R]{\texttt{dev.cur()}}
\index[R]{\texttt{dev.new()}}
\index[R]{\texttt{dev.set()}}

\section{\texorpdfstring{\texttt{par()}}{par()}}\label{par}

Parameters for a graphics device (which may contain several plots) can be accessed and modified using the function \texttt{par()}. Below are important arguments for \texttt{par()}. Some of these can also be specified as arguments in \texttt{plot()}, with different results.

\begin{itemize}
\tightlist
\item
  \texttt{bg} gives the background color for the graphical device. When used in \texttt{plot()} it gives the background color of plotting symbols.
\item
  \texttt{bty} is the box-type to be drawn around the plots. If \texttt{bty} is one of \texttt{"o"} (the default), \texttt{"l"}, \texttt{"7"}, \texttt{"c",\ "u"}, or \texttt{"{]}"} the resulting box resembles the corresponding upper case letter. The value \texttt{"n"} suppresses the box.
\item
  \texttt{fg} gives the foreground color.
\item
  \texttt{font} is an integer that specifies the font typeface. \texttt{1} corresponds to regular text (the default), \texttt{2} to bold face,
  \texttt{3} to italic and \texttt{4} to bold italic.
\item
  \texttt{las} is the style of axis labels: \texttt{0} always parallel to the axis (default), \texttt{1} always horizontal, \texttt{2} always
  perpendicular to the axis, \texttt{3} always vertical.
\item
  \texttt{mar} will have the form \texttt{c(bottom,\ left,\ top,\ right)} and gives the number of lines of margin to be specified on the four sides of the plot. The default is \texttt{c(5,\ 4,\ 4,\ 2)\ +\ 0.1}.
\item
  \texttt{mfrow} will have the form \texttt{c(number\ of\ rows,\ number\ of\ columns)} and the number and position of plots in a graphical layout. Multiple graphs can also be placed into a graphical device with additional control over plot designation to multiple elements in a row and column configuration, using the function \texttt{layout()}.\index[R]{\texttt{layout()}}
\item
  \texttt{oma} specifies the outer margins of a graphical device, given multiple plots, using a vector using a matrix of
  the form: \texttt{c(bottom,\ left,\ top,\ right)}.
\item
  \texttt{usr} will have the form \texttt{c(x1,\ x2,\ y1,\ y2)} giving the extremes of the user coordinates of the plotting
  region.
\end{itemize}

When setting graphical parameters, it is good practice to revert back to the original parameter values. Assume that I want to background of the graphics device to be black. To set this I would type:

\vspace{0.1in}

\begin{Shaded}
\begin{Highlighting}[]
\NormalTok{old.par }\OtherTok{\textless{}{-}} \FunctionTok{par}\NormalTok{(}\AttributeTok{no.readonly =} \ConstantTok{TRUE}\NormalTok{) }\CommentTok{\# save default, for resetting...}
\FunctionTok{par}\NormalTok{(}\AttributeTok{bg =} \StringTok{"black"}\NormalTok{) }\CommentTok{\# change background parameter}
\end{Highlighting}
\end{Shaded}

To return to the default settings for background I would type:

\vspace{0.1in}

\begin{Shaded}
\begin{Highlighting}[]
\FunctionTok{par}\NormalTok{(old.par)}
\end{Highlighting}
\end{Shaded}

Defaults will also be reset by closing the current graphics device, or by opening a new device. For instance, using \texttt{dev.new()}.

Other fundamental properties of the default graphics device, such as device height, width and pointsize, can be adjusted using the \texttt{dev.new()} function. For instance, to create a graphical device 9 inches wide, and 4 inches high, I would type:

\vspace{0.1in}

\begin{Shaded}
\begin{Highlighting}[]
\FunctionTok{dev.new}\NormalTok{(}\AttributeTok{width =} \DecValTok{9}\NormalTok{, }\AttributeTok{height =} \DecValTok{4}\NormalTok{)}
\end{Highlighting}
\end{Shaded}

\begin{example}
\protect\hypertarget{exm:exm-par1}{}\label{exm:exm-par1}

\(\text{}\)\\
Fig \ref{fig:par1} shows an example of applying background and foreground colors using the \texttt{bg} and \texttt{fg} arguments in \texttt{par()}, respectively. Note also the specification of a bold font using the \texttt{par()} argument \texttt{font\ =\ 3}, and expansion of all graphics parameters to slightly larger than their original size, using \texttt{cex\ =\ 1.1}.

\vspace{0.1in}

\begin{Shaded}
\begin{Highlighting}[numbers=left,,]
\NormalTok{old.par }\OtherTok{\textless{}{-}} \FunctionTok{par}\NormalTok{(}\AttributeTok{no.readonly =} \ConstantTok{TRUE}\NormalTok{)}
\FunctionTok{par}\NormalTok{(}\AttributeTok{bg =} \StringTok{"black"}\NormalTok{, }\AttributeTok{fg =} \StringTok{"white"}\NormalTok{, }\AttributeTok{font =} \DecValTok{3}\NormalTok{, }\AttributeTok{cex =} \FloatTok{1.1}\NormalTok{)}
\FunctionTok{plot}\NormalTok{(}\DecValTok{1}\SpecialCharTok{:}\DecValTok{10}\NormalTok{, }\DecValTok{1}\SpecialCharTok{:}\DecValTok{10}\NormalTok{, }\AttributeTok{xlab =} \StringTok{"x"}\NormalTok{, }\AttributeTok{ylab =} \StringTok{"y"}\NormalTok{,}
     \AttributeTok{col.lab =} \StringTok{"white"}\NormalTok{)}
\FunctionTok{par}\NormalTok{(old.par)}
\end{Highlighting}
\end{Shaded}

\begin{figure}[H]

{\centering \includegraphics{figure/minimal-par1-1} 

}

\caption{Use of \texttt{par()} to change background and foreground graphical parameters.}\label{fig:par1}
\end{figure}

\end{example}

\(\blacksquare\)

\begin{example}
\protect\hypertarget{exm:exm-par2}{}\label{exm:exm-par2}

\(\text{}\)\\
Fig \ref{fig:par2} shows how one can place multiple graphs into a single graphical device using the \texttt{mfrow} argument in \texttt{par()}, and control figure margins using the \texttt{par()} argument \texttt{mar} (line two). It also shows some basic plot types resulting from the \texttt{type} argument in \texttt{plot()} (lines 4-7).
\index[terms]{R!graphics!line plots}

\vspace{0.1in}

\begin{Shaded}
\begin{Highlighting}[numbers=left,,]
\NormalTok{old.par }\OtherTok{\textless{}{-}} \FunctionTok{par}\NormalTok{(}\AttributeTok{no.readonly =} \ConstantTok{TRUE}\NormalTok{)}
\FunctionTok{par}\NormalTok{(}\AttributeTok{mfrow =} \FunctionTok{c}\NormalTok{(}\DecValTok{2}\NormalTok{,}\DecValTok{2}\NormalTok{), }\AttributeTok{cex =} \FloatTok{1.1}\NormalTok{, }\AttributeTok{mar =} \FunctionTok{c}\NormalTok{(}\DecValTok{4}\NormalTok{,}\DecValTok{4}\NormalTok{,}\DecValTok{1}\NormalTok{,}\DecValTok{1}\NormalTok{))}
\NormalTok{x }\OtherTok{\textless{}{-}} \DecValTok{1}\SpecialCharTok{:}\DecValTok{10}\NormalTok{; y }\OtherTok{\textless{}{-}} \FunctionTok{sort}\NormalTok{(}\FunctionTok{rnorm}\NormalTok{(}\DecValTok{10}\NormalTok{))}
\FunctionTok{plot}\NormalTok{(x, y)}
\FunctionTok{plot}\NormalTok{(x, y, }\AttributeTok{type =} \StringTok{"l"}\NormalTok{)}
\FunctionTok{plot}\NormalTok{(x, y, }\AttributeTok{type =} \StringTok{"o"}\NormalTok{)}
\FunctionTok{plot}\NormalTok{(x, y, }\AttributeTok{type =} \StringTok{"h"}\NormalTok{)}
\FunctionTok{par}\NormalTok{(old.par)}
\end{Highlighting}
\end{Shaded}

\begin{figure}[H]

{\centering \includegraphics{figure/minimal-par2-1} 

}

\caption{Use of \texttt{par()} to place multiple graphs into a single graphical device. The figure also demonstrates basic plot types, specified using the \texttt{plot()} argument \texttt{type}. Clockwise from the top-left these are: 1) a point plot (scatterplot), 2) a line plot, 3) a histogram-like (high density line) plot, and 4) a plot with a both points and lines.}\label{fig:par2}
\end{figure}

\end{example}

\(\blacksquare\)

\section{Exporting Graphics}\label{exporting-graphics}

To export \textbf{R} graphics, one can generally copy snapshots to a clipboard using pull down menus on graphical device. These can then be pasted into programs (e.g., word processors) as bitmaps (a spatially mapped array of bits) or metafiles, a generic term for a file format that can store multiple types of
(generally graphical) data.

To create the best possible graphs, however, one should save device output using graphical device functions.

For instance, to save a graphics device image as a pdf under the file name \texttt{example.pdf} in the working directory I would type:

\vspace{0.1in}

\begin{Shaded}
\begin{Highlighting}[]
\FunctionTok{pdf}\NormalTok{(}\AttributeTok{file =}\StringTok{"example.pdf"}\NormalTok{)}
\end{Highlighting}
\end{Shaded}

I would then make the plot, for instance

\vspace{0.1in}

\begin{Shaded}
\begin{Highlighting}[]
\FunctionTok{plot}\NormalTok{(}\DecValTok{1}\SpecialCharTok{:}\DecValTok{10}\NormalTok{)}
\end{Highlighting}
\end{Shaded}

The plot will not be shown because the \texttt{png()} graphical device is engaged. As a final step I close the device.

\vspace{0.1in}

\begin{Shaded}
\begin{Highlighting}[]
\FunctionTok{dev.off}\NormalTok{()}
\end{Highlighting}
\end{Shaded}

The graphics file will now be contained in the working directory. If the file argument is unspecified, \texttt{pdf()} will save a file called \texttt{Rplot.pdf}.

By default, the \emph{bitmap graphics formats}\index[terms]{Bitmap}: BMP, JPEG, PNG, and TIFF, have a width and height of 480 pixels, and a ``large'' point size (1/72 inch) in \textbf{R}. This results in a rather coarse 72 ppi (72 points per inch) image resolution.
However, changing the \texttt{res} (resolution) argument in a graphical device function without changing the pointsize, or height and width arguments will generally result in unusable figures.

Because \(500 \approx 72 \cdot 6\), one can generate a TIFF with greater than \(400\) ppi TIFF called \texttt{fig1.tiff} by typing:

\vspace{0.1in}

\begin{Shaded}
\begin{Highlighting}[]
\FunctionTok{tiff}\NormalTok{(}\StringTok{"fig1.tiff"}\NormalTok{, }\AttributeTok{res =} \DecValTok{72} \SpecialCharTok{*} \DecValTok{6}\NormalTok{, }\AttributeTok{height =} \DecValTok{480} \SpecialCharTok{*} \DecValTok{6}\NormalTok{, }\AttributeTok{width =} \DecValTok{480} \SpecialCharTok{*} \DecValTok{6}\NormalTok{)}
\FunctionTok{plot}\NormalTok{(}\DecValTok{1}\SpecialCharTok{:}\DecValTok{10}\NormalTok{)}
\FunctionTok{dev.off}\NormalTok{()}
\end{Highlighting}
\end{Shaded}

With respect to graphical formats, documentation in the \emph{grDevices} package states:

\begin{quote}
``The PNG format is lossless\footnote{Lossless entails data compression without loss of information.} and is best for line diagrams and blocks of color. The JPEG format is lossy\footnote{Lossy refers to data compression in which unnecessary information is discarded.},
but may be useful for image plots, for example. The BMP format is standard on Windows, and supported by most viewers elsewhere. TIFF is a meta-format: the default format written by the default format \texttt{tiff(compression\ =\ none)} is lossless and stores RGB values uncompressed. Such files are widely accepted, which is their main virtue over PNG.''
\end{quote}

\index[terms]{Lossless}
\index[terms]{Lossy}

The \texttt{svg()}, \texttt{cairo\_pdf()} and \texttt{cairo\_ps()} graphical devices apply cairographics and will recognize a large number of symbols and fonts not available for document and image generation in the default setting of the Windows PostScript and PDF devices.

\section{\texorpdfstring{\texttt{text()}, \texttt{points()}, and \texttt{lines()}}{text(), points(), and lines()}}\label{text-points-and-lines}

\index[R]{\texttt{text()}}
\index[R]{\texttt{points()}}
\index[R]{\texttt{lines()}}
The functions \texttt{text()}, \texttt{points()} and \texttt{lines()} can be used to overlay text, points and lines in a plot, respectively. As with \texttt{plot()} the first two arguments of these functions are the \emph{x} and \emph{y} coordinates for the plotted entities. Other arguments concern characteristics of the plotted items. For instance, to plot the text \texttt{"example"} with in an existing plot, at plot coordinates \texttt{x\ =\ 0}, \texttt{y\ =\ 0}, with a large character expansion, I could type:

\vspace{0.1in}

\begin{Shaded}
\begin{Highlighting}[]
\FunctionTok{plot}\NormalTok{(}\SpecialCharTok{{-}}\DecValTok{1}\SpecialCharTok{:}\DecValTok{1}\NormalTok{, }\SpecialCharTok{{-}}\DecValTok{1}\SpecialCharTok{:}\DecValTok{1}\NormalTok{, }\AttributeTok{type =} \StringTok{"n"}\NormalTok{, }\AttributeTok{axes =}\NormalTok{ F, }\AttributeTok{xlab =} \StringTok{""}\NormalTok{, }\AttributeTok{ylab =} \StringTok{""}\NormalTok{) }\CommentTok{\# empty plot}
\FunctionTok{text}\NormalTok{(}\AttributeTok{x =} \DecValTok{0}\NormalTok{, }\AttributeTok{y =} \DecValTok{0}\NormalTok{, }\StringTok{"example"}\NormalTok{, }\AttributeTok{cex =} \DecValTok{9}\NormalTok{)}
\end{Highlighting}
\end{Shaded}

The result is shown in Fig \ref{fig:text1}.

\begin{figure}[H]

{\centering \includegraphics{figure/minimal-text1-1} 

}

\caption{Empty plot (even axes are suppressed) with text overlain.}\label{fig:text1}
\end{figure}

The function \texttt{paste()} can be used to concatenate elements from text strings in plots or output. For instance, try:
\index[R]{\texttt{paste()}}

\vspace{0.1in}

\begin{Shaded}
\begin{Highlighting}[]
\NormalTok{a }\OtherTok{\textless{}{-}} \FunctionTok{c}\NormalTok{(}\StringTok{"a"}\NormalTok{, }\StringTok{"b"}\NormalTok{, }\StringTok{"c"}\NormalTok{)}
\NormalTok{b }\OtherTok{\textless{}{-}} \FunctionTok{c}\NormalTok{(}\StringTok{"d"}\NormalTok{, }\StringTok{"e"}\NormalTok{, }\StringTok{"f"}\NormalTok{)}
\NormalTok{c }\OtherTok{\textless{}{-}} \FunctionTok{paste}\NormalTok{(a, b)}
\NormalTok{c}
\end{Highlighting}
\end{Shaded}

\begin{verbatim}
[1] "a d" "b e" "c f"
\end{verbatim}

Which can be placed in a plot (Fig \ref{fig:text2}) using \texttt{text()}.

\vspace{0.1in}

\begin{Shaded}
\begin{Highlighting}[]
\FunctionTok{plot}\NormalTok{(}\SpecialCharTok{{-}}\DecValTok{1}\SpecialCharTok{:}\DecValTok{1}\NormalTok{, }\SpecialCharTok{{-}}\DecValTok{1}\SpecialCharTok{:}\DecValTok{1}\NormalTok{, }\AttributeTok{type =} \StringTok{"n"}\NormalTok{, }\AttributeTok{axes =}\NormalTok{ F, }\AttributeTok{xlab =} \StringTok{""}\NormalTok{, }\AttributeTok{ylab =} \StringTok{""}\NormalTok{)}
\FunctionTok{text}\NormalTok{(}\AttributeTok{x =} \DecValTok{0}\NormalTok{, }\AttributeTok{y =} \DecValTok{0}\NormalTok{, }\FunctionTok{paste}\NormalTok{(c, }\AttributeTok{collapse =} \StringTok{\textquotesingle{} \textquotesingle{}}\NormalTok{), }\AttributeTok{cex =} \DecValTok{2}\NormalTok{)}
\end{Highlighting}
\end{Shaded}

\begin{figure}[H]

{\centering \includegraphics{figure/minimal-text2-1} 

}

\caption{An empty plot with text overlain.  Note the use \texttt{paste(c, collapse = ' ')} to collapse the string vector \texttt{c} into a single entity.}\label{fig:text2}
\end{figure}

To plot a dashed line between the points \((0, 0)\) and \((3, 2)\), I would type:

\vspace{0.1in}

\begin{Shaded}
\begin{Highlighting}[]
\FunctionTok{lines}\NormalTok{(}\AttributeTok{x =} \FunctionTok{c}\NormalTok{(}\DecValTok{0}\NormalTok{, }\DecValTok{3}\NormalTok{), }\AttributeTok{y =} \FunctionTok{c}\NormalTok{(}\DecValTok{0}\NormalTok{, }\DecValTok{2}\NormalTok{), }\AttributeTok{lty =} \DecValTok{2}\NormalTok{)}
\end{Highlighting}
\end{Shaded}

or

\vspace{0.1in}

\begin{Shaded}
\begin{Highlighting}[]
\FunctionTok{points}\NormalTok{(}\AttributeTok{x =} \FunctionTok{c}\NormalTok{(}\DecValTok{0}\NormalTok{, }\DecValTok{3}\NormalTok{), }\AttributeTok{y =} \FunctionTok{c}\NormalTok{(}\DecValTok{0}\NormalTok{, }\DecValTok{2}\NormalTok{), }\AttributeTok{lty =} \DecValTok{2}\NormalTok{, }\AttributeTok{type =} \StringTok{"l"}\NormalTok{)}
\end{Highlighting}
\end{Shaded}

The result is shown in Fig \ref{fig:text3}.

\begin{figure}[H]

{\centering \includegraphics{figure/minimal-text3-1} 

}

\caption{Plot with dashed line overlain.}\label{fig:text3}
\end{figure}

To place a large, blue, triangle with red outline at the point (1, 1), of an existing plot I would type:

\vspace{0.1in}

\begin{Shaded}
\begin{Highlighting}[]
\FunctionTok{points}\NormalTok{(}\AttributeTok{x =} \DecValTok{0}\NormalTok{, }\AttributeTok{y =} \DecValTok{1}\NormalTok{, }\AttributeTok{pch =} \DecValTok{24}\NormalTok{, }\AttributeTok{col =} \DecValTok{2}\NormalTok{, }\AttributeTok{bg =} \DecValTok{4}\NormalTok{, }\AttributeTok{cex =} \DecValTok{8}\NormalTok{)}
\end{Highlighting}
\end{Shaded}

The resulting plot is shown in Fig \ref{fig:text4}.

\begin{figure}[H]

{\centering \includegraphics{figure/minimal-text4-1} 

}

\caption{Plot with point overlain.}\label{fig:text4}
\end{figure}

\subsection{Plotting Mathematical Text}\label{plotting-mathematical-text}

\textbf{R} has useful functions for the plotting of mathematical expressions. These include the Greek letters, mathematical operators, italicization, and sub- and super-scripts. mathematical text is generally called as an expression in the text argument in the functions \texttt{text()} or \texttt{mtext()}. For example, the formula for the sample variance is overlain in Fig \ref{fig:text4a}.

\vspace{0.1in}

\begin{Shaded}
\begin{Highlighting}[]
\FunctionTok{plot}\NormalTok{(}\SpecialCharTok{{-}}\DecValTok{1}\SpecialCharTok{:}\DecValTok{1}\NormalTok{, }\SpecialCharTok{{-}}\DecValTok{1}\SpecialCharTok{:}\DecValTok{1}\NormalTok{, }\AttributeTok{type =} \StringTok{"n"}\NormalTok{, }\AttributeTok{axes =}\NormalTok{ F, }\AttributeTok{xlab =} \StringTok{""}\NormalTok{, }\AttributeTok{ylab =} \StringTok{""}\NormalTok{)}
\NormalTok{varexp }\OtherTok{\textless{}{-}} \FunctionTok{expression}\NormalTok{(}\FunctionTok{over}\NormalTok{(}\FunctionTok{sum}\NormalTok{(}\FunctionTok{paste}\NormalTok{(}\StringTok{"("}\NormalTok{,}\FunctionTok{italic}\NormalTok{(x[i] }\SpecialCharTok{{-}} \FunctionTok{bar}\NormalTok{(x)),}\StringTok{")"}\SpecialCharTok{\^{}}\DecValTok{2}\NormalTok{),}
                              \FunctionTok{italic}\NormalTok{(i)}\SpecialCharTok{==}\DecValTok{1}\NormalTok{, }\FunctionTok{italic}\NormalTok{(n)),(}\FunctionTok{italic}\NormalTok{(n) }\SpecialCharTok{{-}} \DecValTok{1}\NormalTok{)))}

\FunctionTok{text}\NormalTok{(}\AttributeTok{x =} \DecValTok{0}\NormalTok{, }\AttributeTok{y =} \DecValTok{0}\NormalTok{, varexp, }\AttributeTok{cex =} \DecValTok{3}\NormalTok{)}
\end{Highlighting}
\end{Shaded}

\begin{figure}[H]

{\centering \includegraphics{figure/minimal-text4a-1} 

}

\caption{Empty plot with formula for the sample variance overlain.  Type \texttt{?plotmath} for more information.}\label{fig:text4a}
\end{figure}

\index[R]{\texttt{expression()}}

\subsection{\texorpdfstring{\texttt{mtext()}}{mtext()}}\label{mtext}

To place text in the margin of a plot we can use the function \texttt{mtext()}. As its first argument the \texttt{mtext()} function requires a character string to be written into the plot. The 2nd argument, \texttt{side} defines the plot margin to be written on: \texttt{1} = bottom, \texttt{2} =left, \texttt{3} = top, \texttt{4} = right.
\index[R]{\texttt{mtext()}}
For instance, to place the text \texttt{"Axis\ 2"} on the right hand axis of an existing plot, I would type:

\vspace{0.1in}

\begin{Shaded}
\begin{Highlighting}[]
\FunctionTok{mtext}\NormalTok{(}\StringTok{"Axis 2"}\NormalTok{, }\DecValTok{4}\NormalTok{)}
\end{Highlighting}
\end{Shaded}

\section{Geometric Shapes}\label{geometric-shapes}

Geometric shapes can be drawn using a number of functions including \texttt{rect()} (which draws rectangles) and \texttt{polygon()} (which draws other polygons) based on user-supplied vertices. For instance, to place a purple rectangle with vertices at \((1,1)\), \((1,2)\), \((2,2)\), and \((2,1)\), in an existing plot, I would type:

\vspace{0.1in}

\begin{Shaded}
\begin{Highlighting}[]
\FunctionTok{rect}\NormalTok{(}\AttributeTok{xleft =} \DecValTok{0}\NormalTok{, }\AttributeTok{ybottom =} \DecValTok{1}\NormalTok{, }\AttributeTok{xright =} \DecValTok{2}\NormalTok{, }\AttributeTok{ytop =} \DecValTok{2}\NormalTok{, }\AttributeTok{col =} \DecValTok{6}\NormalTok{)}
\end{Highlighting}
\end{Shaded}

See Fig \ref{fig:text5}.

\begin{figure}[H]

{\centering \includegraphics{figure/minimal-text5-1} 

}

\caption{Plot with rectangle overlain.}\label{fig:text5}
\end{figure}

\index[R]{\texttt{rect()}}
\index[R]{\texttt{polygon()}}

\section{\texorpdfstring{\texttt{axis()}}{axis()}}\label{axis}

The function \texttt{axis()} can be used to create new axes on a plot or to customize axis characteristics. Its first argument (\texttt{side}) specifies the side of the plot that the new axis will occupy \texttt{1} = bottom, \texttt{2} =left, \texttt{3} = top, \texttt{4} = right.

For instance, to create a right hand axis I would type:

\vspace{0.1in}

\begin{Shaded}
\begin{Highlighting}[]
\FunctionTok{axis}\NormalTok{(}\DecValTok{4}\NormalTok{)}
\end{Highlighting}
\end{Shaded}

Other important \texttt{axis()} arguments include a vector of axis labels (argument \texttt{labels}), and the locations of labels (argument \texttt{at}).
\index[R]{\texttt{axis()}}

\begin{example}
\protect\hypertarget{exm:exm-axis1}{}\label{exm:exm-axis1}\(\text{}\)\\
Here I create customized axes with rotated, \(x\)-axis labels, using \texttt{axis()} and \texttt{text()} (Fig \ref{fig:figaxis}).
\vspace{0.1in}

\vspace{0.1in}

\begin{Shaded}
\begin{Highlighting}[numbers=left,,]
\FunctionTok{plot}\NormalTok{(}\DecValTok{1}\SpecialCharTok{:}\DecValTok{3}\NormalTok{, }\AttributeTok{type =} \StringTok{"n"}\NormalTok{, }\AttributeTok{axes =}\NormalTok{ F, }\AttributeTok{xlab =} \StringTok{""}\NormalTok{, }\AttributeTok{ylab =} \StringTok{""}\NormalTok{)}
\FunctionTok{axis}\NormalTok{(}\AttributeTok{side =} \DecValTok{2}\NormalTok{, }\AttributeTok{at =} \DecValTok{1}\SpecialCharTok{:}\DecValTok{3}\NormalTok{, }\AttributeTok{col =} \StringTok{"red"}\NormalTok{)}
\FunctionTok{axis}\NormalTok{(}\AttributeTok{side =} \DecValTok{1}\NormalTok{, }\AttributeTok{at =} \DecValTok{1}\SpecialCharTok{:}\DecValTok{3}\NormalTok{, }\AttributeTok{labels =} \ConstantTok{FALSE}\NormalTok{, }\AttributeTok{col =} \StringTok{"blue"}\NormalTok{)}
\FunctionTok{text}\NormalTok{(}\DecValTok{1}\SpecialCharTok{:}\DecValTok{3}\NormalTok{, }\FunctionTok{rep}\NormalTok{(.}\DecValTok{65}\NormalTok{, }\DecValTok{3}\NormalTok{), }\FunctionTok{c}\NormalTok{(}\StringTok{"Label 1"}\NormalTok{, }\StringTok{"Label 2"}\NormalTok{, }\StringTok{"Label 3"}\NormalTok{), }
     \AttributeTok{srt =} \DecValTok{50}\NormalTok{, }\AttributeTok{xpd =} \ConstantTok{TRUE}\NormalTok{)}
\end{Highlighting}
\end{Shaded}

\begin{figure}[H]

{\centering \includegraphics{figure/minimal-figaxis-1} 

}

\caption{Modifying axes with \texttt{axis()}.}\label{fig:figaxis}
\end{figure}

The argument \texttt{srt\ =\ 50} (Line 5) rotates the text 50 degrees (\texttt{srt} cannot be specified in \texttt{mtext()} or \texttt{axis()}, hence the use of \texttt{text()} here). The specification \texttt{xpd\ =\ TRUE} in \texttt{text()} (Line 5) allows text printing to extend to the plot axis margins.
\end{example}

\(\blacksquare\)

\section{Font Typefaces}\label{font-typefaces}

Font \emph{typefaces}\index[terms]{R!typefaces} can be changed using a number of graphical functions, including \texttt{par()}, via the argument \texttt{family}. The general typeface families: \texttt{"serif"}, \texttt{"mono"}, and \texttt{"sans"}, and the \href{http://en.wikipedia.org/wiki/Hershey_font}{Hershey} family of fonts (type \texttt{?Hershey} for more information) are transferable across all graphics devices employed in \texttt{R}. To change the font in a graphical device from the default sans serif (similar to Arial) to serif (similar to Times New Roman) one
could type:

\vspace{0.1in}

\begin{Shaded}
\begin{Highlighting}[]
\FunctionTok{par}\NormalTok{(}\AttributeTok{family =} \StringTok{"serif"}\NormalTok{)}
\end{Highlighting}
\end{Shaded}

To use a Courier-type monospace font one would use.

\vspace{0.1in}

\begin{Shaded}
\begin{Highlighting}[]
\FunctionTok{par}\NormalTok{(}\AttributeTok{family =} \StringTok{"mono"}\NormalTok{)}
\end{Highlighting}
\end{Shaded}

Many other typeface families are possible, although they may not be transportable to all graphical devices and graphical storage formats.

\begin{example}
\protect\hypertarget{exm:exm-ff1}{}\label{exm:exm-ff1}

\(\text{}\)\\
In the code below I bring in a large number of conventional font families using a function from the Foundational and Applied Statistics for R \href{http://http//www2.cose.isu.edu/~ahoken/book/}{website}. These typefaces (and many others) will typically be available on Windows platform machines, although not all will be supported by non-Windows graphics devices. The result can be seen in Figure \ref{fig:typef} which displays text from ninety-nine Windows typefaces.

\vspace{0.1in}

\begin{Shaded}
\begin{Highlighting}[numbers=left,,]
\FunctionTok{source}\NormalTok{(}\FunctionTok{url}\NormalTok{(}\StringTok{"http://www2.cose.isu.edu/\textasciitilde{}ahoken/book/win\_fonts.R"}\NormalTok{))}
\FunctionTok{png}\NormalTok{(}\StringTok{"fonts.png"}\NormalTok{, }\AttributeTok{res =} \DecValTok{72} \SpecialCharTok{*} \DecValTok{6}\NormalTok{, }\AttributeTok{height =} \DecValTok{480} \SpecialCharTok{*} \DecValTok{6}\NormalTok{, }\AttributeTok{width =} \DecValTok{480} \SpecialCharTok{*} \DecValTok{6}\NormalTok{)}
\NormalTok{  x }\OtherTok{\textless{}{-}} \FunctionTok{rep}\NormalTok{(}\FunctionTok{c}\NormalTok{(}\FloatTok{2.8}\NormalTok{, }\FloatTok{6.4}\NormalTok{, }\FloatTok{9.6}\NormalTok{), }\AttributeTok{each =} \DecValTok{33}\NormalTok{)}
\NormalTok{  y }\OtherTok{\textless{}{-}} \FunctionTok{rep}\NormalTok{(}\FunctionTok{seq}\NormalTok{(}\DecValTok{10}\NormalTok{, }\FloatTok{0.25}\NormalTok{, }\AttributeTok{by =} \SpecialCharTok{{-}}\NormalTok{.}\DecValTok{2965}\NormalTok{), }\DecValTok{3}\NormalTok{)}
\NormalTok{  font.type }\OtherTok{\textless{}{-}} \FunctionTok{paste}\NormalTok{(}\FunctionTok{rep}\NormalTok{(}\StringTok{"f"}\NormalTok{, }\FunctionTok{length}\NormalTok{(fonts)), }\DecValTok{1}\SpecialCharTok{:}\FunctionTok{length}\NormalTok{(fonts), }\AttributeTok{sep =} \StringTok{""}\NormalTok{)}
  \FunctionTok{par}\NormalTok{(}\AttributeTok{mar =} \FunctionTok{c}\NormalTok{(}\FloatTok{0.1}\NormalTok{,}\FloatTok{0.1}\NormalTok{,}\FloatTok{0.1}\NormalTok{,}\FloatTok{0.1}\NormalTok{), }\AttributeTok{cex =} \FloatTok{1.05}\NormalTok{)}
  \FunctionTok{plot}\NormalTok{(}\DecValTok{0}\SpecialCharTok{:}\DecValTok{10}\NormalTok{, }\AttributeTok{type =} \StringTok{"n"}\NormalTok{, }\AttributeTok{xaxt=} \StringTok{"n"}\NormalTok{, }\AttributeTok{yaxt =} \StringTok{"n"}\NormalTok{, }\AttributeTok{xlab =} \StringTok{""}\NormalTok{, }\AttributeTok{ylab =} \StringTok{""}\NormalTok{, }
       \AttributeTok{bty =} \StringTok{"n"}\NormalTok{)}

  \ControlFlowTok{for}\NormalTok{(i }\ControlFlowTok{in} \DecValTok{1}\SpecialCharTok{:}\FunctionTok{length}\NormalTok{(fonts))\{}
    \FunctionTok{text}\NormalTok{(x[i],y[i], }\AttributeTok{labels=}\NormalTok{fonts[i] , }\AttributeTok{family =}\NormalTok{ font.type[i])}
\NormalTok{  \}}
\FunctionTok{dev.off}\NormalTok{()}
\end{Highlighting}
\end{Shaded}

\begin{figure}[H]

{\centering \includegraphics[width=0.8\linewidth,]{fonts} 

}

\caption{Examples of font families that can be used in \textbf{R} graphics.}\label{fig:typef}
\end{figure}

Note that on line 2 in the code above, I use the function \texttt{png()} to generate a high resolution .png graphical file. Thus, running the entirety of the preceding code chunk will create the image file \texttt{fonts.tiff} in your working directory. To save myself from typing an inordinate amount of code, I use a for loop (see Ch \ref{ch8}) to place the fonts one at a time in the graphics device (lines 9-11). Output from closing the graphical device is shown on line 14-15.

Importantly, the typefaces imported from the first line of code in the chunk above will now be available for graphics functions using the Windows graphical device. To see the first six available Windows fonts one can type:

\vspace{0.1in}

\begin{Shaded}
\begin{Highlighting}[]
\FunctionTok{head}\NormalTok{(}\FunctionTok{windowsFonts}\NormalTok{())}
\end{Highlighting}
\end{Shaded}

\begin{verbatim}
$serif
[1] "TT Times New Roman"

$sans
[1] "TT Arial"

$mono
[1] "TT Courier New"

$f1
[1] "Agency FB"

$f2
[1] "Albany AMT"

$f3
[1] "ALGERIAN"
\end{verbatim}

Similarly, one can see the available fonts for PostScript and PDF graphics devices using:

\vspace{0.1in}

\begin{Shaded}
\begin{Highlighting}[]
\FunctionTok{head}\NormalTok{(}\FunctionTok{names}\NormalTok{(}\FunctionTok{pdfFonts}\NormalTok{()))}
\end{Highlighting}
\end{Shaded}

\begin{verbatim}
[1] "serif"      "sans"       "mono"       "AvantGarde" "Bookman"   
[6] "Courier"   
\end{verbatim}

\end{example}

\(\blacksquare\)

\index[R]{\texttt{source()}}
\index[R]{\texttt{png()}}
\index[R]{\texttt{WindowsFonts()}}
\index[R]{\texttt{pdfFonts()}}

\section{Colors}\label{colors}

An enormous number of color choices for \textbf{R} graphics are possible and these can be specified in at least six different ways.

\begin{itemize}
\tightlist
\item
  First, we can specify colors with integers as I did in Figure \ref{fig:sp1}. Additional varieties can be obtained by drawing color elements from the function \texttt{colors()} using \texttt{colors(){[}number{]}} (Fig \ref{fig:col1}).
  \index[R]{\texttt{colors()}}
  \index[terms]{R!graphics!colors}
\end{itemize}

\vspace{0.1in}

\begin{Shaded}
\begin{Highlighting}[]
\NormalTok{e }\OtherTok{\textless{}{-}} \FunctionTok{expand.grid}\NormalTok{(}\DecValTok{1}\SpecialCharTok{:}\DecValTok{20}\NormalTok{, }\DecValTok{1}\SpecialCharTok{:}\DecValTok{32}\NormalTok{)}
\FunctionTok{plot}\NormalTok{(e[,}\DecValTok{1}\NormalTok{], e[,}\DecValTok{2}\NormalTok{], }\AttributeTok{bg =} \FunctionTok{colors}\NormalTok{()[}\DecValTok{1}\SpecialCharTok{:}\DecValTok{640}\NormalTok{], }\AttributeTok{pch =} \DecValTok{22}\NormalTok{, }\AttributeTok{cex =} \FloatTok{2.5}\NormalTok{, }\AttributeTok{xaxt =} \StringTok{"n"}\NormalTok{, }
     \AttributeTok{yaxt =} \StringTok{"n"}\NormalTok{, }\AttributeTok{xlab =} \StringTok{""}\NormalTok{, }\AttributeTok{ylab =} \StringTok{""}\NormalTok{)}
\FunctionTok{text}\NormalTok{(e[,}\DecValTok{1}\NormalTok{], e[,}\DecValTok{2}\NormalTok{], }\DecValTok{1}\SpecialCharTok{:}\DecValTok{640}\NormalTok{, }\AttributeTok{cex =}\NormalTok{ .}\DecValTok{4}\NormalTok{)}
\end{Highlighting}
\end{Shaded}

\begin{figure}[H]

{\centering \includegraphics{figure/minimal-col1-1} 

}

\caption{Color choices from \texttt{colors()}}\label{fig:col1}
\end{figure}

The function \texttt{expand.grid()} creates a dataframe from all combinations of user-supplied supplied vectors. Note that these combinations are used as coordinates in \texttt{plot()}.
\index[R]{\texttt{expand.grid()}}

\begin{itemize}
\item
  Second, we can specify colors using actual color names, e.g., \texttt{"white"}, \texttt{"red"}, \texttt{"yellow"}. For a visual display of essentially all the available named colors in \textbf{R} type: \texttt{example(colors)}.
\item
  Third, we can define colors by requesting red green and blue (RGB) color intensities, along with transparency, using the function \texttt{rgb()} (Fig \ref{fig:rgb}). Usable light intensities can be made to vary individually from 0 to 255 (i.e., within an 8 bit format). Thus, there are \(255^4 = 4,228,250,625\) possible \texttt{rgb()} color combinations. By default, \texttt{red} \texttt{green}, \texttt{blue}, and \texttt{alpha} (transparency) arguments in \texttt{rgb()} are defined to be in (0, 1).
  \index[R]{\texttt{rgb()}}
\end{itemize}

\vspace{0.1in}

\begin{Shaded}
\begin{Highlighting}[numbers=left,,]
\FunctionTok{plot}\NormalTok{(}\DecValTok{1}\SpecialCharTok{:}\DecValTok{10}\NormalTok{, }\AttributeTok{cex =} \DecValTok{15}\NormalTok{, }\AttributeTok{pch =} \DecValTok{19}\NormalTok{, }\AttributeTok{xlab =} \StringTok{""}\NormalTok{, }\AttributeTok{ylab =} \StringTok{""}\NormalTok{,}
     \AttributeTok{col =} \FunctionTok{rgb}\NormalTok{(}\AttributeTok{red =} \FunctionTok{rep}\NormalTok{(}\FloatTok{0.2}\NormalTok{,}\DecValTok{10}\NormalTok{), }\AttributeTok{green =} \FunctionTok{rep}\NormalTok{(}\FloatTok{0.5}\NormalTok{,}\DecValTok{10}\NormalTok{),}
               \AttributeTok{blue =} \FunctionTok{rep}\NormalTok{(}\FloatTok{0.8}\NormalTok{,}\DecValTok{10}\NormalTok{),}
               \AttributeTok{alpha =} \FunctionTok{seq}\NormalTok{(}\FloatTok{0.05}\NormalTok{,}\DecValTok{1}\NormalTok{, }\AttributeTok{length =} \DecValTok{10}\NormalTok{)), }\AttributeTok{axes =}\NormalTok{ F)}
\FunctionTok{box}\NormalTok{()}
\end{Highlighting}
\end{Shaded}

\begin{figure}[H]

{\centering \includegraphics{figure/minimal-rgb-1} 

}

\caption{Demonstration of \texttt{rgb()}, emphasizing changes in transparency.}\label{fig:rgb}
\end{figure}

Note the use of \texttt{box()} on line 5, which places a box around the plot.

\index[R]{\texttt{box()}}

\begin{itemize}
\item
  Fourth, similar to \texttt{rgb()}, we can specify colors using the function \texttt{hcl()} which controls hues, chroma, and luminescence and transparency (see Fig \ref{fig:col2}).
\item
  Fifth, we can define colors using \emph{hexadecimal codes}\footnote{A data coding system that uses 16 symbols: the numbers 1-9, and the letters A-F. Hexadecimals are primarily used to provide a more intuitive representation of binary-coded values (see Ch \ref{ch12}).}, e.g., blue = \texttt{"\#0000FF"}.
  \index[terms]{Hexadecimal}
\item
  Sixth, we can specify colors using \emph{palettes}. Figure \ref{fig:col2} shows six pie plots. Each pie plot uses a different pre-defined color palette. Each pie slice from each pie represents a distinct segment of a distinct palette.
  \index[terms]{R!graphics!color palettes}
\end{itemize}

\vspace{0.1in}

\begin{Shaded}
\begin{Highlighting}[numbers=left,,]
\FunctionTok{layout}\NormalTok{(}\FunctionTok{matrix}\NormalTok{(}\FunctionTok{seq}\NormalTok{(}\DecValTok{1}\NormalTok{,}\DecValTok{6}\NormalTok{),}\DecValTok{3}\NormalTok{,}\DecValTok{2}\NormalTok{))}
\FunctionTok{par}\NormalTok{(}\AttributeTok{mar=}\FunctionTok{c}\NormalTok{(}\DecValTok{1}\NormalTok{,}\DecValTok{1}\NormalTok{,}\DecValTok{1}\NormalTok{,}\DecValTok{1}\NormalTok{))}
\FunctionTok{pie}\NormalTok{(}\FunctionTok{rep}\NormalTok{(}\DecValTok{1}\NormalTok{,}\DecValTok{12}\NormalTok{), }\AttributeTok{col =} \FunctionTok{rainbow}\NormalTok{(}\DecValTok{12}\NormalTok{), }\AttributeTok{main =} \StringTok{"Rainbow colors"}\NormalTok{)}
\FunctionTok{pie}\NormalTok{(}\FunctionTok{rep}\NormalTok{(}\DecValTok{1}\NormalTok{,}\DecValTok{12}\NormalTok{), }\AttributeTok{col =} \FunctionTok{heat.colors}\NormalTok{(}\DecValTok{12}\NormalTok{), }\AttributeTok{main =} \StringTok{"Heat colors"}\NormalTok{)}
\FunctionTok{pie}\NormalTok{(}\FunctionTok{rep}\NormalTok{(}\DecValTok{1}\NormalTok{,}\DecValTok{12}\NormalTok{), }\AttributeTok{col =} \FunctionTok{topo.colors}\NormalTok{(}\DecValTok{12}\NormalTok{), }\AttributeTok{main =} \StringTok{"Topographic colors"}\NormalTok{)}
\FunctionTok{pie}\NormalTok{(}\FunctionTok{rep}\NormalTok{(}\DecValTok{1}\NormalTok{,}\DecValTok{12}\NormalTok{), }\AttributeTok{col =} \FunctionTok{gray}\NormalTok{(}\FunctionTok{seq}\NormalTok{(}\DecValTok{0}\NormalTok{,}\DecValTok{1}\NormalTok{,}\DecValTok{1}\SpecialCharTok{/}\DecValTok{12}\NormalTok{)), }\AttributeTok{main =} \StringTok{"Gray colors"}\NormalTok{)}
\FunctionTok{pie}\NormalTok{(}\FunctionTok{rep}\NormalTok{(}\DecValTok{1}\NormalTok{,}\DecValTok{12}\NormalTok{), }\AttributeTok{col =} \FunctionTok{hcl}\NormalTok{(}\AttributeTok{h=}\FunctionTok{seq}\NormalTok{(}\DecValTok{180}\NormalTok{,}\DecValTok{0}\NormalTok{, }\AttributeTok{length=}\DecValTok{12}\NormalTok{)), }
    \AttributeTok{main =} \StringTok{"Cols from hcl hue"}\NormalTok{)}
\FunctionTok{pie}\NormalTok{(}\FunctionTok{rep}\NormalTok{(}\DecValTok{1}\NormalTok{,}\DecValTok{12}\NormalTok{), }\AttributeTok{col =} \FunctionTok{hcl}\NormalTok{(}\AttributeTok{h=}\FunctionTok{seq}\NormalTok{(}\DecValTok{360}\NormalTok{,}\DecValTok{180}\NormalTok{,}\AttributeTok{length=}\DecValTok{12}\NormalTok{)), }
    \AttributeTok{main =} \StringTok{"Cols from chroma"}\NormalTok{)}
\end{Highlighting}
\end{Shaded}

\begin{figure}[H]

{\centering \includegraphics{figure/minimal-col2-1} 

}

\caption{Examples of color palettes in \textbf{R}. Numbers do not correspond to actual color type designations.}\label{fig:col2}
\end{figure}

Note that the functions \texttt{rainbow()}, \texttt{heat.colors()}, \texttt{topo.colors()} (Lines 3-6) only require an integer specification requesting the number of colors within a particular palette. For instance

\vspace{0.1in}

\begin{Shaded}
\begin{Highlighting}[]
\FunctionTok{rainbow}\NormalTok{(}\DecValTok{5}\NormalTok{)}
\end{Highlighting}
\end{Shaded}

\begin{verbatim}
[1] "#FF0000" "#CCFF00" "#00FF66" "#0066FF" "#CC00FF"
\end{verbatim}

Note that the colors in \texttt{rainbow()} are given in a hexidecimal format.

The function \texttt{palette()} can be used to check and define a number of useful palettes. Colors in the \emph{current} palette can be obtained by typing:

\vspace{0.1in}

\begin{Shaded}
\begin{Highlighting}[]
\FunctionTok{palette}\NormalTok{()}
\end{Highlighting}
\end{Shaded}

\begin{verbatim}
[1] "black"   "#DF536B" "#61D04F" "#2297E6" "#28E2E5" "#CD0BBC" "#F5C710"
[8] "gray62" 
\end{verbatim}

A list of predefined palettes in \texttt{palette()} can be obtained by typing:

\vspace{0.1in}

\begin{Shaded}
\begin{Highlighting}[]
\FunctionTok{palette.pals}\NormalTok{()}
\end{Highlighting}
\end{Shaded}

\begin{verbatim}
 [1] "R3"              "R4"              "ggplot2"        
 [4] "Okabe-Ito"       "Accent"          "Dark 2"         
 [7] "Paired"          "Pastel 1"        "Pastel 2"       
[10] "Set 1"           "Set 2"           "Set 3"          
[13] "Tableau 10"      "Classic Tableau" "Polychrome 36"  
[16] "Alphabet"       
\end{verbatim}

To define the current palette to be the one used by the \emph{ggplot2} package (Ch \ref{ch7}), I could type: \texttt{palette("ggplot2")}.

\index[R]{\texttt{palette()}}
\index[R]{\texttt{palette.pals()}}
A large number of useful pre-defined palettes (including color-blind-safe palettes) can be obtained using the package \emph{RColorBrewer} (Fig \ref{fig:col3}).\index[terms]{R!package!\textit{RColorBrewer}}

\vspace{0.1in}

\begin{Shaded}
\begin{Highlighting}[]
\FunctionTok{library}\NormalTok{(RColorBrewer)}
\FunctionTok{display.brewer.all}\NormalTok{(}\AttributeTok{n =} \DecValTok{7}\NormalTok{, }\AttributeTok{colorblindFriendly =} \ConstantTok{TRUE}\NormalTok{)}
\end{Highlighting}
\end{Shaded}

\begin{figure}[H]

{\centering \includegraphics{figure/minimal-col3-1} 

}

\caption{\textit{RColorBrewer} color-blind-safe, seven category palettes. Top palettes are so-called 'sequential' palettes, middle palettes are 'qualitative', and bottom palettes are 'divergent'.}\label{fig:col3}
\end{figure}

Here are the hexadecimal names for the ``Set2'' palette chunks in Figure \ref{fig:col3}.

\vspace{0.1in}

\begin{Shaded}
\begin{Highlighting}[]
\FunctionTok{brewer.pal}\NormalTok{(}\DecValTok{7}\NormalTok{, }\StringTok{"Set2"}\NormalTok{)}
\end{Highlighting}
\end{Shaded}

\begin{verbatim}
[1] "#66C2A5" "#FC8D62" "#8DA0CB" "#E78AC3" "#A6D854" "#FFD92F" "#E5C494"
\end{verbatim}

Customized palettes can be generated using the \texttt{colorRamp()} function which returns functions that ``interpolate a set of given colors to create new color palettes.'' Important \texttt{colorRampPaltette()} arguments include:
two required arguments.

\begin{itemize}
\tightlist
\item
  \texttt{colors} defines colors to interpolate.
\item
  \texttt{bias} a positive number that controls distinctions among interpolated colors. Larger values indicated greater differences.
\item
  \texttt{space} one of \texttt{"RGB"} or \texttt{"Lab"}, indicating whether RGB or CIELAB\footnote{The CIELAB color space is defined by three values: \texttt{L*} for perceptual lightness and \texttt{a*} and \texttt{b*} for the four unique colors of human vision: red, green, blue and yellow. \citep{cie}. The CIELAB space is intended to be \emph{perceptually uniform}. CIELAB and several other colors spaces are included in the encompassing CIECAM02 color space \citep{wikicielab}.} color spaces are to be used in interpolations.
\end{itemize}

Here I generate and plot a 15 color palette interpolated from the colors red and blue (Fig \ref{fig:crp}).

\vspace{0.1in}

\begin{Shaded}
\begin{Highlighting}[numbers=left,,]
\NormalTok{crp }\OtherTok{\textless{}{-}} \FunctionTok{colorRampPalette}\NormalTok{(}\AttributeTok{colors =} \FunctionTok{c}\NormalTok{(}\StringTok{"red"}\NormalTok{, }\StringTok{"blue"}\NormalTok{))(}\DecValTok{15}\NormalTok{)}
\FunctionTok{plot}\NormalTok{(}\DecValTok{1}\SpecialCharTok{:}\DecValTok{15}\NormalTok{, }\AttributeTok{pch =} \DecValTok{19}\NormalTok{, }\AttributeTok{cex =} \DecValTok{5}\NormalTok{, }\AttributeTok{col =}\NormalTok{ crp, }\AttributeTok{axes =}\NormalTok{ F, }\AttributeTok{xlab =} \StringTok{""}\NormalTok{, }\AttributeTok{ylab =} \StringTok{""}\NormalTok{)}
\FunctionTok{box}\NormalTok{()}
\end{Highlighting}
\end{Shaded}

\begin{figure}[H]

{\centering \includegraphics{figure/minimal-crp-1} 

}

\caption{Color palette generated by the function \texttt{colorRampPalette()}.}\label{fig:crp}
\end{figure}

The function \texttt{box()} (Line 3) places a box around the figure whose axes and axis labels I have intentionally omitted.
\index[R]{\texttt{box()}}
\index[R]{\texttt{colorRampPalette()}}
There are a number of packages for the generation of customized palettes. My current favorite is \emph{colorspace}\index[terms]{R!package!\textit{colorspace}} and its interactive function \texttt{hclwizard()}\index[R]{\texttt{colorspace::hclwizard()}}, which generates the \emph{shiny} GUI (Ch \ref{ch11}) shown in Fig \ref{fig:colspace}.

\vspace{0.1in}

\begin{figure}[H]

{\centering \includegraphics[width=0.75\linewidth,]{figs6/colGUI} 

}

\caption{A GUI for constructed customized palettes generating by the function \texttt{colorspace::hclwizard()}.}\label{fig:colspace}
\end{figure}

\section{Scatterplots}\label{scatterplots}

\index[terms]{R!graphics!scatterplots}
\emph{Scatterplots} project points at the intersection of paired observations describing two quantitative variables. Thus, scatterplots are often presented in conjunction with simple regression analyses \citep{aho2014}.

\begin{example}
\protect\hypertarget{exm:exm-sp1}{}\label{exm:exm-sp1}\(\text{}\)\\
As an example of scatterplot usage we will use the \texttt{Loblolly} dataset in the package \emph{datasets}. Figure \ref{fig:scat1} allows visualization of the relationship of loblolly pine tree age and tree height.

\vspace{0.1in}

\begin{Shaded}
\begin{Highlighting}[]
\FunctionTok{with}\NormalTok{(Loblolly, }\FunctionTok{plot}\NormalTok{(age, height))}
\end{Highlighting}
\end{Shaded}

\begin{figure}[H]

{\centering \includegraphics{figure/minimal-scat1-1} 

}

\caption{Scatterplot of \texttt{height} and \texttt{age} from the \texttt{Loblolly} pine tree dataset.}\label{fig:scat1}
\end{figure}

Now let's fit a simple linear regression for loblolly pine height as a function of age. A regression line will show the best possible linear fit for a response variable as a function of an quantitative explanatory variable \citep{aho2014}. The \textbf{R} function for a linear model is \texttt{lm()}. It encompasses and allows a huge number of statistical procedures, including regression (see Chs. 9-11 in \citep{aho2014}). We have:

\vspace{0.1in}

\begin{Shaded}
\begin{Highlighting}[]
\NormalTok{ha.lm }\OtherTok{\textless{}{-}} \FunctionTok{lm}\NormalTok{(height }\SpecialCharTok{\textasciitilde{}}\NormalTok{ age, }\AttributeTok{data =}\NormalTok{ Loblolly)}
\end{Highlighting}
\end{Shaded}

Note that in the first argument of \texttt{lm()} we define \texttt{height} to be a function of \texttt{age} using the tilde operator. Objects of class \texttt{lm} have their own\texttt{summary} function. This can be called by simply typing:

\vspace{0.1in}

\begin{Shaded}
\begin{Highlighting}[]
\FunctionTok{summary}\NormalTok{(ha.lm)}
\end{Highlighting}
\end{Shaded}

\begin{verbatim}

Call:
lm(formula = height ~ age, data = Loblolly)

Residuals:
   Min     1Q Median     3Q    Max 
-7.021 -2.167 -0.439  2.054  6.855 

Coefficients:
            Estimate Std. Error t value Pr(>|t|)    
(Intercept)  -1.3124     0.6218   -2.11    0.038 *  
age           2.5905     0.0409   63.27   <2e-16 ***
---
Signif. codes:  0 '***' 0.001 '**' 0.01 '*' 0.05 '.' 0.1 ' ' 1

Residual standard error: 2.95 on 82 degrees of freedom
Multiple R-squared:  0.98,  Adjusted R-squared:  0.98 
F-statistic: 4e+03 on 1 and 82 DF,  p-value: <2e-16
\end{verbatim}

The output shows us the \emph{Y}-intercept, -1.31240, and slope, 2.59052, of the fitted regression line, and results from null hypothesis tests, along with a lot of other information.

The \texttt{abline()} function allows the plotting of a line over an existing plot. The first two arguments for \texttt{abline()} are the \emph{Y}-intercept and slope (Fig \ref{fig:scat2}).

\vspace{0.1in}

\begin{Shaded}
\begin{Highlighting}[]
\FunctionTok{with}\NormalTok{(Loblolly, }\FunctionTok{plot}\NormalTok{(age,height, }\AttributeTok{pch=}\DecValTok{2}\NormalTok{, }\AttributeTok{col=}\DecValTok{3}\NormalTok{))}
\FunctionTok{abline}\NormalTok{(}\SpecialCharTok{{-}}\FloatTok{1.312396}\NormalTok{, }\FloatTok{2.590523}\NormalTok{)}
\end{Highlighting}
\end{Shaded}

\begin{figure}[H]

{\centering \includegraphics{figure/minimal-scat2-1} 

}

\caption{Scatterplot with fit overlain.}\label{fig:scat2}
\end{figure}

Note that we could have gotten the same result using \texttt{abline(ha.lm)}.

Finally, we can overlay a 95\% confidence interval for the true regression fitted value (see \citet{aho2014}, Ch 9) using the function \texttt{predict.lm()} (Fig \ref{fig:scat3})

\vspace{0.1in}

\begin{Shaded}
\begin{Highlighting}[numbers=left,,]
\NormalTok{ci }\OtherTok{\textless{}{-}} \FunctionTok{predict}\NormalTok{(ha.lm, }\AttributeTok{interval =} \StringTok{"confidence"}\NormalTok{)}

\NormalTok{o }\OtherTok{\textless{}{-}} \FunctionTok{order}\NormalTok{(Loblolly}\SpecialCharTok{$}\NormalTok{age)}
\NormalTok{ageo }\OtherTok{\textless{}{-}}\NormalTok{ Loblolly}\SpecialCharTok{$}\NormalTok{age[o]}
\NormalTok{cio }\OtherTok{\textless{}{-}}\NormalTok{ ci[o,]}

\FunctionTok{with}\NormalTok{(Loblolly, }\FunctionTok{plot}\NormalTok{(age, height, }\AttributeTok{pch=}\DecValTok{19}\NormalTok{, }\AttributeTok{col=}\DecValTok{1}\NormalTok{))}
\FunctionTok{abline}\NormalTok{(}\SpecialCharTok{{-}}\FloatTok{1.312396}\NormalTok{, }\FloatTok{2.590523}\NormalTok{)}
\FunctionTok{points}\NormalTok{(ageo, cio[,}\DecValTok{2}\NormalTok{], }\AttributeTok{type =} \StringTok{"l"}\NormalTok{, }\AttributeTok{col =} \StringTok{"gray"}\NormalTok{) }\CommentTok{\# lower CI bound}
\FunctionTok{points}\NormalTok{(ageo, cio[,}\DecValTok{3}\NormalTok{], }\AttributeTok{type =} \StringTok{"l"}\NormalTok{, }\AttributeTok{col =} \StringTok{"gray"}\NormalTok{) }\CommentTok{\# upper CI bound}
\end{Highlighting}
\end{Shaded}

\begin{figure}[H]

{\centering \includegraphics{figure/minimal-scat3-1} 

}

\caption{Scatterplot with confidence interval for the true mean of $y$ given $x$, overlain.}\label{fig:scat3}
\end{figure}

The object \texttt{ci}, created on Line 1, is a dataframe containing fits, and corresponding lower and upper confidence interval bounds. The ordering of \(x\)-axis values is established on Line 3 to allow creation of lines that look like functions of \(x\). This ordering is applied to Cartesian coordinates on Lines 4-5.
\end{example}

\(\blacksquare\)
\index[R]{\texttt{abline()}}
\index[R]{\texttt{lm()}}
\index[R]{\texttt{predict.lm()}}

\section{Transformations}\label{transformations}

\index[terms]{Transformation (function)}
Importantly, \texttt{plot()} allows straightforward application of log transformations to axes. For instance, to apply a \(\log_e\) transformation to the \(x\)-axis or \(y\)-axis I could use \texttt{log\ =\ "x"} or \texttt{log\ =\ "y"}, respectively (Fig \ref{fig:logt}).

\vspace{0.1in}

\begin{Shaded}
\begin{Highlighting}[]
\FunctionTok{with}\NormalTok{(Loblolly, }\FunctionTok{plot}\NormalTok{(age, height, }\AttributeTok{log =} \StringTok{"y"}\NormalTok{))}
\end{Highlighting}
\end{Shaded}

\begin{figure}[H]

{\centering \includegraphics{figure/minimal-logt-1} 

}

\caption{Graphical $\log_e$ transformation of the \texttt{height} axis from a scatterplot of the \texttt{Loblolly} pine tree dataset.}\label{fig:logt}
\end{figure}

\section{Multiple Plots}\label{multiple-plots}

As shown earlier, multiple graphs can be placed in a graphics device using the \texttt{mfrow} or \texttt{mfcol} argument in \texttt{par()}. In this section I will try to incorporate a number of the functions discussed so far, including \texttt{plot()}, \texttt{par()}, mathematical formulae with \texttt{text()}, \texttt{points()}, shapes rending with \texttt{rect()}, \texttt{colors()} \texttt{axis()}, and \texttt{mtext()}.

\begin{example}
\protect\hypertarget{exm:exm-multi}{}\label{exm:exm-multi}

\(\text{}\)\\
The dataframe \texttt{C.isotope} in package \emph{asbio} describes variations in \(\delta ^{14}\)C over time in La Jolla California. The term \(\delta ^{14}C\) describes the ratio of carbon 14 to carbon 12 (\(^{14}\)C is unstable, while \(^{12}\)C is a stable isotope of carbon) compared to a standard ratio. We will create a figure with four subplots, with the following characteristics:

\begin{itemize}
\tightlist
\item
  It will have dimensions 8'' x 7''.
\item
  The outer margins (in number of lines) will be bottom = 0.1, left = 0.1, top = 0, right = 0.
\item
  The inner margins (for each plot) will be bottom = 4, left = 4.4, top = 2, right = 2. The plot margins will be light gray. We can specify gray gradations with the function.
\item
  The first plot will show \(\delta ^{14}\)C as a function of date. The plotting area will be dark gray, i.e., \texttt{colors(){[}181{]}}. Points will be white circles with a black border.
\item
  The second plot will be a line plot of atmospheric carbon as a function of date. It will have a light green plotting area: \texttt{colors(){[}363{]}}.
\item
  The third plot will be a scatterplot of \(\delta ^{14}\)C as a function of atmospheric C. Points will be yellow circles with a black border. The plotting area will be light red: \texttt{colors(){[}580{]}}.
\item
  The fourth plot will show the sample variance for atmospheric carbon in the time series. It will have a custom (albeit meaningless) axis, created with \texttt{axis()}, with the labels: \texttt{a}, \texttt{b}, \texttt{c}, and \texttt{d}. It will also have a horizontal axis label inserted with \texttt{mtext()}.
\end{itemize}

The result is shown in Fig \ref{fig:multi1}.

\vspace{0.1in}

\begin{Shaded}
\begin{Highlighting}[numbers=left,,]
\FunctionTok{library}\NormalTok{(asbio)}
\FunctionTok{data}\NormalTok{(C.isotope)}
\FunctionTok{dev.new}\NormalTok{(}\AttributeTok{height =} \DecValTok{8}\NormalTok{, }\AttributeTok{width =} \DecValTok{7}\NormalTok{)}
\NormalTok{op }\OtherTok{\textless{}{-}} \FunctionTok{par}\NormalTok{(}\AttributeTok{mfrow =} \FunctionTok{c}\NormalTok{(}\DecValTok{2}\NormalTok{, }\DecValTok{2}\NormalTok{), }\AttributeTok{oma =} \FunctionTok{c}\NormalTok{(}\FloatTok{0.1}\NormalTok{, }\FloatTok{0.1}\NormalTok{, }\DecValTok{0}\NormalTok{, }\DecValTok{0}\NormalTok{), }\AttributeTok{mar =} \FunctionTok{c}\NormalTok{(}\DecValTok{4}\NormalTok{, }\FloatTok{4.4}\NormalTok{, }\DecValTok{2}\NormalTok{, }\DecValTok{2}\NormalTok{),}
          \AttributeTok{bg =} \FunctionTok{gray}\NormalTok{(.}\DecValTok{97}\NormalTok{))}
\CommentTok{\#{-}{-}{-}{-}{-}{-}{-}{-}{-}{-}{-}{-}{-}{-}{-}{-}{-}{-}{-}{-}{-}{-}{-}{-}{-}{-}{-}{-}{-}{-}{-}{-} plot 1 {-}{-}{-}{-}{-}{-}{-}{-}{-}{-}{-}{-}{-}{-}{-}{-}{-}{-}{-}{-}{-}{-}{-}{-}{-}{-}{-}{-}{-}{-}{-}\#}
\FunctionTok{with}\NormalTok{(C.isotope, }\FunctionTok{plot}\NormalTok{(Decimal.date, D14C, }\AttributeTok{xlab =} \StringTok{"Date"}\NormalTok{, }\AttributeTok{ylab =} 
                       \FunctionTok{expression}\NormalTok{(}\FunctionTok{paste}\NormalTok{(delta}\SpecialCharTok{\^{}}\DecValTok{14}\NormalTok{,}\StringTok{"C (per mille)"}\NormalTok{)), }
                     \AttributeTok{type =} \StringTok{"n"}\NormalTok{))}

\FunctionTok{rect}\NormalTok{(}\FunctionTok{par}\NormalTok{(}\StringTok{"usr"}\NormalTok{)[}\DecValTok{1}\NormalTok{], }\FunctionTok{par}\NormalTok{(}\StringTok{"usr"}\NormalTok{)[}\DecValTok{3}\NormalTok{], }\FunctionTok{par}\NormalTok{(}\StringTok{"usr"}\NormalTok{)[}\DecValTok{2}\NormalTok{], }\FunctionTok{par}\NormalTok{(}\StringTok{"usr"}\NormalTok{)[}\DecValTok{4}\NormalTok{],}
     \AttributeTok{col =} \FunctionTok{colors}\NormalTok{()[}\DecValTok{181}\NormalTok{])}

\FunctionTok{with}\NormalTok{(C.isotope, }\FunctionTok{points}\NormalTok{(Decimal.date, D14C, }\AttributeTok{pch =} \DecValTok{21}\NormalTok{, }\AttributeTok{bg =} \StringTok{"white"}\NormalTok{))}

\CommentTok{\#{-}{-}{-}{-}{-}{-}{-}{-}{-}{-}{-}{-}{-}{-}{-}{-}{-}{-}{-}{-}{-}{-}{-}{-}{-}{-}{-}{-}{-}{-}{-}{-} plot 2 {-}{-}{-}{-}{-}{-}{-}{-}{-}{-}{-}{-}{-}{-}{-}{-}{-}{-}{-}{-}{-}{-}{-}{-}{-}{-}{-}{-}{-}{-}{-}\#}
\FunctionTok{with}\NormalTok{(C.isotope, }\FunctionTok{plot}\NormalTok{(Decimal.date, CO2, }\AttributeTok{xlab =} \StringTok{"Date"}\NormalTok{, }\AttributeTok{ylab =}
                       \FunctionTok{expression}\NormalTok{(}\FunctionTok{paste}\NormalTok{(CO[}\DecValTok{2}\NormalTok{],}\StringTok{" (ppm)"}\NormalTok{)), }
                     \AttributeTok{type =} \StringTok{"n"}\NormalTok{))}

\FunctionTok{rect}\NormalTok{(}\FunctionTok{par}\NormalTok{(}\StringTok{"usr"}\NormalTok{)[}\DecValTok{1}\NormalTok{], }\FunctionTok{par}\NormalTok{(}\StringTok{"usr"}\NormalTok{)[}\DecValTok{3}\NormalTok{], }\FunctionTok{par}\NormalTok{(}\StringTok{"usr"}\NormalTok{)[}\DecValTok{2}\NormalTok{], }\FunctionTok{par}\NormalTok{(}\StringTok{"usr"}\NormalTok{)[}\DecValTok{4}\NormalTok{],}
     \AttributeTok{col =} \FunctionTok{colors}\NormalTok{()[}\DecValTok{363}\NormalTok{])}

\FunctionTok{with}\NormalTok{(C.isotope, }\FunctionTok{points}\NormalTok{(Decimal.date, CO2, }\AttributeTok{type =} \StringTok{"l"}\NormalTok{))}

\CommentTok{\#{-}{-}{-}{-}{-}{-}{-}{-}{-}{-}{-}{-}{-}{-}{-}{-}{-}{-}{-}{-}{-}{-}{-}{-}{-}{-}{-}{-}{-}{-}{-}{-} plot 3 {-}{-}{-}{-}{-}{-}{-}{-}{-}{-}{-}{-}{-}{-}{-}{-}{-}{-}{-}{-}{-}{-}{-}{-}{-}{-}{-}{-}{-}{-}{-}\#}
\FunctionTok{with}\NormalTok{(C.isotope, }\FunctionTok{plot}\NormalTok{(CO2, D14C, }\AttributeTok{xlab =} \FunctionTok{expression}\NormalTok{(}\FunctionTok{paste}\NormalTok{(CO[}\DecValTok{2}\NormalTok{], }\StringTok{" (ppm)"}\NormalTok{)), }
                     \AttributeTok{ylab =} \FunctionTok{expression}\NormalTok{(}\FunctionTok{paste}\NormalTok{(delta}\SpecialCharTok{\^{}}\DecValTok{14}\NormalTok{,}\StringTok{"C (per mille)"}\NormalTok{)), }
                     \AttributeTok{type =} \StringTok{"n"}\NormalTok{))}

\FunctionTok{rect}\NormalTok{(}\FunctionTok{par}\NormalTok{(}\StringTok{"usr"}\NormalTok{)[}\DecValTok{1}\NormalTok{], }\FunctionTok{par}\NormalTok{(}\StringTok{"usr"}\NormalTok{)[}\DecValTok{3}\NormalTok{], }\FunctionTok{par}\NormalTok{(}\StringTok{"usr"}\NormalTok{)[}\DecValTok{2}\NormalTok{], }\FunctionTok{par}\NormalTok{(}\StringTok{"usr"}\NormalTok{)[}\DecValTok{4}\NormalTok{],}
     \AttributeTok{col =} \FunctionTok{colors}\NormalTok{()[}\DecValTok{580}\NormalTok{])}

\FunctionTok{with}\NormalTok{(C.isotope, }\FunctionTok{points}\NormalTok{(CO2, D14C, }\AttributeTok{pch =} \DecValTok{21}\NormalTok{, }\AttributeTok{bg =} \StringTok{"yellow"}\NormalTok{))}

\CommentTok{\#{-}{-}{-}{-}{-}{-}{-}{-}{-}{-}{-}{-}{-}{-}{-}{-}{-}{-}{-}{-}{-}{-}{-}{-}{-}{-}{-}{-}{-}{-}{-}{-} plot 4 {-}{-}{-}{-}{-}{-}{-}{-}{-}{-}{-}{-}{-}{-}{-}{-}{-}{-}{-}{-}{-}{-}{-}{-}{-}{-}{-}{-}{-}{-}{-}\#}
\FunctionTok{plot}\NormalTok{(}\DecValTok{1}\SpecialCharTok{:}\DecValTok{10}\NormalTok{, }\DecValTok{1}\SpecialCharTok{:}\DecValTok{10}\NormalTok{, }\AttributeTok{xlab =} \StringTok{""}\NormalTok{, }\AttributeTok{ylab =} \StringTok{""}\NormalTok{, }\AttributeTok{xaxt =} \StringTok{"n"}\NormalTok{, }\AttributeTok{yaxt =} \StringTok{"n"}\NormalTok{, }
     \AttributeTok{type =} \StringTok{"n"}\NormalTok{)}

\FunctionTok{rect}\NormalTok{(}\FunctionTok{par}\NormalTok{(}\StringTok{"usr"}\NormalTok{)[}\DecValTok{1}\NormalTok{], }\FunctionTok{par}\NormalTok{(}\StringTok{"usr"}\NormalTok{)[}\DecValTok{3}\NormalTok{], }\FunctionTok{par}\NormalTok{(}\StringTok{"usr"}\NormalTok{)[}\DecValTok{2}\NormalTok{], }\FunctionTok{par}\NormalTok{(}\StringTok{"usr"}\NormalTok{)[}\DecValTok{4}\NormalTok{], }
     \AttributeTok{col =} \StringTok{"white"}\NormalTok{)}
\FunctionTok{text}\NormalTok{(}\FloatTok{5.5}\NormalTok{, }\FloatTok{5.5}\NormalTok{, }\FunctionTok{expression}\NormalTok{(}\FunctionTok{paste}\NormalTok{(}\FunctionTok{over}\NormalTok{(}
  \FunctionTok{sum}\NormalTok{(}\FunctionTok{paste}\NormalTok{(}\StringTok{"("}\NormalTok{,}\FunctionTok{italic}\NormalTok{(x[i] }\SpecialCharTok{{-}} \FunctionTok{bar}\NormalTok{(x)),}\StringTok{")"}\SpecialCharTok{\^{}}\DecValTok{2}\NormalTok{), }
      \FunctionTok{italic}\NormalTok{(i)}\SpecialCharTok{==}\DecValTok{1}\NormalTok{, }\FunctionTok{italic}\NormalTok{(n)),(}\FunctionTok{italic}\NormalTok{(n) }\SpecialCharTok{{-}} \DecValTok{1}\NormalTok{)),}\StringTok{" = 78.4"}\NormalTok{)), }
  \AttributeTok{cex =} \FloatTok{1.5}\NormalTok{)}
\FunctionTok{axis}\NormalTok{(}\AttributeTok{side =} \DecValTok{1}\NormalTok{, }\AttributeTok{at =} \FunctionTok{c}\NormalTok{(}\DecValTok{2}\NormalTok{, }\DecValTok{4}\NormalTok{, }\DecValTok{6}\NormalTok{, }\DecValTok{8}\NormalTok{), }\AttributeTok{labels =} \FunctionTok{c}\NormalTok{(}\StringTok{"a"}\NormalTok{, }\StringTok{"b"}\NormalTok{, }\StringTok{"c"}\NormalTok{, }\StringTok{"d"}\NormalTok{))}
\FunctionTok{mtext}\NormalTok{(}\AttributeTok{side =} \DecValTok{1}\NormalTok{, }
      \FunctionTok{expression}\NormalTok{(}\FunctionTok{paste}\NormalTok{(}\StringTok{"Variance of "}\NormalTok{, CO[}\DecValTok{2}\NormalTok{], }\StringTok{" concentration"}\NormalTok{)),}
      \AttributeTok{line =} \DecValTok{3}\NormalTok{)}
\FunctionTok{par}\NormalTok{(op)}
\end{Highlighting}
\end{Shaded}

\begin{figure}[H]

{\centering \includegraphics{figure/minimal-multi1-1} 

}

\caption{Figure resulting from summative example code.}\label{fig:multi1}
\end{figure}

\end{example}

\(\blacksquare\)
\index[terms]{R!package!\textit{asbio}}

\section{Histograms}\label{histograms}

\index[terms]{R!graphics!histograms}
\emph{Histograms} are vital for considering the distributional characteristics of quantitative data. They consist of rectangles whose height is proportional or equivalent to the frequency of particular numeric intervals (\emph{bins}) describing that variable.

\begin{example}
\protect\hypertarget{exm:exm-bryce}{}\label{exm:exm-bryce}

\(\text{}\)\\
The \texttt{brycesite} dataset from package \emph{labdsv} consists of environmental variables recorded at, or calculated for, each of 160 plots in Bryce Canyon National Park in Southern Utah.

\vspace{0.1in}

\begin{Shaded}
\begin{Highlighting}[]
\FunctionTok{library}\NormalTok{(labdsv)}
\FunctionTok{data}\NormalTok{(brycesite)}
\end{Highlighting}
\end{Shaded}

The histogram in Fig \ref{fig:hist1} shows the distribution of the aspect (in degrees) of sites in the dataset.

\vspace{0.1in}

\begin{Shaded}
\begin{Highlighting}[]
\FunctionTok{with}\NormalTok{(brycesite, }\FunctionTok{hist}\NormalTok{(asp, }\AttributeTok{xlab =} \StringTok{"Aspect (Degrees)"}\NormalTok{, }\AttributeTok{main =} \StringTok{""}\NormalTok{))}
\end{Highlighting}
\end{Shaded}

\begin{figure}[H]

{\centering \includegraphics{figure/minimal-hist1-1} 

}

\caption{Histogram of raw aspect measures from the \texttt{brycesite} dataset.}\label{fig:hist1}
\end{figure}

\index[R]{\texttt{hist()}}
The distribution appears remarkably uniform.

Consideration of raw aspect values in analyses is problematic because the measurements are circular. As a result the values 1 and 360 are numerically 359 units apart, although they in fact only differ by one degree. One solution is to use the transformation \([1 - \cos(\text{aspect}^{\text{o}} - 45)]/2\). This index will have highest values on southwest slopes (at 225 degrees), and lowest values on northeast facing slopes (at 45 degrees). This acknowledges the fact that highest temperatures in the Northern Hemisphere occur on Southwest facing slopes because they receive ambient warming during the morning, coupled with late afternoon direct radiation. We have:

\vspace{0.1in}

\begin{Shaded}
\begin{Highlighting}[]
\NormalTok{asp.val }\OtherTok{\textless{}{-}}\NormalTok{ (}\DecValTok{1} \SpecialCharTok{{-}} \FunctionTok{cos}\NormalTok{(((brycesite}\SpecialCharTok{$}\NormalTok{asp }\SpecialCharTok{{-}} \DecValTok{45}\NormalTok{) }\SpecialCharTok{*}\NormalTok{ pi)}\SpecialCharTok{/}\DecValTok{180}\NormalTok{))}\SpecialCharTok{/}\DecValTok{2}
\end{Highlighting}
\end{Shaded}

Fig \ref{fig:hist2} shows the distribution of the transformed aspects which now appears bimodal.

\vspace{0.1in}

\begin{Shaded}
\begin{Highlighting}[]
\FunctionTok{hist}\NormalTok{(asp.val, }\AttributeTok{xlab =} \StringTok{"Aspect idex"}\NormalTok{, }\AttributeTok{main =} \StringTok{""}\NormalTok{)}
\end{Highlighting}
\end{Shaded}

\begin{figure}[H]

{\centering \includegraphics{figure/minimal-hist2-1} 

}

\caption{Histogram of transformed aspect measures from the \texttt{brycesite} dataset.}\label{fig:hist2}
\end{figure}

\end{example}

\(\blacksquare\)

\index[terms]{R!package!\textit{labdsv}}

\section{Controlling Graphical Features using Vectors}\label{controlling-graphical-features-using-vectors}

It is often useful to add information to graphical elements using a variable.

\begin{example}
\protect\hypertarget{exm:exm-bryce2}{}\label{exm:exm-bryce2}\(\text{}\)\\
The \texttt{brycesite} contains information on incident radiation received by sites, measured in Langleys. A Langley (Ly) is a measure of energy per unit area, per unit time. To be precise, one Ly = 1 calorie m\(^{-2}\) min\(^{-1}\). In SI units 1Ly = 41840.00 J m\(^{-2}\). Fig \ref{fig:scatbryce} is a scatterplot of Langleys as a function of aspect index values. In addition five topographic positions from \texttt{brycesite\$pos} are distinguished using both point color and shape. For clarity I also create a legend. Note that ridge top sites have mostly northeastern aspect, and hence have lower radiation inputs.

\vspace{0.1in}

\begin{Shaded}
\begin{Highlighting}[numbers=left,,]
\FunctionTok{with}\NormalTok{(brycesite, }\FunctionTok{plot}\NormalTok{(asp.val, annrad, }\AttributeTok{xlab =} \StringTok{"Aspect value"}\NormalTok{,}
                     \AttributeTok{ylab =} \StringTok{"Annual radiation (Langleys)"}\NormalTok{,}
                     \AttributeTok{col =} \FunctionTok{as.numeric}\NormalTok{(pos), }\AttributeTok{pch =} \FunctionTok{as.numeric}\NormalTok{(pos)))}

\FunctionTok{legend}\NormalTok{(}\StringTok{"bottomright"}\NormalTok{, }\AttributeTok{legend =} \FunctionTok{levels}\NormalTok{(brycesite}\SpecialCharTok{$}\NormalTok{pos), }\AttributeTok{pch =} \DecValTok{1}\SpecialCharTok{:}\DecValTok{5}\NormalTok{, }\AttributeTok{col =} \DecValTok{1}\SpecialCharTok{:}\DecValTok{5}\NormalTok{)}
\end{Highlighting}
\end{Shaded}

\begin{figure}[H]

{\centering \includegraphics{figure/minimal-scatbryce-1} 

}

\caption{Scatterplot of aspect index value versus annual radiation with topographic positions indicated
 from the \texttt{brycesite} dataset.}\label{fig:scatbryce}
\end{figure}

Note that to assign colors and plotting characters appropriately, I coerce the categorical topographic position vector, \texttt{brycesite\$pos}, to be numeric with \texttt{as.numeric()} (Line 3). The result is:

\vspace{0.1in}

\begin{Shaded}
\begin{Highlighting}[]
\FunctionTok{as.numeric}\NormalTok{(brycesite}\SpecialCharTok{$}\NormalTok{pos)}
\end{Highlighting}
\end{Shaded}

\begin{verbatim}
  [1] 4 3 3 4 5 3 3 5 3 3 2 2 3 4 3 3 3 1 2 2 2 5 4 4 3 5 4 3 5 3 5 3 2 5 5
 [36] 4 1 1 2 4 4 3 3 3 3 4 3 5 3 3 3 2 5 3 5 3 3 5 5 4 3 3 5 2 3 3 5 2 2 5
 [71] 2 2 3 3 3 2 2 3 3 2 4 3 4 2 5 3 3 2 2 3 5 5 3 5 5 3 3 3 3 5 5 3 3 3 3
[106] 5 1 2 4 1 2 1 2 3 5 1 5 3 3 3 3 1 3 2 2 5 2 1 2 2 1 2 1 1 1 1 1 1 2 1
[141] 1 4 5 5 5 4 5 2 2 4 1 5 5 5 3 2 2 1 5 4
\end{verbatim}

Ones correspond to the first alphanumeric level in \texttt{pos}, \texttt{bottom}, whereas fives correspond to the last alphanumeric level, \texttt{up\_slope}. The color and symbols assignments are made within the plot on Line three. Base graphics legends can be created using the function \texttt{legend()} (Line 5). The first argument(s) will be a specific \emph{x}, \emph{y} position in the plot for the legend, or one of: \texttt{"bottomright"}, \texttt{"bottom"}, \texttt{"bottomleft"}, \texttt{"left"}, \texttt{"topleft"}, \texttt{"top"}, \texttt{"topright"}, \texttt{"right"}, or \texttt{"center"}. The \texttt{legend} argument names the categories to be depicted. The function \texttt{levels()} used in the \texttt{legend} argument lists the categories in a vector of class factor, alphanumerically.
\end{example}

\(\blacksquare\)

\index[R]{\texttt{legend()}}
\index[R]{\texttt{levels()}}

\section{Secondary Axes}\label{secondary-axes}

For many graphical summaries it may be necessary to add additional axes. For base graphics this will involve laying one plot on top of another, by specifying \texttt{par(new\ =\ TRUE)}, and defining \texttt{axes\ =\ FALSE}, and depending on whether we want extra vertical or horizontal axes, \texttt{xlab\ =\ FALSE} or \texttt{ylab\ =\ FALSE}, and \texttt{ylab\ =\ ""} or \texttt{xlab\ =\ ""} in the arguments of the second plot.

\begin{example}
\protect\hypertarget{exm:exm-axis2}{}\label{exm:exm-axis2}\(\text{}\)\\
In this example I make a scatterplot that considers both \texttt{brycesite} annual radiation and annual growing season radiation as a function of aspect value (Fig \ref{fig:secaxis}).

\vspace{0.1in}

\begin{Shaded}
\begin{Highlighting}[numbers=left,,]
\NormalTok{op }\OtherTok{\textless{}{-}} \FunctionTok{par}\NormalTok{(}\AttributeTok{mar =} \FunctionTok{c}\NormalTok{(}\DecValTok{5}\NormalTok{,}\FloatTok{4.5}\NormalTok{,}\DecValTok{1}\NormalTok{,}\FloatTok{4.5}\NormalTok{), }\AttributeTok{cex =} \FloatTok{1.2}\NormalTok{)}
\FunctionTok{with}\NormalTok{(brycesite, }\FunctionTok{plot}\NormalTok{(asp.val, annrad, }\AttributeTok{xlab =} \StringTok{"Aspect value"}\NormalTok{,}
                     \AttributeTok{ylab =} \StringTok{"Annual radiation (Langleys)"}\NormalTok{))}
\FunctionTok{par}\NormalTok{(}\AttributeTok{new =} \ConstantTok{TRUE}\NormalTok{)}
\FunctionTok{with}\NormalTok{(brycesite, }\FunctionTok{plot}\NormalTok{(asp.val, grorad, }\AttributeTok{pch =} \DecValTok{19}\NormalTok{, }\AttributeTok{axes =} \ConstantTok{FALSE}\NormalTok{, }\AttributeTok{xlab =} \StringTok{""}\NormalTok{,}
                     \AttributeTok{ylab =} \StringTok{""}\NormalTok{))}
\FunctionTok{axis}\NormalTok{(}\DecValTok{4}\NormalTok{)}
\FunctionTok{mtext}\NormalTok{(}\AttributeTok{side =} \DecValTok{4}\NormalTok{,}\StringTok{"Growing season radiation (Langleys)"}\NormalTok{, }\AttributeTok{line =} \DecValTok{3}\NormalTok{, }\AttributeTok{cex =} \FloatTok{1.2}\NormalTok{)}
\FunctionTok{legend}\NormalTok{(}\StringTok{"bottomright"}\NormalTok{, }\AttributeTok{pch=}\FunctionTok{c}\NormalTok{(}\DecValTok{1}\NormalTok{, }\DecValTok{19}\NormalTok{), }\AttributeTok{legend =} \FunctionTok{c}\NormalTok{(}\StringTok{"Annual radiation"}\NormalTok{,}
                                            \StringTok{"Growing season radiation"}\NormalTok{))}
\FunctionTok{par}\NormalTok{(op)}
\end{Highlighting}
\end{Shaded}

\begin{figure}[H]

{\centering \includegraphics{figure/minimal-secaxis-1} 

}

\caption{Scatterplot of annual radiation and growing season radiation as function of the aspect value index for the \texttt{brycesite} dataset.}\label{fig:secaxis}
\end{figure}

Note the extra room given to the right hand margin (Line 1) to contain a labeling for a secondary vertical axis. The code \texttt{par(new\ =\ TRUE)} (Line 4) tells \textbf{R} not to \emph{clean} the graphical device before drawing a new plot. The code \texttt{axis(4)} (Line 7) creates labeling for the right hand axis. The argument \texttt{axes\ =\ FALSE} in the second plot, suppresses default plot plotting of axis units on the left and bottom axes.
\end{example}

\(\blacksquare\)

\section{Barplots}\label{barplots}

\index[terms]{R!graphics!barplots}
\emph{Barplots} are frequently used to compare single number summaries (e.g., sum, median, mean, etc.) of categorical levels.

\begin{example}
\protect\hypertarget{exm:exm-bp}{}\label{exm:exm-bp}

\(\text{}\)\\
Of great concern to both citizens and scientists are rising global levels of atmospheric greenhouse gasses. Atmospheric CO\(_2\) concentrations have increased more than 40\% since the start of the industrial revolution while the more potent greenhouse gasses CH\(_4\) and NO\(_2\) have increased approximately 150\% and 23\%, respectively \citep{brink}. We will take a detailed look at recent global patterns of CO\(_2\) emissions and human population numbers in this example, while creating different sorts of barplots, and applying some of the data management techniques introduced in Chapter \ref{ch4}. Tidyverse data management approaches will be used for creating \emph{ggplot2} graphics in Chapter \ref{ch4}. We will use the \texttt{world.emissions} dataframe from \emph{asbio} as our data source.

\vspace{0.1in}

\begin{Shaded}
\begin{Highlighting}[numbers=left,,]
\FunctionTok{library}\NormalTok{(asbio)}
\FunctionTok{data}\NormalTok{(world.emissions)}

\NormalTok{nred }\OtherTok{\textless{}{-}}\NormalTok{ world.emissions[world.emissions}\SpecialCharTok{$}\NormalTok{continent }\SpecialCharTok{!=} \StringTok{"Redundant"}\NormalTok{,]}
\NormalTok{co2 }\OtherTok{\textless{}{-}} \FunctionTok{with}\NormalTok{(nred, }\FunctionTok{tapply}\NormalTok{(co2, country, }\ControlFlowTok{function}\NormalTok{(x)\{}\FunctionTok{mean}\NormalTok{(x, }\AttributeTok{na.rm =}\NormalTok{ T)\}))}
\NormalTok{n }\OtherTok{\textless{}{-}} \FunctionTok{with}\NormalTok{(nred, }\FunctionTok{tapply}\NormalTok{(co2, country, }\ControlFlowTok{function}\NormalTok{(x)\{}\FunctionTok{length}\NormalTok{(x)\}))}

\NormalTok{co2n }\OtherTok{\textless{}{-}} \FunctionTok{data.frame}\NormalTok{(}\FunctionTok{cbind}\NormalTok{(co2, n))}
\NormalTok{co2n.sub }\OtherTok{\textless{}{-}}\NormalTok{ co2n[}\FunctionTok{which}\NormalTok{(}\FunctionTok{row.names}\NormalTok{(co2) }\SpecialCharTok{\%in\%} \FunctionTok{c}\NormalTok{(}\StringTok{"Canada"}\NormalTok{, }\StringTok{"China"}\NormalTok{, }\StringTok{"Finland"}\NormalTok{,}
\StringTok{"Japan"}\NormalTok{, }\StringTok{"Kenya"}\NormalTok{, }\StringTok{"United States"}\NormalTok{)),]}

\NormalTok{labels }\OtherTok{\textless{}{-}} \FunctionTok{paste}\NormalTok{(}\FunctionTok{rownames}\NormalTok{(co2n.sub), }\StringTok{" ("}\NormalTok{, co2n.sub}\SpecialCharTok{$}\NormalTok{n, }\StringTok{")"}\NormalTok{, }\AttributeTok{sep =} \StringTok{""}\NormalTok{)}
\end{Highlighting}
\end{Shaded}

In the code above, CO\(_2\) annual means and sample sizes for each country are computed on Lines 4-5. A subset dataset of six countries is created on Lines 7-8. Country names for this subset and the number of years of data collection are combined in an object called \texttt{labels} on Line 10. Here is the barplot code.
\vspace{0.1in}

\begin{Shaded}
\begin{Highlighting}[numbers=left,,]
\NormalTok{cols }\OtherTok{\textless{}{-}} \FunctionTok{c}\NormalTok{(}\StringTok{"\#5D3003"}\NormalTok{,}\StringTok{"\#347C62"}\NormalTok{,}\StringTok{"\#A79DBE"}\NormalTok{,}\StringTok{"\#DFCCC2"}\NormalTok{,}\StringTok{"\#994E58"}\NormalTok{, }\StringTok{"\#F1F1F1"}\NormalTok{)}
\FunctionTok{barplot}\NormalTok{(co2n.sub}\SpecialCharTok{$}\NormalTok{co2, }\AttributeTok{las =} \DecValTok{2}\NormalTok{, }\AttributeTok{ylab =} \StringTok{""}\NormalTok{, }\AttributeTok{yaxt =} \StringTok{"n"}\NormalTok{, }\AttributeTok{names =}\NormalTok{ labels,}
        \AttributeTok{log =} \StringTok{"y"}\NormalTok{, }\AttributeTok{col =}\NormalTok{ cols)}
\FunctionTok{axis}\NormalTok{(}\DecValTok{2}\NormalTok{)}
\FunctionTok{mtext}\NormalTok{(}\AttributeTok{side =} \DecValTok{2}\NormalTok{, }
      \FunctionTok{expression}\NormalTok{(}\FunctionTok{paste}\NormalTok{(CO[}\DecValTok{2}\NormalTok{], }\StringTok{" Emissions (metric tons x "}\NormalTok{, }\DecValTok{10}\SpecialCharTok{\^{}}\DecValTok{6}\NormalTok{, }\StringTok{")"}\NormalTok{)),}
      \AttributeTok{line =} \FloatTok{2.5}\NormalTok{)}
\end{Highlighting}
\end{Shaded}

The color palette on Line 1 in the code above was generated using \texttt{colorspace::hclwizard()}. Note that rotated \emph{x}-axis labels (\texttt{las\ =\ 2}) and log-scale \emph{y}-axis are are specified on the call to \texttt{barplot()} on Lines 2-3. A customized \emph{y}-axis is constructed on Lines 4-7. This is done largely to force the axis tick labels to be have a default vertical format. They would be vertical otherwise because of the use of \texttt{las\ =\ 2} in \texttt{barplot()}. Figure \ref{fig:bp1} shows the shows the final result.

\begin{figure}[H]

{\centering \includegraphics{figure/minimal-bp1-1} 

}

\caption{Barplot of mean annual CO$_2$ emmission levels for six countries. Number of years used in computing means indicated in parentheses.}\label{fig:bp1}
\end{figure}

To depict trends since the year 2000, we can use a stacked barplot (Fig \ref{fig:bp1stb}), or a side by side barplot (Fig \ref{fig:bp1ssb}) by applying \texttt{barplot()} to a matrix with columns representing categories.

In the code below we subset the data by country (Lines 1-4) and year (Line 6), and create a dataframe containing CO\(_2\) and country data (line 8), which is converted to a wide format matrix using \texttt{unstack()} (Line 9).

\vspace{0.1in}

\begin{Shaded}
\begin{Highlighting}[numbers=left,,]
\NormalTok{csub }\OtherTok{\textless{}{-}}\NormalTok{ world.emissions[}
  \FunctionTok{which}\NormalTok{(world.emissions}\SpecialCharTok{$}\NormalTok{country }\SpecialCharTok{\%in\%} 
          \FunctionTok{c}\NormalTok{(}\StringTok{"Canada"}\NormalTok{, }\StringTok{"China"}\NormalTok{, }\StringTok{"Finland"}\NormalTok{, }
            \StringTok{"Japan"}\NormalTok{, }\StringTok{"Kenya"}\NormalTok{, }\StringTok{"United States"}\NormalTok{)),]}

\NormalTok{ysub }\OtherTok{\textless{}{-}}\NormalTok{ csub[}\FunctionTok{which}\NormalTok{(csub}\SpecialCharTok{$}\NormalTok{year }\SpecialCharTok{\textgreater{}=} \DecValTok{2000}\NormalTok{),]}

\NormalTok{dat }\OtherTok{\textless{}{-}} \FunctionTok{data.frame}\NormalTok{(}\AttributeTok{co2 =}\NormalTok{ ysub}\SpecialCharTok{$}\NormalTok{co2, }\AttributeTok{country =}\NormalTok{ ysub}\SpecialCharTok{$}\NormalTok{country)}
\NormalTok{dat1 }\OtherTok{\textless{}{-}} \FunctionTok{as.matrix}\NormalTok{(}\FunctionTok{unstack}\NormalTok{(dat))}
\end{Highlighting}
\end{Shaded}

In the code below, hexadecimal colors generated by \texttt{colorspace::hclwizard()} are brought in (Line 1) and modified, i.e., unlisted, coerced to be a character vector and reversed (Line 2), preceding creation of the barplot (Lines 4-8).

\vspace{0.1in}

\begin{Shaded}
\begin{Highlighting}[numbers=left,,]
\NormalTok{cols }\OtherTok{\textless{}{-}} \FunctionTok{read.table}\NormalTok{(}\StringTok{"colormap\_hex.txt"}\NormalTok{) }\CommentTok{\# file from hclwizard()}
\NormalTok{cols }\OtherTok{\textless{}{-}} \FunctionTok{rev}\NormalTok{(}\FunctionTok{as.character}\NormalTok{(}\FunctionTok{unlist}\NormalTok{(cols)))}

\FunctionTok{barplot}\NormalTok{(dat1, }\AttributeTok{log =} \StringTok{"y"}\NormalTok{, }\AttributeTok{col =}\NormalTok{ cols, }\AttributeTok{yaxt =} \StringTok{"n"}\NormalTok{, }\AttributeTok{las =} \DecValTok{2}\NormalTok{, }\AttributeTok{names =}\NormalTok{ labels)}
\FunctionTok{axis}\NormalTok{(}\DecValTok{2}\NormalTok{)}
\FunctionTok{mtext}\NormalTok{(}\AttributeTok{side =} \DecValTok{2}\NormalTok{, }
      \FunctionTok{expression}\NormalTok{(}\FunctionTok{paste}\NormalTok{(CO[}\DecValTok{2}\NormalTok{], }\StringTok{" Emissions (metric tons x "}\NormalTok{, }\DecValTok{10}\SpecialCharTok{\^{}}\DecValTok{6}\NormalTok{, }\StringTok{")"}\NormalTok{)),}
      \AttributeTok{line =} \FloatTok{2.5}\NormalTok{, }\AttributeTok{cex =} \FloatTok{1.2}\NormalTok{)}
\end{Highlighting}
\end{Shaded}

\begin{figure}[H]

{\centering \includegraphics{figure/minimal-bp1stb-1} 

}

\caption{Stacked barplot of mean annual CO$_2$ emmission levels for six countries from 2000-2019.}\label{fig:bp1stb}
\end{figure}

Side by side barplots are generated by specifying \texttt{beside\ =\ TRUE} in \texttt{barplot()} (Line 2 in code below).

\vspace{0.1in}

\begin{Shaded}
\begin{Highlighting}[numbers=left,,]
\FunctionTok{barplot}\NormalTok{(dat1, }\AttributeTok{log =} \StringTok{"y"}\NormalTok{, }\AttributeTok{beside =} \ConstantTok{TRUE}\NormalTok{, }\AttributeTok{col =}\NormalTok{ cols, }\AttributeTok{yaxt =} \StringTok{"n"}\NormalTok{, }\AttributeTok{las =} \DecValTok{2}\NormalTok{)}

\FunctionTok{axis}\NormalTok{(}\DecValTok{2}\NormalTok{)}
\FunctionTok{mtext}\NormalTok{(}\AttributeTok{side =} \DecValTok{2}\NormalTok{, }
      \FunctionTok{expression}\NormalTok{(}\FunctionTok{paste}\NormalTok{(CO[}\DecValTok{2}\NormalTok{], }\StringTok{" Emissions (metric tons x "}\NormalTok{, }\DecValTok{10}\SpecialCharTok{\^{}}\DecValTok{6}\NormalTok{, }\StringTok{")"}\NormalTok{)),}
      \AttributeTok{line =} \FloatTok{2.5}\NormalTok{, }\AttributeTok{cex =} \FloatTok{1.2}\NormalTok{)}
\end{Highlighting}
\end{Shaded}

\begin{figure}[H]

{\centering \includegraphics{figure/minimal-bp1ssb-1} 

}

\caption{Side by side barplot of mean annual CO$_2$ emmission levels for six countries from 2000-2019.}\label{fig:bp1ssb}
\end{figure}

\end{example}

\(\blacksquare\)

\index[terms]{R!package!\textit{asbio}}
\index[R]{\texttt{barplot()}}
\index[R]{\texttt{colorspace::hclwizard()}}
\index[R]{\texttt{unstack()}}

\section{Boxplots}\label{boxplots}

\index[terms]{R!graphics!boxplots}
\emph{Boxplots} or \emph{box and whisker plots} and their variants are an excellent way to quickly summarize and compare the distributions of levels in a categorical variable with respect to a quantitative variable. The function \texttt{boxplot()} does this by graphically providing a five number summary for factor levels (Fig \ref{fig:box}). Specifically, the upper and lower hinges of boxes from boxplot show the 1st and 3rd quartiles (thus the box contains the central 50\% of the data). The black stripe in the middle of each box shows the median. The whiskers extend to the most extreme data point which is no more than \texttt{coef} times the length of the box away from a hinge, where \texttt{coef} is defined in the arguments for \texttt{boxplot()} (by default \texttt{coef\ =\ 1.5}). Circle symbols outside of whiskers can be considered outliers \citep[\emph{cf}.,][]{tukey1977}.

\begin{figure}[H]

{\centering \includegraphics[width=0.8\linewidth,]{figs6/box} 

}

\caption{A summary of boxplot characteristics.}\label{fig:box}
\end{figure}

\index[R]{\texttt{boxplot()}}

\begin{example}
\protect\hypertarget{exm:exm-bxp1}{}\label{exm:exm-bxp1}

\(\text{}\)\\
Here we reconsider the \texttt{world.emissions} data using boxplots. Recall our approach from the previous exercise:

\vspace{0.1in}

\begin{Shaded}
\begin{Highlighting}[numbers=left,,]
\NormalTok{csub }\OtherTok{\textless{}{-}}\NormalTok{ world.emissions[}
  \FunctionTok{which}\NormalTok{(world.emissions}\SpecialCharTok{$}\NormalTok{country }\SpecialCharTok{\%in\%} \FunctionTok{c}\NormalTok{(}\StringTok{"Canada"}\NormalTok{, }\StringTok{"China"}\NormalTok{, }\StringTok{"Finland"}\NormalTok{,}
                                     \StringTok{"Japan"}\NormalTok{, }\StringTok{"Kenya"}\NormalTok{, }\StringTok{"United States"}\NormalTok{)),]}
\NormalTok{ysub }\OtherTok{\textless{}{-}}\NormalTok{ csub[}\FunctionTok{which}\NormalTok{(csub}\SpecialCharTok{$}\NormalTok{year }\SpecialCharTok{\textgreater{}=} \DecValTok{2000}\NormalTok{),]}

\NormalTok{dat }\OtherTok{\textless{}{-}} \FunctionTok{data.frame}\NormalTok{(}\AttributeTok{co2 =}\NormalTok{ ysub}\SpecialCharTok{$}\NormalTok{co2, }\AttributeTok{country =}\NormalTok{ ysub}\SpecialCharTok{$}\NormalTok{country)}
\end{Highlighting}
\end{Shaded}

We can use \texttt{plot(q\ \textasciitilde{}\ c)} (Fig \ref{fig:d2}) or \texttt{boxplot(q\ \textasciitilde{}\ c)} to make boxplots, where \texttt{q} is a vector of quantitative data and \texttt{c} is a corresponding vector of categorical data.

\vspace{0.1in}

\begin{Shaded}
\begin{Highlighting}[numbers=left,,]
\NormalTok{cols1 }\OtherTok{\textless{}{-}} \FunctionTok{rev}\NormalTok{(}\FunctionTok{c}\NormalTok{(}\StringTok{"\#5D3003"}\NormalTok{,}\StringTok{"\#347C62"}\NormalTok{,}\StringTok{"\#A79DBE"}\NormalTok{,}
               \StringTok{"\#DFCCC2"}\NormalTok{,}\StringTok{"\#994E58"}\NormalTok{, }\StringTok{"\#F1F1F1"}\NormalTok{))}
\FunctionTok{with}\NormalTok{(dat, }
     \FunctionTok{boxplot}\NormalTok{(co2 }\SpecialCharTok{\textasciitilde{}}\NormalTok{ country, }\AttributeTok{col =}\NormalTok{ cols1, }\AttributeTok{las =} \DecValTok{2}\NormalTok{, }\AttributeTok{xlab =} \StringTok{""}\NormalTok{, }
             \AttributeTok{ylab =} 
               \FunctionTok{expression}\NormalTok{(}
                 \FunctionTok{paste}\NormalTok{(CO[}\DecValTok{2}\NormalTok{], }\StringTok{" Emissions (metric tons x "}\NormalTok{, }\DecValTok{10}\SpecialCharTok{\^{}}\DecValTok{6}\NormalTok{, }\StringTok{")"}\NormalTok{))))}
\end{Highlighting}
\end{Shaded}

\begin{figure}[H]

{\centering \includegraphics{figure/minimal-bxp-1} 

}

\caption{Boxplots of annual CO$_2$ emmission levels for six countries from 2000-2019.}\label{fig:bxp}
\end{figure}

\end{example}

\(\blacksquare\)

\subsection{Violin Plots}\label{violin-plots}

\index[terms]{R!graphics!violin plots}
Another graphical tool for comparing probability distributions is a \emph{violin plot}. It contains contains similar components to a boxplots, including designation of boxes and whiskers, along with a rotated kernel density plot on each side. Thus, it allows additional consideration of the skew, kurtosis and potential multimodality of distributions. The function \texttt{vioplot} from the package \emph{vioplot} allows base graphics generation of violin plots.
\index[R]{\texttt{vioplot::vioplot()}}\index[terms]{R!package!\textit{vioplot}}

\begin{example}
\protect\hypertarget{exm:exm-vio1}{}\label{exm:exm-vio1}

\(\text{}\)\\
As an example we will compare violin plots based on random sampling of a bimodal, uniform and normal distribution (see \texttt{?vioplot}). Note the kernel density generator fits a oblique sphere, although the uniform PDF is a rectangle (Fig \ref{fig:vio}).

\vspace{0.1in}

\begin{Shaded}
\begin{Highlighting}[numbers=left,,]
\FunctionTok{library}\NormalTok{(vioplot)}

\NormalTok{mu }\OtherTok{\textless{}{-}} \DecValTok{2}
\NormalTok{sig }\OtherTok{\textless{}{-}} \FloatTok{0.6}
\NormalTok{bimodal }\OtherTok{\textless{}{-}} \FunctionTok{c}\NormalTok{(}\FunctionTok{rnorm}\NormalTok{(}\DecValTok{1000}\NormalTok{,}\SpecialCharTok{{-}}\NormalTok{mu, sig), }\FunctionTok{rnorm}\NormalTok{(}\DecValTok{1000}\NormalTok{, mu, sig))}
\NormalTok{uniform }\OtherTok{\textless{}{-}} \FunctionTok{runif}\NormalTok{(}\DecValTok{2000}\NormalTok{, }\SpecialCharTok{{-}}\DecValTok{4}\NormalTok{, }\DecValTok{4}\NormalTok{)}
\NormalTok{normal }\OtherTok{\textless{}{-}} \FunctionTok{rnorm}\NormalTok{(}\DecValTok{2000}\NormalTok{, }\DecValTok{0}\NormalTok{, }\DecValTok{3}\NormalTok{)}
\FunctionTok{vioplot}\NormalTok{(bimodal, uniform, normal, }\AttributeTok{col =}\NormalTok{ cols1[}\DecValTok{1}\SpecialCharTok{:}\DecValTok{3}\NormalTok{],}
        \AttributeTok{names =} \StringTok{"Bimodal"}\NormalTok{, }\StringTok{"Uniform"}\NormalTok{, }\StringTok{"Normal"}\NormalTok{)}
\end{Highlighting}
\end{Shaded}

\begin{figure}[H]

{\centering \includegraphics{figure/minimal-vio-1} 

}

\caption{Violin plots based on random sampling from a bimodal, uniform, and normal distribution.}\label{fig:vio}
\end{figure}

\end{example}

\(\blacksquare\)

\section{Interval Plots}\label{interval-plots}

\index[terms]{R!graphics!interval plots}
\emph{Interval plots} display location measures (e.g.~means, medians, etc.), typically as bars, along with error bars representing measures of data dispersion (e.g., standard errors, standard deviations, confidence intervals, interquartile ranges, etc.). Thus, barplots and boxplots can be considered special types of interval plots.

\begin{example}
\protect\hypertarget{exm:exm-iplot1}{}\label{exm:exm-iplot1}

\(\text{}\)\\
As an example, we will create an interval plot by hand using a classic dataset from R.A. Fisher that records the yield of different varieties of potatoes. The data are in the dataframe \texttt{asbio::potato}. Here are the means and the standard errors of the mean.
\index[terms]{R!package!\textit{asbio}}

\vspace{0.1in}

\begin{Shaded}
\begin{Highlighting}[]
\FunctionTok{data}\NormalTok{(potato)}
\NormalTok{means }\OtherTok{\textless{}{-}} \FunctionTok{with}\NormalTok{(potato, }\FunctionTok{tapply}\NormalTok{(Yield, Variety, mean))}
\FunctionTok{head}\NormalTok{(means)}
\end{Highlighting}
\end{Shaded}

\begin{verbatim}
         Ajax Arran comrade British queen  Duke of York       Epicure 
       3.3400        2.2622        3.1367        1.7778        2.1600 
   Great Scot 
       3.4033 
\end{verbatim}

\vspace{0.1in}

\begin{Shaded}
\begin{Highlighting}[]
\NormalTok{ses }\OtherTok{\textless{}{-}} \FunctionTok{with}\NormalTok{(potato, }
            \FunctionTok{tapply}\NormalTok{(Yield, Variety, }\ControlFlowTok{function}\NormalTok{(x)\{}\FunctionTok{sd}\NormalTok{(x)}\SpecialCharTok{/}\FunctionTok{sqrt}\NormalTok{(}\FunctionTok{length}\NormalTok{(x))\}))}
\FunctionTok{head}\NormalTok{(ses)}
\end{Highlighting}
\end{Shaded}

\begin{verbatim}
         Ajax Arran comrade British queen  Duke of York       Epicure 
     0.305941      0.070902      0.181184      0.148313      0.146771 
   Great Scot 
     0.140929 
\end{verbatim}

We will plot the means using a barplot and save the horizontal locations of bars as a object \texttt{bloc}.

\vspace{0.1in}

\begin{Shaded}
\begin{Highlighting}[]
\NormalTok{bloc }\OtherTok{\textless{}{-}} \FunctionTok{barplot}\NormalTok{(means, }\AttributeTok{las =} \DecValTok{2}\NormalTok{, }\AttributeTok{ylab =} \StringTok{"Yield (lbs per plant)"}\NormalTok{, }\AttributeTok{col =}\NormalTok{ cols)}
\end{Highlighting}
\end{Shaded}

We will then overlay error bars using the function \texttt{segments()} or \texttt{arrows()}.

\vspace{0.1in}

\begin{Shaded}
\begin{Highlighting}[]
\FunctionTok{segments}\NormalTok{(}\AttributeTok{x0 =}\NormalTok{ bloc, }\AttributeTok{y0 =}\NormalTok{ means }\SpecialCharTok{{-}}\NormalTok{ ses, }\AttributeTok{y1 =}\NormalTok{ means }\SpecialCharTok{+}\NormalTok{ ses, }\AttributeTok{x1 =}\NormalTok{ bloc)}
\end{Highlighting}
\end{Shaded}

The result is shown in Fig \ref{fig:iplot}.

\begin{figure}[H]

{\centering \includegraphics{figure/minimal-iplot-1} 

}

\caption{Interval plot of the Fisher potato dataset.  Bar heights are means, error bars represent $\bar{x} \pm \hat{\sigma}_{\bar{x}}$.}\label{fig:iplot}
\end{figure}

\end{example}

\(\blacksquare\)

\index[R]{\texttt{segments()}} \index[R]{\texttt{arrows()}}

The function \texttt{asbio::bplot} allows straightforward creation of interval plots from specified explanatory and response variables. A large number of location and dispersion measures can be specified in the function arguments. To recreate Fig \ref{fig:iplot} one could simply type:
\index[R]{\texttt{asbio::bplot}}

\vspace{0.1in}

\begin{Shaded}
\begin{Highlighting}[]
\FunctionTok{with}\NormalTok{(potato, }\FunctionTok{bplot}\NormalTok{(}\AttributeTok{y =}\NormalTok{ Yield, }\AttributeTok{x =}\NormalTok{ Variety, }\AttributeTok{bar.col =}\NormalTok{ cols, }\AttributeTok{border =}\NormalTok{ cols))}
\end{Highlighting}
\end{Shaded}

\subsection{Pairwise Comparisons}\label{pairwise-comparisons}

An important component of many biological analyses are multiple pairwise comparisons of means (or other location measures). These tests will often require control of Family-Wise type I Error Rate (FWER), that is, the probability of incorrectly rejecting at least one true null hypothesis in a family of related tests. The most powerful method for controlling FWER in a \emph{post hoc} family of pairwise tests, following an omnibus ANalysis Of VAriance (ANOVA), is Tukey's honest significant difference (see \citet{aho2014}).

\begin{example}
\protect\hypertarget{exm:exm-iplotb}{}\label{exm:exm-iplotb}\(\text{}\)\\
\citet{zelazo1972} performed a series of experiments to determine whether certain exercises could allow infants to learn to walk at a younger age. The experimental treatments were: Active Exercise (AE), Passive Exercise (PE), Test-Only (TO), and Control (C). The data are in the dataframe \texttt{asbio::baby.walk}. For more information type \texttt{?baby.walk}.

Rejection of the omnibus ANOVA null hypothesis of no mean treatment differences, allowed pairwise comparison of treatment means using Tukey's procedure. We will use the function \texttt{asbio::pairw.anova()} to run this analysis.

\vspace{0.1in}

\begin{Shaded}
\begin{Highlighting}[]
\FunctionTok{data}\NormalTok{(baby.walk)}
\NormalTok{tukey }\OtherTok{\textless{}{-}} \FunctionTok{with}\NormalTok{(baby.walk, }\FunctionTok{pairw.anova}\NormalTok{(}\AttributeTok{y =}\NormalTok{ date, }\AttributeTok{x =}\NormalTok{ treatment))}
\NormalTok{tukey}
\end{Highlighting}
\end{Shaded}

\begin{verbatim}

95% Tukey-Kramer confidence intervals 

              Diff    Lower    Upper  Decision Adj. p-value
muAE-muC    -2.225 -4.35648 -0.09352 Reject H0     0.038997
muAE-muPE   -0.525 -2.65648  1.60648    FTR H0     0.897224
muC-muPE       1.7 -0.52625  3.92625    FTR H0     0.172932
muAE-muTO -1.58333 -3.61562  0.44895    FTR H0     0.160457
muC-muTO   0.64167 -1.48981  2.77314    FTR H0     0.829542
muPE-muTO -1.05833 -3.18981  1.07314    FTR H0     0.513366
\end{verbatim}

Interval plots can be used to summarize these comparisons. The \texttt{plot} method for objects of class \texttt{pairw} calls \texttt{bplot()} for this purpose. In particular, we have:

\vspace{0.1in}

\begin{Shaded}
\begin{Highlighting}[]
\FunctionTok{plot}\NormalTok{(tukey, }\AttributeTok{ylab =} \StringTok{"Months until walking"}\NormalTok{, }\AttributeTok{cex.lett =} \FloatTok{1.2}\NormalTok{)}
\end{Highlighting}
\end{Shaded}

\begin{verbatim}

Bars are means.  Errors are SEs.

The population means of factor levels with the same letter are not
significantly different at alpha = 0.05 using the Tukey HSD method.
\end{verbatim}

\begin{figure}[H]

{\centering \includegraphics{figure/minimal-iplotb-1} 

}

\caption{An interval plot summarizing the results of pairwise comparisons for the \texttt{baby.walk} example.}\label{fig:iplotb}
\end{figure}

As stated in the \texttt{plot.pairw()} function output (Fig \ref{fig:iplotb}), letters above bars summarize the result of pairwise comparisons. In particular, factor levels means with the same letter are not significantly different using the conventional FWER \(\alpha = 0.05\).
\end{example}

\(\blacksquare\)

\index[R]{\texttt{asbio::pairw.anova()}}
\index[R]{\texttt{asbio::bplot()}}

We will look at more sophisticated graphical methods for pairwise comparisons in Ch \ref{ch7}.

\section{\texorpdfstring{\texttt{matplot()}}{matplot()}}\label{matplot}

The function \texttt{matplot()} allows one to plot the columns of one matrix against the columns of another. There is no clear \emph{ggplot2} alternative to \texttt{matplot()} because \emph{tidyverse} functions require data to be a long table format, whereas \texttt{matplot()} works best with data in a wide table format.
\index[R]{\texttt{matplot()}}

\begin{example}
\protect\hypertarget{exm:exm-mplot}{}\label{exm:exm-mplot}

\(\text{}\)\\
To demonstrate \texttt{matplot()} we will use the \texttt{dat1} dataset, used to create Fig \ref{fig:bp1stb}, which contains annual CO\(_2\) levels from 2000-2019 for six countries.

\vspace{0.1in}

\begin{Shaded}
\begin{Highlighting}[numbers=left,,]
\FunctionTok{par}\NormalTok{(}\AttributeTok{mar =} \FunctionTok{c}\NormalTok{(}\DecValTok{3}\NormalTok{,}\FloatTok{4.5}\NormalTok{,}\DecValTok{5}\NormalTok{,}\DecValTok{2}\NormalTok{), }\AttributeTok{cex =} \FloatTok{1.1}\NormalTok{)}
\FunctionTok{matplot}\NormalTok{(}\AttributeTok{x =} \DecValTok{2000}\SpecialCharTok{:}\DecValTok{2019}\NormalTok{, }\AttributeTok{y =}\NormalTok{ dat1, }\AttributeTok{col =}\NormalTok{ cols1, }\AttributeTok{type =} \StringTok{"l"}\NormalTok{, }\AttributeTok{lwd =} \FloatTok{1.5}\NormalTok{,}
        \AttributeTok{log =} \StringTok{"y"}\NormalTok{, }
        \AttributeTok{ylab =} \FunctionTok{expression}\NormalTok{(}
          \FunctionTok{paste}\NormalTok{(CO[}\DecValTok{2}\NormalTok{], }\StringTok{" Emissions (metric tons x "}\NormalTok{, }\DecValTok{10}\SpecialCharTok{\^{}}\DecValTok{6}\NormalTok{, }\StringTok{")"}\NormalTok{)))}

\FunctionTok{legend}\NormalTok{(}\AttributeTok{x =} \DecValTok{2001}\NormalTok{, }\AttributeTok{y =} \DecValTok{80000}\NormalTok{, }\AttributeTok{xpd =} \ConstantTok{TRUE}\NormalTok{,}
       \AttributeTok{lty =} \DecValTok{1}\SpecialCharTok{:}\DecValTok{5}\NormalTok{, }\AttributeTok{ncol =} \DecValTok{2}\NormalTok{, }\AttributeTok{lwd =} \FloatTok{1.5}\NormalTok{, }\AttributeTok{bty =} \StringTok{"n"}\NormalTok{,}
       \AttributeTok{col =}\NormalTok{ cols, }\AttributeTok{legend =}
         \FunctionTok{c}\NormalTok{(}\StringTok{"Canada"}\NormalTok{, }\StringTok{"China"}\NormalTok{, }\StringTok{"Finland"}\NormalTok{,}
           \StringTok{"Japan"}\NormalTok{, }\StringTok{"Kenya"}\NormalTok{, }\StringTok{"United States"}\NormalTok{))}
\end{Highlighting}
\end{Shaded}

In the code above, note that I allocate additional room in the top of the graph for a legend (Line 1). Note that the response variable is a matrix of CO\(_2\) values whose columns delimit countries (Line 2). The \texttt{xpd} argument in \texttt{legend()} allows plotting to be clipped to the device region which will generally exceed the plot region (Line 6). The result is shown in Fig \ref{fig:mp}.

\begin{figure}[H]

{\centering \includegraphics{figure/minimal-mp-1} 

}

\caption{A matrix line plot.}\label{fig:mp}
\end{figure}

\end{example}

\(\blacksquare\)

\section{Interactivity}\label{interactivity}

As noted earlier, \textbf{R} graphics are generally non-interactive. Some \emph{graphical interactivity} is allowed via the function \texttt{locator()}, which returns graphical coordinates where a mouse click occurred in plot, and \texttt{identify()}, which can be used to add labels and symbols to mouse click locations. For instance, try:

\vspace{0.1in}

\begin{Shaded}
\begin{Highlighting}[]
\FunctionTok{dev.new}\NormalTok{(}\AttributeTok{RStudioGD =} \ConstantTok{FALSE}\NormalTok{) }\CommentTok{\# If one is using RStudio}
\FunctionTok{plot}\NormalTok{(}\DecValTok{1}\SpecialCharTok{:}\DecValTok{10}\NormalTok{)}
\FunctionTok{identify}\NormalTok{(}\DecValTok{1}\SpecialCharTok{:}\DecValTok{10}\NormalTok{, }\AttributeTok{labels =} \DecValTok{1}\SpecialCharTok{:}\DecValTok{10}\NormalTok{)}
\end{Highlighting}
\end{Shaded}

\index[terms]{Graphical interactivity}

For Windows, X Window, and Cairo graphics devices, more sophisticated methods exist for interactivity. In these settings, the function \texttt{setGraphicsEventHandlers()} can be used to call functions when events such as mouse clicks or keystrokes occur. For instance, open an appropriate graphics device and try:

\vspace{0.1in}

\begin{Shaded}
\begin{Highlighting}[]
\FunctionTok{example}\NormalTok{(}\StringTok{"getGraphicsEvent"}\NormalTok{)}
\end{Highlighting}
\end{Shaded}

Still other interactive options are possible using animations and hand rotatable graphics. These approaches, which are often transferable to Markdown HTMLs, are considered briefly in the next two sections of this chapter. Animations using the package \emph{ggplot2} are considered in Ch \ref{ch7}. GUI driven graphics interactivity is also possible, and is described in Ch \ref{ch11}.
\index[terms]{R!package!\textit{ggplot2}}

\section{Three Dimensional Graphics}\label{three-dimensional-graphics}

\index[terms]{R!graphics!3D plots}
It is often necessary to consider more than two variables in biological graphics. This can be done in a number of different ways, including the use of additional axes (e.g., Fig \ref{fig:secaxis}) (including 3D plots), additional colors, multiple line or symbol types (Fig \ref{fig:scatbryce}), or even multiple symbol sizes.

\begin{example}
\protect\hypertarget{exm:exm-thd}{}\label{exm:exm-thd}

\(\text{}\)\\
To consider three dimensional plotting we will use two datasets from the package \emph{vegan} which describe taiga/tundra ecosystems at particular Scandinavian sites. Vegetation data are contained in the dataset \texttt{varespec} while soil chemistry data for the same sites are contained in the dataset \texttt{varechem}.

\vspace{0.1in}

\begin{Shaded}
\begin{Highlighting}[]
\FunctionTok{library}\NormalTok{(vegan)}
\FunctionTok{data}\NormalTok{(varespec)}
\FunctionTok{data}\NormalTok{(varechem)}
\end{Highlighting}
\end{Shaded}

In Fig \ref{fig:sym} we examine the distribution of the heath plant \emph{Vaccinium vitis-idaea} (a common species in boreal forest understories) with respect to both pH and soil percent nitrogen. This is done by making symbol sizes change with the abundance of \emph{V. vitis-idaea}.

\vspace{0.1in}

\begin{Shaded}
\begin{Highlighting}[]
\FunctionTok{with}\NormalTok{(varechem, }\FunctionTok{plot}\NormalTok{(N, pH, }\AttributeTok{xlab =} \StringTok{"\% soil N"}\NormalTok{, }\AttributeTok{pch =} \DecValTok{16}\NormalTok{,}
                    \AttributeTok{cex =}\NormalTok{ varespec}\SpecialCharTok{$}\NormalTok{Vaccviti}\SpecialCharTok{/}\DecValTok{100} \SpecialCharTok{*} \DecValTok{15}\NormalTok{))}
\end{Highlighting}
\end{Shaded}

\begin{figure}[H]

{\centering \includegraphics{figure/minimal-sym-1} 

}

\caption{Cover of \textit{Vaccinium vitis-idaea} with respect to pH and \% soil
nitrogen. Larger symbols indicate higher percent plant cover.}\label{fig:sym}
\end{figure}

\emph{Vaccinium vitis-idaea} appears to prefer intermediate to low levels of soil N, and acidic soils. The somewhat negative association between soil N and pH is probably due to soil leaching, because H\(^+\) (and Al\(^{3+}\)) cations are more strongly adsorbed by soil colloids than bases in poorly drained soils.

A 3D plot of the same associations can be created using the \texttt{scatterplot3d()} function from the package \emph{scatterplot3d} .

\vspace{0.1in}

\begin{Shaded}
\begin{Highlighting}[numbers=left,,]
\FunctionTok{library}\NormalTok{(scatterplot3d)}
\NormalTok{Fig }\OtherTok{\textless{}{-}} \ControlFlowTok{function}\NormalTok{(}\AttributeTok{angle =} \DecValTok{55}\NormalTok{)\{}
\NormalTok{s3d }\OtherTok{\textless{}{-}} \FunctionTok{scatterplot3d}\NormalTok{(}\FunctionTok{cbind}\NormalTok{(varechem}\SpecialCharTok{$}\NormalTok{N, varechem}\SpecialCharTok{$}\NormalTok{pH, varespec}\SpecialCharTok{$}\NormalTok{Vaccviti),}
\AttributeTok{type =} \StringTok{"h"}\NormalTok{, }\AttributeTok{highlight.3d =} \ConstantTok{TRUE}\NormalTok{, }\AttributeTok{angle =}\NormalTok{ angle, }\AttributeTok{scale =}\NormalTok{ .}\DecValTok{7}\NormalTok{, }\AttributeTok{pch =} \DecValTok{16}\NormalTok{,}
\AttributeTok{xlab =} \StringTok{"N"}\NormalTok{, }\AttributeTok{ylab =} \StringTok{"pH"}\NormalTok{, }\AttributeTok{zlab =}
  \FunctionTok{expression}\NormalTok{(}\FunctionTok{paste}\NormalTok{(}\FunctionTok{italic}\NormalTok{(Vaccinium), }\StringTok{" "}\NormalTok{, }\FunctionTok{italic}\NormalTok{(vitis}\SpecialCharTok{{-}}\NormalTok{idaea), }
                   \StringTok{" \% cover"}\NormalTok{)))}

\NormalTok{lm1 }\OtherTok{\textless{}{-}}\FunctionTok{lm}\NormalTok{(varespec}\SpecialCharTok{$}\NormalTok{Vaccviti }\SpecialCharTok{\textasciitilde{}}\NormalTok{ varechem}\SpecialCharTok{$}\NormalTok{N }\SpecialCharTok{+}\NormalTok{ varechem}\SpecialCharTok{$}\NormalTok{pH)}
\NormalTok{s3d}\SpecialCharTok{$}\FunctionTok{plane3d}\NormalTok{(lm1)}
\NormalTok{\}}
\FunctionTok{Fig}\NormalTok{()}
\end{Highlighting}
\end{Shaded}

In the code above, I define the figure to be a function (named \texttt{Fig}) to allow the angle of rotation in the 3D scatterplot to be easily changed using the \texttt{angle} argument in \texttt{Fig} (Line 2). Functions will be addressed in detail in Ch 8. By stipulating \texttt{highlight.3d\ =\ TRUE} (Line 4), objects that are closer to the viewer with respect to the \texttt{x} plane are given warmer colors. A regression ``plane'' is also overlaid on the graph (Lines 9-10). The fitted plane is produced from a multiple regression model created by the function \texttt{lm()}.
\index[R]{\texttt{lm()}}

The result is shown in Fig \ref{fig:tds}.

\begin{figure}[H]

{\centering \includegraphics{figure/minimal-tds-1} 

}

\caption{Cover of \textit{Vaccinium vitis-idaea} with respect to pH and \% soil
nitrogen, depicted in a 3D scatterplot.}\label{fig:tds}
\end{figure}

\end{example}

\(\blacksquare\)

\index[terms]{R!package!\textit{vegan}}
\index[R]{\texttt{scatterplot3d::scatterplot3d()}}
\index[terms]{R!package!\textit{scatterplot3d}}

\section{Animation}\label{animation}

\index[terms]{R!graphics!animation}
\emph{Animations} can be created in \textbf{R} to illustrate a wide range of processes \citep{xie2013, xie2018}. Functions with animation are generally based on loops (Section \ref{loops}) with some method of slowing the loop; usually the function \texttt{Sys.sleep()}.
\index[R]{\texttt{Sys.sleep()}}

\begin{example}
\protect\hypertarget{exm:exm-anm}{}\label{exm:exm-anm}

\(\text{}\)\\
Here we add animation to the 3D scatterplot shown in Fig \ref{fig:tds}. This will be facilitated by the fact that the plot is a function with an argument whose alteration results in modification of the graph.

\vspace{0.1in}

\begin{Shaded}
\begin{Highlighting}[numbers=left,,]
\NormalTok{fig.rot }\OtherTok{\textless{}{-}} \ControlFlowTok{function}\NormalTok{()\{}
\FunctionTok{lapply}\NormalTok{(}\FunctionTok{seq}\NormalTok{(}\DecValTok{1}\NormalTok{, }\DecValTok{360}\NormalTok{), }\ControlFlowTok{function}\NormalTok{(i)\{}
  \FunctionTok{Fig}\NormalTok{(i)}
  \FunctionTok{Sys.sleep}\NormalTok{(.}\DecValTok{1}\NormalTok{)}
\NormalTok{  \})}
\NormalTok{\}}

\FunctionTok{fig.rot}\NormalTok{()}
\CommentTok{\# save frames into one GIF:}
\FunctionTok{library}\NormalTok{(animation)}
\FunctionTok{saveGIF}\NormalTok{(}\FunctionTok{fig.rot}\NormalTok{(), }\AttributeTok{interval =} \DecValTok{1}\NormalTok{, }\AttributeTok{movie.name =} \StringTok{"vaccinium.gif"}\NormalTok{)}
\end{Highlighting}
\end{Shaded}

Recall that \texttt{lapply()} returns a list of the same length as its first argument \texttt{X}, whose elements result from applying a function, given in the second argument, to corresponding elements of \texttt{X}. In the code above, an argument-less function is created containing a loop run by \texttt{lapply()} (Lines 2 - 5). As the loop index \texttt{i} changes from \texttt{i\ =\ 1} to \texttt{i\ =\ 360} (Line 2) this changes the \texttt{angle} argument in the function \texttt{fig()}, used in creating Fig \ref{fig:tds}. At the end of each step in the loop, the system is paused for a tenth of second (Line 4) with the function \texttt{Sys.sleep()} to allow each ``frame'' of the animation to be viewed separately. In the (optional) last two lines of code, the \textbf{R} package \emph{animation} is loaded, and the function \texttt{animation::saveGIF()} is used to save the animation in a GIF file format\footnote{Use of \texttt{saveGIF} requires installation of open source software \href{http://www.imagemagick.org/script/convert.php}{ImageMagick} or \href{http://www.graphicsmagick.org}{GraphicsMagick} (see \texttt{?saveGIF}).}.

The animation result is shown in \ref{fig:tdsa1}.

\begin{figure}[H]

{\centering \animategraphics[,controls,loop]{10}{figure/minimal-tdsa1-}{1}{360}

}

\caption{Animated version of the 3D scatterplot from Fig \ref{fig:tds}. Animation controls are provided by the LaTeX package \textit{animation}.}\label{fig:tdsa1}
\end{figure}

\end{example}

\(\blacksquare\)

Working animations generated in \textbf{R} can be placed into HTML documents created under \textbf{R} Markdown, or PDF documents created using Sweave-alike approaches (Section \ref{workflow}). The former approach currently requires installation of the \emph{gifski} \textbf{R} package \index[terms]{R!package!\textit{gifski}} and the specification: \texttt{animation.hook\ =\ "gifski"} among the chunk options for the animation. The latter approach requires loading of the \emph{animate} LaTeX package and using the chunk option \texttt{fig.show\ =\ "animate"}. PDF animations can viewed using a number of PDF viewers including the Foxit\(^{\circledR}\) and Adobe\(^{\circledR}\) Acrobat Readers.

\begin{example}
\protect\hypertarget{exm:exm-rgl}{}\label{exm:exm-rgl}

\(\text{}\)\\
We can also create hand-rotatable 3D figures under the \emph{rgl} real-time rendering system.

\vspace{0.1in}

\vspace{0.1in}

\begin{Shaded}
\begin{Highlighting}[numbers=left,,]
\NormalTok{expg }\OtherTok{\textless{}{-}} \FunctionTok{expand.grid}\NormalTok{(varechem}\SpecialCharTok{$}\NormalTok{pH, varechem}\SpecialCharTok{$}\NormalTok{N)}
\NormalTok{subs }\OtherTok{\textless{}{-}} \FunctionTok{cbind}\NormalTok{(varechem}\SpecialCharTok{$}\NormalTok{pH, varechem}\SpecialCharTok{$}\NormalTok{N)}
\NormalTok{tf }\OtherTok{\textless{}{-}}\NormalTok{ (expg[,}\DecValTok{1}\NormalTok{] }\SpecialCharTok{==}\NormalTok{ subs[,}\DecValTok{1}\NormalTok{]) }\SpecialCharTok{\&}\NormalTok{ (expg[,}\DecValTok{2}\NormalTok{] }\SpecialCharTok{==}\NormalTok{ subs[,}\DecValTok{2}\NormalTok{])}
\NormalTok{y }\OtherTok{\textless{}{-}} \FunctionTok{ifelse}\NormalTok{(tf }\SpecialCharTok{==} \ConstantTok{TRUE}\NormalTok{, varespec}\SpecialCharTok{$}\NormalTok{Vaccviti, }\ConstantTok{NA}\NormalTok{)}
\NormalTok{surface }\OtherTok{\textless{}{-}} \FunctionTok{data.frame}\NormalTok{(}\AttributeTok{N =}\NormalTok{ expg[,}\DecValTok{1}\NormalTok{], }\AttributeTok{pH =}\NormalTok{ expg[,}\DecValTok{2}\NormalTok{], }\AttributeTok{vac.vit =}\NormalTok{ y)}

\FunctionTok{library}\NormalTok{(car)}
\FunctionTok{scatter3d}\NormalTok{(vac.vit }\SpecialCharTok{\textasciitilde{}}\NormalTok{ N }\SpecialCharTok{+}\NormalTok{ pH, }\AttributeTok{data =}\NormalTok{ surface, }\AttributeTok{surface =} \ConstantTok{TRUE}\NormalTok{, }\AttributeTok{fit =} \StringTok{"linear"}\NormalTok{,}
\AttributeTok{zlab =} \StringTok{"N"}\NormalTok{, }\AttributeTok{xlab =} \StringTok{"pH"}\NormalTok{, }\AttributeTok{ylab =} \StringTok{"Vaccinium vitilus (\% cover)"}\NormalTok{)}
\end{Highlighting}
\end{Shaded}

In the code above, a initial surface is created that considers all possible combinations of pH and N outcomes (Line 1) and actual occurrences of \texttt{varespec\$Vaccviti} at \emph{observed} combinations (Lines 3 and 4). The function \texttt{scatter3d()} in the package \emph{car} uses tools from the \emph{rgl} package to render a three dimensional scatterplot. The scatterplot will be rotatable within an \textbf{R} session, and can be rendered as a rotatable graphic in an \textbf{R} Markdown HTML\footnote{See \texttt{rgl::playwidget()} if you you are reading this document as a pdf.}. The form of the plot is shown in Fig \ref{fig:figrgl1}, although its rotatability will require an interactive \textbf{R} environment or a ammenable HTML/PDF framework. Plots from \emph{rgl} can also be rendered and manipulated in Shiny apps (see Ch \ref{ch11}).

\begin{figure}[H]

{\centering \includegraphics[width=0.7\linewidth,]{figs6/sprgl} 

}

\caption{A hand rotatable graphics object (within an \textbf{R} interactive or suitable PDF/HTML environment).}\label{fig:figrgl1}
\end{figure}

\end{example}

\(\blacksquare\)

\index[terms]{R!package!\textit{car}}
\index[terms]{R!package!\textit{rgl}}
\index[R]{\texttt{car::scatter3d()}}
\index[R]{\texttt{lapply()}}
\index[R]{\texttt{Sys.sleep()}}
\index[terms]{R!package!\textit{animation}}
\index[R]{\texttt{animation::saveGIF()}}
\index[terms]{Loop (programming)}

\section*{Exercises}\label{exercises-5}
\addcontentsline{toc}{section}{Exercises}

\begin{enumerate}
\def\labelenumi{\arabic{enumi}.}
\tightlist
\item
  Consider the variables: \texttt{x\ \textless{}-\ c(1,2,2.5,3,4,3,5)} and \texttt{y\ \textless{}-\ c(6,4.3,3,3.1,2,1.7,1)}.

  \begin{enumerate}
  \def\labelenumii{(\alph{enumii})}
  \tightlist
  \item
    Make a plot with \texttt{x} defining the \emph{x}-axis and \texttt{y} on the \emph{y}-axis.
  \item
    Make every point in the scatterplot a different color.
  \item
    Make every point a different shape.
  \item
    Create a legend describing all the shape and color combinations of all points one through seven (call them Point 1, Point 2, etc.).
  \item
    Convert from a point to an overplotted line and point plot.
  \item
    Change the label of the \emph{x}-axis to ``\emph{Abscissa} axis'' and the label of the \emph{y}-axis to be ``\emph{Ordinate} axis'' using a \texttt{plotmath} approach. This will require use of the functions \texttt{expression()}, \texttt{paste()} and \texttt{italic()}.
  \item
    Place the text ``\emph{y} = -1.203\emph{x} + 6.539'' at coordinates \texttt{x\ =\ 2}, \texttt{y\ =\ 2.5} using the function \texttt{text()}. Italicize as indicated.
  \item
    Place a line with a slope of -1.203 and an \emph{y}-intercept of 6.539 on the plot using the function \texttt{abline()}.
    \vspace{0.1in}
  \end{enumerate}
\item
  The \texttt{Indometh} dataframe from the package \emph{datasets} describes pharmacokinetics of the drug indomethacin following intravenous injections for six human subjects.

  \begin{enumerate}
  \def\labelenumii{(\alph{enumii})}
  \tightlist
  \item
    Create a histogram for the variable \texttt{conc}, which gives plasma concentrations of indomethacin in (mcg/ml) in subjects over time. Use an appropriate \emph{x}-axis label.
  \item
    Create a scatterplot of \texttt{conc} as a function of \texttt{time} (in hours). Create appropriate axis labels.
  \item
    Change symbols and colors of points in \textbf{(b)} based on levels in \texttt{Subject}.
  \item
    Create a wide table format for \texttt{Indometh} using: \texttt{wide\ \textless{}-\ unstack(Indometh,\ conc\ \textasciitilde{}\ Subject)} and \texttt{names(wide)\ \textless{}-\ paste("Subject",\ c(1,4,2,5,6,3))}.
  \item
    Create a stacked barplot and a side by side barplot based on: \texttt{barplot(as.matrix(wide))}.
  \item
    Use appropriate \emph{y}-axis labels.
  \item
    Create a multiple line plot (with a line for each subject) using:
    \texttt{time\ \textless{}-\ c(0.25,\ 0.50,\ 0.75,\ 1.00,\ 1.25,\ 2.00,\ 3.00,\ 4.00,\ 5.00,\ 6.00,\ 8.00)} and \texttt{matplot(x\ =\ time,\ y\ =\ wide,\ type\ =\ "l")}
  \item
    Generate appropriate axis labels for the plot.
  \item
    Create an appropriate legend for the plot created in (h). The colors and line types used by \texttt{matplot()} will be 1:6. The order of subjects is 1, 4, 2, 5, 6, 3.
    \vspace{0.1in}
  \end{enumerate}
\item
  The dataframe \texttt{life.exp} from \emph{asbio} compares life expectancy of field mice given five different diets.

  \begin{enumerate}
  \def\labelenumii{(\alph{enumii})}
  \tightlist
  \item
    Make and interpret a boxplot showing \texttt{lifespan} as a function of \texttt{treatment}.
  \item
    Make an interval plot by hand showing \texttt{lifespan} as a function of \texttt{treatment} using means as measures of location, and standard deviations to generate error bars.
    \vspace{0.1in}
  \end{enumerate}
\item
  (Advanced) Conduct an ANOVA and a \emph{post hoc} pairwise comparison of means with Tukey's HSD using: \texttt{anova(lm(lifespan\ \textasciitilde{}\ treatment,\ data\ =\ life.exp))}, \texttt{tukey\ \textless{}-\ with(life.exp,\ pairw.anova(lifespan,\ treatment))}.

  \begin{enumerate}
  \def\labelenumii{(\alph{enumii})}
  \tightlist
  \item
    Create an interval plot summarizing these results using: \texttt{plot(tukey)}.
  \item
    Interpret (a).
    \vspace{0.1in}
  \end{enumerate}
\item
  Load the \texttt{C.isotope} dataframe from the package \emph{asbio}. Using \texttt{par(mfrow())}, create a graphical device holding three plots in a single row, i.e., the three plots will be side by side.

  \begin{enumerate}
  \def\labelenumii{(\alph{enumii})}
  \tightlist
  \item
    In the first plot, show \(\delta ^{14}\)C as a function of time (\texttt{decimal.date}) using a line plot. Use appropriate axis labels.
  \item
    In the second plot, show CO\(_2\) concentration as a function of time in a scatterplot.
  \item
    In the third plot, show measurement precision (column four in the dataset) as a function of \(\delta ^{14}\)C.
    \vspace{0.1in}
  \end{enumerate}
\item
  Load the \texttt{goats} dataframe from package \emph{asbio}.

  \begin{enumerate}
  \def\labelenumii{(\alph{enumii})}
  \tightlist
  \item
    Create a scatterplot of \texttt{NO3} as a function of \texttt{feces}.
  \item
    Make a plot showing \texttt{NO3} and \texttt{organic.matter} as a simultaneous function of \texttt{feces} by adding a second \emph{y}-axis.
  \item
    Change symbol sizes in (a) to reflect the values in organic.matter.
  \item
    Create a 3D scatterplot with \texttt{scatterplot3d::scatterplot3d}, depicting \texttt{NO3} as a function of \texttt{organic.matter} and \texttt{feces}.
  \end{enumerate}
\end{enumerate}

\chapter{Grid Graphics, Including ggplot2}\label{ch7}

\begin{quote}
\emph{``If you think you can learn all of \textbf{R}, you are wrong. For the foreseeable future you will not even be able to keep up with the new additions."}
\end{quote}

- \textbf{Patrick Burns}, \emph{CambR User Group Meeting, Cambridge (May 2012)}

\section{Grid Graphics}\label{grid-graphics}

There are a large number of auxiliary \textbf{R} packages specifically for graphics. Many of these utilize or extend the base \textbf{R} graphics approaches described in Chapter \ref{ch6}. Several successful newer packages, however, rely on the \textbf{R} grid graphics system (see \citet{murrell2019}), codified in the package \emph{grid} \citep{R}. \index[terms]{R!package!\textit{grid}}\index[terms]{Grid graphics} The grid graphics system itself provides only low-level facilities with no high-level functions to generate complete plots. Nonetheless, several successful packages have built high-level functions on grid foundations including \emph{lattice}, \emph{gridGraphics} --which converts plots drawn with the base \textbf{R} graphics, e.g., \texttt{plot()}, to identical grid output-- and the highly popular \emph{ggplot2} package. Functions from the latter package are the major focus of this chapter.
\index[terms]{R!package!\textit{lattice}}\index[terms]{R!package!\textit{gridGraphics}}

\section{\texorpdfstring{\emph{lattice}}{lattice}}\label{lattice}

\index[terms]{R!graphics!trellis plots}
Among other applications, the \emph{lattice} package \citep{lattice} contains functions for implementing the trellis graphical system\footnote{The \textbf{R} trellis graphics system was originally developed for \textbf{S} and \textbf{S}-Plus at Bell Labs (see \citet{becker1996}). The \emph{lattice} package can be considered a re-implementation of this original system.}, so-called because it often utilizes a rectangular array of plots resembling a garden trellis \citep{ryan1993}.\index[terms]{Trellis graphics} Trellis plots, generated from \emph{lattice}, are an important component of several widely-used \textbf{R} packages, including \emph{nlme}\index[terms]{R!package!\textit{nlme}}, which allows the generation of linear and nonlinear mixed effects models (see \citet{aho2014}, Ch 10).

\begin{example}
\protect\hypertarget{exm:exm-lat1}{}\label{exm:exm-lat1}

\(\text{}\)\\
A simple example of a call to trellis plotting is shown in Fig \ref{fig:trellI}. The \texttt{datasets::Indometh} dataframe, previously used in Examples in Ch \ref{ch6}, records pharmacokinetics of the drug indomethacin, following intravenous injections given to human subjects. The dataframe belongs to several grouped object classes, defined in \emph{nlme}, which have their own plotting methods (Ch \ref{ch8}), and are implemented through a generic call to \texttt{plot()}. We are, of course, more familiar with the use of \texttt{plot()} in base \textbf{R} graphics approaches, implemented via the \emph{graphics} package.

\vspace{0.1in}

\begin{Shaded}
\begin{Highlighting}[]
\FunctionTok{library}\NormalTok{(nlme)}
\FunctionTok{plot}\NormalTok{(Indometh)}
\end{Highlighting}
\end{Shaded}

\begin{figure}[H]

{\centering \includegraphics{figure/minimal-trellI-1} 

}

\caption{Example of a trellis plot. Indomethacin levels are tracked in six human subjects over eight hours following intravenous injections.}\label{fig:trellI}
\end{figure}

\end{example}

\(\blacksquare\)

The \emph{lattice} package contains several high level plotting functions that can be considered analogues of base \textbf{R} \emph{graphics} functions. These include:

\begin{itemize}
\tightlist
\item
  \texttt{lattice::xyplot()}, which is similar to \texttt{graphics::plot()} in its default \texttt{type\ =\ "p"} mode,
\item
  \texttt{lattice::histogram()}, which is analogous to \texttt{graphics::hist()},
\item
  \texttt{lattice::barchart()}, which is similar to \texttt{graphics::barplot()},
\item
  \texttt{lattice::levelplot()}, which is analogous to \texttt{graphics::image()}, and
\item
  \texttt{lattice::wireframe()}, which is similar to \texttt{graphics::persp()}.
\end{itemize}

In general, trellis plots in \emph{lattice} can be created using a conditional formula as the first argument of its functions. This will have the form \texttt{y\ \textasciitilde{}\ x\textbar{}z}, which signifies \texttt{y} is a function of \texttt{x}, given levels in \texttt{z}.

\begin{example}
\protect\hypertarget{exm:exm-latp1}{}\label{exm:exm-latp1}\(\text{}\)\\
Consider a summarization of the association of age and tobacco use (\texttt{LOW} and \texttt{HIGH}) and esophageal cancer cases using the \texttt{e.cancer} dataset \citep{breslow1980} from \emph{asbio} (Fig \ref{fig:ecancer}).

\vspace{0.1in}

\begin{Shaded}
\begin{Highlighting}[numbers=left,,]
\FunctionTok{data}\NormalTok{(e.cancer)}

\FunctionTok{library}\NormalTok{(tidyverse)}
\NormalTok{means }\OtherTok{\textless{}{-}}\NormalTok{ e.cancer }\SpecialCharTok{|\textgreater{}}  \CommentTok{\# obtain means}
  \FunctionTok{group\_by}\NormalTok{(age.grp, tobacco) }\SpecialCharTok{|\textgreater{}}
  \FunctionTok{summarize}\NormalTok{(}\AttributeTok{cases =} \FunctionTok{mean}\NormalTok{(cases))}

\FunctionTok{library}\NormalTok{(lattice)}
\FunctionTok{barchart}\NormalTok{(cases }\SpecialCharTok{\textasciitilde{}}\NormalTok{ age.grp}\SpecialCharTok{|}\NormalTok{tobacco, }\AttributeTok{data =}\NormalTok{ means, }\AttributeTok{xlab =} \StringTok{"Age"}\NormalTok{, }
         \AttributeTok{ylab =} \StringTok{"No. of cases"}\NormalTok{)}
\end{Highlighting}
\end{Shaded}

\begin{figure}[H]

{\centering \includegraphics{figure/minimal-ecancer-1} 

}

\caption{Use of \texttt{lattice::barchart()} to illustrate changes in esophogeal cancer cases with subject tobacco use and age. Bar heights are means.}\label{fig:ecancer}
\end{figure}

Note the use of pipes and \emph{tidyverse} functions (Ch \ref{ch5}) to obtain mean numbers of \texttt{cases} for combinations of levels in \texttt{age.grp} and \texttt{tobacco} (Lines 3-6). The formula\texttt{cases\ \textasciitilde{}\ age.grp\textbar{}tobacco} (Line 8) indicates that the mean \texttt{cases} should considered as a function of levels in \texttt{age.grp}, given levels in \texttt{tobacco}.
\end{example}

\(\blacksquare\)

\index[R]{\texttt{barplot()}}
\index[R]{\texttt{plot()}}
\index[R]{\texttt{hist()}}
\index[R]{\texttt{image()}}
\index[R]{\texttt{persp()}}
\index[R]{\texttt{lattice::barchart()}}
\index[R]{\texttt{lattice::xyplot()}}
\index[R]{\texttt{lattice::wireframe()}}
\index[R]{\texttt{lattice::histogram()}}
\index[R]{\texttt{lattice::levelplot()}}

Approaches in \emph{lattice} can be used in may non-trellis applications.

\begin{example}
\protect\hypertarget{exm:exm-lat2}{}\label{exm:exm-lat2}

\(\text{}\)\\
Figure \ref{fig:lattice} provides examples of three dimensional graphics generation using the \emph{lattice} functions \index[terms]{R!graphics!3D plots} \texttt{levelplot()}, \texttt{contourplot()}, and \texttt{wireframe()}. The functions are easiest to use when data are in a spatial grid format with row and column numbers defining evenly spaced intervals from some reference point, and cell responses themselves constitute ``heights'' for the \emph{z} (vertical) axis. The popular \texttt{volcano} dataset, used in the figure, describes the topography of Maungawhau / Mount Eden, a scoria cone in the Mount Eden suburb of Auckland, New Zealand. In this case, rows and columns represent 10m Cartesian intervals. The first row contains elevations (in meters above sea level) for northernmost points, whereas the first column contains elevations of westernmost points. The argument \texttt{split} in \texttt{plot.trellis()} is used with both graphs in Fig \ref{fig:lattice} (Lines 5 and 7). It is a vector of 4 integers \texttt{c(x,\ y,\ nx,\ ny)} that indicate where to position the current plot at the \texttt{x}, \texttt{y} position in a regular array of \texttt{nx} by \texttt{ny} plots.

\vspace{0.1in}

\begin{Shaded}
\begin{Highlighting}[]
\FunctionTok{library}\NormalTok{(lattice)}
\FunctionTok{plot}\NormalTok{(}\FunctionTok{levelplot}\NormalTok{(volcano, }\AttributeTok{col.regions =}\NormalTok{ heat.colors, }\AttributeTok{xlab =} \StringTok{"x"}\NormalTok{, }\AttributeTok{ylab =} \StringTok{"y"}\NormalTok{),}
     \AttributeTok{split =} \FunctionTok{c}\NormalTok{(}\DecValTok{1}\NormalTok{, }\DecValTok{1}\NormalTok{, }\DecValTok{1}\NormalTok{, }\DecValTok{2}\NormalTok{), }\AttributeTok{more =} \ConstantTok{TRUE}\NormalTok{,}
     \AttributeTok{panel.width =} \FunctionTok{list}\NormalTok{(}\AttributeTok{x =} \FloatTok{5.4}\NormalTok{, }\AttributeTok{units =} \StringTok{"inches"}\NormalTok{))}

\FunctionTok{plot}\NormalTok{(}\FunctionTok{wireframe}\NormalTok{(volcano, }\AttributeTok{panel.aspect =} \FloatTok{0.7}\NormalTok{, }\AttributeTok{zoom =} \DecValTok{1}\NormalTok{, }\AttributeTok{lwd =} \FloatTok{0.01}\NormalTok{, }
               \AttributeTok{xlab =} \StringTok{"x"}\NormalTok{, }\AttributeTok{ylab =} \StringTok{"y"}\NormalTok{, }\AttributeTok{zlab =} \StringTok{"z"}\NormalTok{), }
     \AttributeTok{split =} \FunctionTok{c}\NormalTok{(}\DecValTok{1}\NormalTok{, }\DecValTok{2}\NormalTok{, }\DecValTok{1}\NormalTok{, }\DecValTok{2}\NormalTok{), }\AttributeTok{more =} \ConstantTok{FALSE}\NormalTok{,}
     \AttributeTok{panel.width =} \FunctionTok{list}\NormalTok{(}\AttributeTok{x =} \FloatTok{5.4}\NormalTok{, }\AttributeTok{units =} \StringTok{"inches"}\NormalTok{))}
\end{Highlighting}
\end{Shaded}

\begin{figure}[H]

{\centering \includegraphics{figure/minimal-lattice-1} 

}

\caption{Representations of Maungawhau (Mt Eden) using \textit{lattice} functions.}\label{fig:lattice}
\end{figure}

\end{example}

\(\blacksquare\)

\index[R]{\texttt{lattice::levelplot()}}
\index[R]{\texttt{lattice::contourplot()}} \index[R]{\texttt{lattice::wireframe()}}
\index[R]{\texttt{lattice::plot.trellis()}}

While \emph{lattice} can be used to generate nice graphs, many users have found its coding requirements to be burdensome and non-intuitive. This issue, coupled with the desirable characteristics of the grid graphics system, prompted the development of the package \emph{ggplot2}, one of the \emph{tidyverse} collection of packages (Ch \ref{ch5}).

\section{\texorpdfstring{\emph{ggplot2}}{ggplot2}}\label{ggplot2}

The \emph{ggplot2} package (formerly \emph{ggplot}) emulates the ``grammar of graphics'', that underlies all statistical graphics \citep{wilkinson2012}. According to its developer, ``\emph{ggplot2} \(\ldots\) tries to take the good parts of \emph{base} and \emph{lattice}
graphics and none of the bad.'' The success of the \emph{ggplot2} package is evident from its rich ecosystem of contributed \href{https://exts.ggplot2.tidyverse.org/gallery/}{extension packages}. Detailed descriptions of the \emph{ggplot2} package can be found in \citet{wickham2010}, and \citet{wickham2016}. Helpful \(ggplot2\) ``cheatsheets'' can be found \href{https://rstudio.github.io/cheatsheets/html/data-visualization.html}{here}. Like most grid packages, \emph{ggplot2} does not play well with base \textbf{R} graphics. In fact, \emph{ggplot2} is based on its own unique object oriented system, the \emph{ggproto} system \footnote{For more information type: \texttt{?ggplot2::ggproto}.}.
\index[terms]{R!package!\textit{ggplot2}}
\index[terms]{ggproto}

\subsection{\texorpdfstring{\texttt{ggplot()}}{ggplot()}}\label{ggplot}

The function \texttt{ggplot()} is used to initialize essentially all plotting procedures in \emph{ggplot2}. There are three common approaches:

\begin{enumerate}
\def\labelenumi{\arabic{enumi}.}
\tightlist
\item
  \texttt{ggplot(df,\ aes(x,\ y,\ other\ aesthetics))}

  \begin{itemize}
  \tightlist
  \item
    Here \texttt{df} is a tibble or dataframe. and \texttt{aes()} represents aesthetic mappings. This approach is recommended if all layers use the same data and aesthetics.
  \end{itemize}
\item
  \texttt{ggplot(df)}

  \begin{itemize}
  \tightlist
  \item
    Here only the dataframe or tibble to be used is identified up-front. This approach is useful if graphical layers use different \(x\) and \(y\) coordinates, drawn from the same dataset, \texttt{df}.
  \end{itemize}
\item
  \texttt{ggplot()}

  \begin{itemize}
  \tightlist
  \item
    Here a ggplot skeleton is initialized that is fleshed out as layers are added. This approach is recommended if more than more than one dataset is used in the creation of graphical layers.
  \end{itemize}
\end{enumerate}

\index[R]{\texttt{ggplot2::ggplot()}}
One of these three approaches will be used as the first line of code when creating a \emph{ggplot2} graphic. Layers will then be added representing geoms, themes, and aesthetics (see Section \ref{gat} immediately below). To clarify coding steps, this is typically done by separating layers into lines, connected with the continuation prompt \texttt{+}.
\index[terms]{R!continuation prompt}

\begin{example}
\protect\hypertarget{exm:exm-gg1}{}\label{exm:exm-gg1}

\(\text{}\)\\
In the code below I have initiated a ggplot, under Approach 1 discussed above, using data from a dataframe called \texttt{df} that contains variables named \texttt{x} and \texttt{y}, that will define the \(x\) and \(y\) coordinates for the plot. I have also added two geom layers and a theme, via the imaginary functions \texttt{geom1}, \texttt{geom2} and \texttt{theme1}.

\vspace{0.1in}

\begin{Shaded}
\begin{Highlighting}[]
\FunctionTok{ggplot}\NormalTok{(df, }\FunctionTok{aes}\NormalTok{(}\AttributeTok{x =}\NormalTok{ x, }\AttributeTok{y =}\NormalTok{ y)) }\SpecialCharTok{+}
  \FunctionTok{geom1}\NormalTok{() }\SpecialCharTok{+}
  \FunctionTok{geom2}\NormalTok{() }\SpecialCharTok{+}
  \FunctionTok{theme1}\NormalTok{()}
\end{Highlighting}
\end{Shaded}

\end{example}

\(\blacksquare\)

\subsection{Geoms, Aesthetics, and Themes}\label{gat}

The \emph{ggplot2} package facilitates the generation of overlays with \emph{geoms}, short for ``geometric objects'', aesthetics, and themes. A number of \emph{ggplot2} geom functions are shown in Table \ref{tab:geomfunctions}. Note that arguments in geom functions are fairly consistent. The argument \texttt{mapping} refers to aesthetic mappings, often specified with the \emph{ggplot2} function \texttt{aes()}. A few aesthetic mapping functions are shown in Table \ref{tab:aes}. An explicit definition for the \texttt{stat} argument is required by several geoms, e.g., \texttt{geom\_col()} and \texttt{geom\_bar()}, and can take the form \texttt{stat\ =\ "identity"}, indicating that raw unsummarized data are to be plotted. The \emph{ggplot2} package also allows specification of general graphical themes including user-defined themes, via the function \texttt{theme()}. An exhaustive list of \(> 90\) potential \texttt{theme()} arguments can be found by typing \texttt{?theme}. Pre-defined \emph{ggplot2} theme frameworks include \texttt{theme\_gray()}, the signature \emph{ggplot2} theme (with a grey background and white gridlines), \texttt{theme\_bw()}, \texttt{theme\_classic()}, \texttt{theme\_dark()}, \texttt{theme\_minimal()}, and many others.
\index[terms]{R!package!\textit{ggplot2}!aesthetic}
\index[terms]{R!package!\textit{ggplot2}!mapping}
\index[terms]{R!package!\textit{ggplot2}!theme}
\index[R]{\texttt{ggplot2::theme()}}
\index[R]{\texttt{ggplot2::theme\_dark()}}
\index[R]{\texttt{ggplot2::theme\_bw()}}
\index[R]{\texttt{ggplot2::theme\_classic()}}
\index[R]{\texttt{ggplot2::theme\_minimal()}}

\vspace{0.1in}

\begin{longtable}{lll}
\caption{A few geom alternatives.} \\ 
  \toprule
Geom function & Usage & Impt. arguments \\ 
  \midrule
\texttt{geom\_abline()} & Add reference lines  &  \texttt{mapping} \\ 
  \texttt{geom\_hline()} &  &  \texttt{data} \\ 
  \texttt{geom\_vline()} &  &  \texttt{slope} \\ 
   &  &  \texttt{intercept}  \\ 
   \midrule
\texttt{geom\_segment()} & Add lines and curves  &  \texttt{mapping} \\ 
  \texttt{geom\_curve()} &  &  \texttt{data} \\ 
   &  &  \texttt{position}  \\ 
   \midrule
\texttt{geom\_area()} & Area and ribbon charts   &  \texttt{mapping} \\ 
  \texttt{geom\_ribbon()} &  &  \texttt{data} \\ 
   &  &  \texttt{stat}    \\ 
   \midrule
\texttt{geom\_bar()} & Bar charts   &  \texttt{mapping} \\ 
  \texttt{geom\_col()} &  &  \texttt{data} \\ 
   &  &  \texttt{stat}    \\ 
   \midrule
\texttt{geom\_bin2d()} & Heatmap of bin counts   &   \texttt{mapping} \\ 
   &  &  \texttt{data} \\ 
   &  &  \texttt{stat}   \\ 
   \midrule
\texttt{geom\_boxplot()} & Boxplots   &   \texttt{mapping} \\ 
   &  &  \texttt{data} \\ 
   &  &  \texttt{stat}   \\ 
   \midrule
\texttt{geom\_contour\_filled()} & Generate 2D contours of 3D surface   &  \texttt{mapping} \\ 
   &  &  \texttt{data} \\ 
   &  &  \texttt{stat}   \\ 
   \midrule
\texttt{geom\_count()} & Count overlapping points  &  \texttt{mapping} \\ 
  \texttt{geom\_sum()} &  &  \texttt{data} \\ 
   &  &  \texttt{stat}    \\ 
   \midrule
\texttt{geom\_crossbar()} & Add lines, crossbars, error bars   &  \texttt{mapping} \\ 
  \texttt{geom\_errorbar()} &  &  \texttt{data} \\ 
  \texttt{geom\_linerange()} &  &  \texttt{stat}    \\ 
  \texttt{geom\_pointsrange()} &  &  \\ 
   \midrule
\texttt{geom\_density()} & Smoothed densities   &  \texttt{mapping} \\ 
   &  &  \texttt{data} \\ 
   &  &  \texttt{stat}   \\ 
   \midrule
\texttt{geom\_density\_2d()} & Contours of 2D densities   &  \texttt{mapping} \\ 
  \texttt{geom\_density\_2d\_filled()} &  &  \texttt{data} \\ 
   &  &  \texttt{stat}   \\ 
   \midrule
\texttt{geom\_dotplot()} & Dot plots   &  \texttt{mapping} \\ 
   &  &  \texttt{data} \\ 
   &  &  \texttt{position}   \\ 
   \midrule
\texttt{geom\_errorbarh()} & Horizontal error bars   &  \texttt{mapping} \\ 
   &  &  \texttt{data} \\ 
   &  &  \texttt{stat}  \\ 
   \midrule
\texttt{geom\_freqpoly()} & Histograms   &  \texttt{mapping} \\ 
  \texttt{geom\_histogram()} &  &  \texttt{data} \\ 
   &  &  \texttt{stat}   \\ 
   \midrule
\texttt{geom\_function()} & Draw curve from function   &  \texttt{mapping} \\ 
   &  &  \texttt{data} \\ 
   &  &  \texttt{stat}   \\ 
   \midrule
\texttt{geom\_hex()} & Hexagonal heat map   &  \texttt{mapping} \\ 
   &  &  \texttt{data} \\ 
   &  &  \texttt{stat}    \\ 
   \midrule
\texttt{geom\_jitter()} & Jittered points   &  \texttt{mapping} \\ 
   &  &  \texttt{data} \\ 
   &  &  \texttt{stat}   \\ 
   \midrule
\texttt{geom\_text()} & Add text &  \texttt{mapping} \\ 
   &  &  \texttt{data} \\ 
   &  &  \texttt{stat}   \\ 
   \midrule
\texttt{geom\_point()} & Add points &  \texttt{mapping} \\ 
   &  &  \texttt{data} \\ 
   &  & \texttt{stat}   \\ 
   \bottomrule
\label{tab:geomfunctions}
\end{longtable}

\index[R]{\texttt{ggplot2::geom\_abline()}}
\index[R]{\texttt{ggplot2::geom\_hline()}}
\index[R]{\texttt{ggplot2::geom\_vline()}}
\index[R]{\texttt{ggplot2::geom\_segment()}}
\index[R]{\texttt{ggplot2::geom\_curve()}}
\index[R]{\texttt{ggplot2::geom\_area()}}
\index[R]{\texttt{ggplot2::geom\_ribbon()}}
\index[R]{\texttt{ggplot2::geom\_bar()}}
\index[R]{\texttt{ggplot2::geom\_col()}}
\index[R]{\texttt{ggplot2::geom\_bin2d()}}
\index[R]{\texttt{ggplot2::geom\_boxplot()}}
\index[R]{\texttt{ggplot2::geom\_contour\_filled()}}
\index[R]{\texttt{ggplot2::geom\_count()}}
\index[R]{\texttt{ggplot2::geom\_sum()}}
\index[R]{\texttt{ggplot2::geom\_crossbar()}}
\index[R]{\texttt{ggplot2::geom\_errorbar()}}
\index[R]{\texttt{ggplot2::geom\_linerange()}}
\index[R]{\texttt{ggplot2::geom\_pointsrange()}}
\index[R]{\texttt{ggplot2::geom\_density()}}
\index[R]{\texttt{ggplot2::geom\_density\_2d()}}
\index[R]{\texttt{ggplot2::geom\_density\_2d\_filled()}}
\index[R]{\texttt{ggplot2::geom\_dotplot()}}
\index[R]{\texttt{ggplot2::geom\_errorbarh()}}
\index[R]{\texttt{ggplot2::geom\_freqpoly()}}
\index[R]{\texttt{ggplot2::geom\_histogram()}}
\index[R]{\texttt{ggplot2::geom\_function()}}
\index[R]{\texttt{ggplot2::geom\_hex()}}
\index[R]{\texttt{ggplot2::geom\_jitter()}}
\index[R]{\texttt{ggplot2::geom\_text()}}
\index[R]{\texttt{ggplot2::geom\_point()}}

\vspace{0.1in}

\begin{table}[ht]
\centering
\caption{A few example of \textit{ggplot2} aesthetic functions.} 
\label{tab:aes}
\begin{tabular}{lll}
  \toprule
Function & Usage & Arguments \\ 
  \midrule
\texttt{aes()} & Aesthetics of geoms & \texttt{x}, \texttt{y} \\ 
   \midrule
\texttt{colour()} & Color related aesthetics & See \texttt{?aes\_colour\_fill\_alpha} \\ 
  \texttt{fill()} &  &  \\ 
  \texttt{alpha()} &  &  \\ 
   \midrule
\texttt{linetype()} & Line type, size, shape & See\texttt{ ?aes\_linetype\_size\_shape} \\ 
  \texttt{size()} &  &  \\ 
  \texttt{shape()} &  &  \\ 
   \midrule
\texttt{group()} & Grouping & See \texttt{?aes\_group\_order} \\ 
   \bottomrule
\end{tabular}
\end{table}

\index[R]{\texttt{ggplot2::aes()}}
\index[R]{\texttt{ggplot2::colour()}}
\index[R]{\texttt{ggplot2::linetype()}}
\index[R]{\texttt{ggplot2::group()}}

\subsection{Boxplots}\label{boxplots-1}

\index[terms]{R!graphics!boxplots}
Figure \ref{fig:potato} shows a boxplot of R.A. Fisher's classic potato dataset from the Rothamsted Experimental Station \citep{fisher1923studies}. There are three important coding features that should be recognized in the chunk below. First, the plot was initialized using the first approach described in the previous section: \texttt{ggplot(df,\ aes(x,\ y,\ other\ aesthetics))} (Line 2). Specifically, the dataframe to be used, \texttt{potato}, was identified, and the coordinates for the plot were defined inside \texttt{aes()}. Second, plot modifications are added with the functions \texttt{theme()} (which includes a call to the function \texttt{element\_text()} to change the angle of text on the \emph{x}-axis), \texttt{xlab()}, and finally, \texttt{geom\_boxplot()} (Lines 3-5). Third, the continuation prompt, \texttt{+}, is placed at the end of lines of code to indicate that another graphical layer is being added to the plot. In \emph{ggplot2}, \texttt{+}, is somewhat analogous to the forward pipe operator, \texttt{\textbar{}\textgreater{}}, used in the \emph{tiddyverse} (Ch \ref{ch5}). Specifically, it denotes the continuation of \emph{ggplot2} plotting commands for a particular graphic. This continuation is broken with a line break (Line 6).

\begin{Shaded}
\begin{Highlighting}[numbers=left,,]
\FunctionTok{data}\NormalTok{(potato) }\CommentTok{\# in asbio}
\FunctionTok{ggplot}\NormalTok{(potato, }\FunctionTok{aes}\NormalTok{(}\AttributeTok{x =} \FunctionTok{factor}\NormalTok{(Variety), }\AttributeTok{y =}\NormalTok{ Yield)) }\SpecialCharTok{+}
  \FunctionTok{theme}\NormalTok{(}\AttributeTok{axis.text.x =} \FunctionTok{element\_text}\NormalTok{(}\AttributeTok{angle =} \DecValTok{50}\NormalTok{, }\AttributeTok{hjust =} \DecValTok{1}\NormalTok{, }\AttributeTok{vjust =} \FloatTok{0.9}\NormalTok{)) }\SpecialCharTok{+}
  \FunctionTok{xlab}\NormalTok{(}\StringTok{"Variety"}\NormalTok{) }\SpecialCharTok{+}
  \FunctionTok{geom\_boxplot}\NormalTok{()}
\end{Highlighting}
\end{Shaded}

\index[R]{\texttt{ggplot2::xlab()}}
\index[R]{\texttt{ggplot2::element\_text()}}
\index[R]{\texttt{ggplot2::geom\_boxplot()}}
\index[R]{\texttt{ggplot2::+}}

\vspace{0.1in}

\begin{figure}[H]

{\centering \includegraphics{figure/minimal-potato-1} 

}

\caption{An example of a \textit{ggplot2} boxplot. This is the signature appearance of ggplot2 graphs: a grey background and white grid lines.}\label{fig:potato}
\end{figure}

\subsection{Saving Plots}\label{saving-plots}

Plots can be saved using the function \texttt{ggsave()} or with a graphical device function, e.g., \texttt{pdf()}, \texttt{png()}, as described in Ch \ref{ch6}.

\vspace{0.1in}

\begin{Shaded}
\begin{Highlighting}[numbers=left,,]
\NormalTok{g }\OtherTok{\textless{}{-}} \FunctionTok{ggplot}\NormalTok{(potato, }\FunctionTok{aes}\NormalTok{(}\AttributeTok{x =} \FunctionTok{factor}\NormalTok{(Variety), }\AttributeTok{y =}\NormalTok{ Yield)) }\SpecialCharTok{+}
  \FunctionTok{theme}\NormalTok{(}\AttributeTok{axis.text.x =} \FunctionTok{element\_text}\NormalTok{(}\AttributeTok{angle =} \DecValTok{50}\NormalTok{, }\AttributeTok{hjust =} \DecValTok{1}\NormalTok{, }\AttributeTok{vjust =} \FloatTok{0.9}\NormalTok{)) }\SpecialCharTok{+}
  \FunctionTok{xlab}\NormalTok{(}\StringTok{"Variety"}\NormalTok{) }\SpecialCharTok{+}
  \FunctionTok{geom\_boxplot}\NormalTok{()}

\FunctionTok{pdf}\NormalTok{(}\StringTok{"potato.pdf"}\NormalTok{)}
\FunctionTok{print}\NormalTok{(g)}
\FunctionTok{dev.off}\NormalTok{()}
\end{Highlighting}
\end{Shaded}

\begin{verbatim}
pdf 
  2 
\end{verbatim}

Note that with the first line of code the ggplot is converted into an object called \texttt{g}. In Lines 5-7, the graph is rendered, using \texttt{print.ggplot(g)} or \texttt{plot.ggplot(g)}, and compiled.

The object \texttt{g} has \emph{ggplot2}-specific classes \texttt{"gg"} and \texttt{"ggplot"},
\vspace{0.1in}

\begin{Shaded}
\begin{Highlighting}[]
\FunctionTok{class}\NormalTok{(g)}
\end{Highlighting}
\end{Shaded}

\begin{verbatim}
[1] "gg"     "ggplot"
\end{verbatim}

and base type `\texttt{list}
\vspace{0.1in}

\begin{Shaded}
\begin{Highlighting}[]
\FunctionTok{typeof}\NormalTok{(g)}
\end{Highlighting}
\end{Shaded}

\begin{verbatim}
[1] "list"
\end{verbatim}

\index[terms]{R!base type!list}
\index[R]{\texttt{dev.off()}}
\index[R]{\texttt{pdf()}}
\index[R]{\texttt{ggplot::print.ggplot()}}
\index[R]{\texttt{ggplot::plot.ggplot()}}

\subsection{Line Plots}\label{line-plots}

Line plots are generally rendered using the function \texttt{geom\_line()}.
\index[terms]{R!graphics!line plots}
\index[R]{\texttt{ggplot2::geom\_line()}}

In Fig \ref{fig:glpotato} we consider the Fisher's potato data under a line plot approach. This presentation allows us to consider both potato variety and fertilizer levels. Note that I distinguish categories in the variable \texttt{Fert} using \texttt{colour} and \texttt{lty} arguments in \texttt{aes()} function calls (Line 1).

\vspace{0.1in}

\begin{Shaded}
\begin{Highlighting}[numbers=left,,]
\FunctionTok{ggplot}\NormalTok{(potato, }\FunctionTok{aes}\NormalTok{(}\AttributeTok{x =}\NormalTok{ Variety, }\AttributeTok{y =}\NormalTok{ Yield, }\AttributeTok{colour =}\NormalTok{ Fert)) }\SpecialCharTok{+}
  \FunctionTok{geom\_line}\NormalTok{(}\FunctionTok{aes}\NormalTok{(}\AttributeTok{group =}\NormalTok{ Fert, }\AttributeTok{lty =}\NormalTok{ Fert),}
       \AttributeTok{alpha =}\NormalTok{ .}\DecValTok{7}\NormalTok{, }\AttributeTok{linewidth =} \FloatTok{1.1}\NormalTok{) }\SpecialCharTok{+}
  \FunctionTok{theme\_classic}\NormalTok{() }\SpecialCharTok{+}
  \FunctionTok{theme}\NormalTok{(}\AttributeTok{axis.text.x =} \FunctionTok{element\_text}\NormalTok{(}\AttributeTok{angle =} \DecValTok{50}\NormalTok{, }\AttributeTok{hjust =} \DecValTok{1}\NormalTok{, }\AttributeTok{vjust =} \FloatTok{0.9}\NormalTok{))}
\end{Highlighting}
\end{Shaded}

\begin{figure}[H]

{\centering \includegraphics{figure/minimal-glpotato-1} 

}

\caption{Line plot representation of the \texttt{potato} dataset.}\label{fig:glpotato}
\end{figure}

\subsection{Scatterplots}\label{scatterplots-1}

\index[terms]{R!graphics!scatterplots}
Here we summarize the \texttt{asbio::world.emissions} CO\(_2\) and gross domestic product data, using a \emph{tidyverse} approach.

\vspace{0.1in}

\begin{Shaded}
\begin{Highlighting}[numbers=left,,]
\FunctionTok{library}\NormalTok{(asbio)}
\FunctionTok{data}\NormalTok{(world.emissions)}
\FunctionTok{library}\NormalTok{(dplyr)}
\NormalTok{country.data }\OtherTok{\textless{}{-}}\NormalTok{ world.emissions }\SpecialCharTok{|\textgreater{}}
  \FunctionTok{filter}\NormalTok{(continent }\SpecialCharTok{!=} \StringTok{"Redundant"}\NormalTok{) }\SpecialCharTok{|\textgreater{}}
  \FunctionTok{group\_by}\NormalTok{(country) }\SpecialCharTok{|\textgreater{}}
  \FunctionTok{summarize}\NormalTok{(}\AttributeTok{co2 =} \FunctionTok{mean}\NormalTok{(co2, }\AttributeTok{na.rm =} \ConstantTok{TRUE}\NormalTok{),}
            \AttributeTok{gdp =} \FunctionTok{mean}\NormalTok{(gdp, }\AttributeTok{na.rm =} \ConstantTok{TRUE}\NormalTok{))}
\end{Highlighting}
\end{Shaded}

I don't like the default \texttt{ggplot2} margins. Specifically, I feel that the axis label font size is too small and placed too close to the axes. Thus, prior to making a scatterplot of these variables, I make my own margin theme, as a function, that calls \texttt{theme()}.
\index[R]{\texttt{ggplot2::theme()}}

\vspace{0.1in}

\begin{Shaded}
\begin{Highlighting}[numbers=left,,]
\NormalTok{margin\_theme }\OtherTok{\textless{}{-}} \ControlFlowTok{function}\NormalTok{()\{}
  \FunctionTok{theme}\NormalTok{(}\AttributeTok{axis.title.x =} \FunctionTok{element\_text}\NormalTok{(}\AttributeTok{vjust=}\SpecialCharTok{{-}}\DecValTok{6}\NormalTok{, }\AttributeTok{size =} \DecValTok{12}\NormalTok{),}
        \AttributeTok{axis.title.y =} \FunctionTok{element\_text}\NormalTok{(}\AttributeTok{vjust=}\DecValTok{6}\NormalTok{, }\AttributeTok{size =} \DecValTok{12}\NormalTok{),}
        \AttributeTok{axis.text =} \FunctionTok{element\_text}\NormalTok{(}\AttributeTok{size =} \DecValTok{10}\NormalTok{),}
        \AttributeTok{plot.margin =} \FunctionTok{margin}\NormalTok{(}\AttributeTok{t =} \FloatTok{7.5}\NormalTok{, }\AttributeTok{r =} \FloatTok{7.5}\NormalTok{, }\AttributeTok{b =} \DecValTok{22}\NormalTok{, }\AttributeTok{l =} \DecValTok{22}\NormalTok{))}
\NormalTok{\}}
\end{Highlighting}
\end{Shaded}

We can call this custom theme within ggplot code (Fig \ref{fig:gscat1}).

\vspace{0.1in}

\begin{Shaded}
\begin{Highlighting}[numbers=left,,]
\NormalTok{g }\OtherTok{\textless{}{-}} \FunctionTok{ggplot}\NormalTok{(country.data, }\FunctionTok{aes}\NormalTok{(}\AttributeTok{x =}\NormalTok{ gdp, }\AttributeTok{y =}\NormalTok{ co2)) }\SpecialCharTok{+}
  \FunctionTok{ylab}\NormalTok{(expression}
\NormalTok{       (}\FunctionTok{paste}\NormalTok{(CO[}\DecValTok{2}\NormalTok{], }\StringTok{" Emissions (metric tons x "}\NormalTok{, }\DecValTok{10}\SpecialCharTok{\^{}}\DecValTok{6}\NormalTok{, }\StringTok{")"}\NormalTok{))) }\SpecialCharTok{+}
  \FunctionTok{xlab}\NormalTok{(}\StringTok{"GDP (international dollars)"}\NormalTok{) }\SpecialCharTok{+}
  \FunctionTok{geom\_point}\NormalTok{(}\AttributeTok{size =} \DecValTok{2}\NormalTok{) }\SpecialCharTok{+}
  \FunctionTok{scale\_y\_continuous}\NormalTok{(}\AttributeTok{trans=} \StringTok{"log10"}\NormalTok{) }\SpecialCharTok{+}
  \FunctionTok{scale\_x\_continuous}\NormalTok{(}\AttributeTok{trans=} \StringTok{"log10"}\NormalTok{) }\SpecialCharTok{+}
  \FunctionTok{theme\_classic}\NormalTok{() }\SpecialCharTok{+}
  \FunctionTok{margin\_theme}\NormalTok{()}
\NormalTok{g}
\end{Highlighting}
\end{Shaded}

\begin{figure}[H]

{\centering \includegraphics{figure/minimal-gscat1-1} 

}

\caption{An example of a \textit{ggplot2} scatterplot using the \texttt{world.emissions} dataset.}\label{fig:gscat1}
\end{figure}

I also called the complete \emph{ggplot2} theme \texttt{theme\_classic()} to generate an uncluttered graph with no grid lines.

Several other code steps are worth mentioning in Fig \ref{fig:gscat1}. First, note the use of plotmath code using calls to \texttt{expression()} in \texttt{xlab()} and \texttt{ylab()}. As an alternative, I could have used \texttt{labs(x,y)} where the arguments \texttt{x} and \texttt{y} would contain code for \texttt{xlab()} and \texttt{ylab()}. Second, \(\log_{10}\) transformations were applied to both axes using the \emph{ggplot2} functions \texttt{scale\_x\_continuous()} and \texttt{scale\_y\_continuous()}. As an alternative, I could have used the functions \texttt{scale\_x\_log10()} and \texttt{scale\_y\_log10()}.

We can also require specific tick locations using \texttt{scale\_y\_continuous} (Fig \ref{fig:weirdticks}).

\begin{Shaded}
\begin{Highlighting}[]
\NormalTok{g }\SpecialCharTok{+} \FunctionTok{scale\_y\_continuous}\NormalTok{(}\AttributeTok{breaks =} \FunctionTok{c}\NormalTok{(}\DecValTok{1}\NormalTok{, }\DecValTok{50}\NormalTok{, }\DecValTok{150}\NormalTok{, }\DecValTok{500}\NormalTok{, }\DecValTok{750}\NormalTok{, }\DecValTok{1600}\NormalTok{),}
                       \AttributeTok{trans=} \StringTok{"log10"}\NormalTok{)}
\end{Highlighting}
\end{Shaded}

\begin{figure}[H]

{\centering \includegraphics{figure/minimal-weirdticks-1} 

}

\caption{Custom tick mark locations overlaid on he $y$-axis of Fig \ref{fig:gscat1}.}\label{fig:weirdticks}
\end{figure}

\index[R]{\texttt{ggplot2::ylab()}}
\index[R]{\texttt{ggplot2::scale\_y\_continuous()}}
\index[R]{\texttt{ggplot2::scale\_x\_continuous()}}
\index[R]{\texttt{ggplot2::scale\_x\_log10()}}
\index[R]{\texttt{ggplot2::scale\_y\_log10()}}
\index[R]{\texttt{ggplot2::theme\_classic()}}
\index[R]{\texttt{ggplot2::labs()}}

\subsection{Transformations}\label{ggtrans}

\index[terms]{Transformation (function)}
Other than \(\log_{10}\) transformations (Fig \ref{fig:gscat1}, several other graphical transformation can be readily applied in the \texttt{transform} function of \texttt{scale\_x\_continuous()} and \texttt{scale\_y\_continuous()} including \texttt{"asn"} (arcsine), \texttt{"atanh"} (the inverse hyperbolic tangent), \texttt{"boxcox"}, i.e., the optimal power transform for the response variable in a linear model (see \citet{aho2014}), \texttt{"date"}, \texttt{"exp"}, \texttt{"hms"}, \texttt{"log"} (\(\log_e\) transform), \texttt{"log1p"} (\(\log_e\) transform, following the addition of 1 to prevent undefined logarithms of zeroes), \texttt{"log2"}, \texttt{"logit"} (i.e., the log odds for a probability), \texttt{"modulus"}, \texttt{"probit"}, \texttt{"pseudo\_log"} (\(\log_e\) transform, \texttt{NA}s resulting from undefined logarithms of zeroes, are given the value zero), \texttt{"reciprocal"}, \texttt{"reverse"}, \texttt{"sqrt"} and \texttt{"time"}.

\subsection{Adding Model Fits}\label{adding-model-fits}

It is straightforward to add fits from statistical models to a ggplot object, for instance the object \texttt{g} created in Fig \ref{fig:gscat1}. These can include conventional general linear models and locally fitted models, like Generalized Additive Models (GAMs) and locally weighted scatterplot smoothers (LOWESS), that allow the association between \emph{x} and \emph{y} to ``speak for itself'' without the assumption of underlying global linear association \citep{aho2014}. By default, \texttt{geom\_smooth()} provides a LOWESS fit using the function \texttt{loess()} from the \textbf{R} distribution \emph{stats} package. The code \texttt{geom\_smooth(method\ =\ "lm")} fits a general linear model, in this case, a simple linear regression (Fig \ref{fig:gsmooth}). By default, error polygons are included with fits that represent 95\% confidence intervals for the true fitted value. These can be turned off by specifying \texttt{geom\_smooth(se\ =\ FALSE)}.
\index[R]{\texttt{ggplot2::geom\_smooth()}}
\index[terms]{LOWESS}\index[R]{\texttt{loess()}}
\index[terms]{Generalized Additive Model (GAM)}
\index[terms]{General linear model}
\vspace{0.1in}

\begin{Shaded}
\begin{Highlighting}[]
\NormalTok{g }\SpecialCharTok{+} \FunctionTok{geom\_smooth}\NormalTok{(}\AttributeTok{method =} \StringTok{"lm"}\NormalTok{) }\CommentTok{\# call to lm fit}
\end{Highlighting}
\end{Shaded}

\begin{figure}[H]

{\centering \includegraphics{figure/minimal-gsmooth-1} 

}

\caption{A regression model overlaid on Fig \@ref{fig:gscat1}.}\label{fig:gsmooth}
\end{figure}

\subsection{Annotations in Graphs}\label{annotations-in-graphs}

The \emph{ggplot2} package has nice functions for graph annotation. In Fig \ref{fig:ann1} we use the function \texttt{geom\_label()} to label countries. The arguments \texttt{nudge\_y} and \texttt{nudge\_x} allow adjustments to label locations.
\index[R]{\texttt{ggplot2::geom\_label()}}

\vspace{0.1in}

\begin{Shaded}
\begin{Highlighting}[numbers=left,,]
\NormalTok{sub }\OtherTok{\textless{}{-}}\NormalTok{ country.data }\SpecialCharTok{|\textgreater{}}
  \FunctionTok{filter}\NormalTok{(country }\SpecialCharTok{\%in\%} \FunctionTok{c}\NormalTok{(}\StringTok{"Canada"}\NormalTok{, }\StringTok{"Finland"}\NormalTok{,}
\StringTok{"Japan"}\NormalTok{, }\StringTok{"Kenya"}\NormalTok{, }\StringTok{"United States"}\NormalTok{))}

\NormalTok{g }\SpecialCharTok{+} \FunctionTok{geom\_point}\NormalTok{(}\AttributeTok{size =} \DecValTok{3}\NormalTok{, }\AttributeTok{shape =} \DecValTok{1}\NormalTok{, }\AttributeTok{data =}\NormalTok{ sub, }\AttributeTok{col =} \StringTok{"orange"}\NormalTok{) }\SpecialCharTok{+}
  \FunctionTok{geom\_label}\NormalTok{(}\FunctionTok{aes}\NormalTok{(}\AttributeTok{label =}\NormalTok{ country), }\AttributeTok{data =}\NormalTok{ sub, }\AttributeTok{nudge\_y =}\NormalTok{ .}\DecValTok{25}\NormalTok{,}
             \AttributeTok{nudge\_x =} \SpecialCharTok{{-}}\NormalTok{.}\DecValTok{25}\NormalTok{, }\AttributeTok{alpha =}\NormalTok{ .}\DecValTok{9}\NormalTok{, }\AttributeTok{colour  =} \StringTok{"orange"}\NormalTok{)}
\end{Highlighting}
\end{Shaded}

\begin{figure}[H]

{\centering \includegraphics{figure/minimal-ann1-1} 

}

\caption{Country annotations added to Fig \ref{fig:gscat1}.}\label{fig:ann1}
\end{figure}

The \emph{ggpmisc} package allows annotation of statistical models in a ggplot. This is accomplished using the function \texttt{ggpmisc::stat\_poly\_eq()} which fits a model using\texttt{stats::lm()}, computes model quantities and prepares tidy text summaries of the model including the model equation, test statistic values and \(p\)-values. Computed terms are called using the \emph{ggplot2} function \texttt{after\_stat()}, which delays aesthetic mapping in \texttt{aes()} until after statistic calculation.
\index[terms]{R!package!\textit{ggpmisc}}
\index[R]{\texttt{ggplot2::after\_stat}}
\index[R]{\texttt{ggpmisc::stat\_poly\_eq()}}

In Fig \ref{fig:ggpmisc}, the equation for the \texttt{world.emissions} regression model is placed in the figure with \texttt{after\_stat(eq.label)}, and the adjusted \(R^2\) \citep{aho2014} is placed using \texttt{after\_stat(adj.rr.label)}. A complete list of available computed terms, including \texttt{eq.label} and \texttt{adj.rr.label}, is given in the documentation for \texttt{stat\_poly\_eq()}.

\vspace{0.1in}

\begin{Shaded}
\begin{Highlighting}[numbers=left,,]
\FunctionTok{library}\NormalTok{(ggpmisc)}

\NormalTok{g }\SpecialCharTok{+} \FunctionTok{geom\_smooth}\NormalTok{(}\AttributeTok{method =} \StringTok{"lm"}\NormalTok{) }\SpecialCharTok{+}
\FunctionTok{stat\_poly\_eq}\NormalTok{(}\FunctionTok{aes}\NormalTok{(}\AttributeTok{label =} \FunctionTok{paste}\NormalTok{(}\FunctionTok{after\_stat}\NormalTok{(eq.label),}
                                  \FunctionTok{after\_stat}\NormalTok{(adj.rr.label),}
                                  \AttributeTok{sep =} \StringTok{"*}\SpecialCharTok{\textbackslash{}"}\StringTok{, }\SpecialCharTok{\textbackslash{}"}\StringTok{*"}\NormalTok{)))}
\end{Highlighting}
\end{Shaded}

\begin{figure}[H]

{\centering \includegraphics{figure/minimal-ggpmisc-1} 

}

\caption{Regression model summaries overlaid on Fig \ref{fig:gscat1}.}\label{fig:ggpmisc}
\end{figure}

A potential criticism of \emph{ggplot2} is that its graphical rendering approaches are not readily accessible (as code or output), and statistical summaries are often not adequately or clearly described (in documentation or output). For instance, when using \texttt{geom\_smooth()} it is unclear which default smoothing parameters are actually being used, although these can be set within \texttt{geom\_smooth()} The function \texttt{ggplot2::ggplot\_build()} provides underlying plotting details. For example

\vspace{0.1in}

\begin{Shaded}
\begin{Highlighting}[]
\FunctionTok{head}\NormalTok{(}\FunctionTok{ggplot\_build}\NormalTok{(g)}\SpecialCharTok{$}\NormalTok{data[[}\DecValTok{1}\NormalTok{]])}
\end{Highlighting}
\end{Shaded}

\begin{verbatim}
         y      x PANEL group shape colour size fill alpha stroke
1  0.40732 10.494     1    -1    19  black    2   NA    NA    0.5
2  0.51149 10.085     1    -1    19  black    2   NA    NA    0.5
3  1.63089 11.428     1    -1    19  black    2   NA    NA    0.5
4 -0.31447     NA     1    -1    19  black    2   NA    NA    0.5
5  1.00675 10.642     1    -1    19  black    2   NA    NA    0.5
6 -0.95782     NA     1    -1    19  black    2   NA    NA    0.5
\end{verbatim}

\index[R]{\texttt{ggplot2::ggplot\_build()}}

The \emph{gginnards} package can be used to generate accessible ggplot analytical information. This can be done by calling \texttt{gginnards::geom\_debug()} within \texttt{ggpmisc::stat\_poly\_eq()}.

\index[R]{\texttt{gginnards::geom\_debug()}}
\index[terms]{R!package!\textit{gginnards}}
As an example, recall that a \(\log_{10}-\log_{10}\) transformation was used to generate the scatterplot object, \texttt{g}, used to project Fig \ref{fig:gscat1}.
I can summarize the linear model, overlaid in Fig \ref{fig:gsmooth}, with the code:

\vspace{0.1in}

\begin{Shaded}
\begin{Highlighting}[numbers=left,,]
\FunctionTok{library}\NormalTok{(gginnards)}
\NormalTok{g }\SpecialCharTok{+} \FunctionTok{stat\_poly\_eq}\NormalTok{(}\AttributeTok{formula =}\NormalTok{ y }\SpecialCharTok{\textasciitilde{}}\NormalTok{ x, }\AttributeTok{geom =} \StringTok{"debug"}\NormalTok{,}
                 \AttributeTok{output.type =} \StringTok{"numeric"}\NormalTok{,}
                 \AttributeTok{summary.fun =} \ControlFlowTok{function}\NormalTok{(x) x[[}\StringTok{"coef.ls"}\NormalTok{]])}
\end{Highlighting}
\end{Shaded}

\scriptsize

\begin{verbatim}
[1] "PANEL 1; group(s) -1; 'draw_function()' input 'data' (head):"
  npcx npcy label
1   NA   NA      
                                                                                           coef.ls
1 -1.0801e+01, 1.1109e+00, 2.8430e-01, 2.6794e-02, -3.7990e+01, 4.1462e+01, 1.3821e-82, 3.6215e-88
             coefs r.squared rr.confint.level rr.confint.low
1 -10.8005, 1.1109   0.91388             0.95        0.89056
  rr.confint.high adj.r.squared f.value f.df1 f.df2    p.value    AIC
1         0.92912       0.91335  1719.1     1   162 3.6215e-88 30.353
     BIC   n rr.label b_0.constant     b_0    b_1 fm.method fm.class
1 39.652 164                 FALSE -10.801 1.1109     lm:qr       lm
  fm.formula fm.formula.chr      x      y PANEL group orientation
1      y ~ x          y ~ x 8.3836 3.2706     1    -1           x
\end{verbatim}

\normalsize

This is in accordance with the linear model \(\log_{10}(\text{CO}_2) = -10.800518+ 1.110939\log_{10}(\text{GDP})\) obtained using the base function \texttt{lm()}.

\vspace{0.1in}

\begin{Shaded}
\begin{Highlighting}[]
\NormalTok{model }\OtherTok{\textless{}{-}} \FunctionTok{lm}\NormalTok{(}\FunctionTok{log}\NormalTok{(co2, }\AttributeTok{base =} \DecValTok{10}\NormalTok{) }\SpecialCharTok{\textasciitilde{}} \FunctionTok{log}\NormalTok{(gdp, }\AttributeTok{base =} \DecValTok{10}\NormalTok{), }\AttributeTok{data =}\NormalTok{ country.data)}
\FunctionTok{coef}\NormalTok{(model)}
\end{Highlighting}
\end{Shaded}

\begin{verbatim}
        (Intercept) log(gdp, base = 10) 
           -10.8005              1.1109 
\end{verbatim}

\subsection{Secondary Axes}\label{secondary-axes-1}

Secondary axes can be difficult to implement in \emph{ggplot2} because they require user specification of a one-to-one transformation for the primary axis.

\begin{example}
\protect\hypertarget{exm:exm-2axis}{}\label{exm:exm-2axis}

\(\text{}\)\\
To demonstrate the generation of ggplot secondary axes, we will examine two datasets published by \citet{rubino2013} concerning CO\(_2\) and \(\delta ^{13}\text{C}\) trapped in Antarctic ice layers. We wish to simultaneously plot CO\(_2\) and \(\delta^{13}\text{C}\) of as a function of the age of the depositional layer. We will use the primary (left-hand) vertical axis to plot CO\(_2\) and the use the right hand axis for \(\delta ^{13}\text{C}\). We first create a composite dataset for years in which both CO\(_2\) and \(\delta ^{13}\text{C}\) were measured.

\vspace{0.1in}

\begin{Shaded}
\begin{Highlighting}[numbers=left,,]
\FunctionTok{data}\NormalTok{(Rabino\_CO2); }\FunctionTok{data}\NormalTok{(Rabino\_del13C)}
\CommentTok{\# Match 1st dataset with 2nd}
\NormalTok{w }\OtherTok{\textless{}{-}} \FunctionTok{which}\NormalTok{(Rabino\_CO2}\SpecialCharTok{$}\NormalTok{effective.age }\SpecialCharTok{\%in\%}\NormalTok{ Rabino\_del13C}\SpecialCharTok{$}\NormalTok{effective.age)}
\NormalTok{R.C }\OtherTok{\textless{}{-}}\NormalTok{ Rabino\_CO2[w,]}
\CommentTok{\# match 2nd dataset with 1st}
\NormalTok{w }\OtherTok{\textless{}{-}} \FunctionTok{which}\NormalTok{(Rabino\_del13C}\SpecialCharTok{$}\NormalTok{effective.age }\SpecialCharTok{\%in\%}\NormalTok{ R.C}\SpecialCharTok{$}\NormalTok{effective.age)}
\NormalTok{R.d }\OtherTok{\textless{}{-}}\NormalTok{ Rabino\_del13C[w,]}
\NormalTok{data.C }\OtherTok{\textless{}{-}} \FunctionTok{data.frame}\NormalTok{(}\AttributeTok{CO2 =} \FunctionTok{tapply}\NormalTok{(R.C}\SpecialCharTok{$}\NormalTok{CO2, R.C}\SpecialCharTok{$}\NormalTok{effective.age, mean),}
                     \AttributeTok{d13C =} \FunctionTok{tapply}\NormalTok{(R.d}\SpecialCharTok{$}\NormalTok{d13C.CO2, R.d}\SpecialCharTok{$}\NormalTok{effective.age, mean),}
                     \AttributeTok{year =} \FunctionTok{as.numeric}\NormalTok{(}\FunctionTok{levels}\NormalTok{(}\FunctionTok{factor}\NormalTok{(R.d}\SpecialCharTok{$}\NormalTok{effective.age))))}
\end{Highlighting}
\end{Shaded}

For the years (ice depths) under consideration, CO\(_2\) levels vary between approximately 271 and 368 ppm. A range of around 100 ppm.

\vspace{0.1in}

\begin{Shaded}
\begin{Highlighting}[]
\NormalTok{data.C }\SpecialCharTok{|\textgreater{}}
  \FunctionTok{reframe}\NormalTok{(}\AttributeTok{range\_ppm =} \FunctionTok{range}\NormalTok{(CO2, }\AttributeTok{na.rm =}\NormalTok{ T))}
\end{Highlighting}
\end{Shaded}

\begin{verbatim}
  range_ppm
1    277.16
2    368.02
\end{verbatim}

Experimentation using simple linear transformations (additions and/or multiplications to a variable), reveals that a similar range can be generated for \(\delta ^{13}\text{C}\) following the transformation: \(y' = 56 \cdot y + 729\).

\vspace{0.1in}

\begin{Shaded}
\begin{Highlighting}[]
\NormalTok{data.C }\SpecialCharTok{|\textgreater{}}
  \FunctionTok{reframe}\NormalTok{(}\AttributeTok{range\_ppm =} \FunctionTok{range}\NormalTok{((d13C }\SpecialCharTok{*} \DecValTok{56}\NormalTok{) }\SpecialCharTok{+} \DecValTok{729}\NormalTok{, }\AttributeTok{na.rm =}\NormalTok{ T))}
\end{Highlighting}
\end{Shaded}

\begin{verbatim}
  range_ppm
1    277.11
2    373.34
\end{verbatim}

Thus, we create:
\vspace{0.1in}

\begin{Shaded}
\begin{Highlighting}[numbers=left,,]
\NormalTok{data.C}\SpecialCharTok{$}\NormalTok{td13C }\OtherTok{\textless{}{-}}\NormalTok{ data.C}\SpecialCharTok{$}\NormalTok{d13C }\SpecialCharTok{*} \DecValTok{56} \SpecialCharTok{+} \DecValTok{730}
\end{Highlighting}
\end{Shaded}

and use it in the ggplot code below. A scatterplot of the data is shown in Fig \ref{fig:rubino}.

\vspace{0.1in}

\begin{Shaded}
\begin{Highlighting}[numbers=left,,firstnumber=2,]
\FunctionTok{ggplot}\NormalTok{(data.C, }\FunctionTok{aes}\NormalTok{(}\AttributeTok{x =}\NormalTok{ year, }\AttributeTok{y =}\NormalTok{ CO2)) }\SpecialCharTok{+}
  \FunctionTok{geom\_point}\NormalTok{(}\AttributeTok{colour =} \StringTok{"blue"}\NormalTok{, }\AttributeTok{size =} \FloatTok{2.7}\NormalTok{, }\AttributeTok{alpha =} \FloatTok{0.2}\NormalTok{) }\SpecialCharTok{+}
  \FunctionTok{theme\_classic}\NormalTok{() }\SpecialCharTok{+}
  \FunctionTok{margin\_theme}\NormalTok{() }\SpecialCharTok{+}
  \FunctionTok{ylab}\NormalTok{(}\FunctionTok{expression}\NormalTok{(}\FunctionTok{paste}\NormalTok{(}\StringTok{"C"}\NormalTok{,O[}\DecValTok{2}\NormalTok{], }\StringTok{" (ppm)"}\NormalTok{))) }\SpecialCharTok{+}
  \FunctionTok{geom\_point}\NormalTok{(}\AttributeTok{data =}\NormalTok{ data.C, }\FunctionTok{aes}\NormalTok{(}\AttributeTok{x =}\NormalTok{ year, }\AttributeTok{y =}\NormalTok{ td13C), }\AttributeTok{colour =} \StringTok{"red"}\NormalTok{,}
             \AttributeTok{size =} \FloatTok{2.7}\NormalTok{, }\AttributeTok{alpha =} \FloatTok{0.2}\NormalTok{) }\SpecialCharTok{+}
  \FunctionTok{scale\_y\_continuous}\NormalTok{(}\AttributeTok{sec.axis =} 
                       \FunctionTok{sec\_axis}\NormalTok{(}\SpecialCharTok{\textasciitilde{}}\NormalTok{ (. }\SpecialCharTok{{-}} \DecValTok{730}\NormalTok{)}\SpecialCharTok{/}\DecValTok{56}\NormalTok{, }
                                \AttributeTok{name =} \FunctionTok{expression}\NormalTok{(}\FunctionTok{paste}\NormalTok{(delta}\SpecialCharTok{\^{}}\DecValTok{13}\NormalTok{, }
                                                        \StringTok{"C (\textbackslash{}u2030)"}\NormalTok{)))) }\SpecialCharTok{+}
\FunctionTok{theme}\NormalTok{(}\AttributeTok{axis.text.y.right =} \FunctionTok{element\_text}\NormalTok{(}\AttributeTok{colour =} \StringTok{"red"}\NormalTok{)) }\SpecialCharTok{+}
\FunctionTok{theme}\NormalTok{(}\AttributeTok{axis.text.y.left =} \FunctionTok{element\_text}\NormalTok{(}\AttributeTok{colour =} \StringTok{"blue"}\NormalTok{)) }\SpecialCharTok{+}
\FunctionTok{theme}\NormalTok{(}\AttributeTok{axis.title.y.right =} \FunctionTok{element\_text}\NormalTok{(}\AttributeTok{colour =} \StringTok{"red"}\NormalTok{)) }\SpecialCharTok{+}
\FunctionTok{theme}\NormalTok{(}\AttributeTok{axis.title.y.left =} \FunctionTok{element\_text}\NormalTok{(}\AttributeTok{colour =} \StringTok{"blue"}\NormalTok{)) }\SpecialCharTok{+}
\FunctionTok{theme}\NormalTok{(}\AttributeTok{axis.line.y.right =} \FunctionTok{element\_line}\NormalTok{(}\AttributeTok{colour =} \StringTok{"red"}\NormalTok{)) }\SpecialCharTok{+}
\FunctionTok{theme}\NormalTok{(}\AttributeTok{axis.line.y.left =} \FunctionTok{element\_line}\NormalTok{(}\AttributeTok{colour =} \StringTok{"blue"}\NormalTok{)) }\SpecialCharTok{+}
\FunctionTok{theme}\NormalTok{(}\AttributeTok{axis.ticks.y.right =} \FunctionTok{element\_line}\NormalTok{(}\AttributeTok{colour =} \StringTok{"red"}\NormalTok{)) }\SpecialCharTok{+}
\FunctionTok{xlab}\NormalTok{(}\StringTok{"Year"}\NormalTok{)}
\end{Highlighting}
\end{Shaded}

\begin{figure}[H]

{\centering \includegraphics{figure/minimal-rubino-1} 

}

\caption{A graphical representation of data published by \citet{rubino2013}, using two vertical axes.}\label{fig:rubino}
\end{figure}

There were two vital steps for creating the secondary axis.

\begin{itemize}
\tightlist
\item
  First, as a preliminary step we transformed the raw \(\delta^{13}\)C data to allow plotting \(\delta^{13}\)C points in the range of CO\(_2\) observations (Line 1). The result is the object \texttt{data.C\$td13C}.
\item
  Second, in the figure code above we scale the secondary axis based on a back-transformation of the transformed data. That is, we solved for \(y\) in \(y' = 56 \cdot y + 730\) and found \(y = (y' - 730)/56\). This is what underlies the code on Line 9:
  \texttt{sec.axis\ =\ sec\_axis(\textasciitilde{}\ (.\ -\ 730)/56,} in the first argument of \texttt{sec\_axis()}.
  Note that axis components were painstakingly colored using \texttt{ggplot::theme()} (Lines 12-19).
\end{itemize}

\end{example}

\(\blacksquare\)

\index[R]{\texttt{ggplot2::sec\_axis()}}
\index[R]{\texttt{dplyr::reframe()}}

\subsection{Defining Graphical Features using Vectors}\label{defining-graphical-features-using-vectors}

As we have already seen, it is straightforward to define figure plotting characteristics (symbols, symbol sizes, colors, line types, etc.) using relevant data.

\begin{example}
\protect\hypertarget{exm:exm-pp}{}\label{exm:exm-pp}

\(\text{}\)\\
In Fig \ref{fig:gfly} we change symbols and colors for a representation of the \texttt{asbio::fly.sex} dataset based on experimental treatments:

\begin{figure}[H]

{\centering \includegraphics{figure/minimal-gfly-1} 

}

\caption{A representation of the \texttt{fly.sex} dataset.}\label{fig:gfly}
\end{figure}

Note that the linear fits in Fig \ref{fig:gfly} are actually for separate regression models, \texttt{longevity\ \textasciitilde{}\ thorax}, for each level in \texttt{fly.sex\$Treatment}. They are \emph{not} from the single ANCOVA model: \texttt{lm(longevity\ \textasciitilde{}\ thorax\ *\ Treatment)}, although this is not clear at all from the \texttt{ggplot} graph. It is, however, revealed from:

\vspace{0.1in}

\begin{Shaded}
\begin{Highlighting}[]
\NormalTok{g1 }\SpecialCharTok{+} \FunctionTok{stat\_poly\_eq}\NormalTok{(}\AttributeTok{formula =}\NormalTok{ y }\SpecialCharTok{\textasciitilde{}}\NormalTok{ x, }\AttributeTok{geom =} \StringTok{"debug"}\NormalTok{,}
                 \AttributeTok{output.type =} \StringTok{"numeric"}\NormalTok{,}
                 \AttributeTok{summary.fun =} \ControlFlowTok{function}\NormalTok{(x) x[[}\StringTok{"coef.ls"}\NormalTok{]])}
\end{Highlighting}
\end{Shaded}

\scriptsize

\begin{verbatim}
`geom_smooth()` using formula = 'y ~ x'
\end{verbatim}

\begin{verbatim}
[1] "PANEL 1; group(s) 1, 2, 3, 4, 5; 'draw_function()' input 'data' (head):"
  npcx npcy label
1   NA   NA      
2   NA   NA      
3   NA   NA      
4   NA   NA      
5   NA   NA      
                                                                                           coef.ls
1 -5.5699e+01, 1.4779e+02, 1.6834e+01, 2.0795e+01, -3.3087e+00, 7.1071e+00, 3.0654e-03, 3.0692e-07
2 -5.0242e+01, 1.3613e+02, 2.4519e+01, 2.9186e+01, -2.0491e+00, 4.6640e+00, 5.2023e-02, 1.0753e-04
3    -43.7248157, 131.4496314, 31.3250601, 37.8123482, -1.3958414, 3.4763678, 0.1760921, 0.0020423
4 -5.7992e+01, 1.3700e+02, 2.8260e+01, 3.3625e+01, -2.0521e+00, 4.0744e+00, 5.1714e-02, 4.6763e-04
5 -6.1280e+01, 1.2500e+02, 1.5225e+01, 1.8944e+01, -4.0250e+00, 6.5983e+00, 5.2871e-04, 9.8692e-07
             coefs r.squared rr.confint.level rr.confint.low
1 -55.699, 147.790   0.68712             0.95       0.418224
2 -50.242, 136.127   0.48607             0.95       0.169020
3 -43.725, 131.450   0.34445             0.95       0.058492
4 -57.992, 137.001   0.41920             0.95       0.110089
5   -61.28, 125.00   0.65433             0.95       0.370279
  rr.confint.high adj.r.squared f.value f.df1 f.df2    p.value    AIC
1         0.79674       0.67352  50.511     1    23 3.0692e-07 180.72
2         0.66239       0.46373  21.753     1    23 1.0753e-04 199.31
3         0.56048       0.31595  12.085     1    23 2.0423e-03 202.90
4         0.61539       0.39395  16.600     1    23 4.6763e-04 197.51
5         0.77529       0.63930  43.538     1    23 9.8692e-07 174.04
     BIC  n rr.label b_0.constant     b_0    b_1 fm.method fm.class
1 184.38 25                 FALSE -55.699 147.79     lm:qr       lm
2 202.96 25                 FALSE -50.242 136.13     lm:qr       lm
3 206.56 25                 FALSE -43.725 131.45     lm:qr       lm
4 201.16 25                 FALSE -57.992 137.00     lm:qr       lm
5 177.69 25                 FALSE -61.280 125.00     lm:qr       lm
  fm.formula fm.formula.chr    x     y group PANEL orientation
1      y ~ x          y ~ x 0.64 97.00     1     1           x
2      y ~ x          y ~ x 0.64 92.95     2     1           x
3      y ~ x          y ~ x 0.64 88.90     3     1           x
4      y ~ x          y ~ x 0.64 84.85     4     1           x
5      y ~ x          y ~ x 0.64 80.80     5     1           x
\end{verbatim}

\normalsize

\end{example}

\(\blacksquare\)

\subsection{Modifying Legends}\label{modifying-legends}

Note that a legend was created for Fig \ref{fig:gfly} because of designation of groups in the initial aesthetics. Legend characteristics generally need to be modified using \texttt{theme()}. For instance, to change the legend location from the right-hand side of the plot to the the left-hand side, I could use:

\vspace{0.1in}

\begin{Shaded}
\begin{Highlighting}[]
\NormalTok{g1 }\SpecialCharTok{+} \FunctionTok{theme}\NormalTok{(}\AttributeTok{legend.position =} \StringTok{"left"}\NormalTok{)}
\end{Highlighting}
\end{Shaded}

\subsection{Multiple plots}\label{multiple-plots-1}

We can place multiple ggplots into a single graphics device using several approaches. I consider two here: 1) facet functions from the \emph{ggplot2} package, and 2) ggplot extension functions from the package \emph{cowplot}.

\subsubsection{Faceting}\label{faceting}

The functions \texttt{facet\_wrap()} and \texttt{facet\_grid()} can be used to generate a sequence of plot panels.
\index[R]{\texttt{ggplot2::facet\_wrap()}}\index[R]{\texttt{facet\_grid()}}

\begin{example}
\protect\hypertarget{exm:exm-fac}{}\label{exm:exm-fac}\(\text{}\)\\
I will modify Fig \ref{fig:gfly} to demonstrate the use of \texttt{facet\_wrap()}.

\vspace{0.1in}

\begin{Shaded}
\begin{Highlighting}[numbers=left,,]
\NormalTok{g1 }\OtherTok{\textless{}{-}} \FunctionTok{ggplot}\NormalTok{(fly.sex, }\FunctionTok{aes}\NormalTok{(}\AttributeTok{y =}\NormalTok{ longevity, }\AttributeTok{x =}\NormalTok{ thorax,}
                         \AttributeTok{group =}\NormalTok{ Treatment)) }\SpecialCharTok{+}
  \FunctionTok{geom\_point}\NormalTok{(}\FunctionTok{aes}\NormalTok{(}\AttributeTok{colour =}\NormalTok{ Treatment, }\AttributeTok{shape =}\NormalTok{ Treatment)) }\SpecialCharTok{+}
  \FunctionTok{facet\_wrap}\NormalTok{(}\FunctionTok{vars}\NormalTok{(Treatment)) }\SpecialCharTok{+}
  \FunctionTok{theme\_classic}\NormalTok{() }\SpecialCharTok{+}
  \FunctionTok{margin\_theme}\NormalTok{() }\SpecialCharTok{+}
  \FunctionTok{geom\_smooth}\NormalTok{(}\AttributeTok{method =} \StringTok{"lm"}\NormalTok{, }\AttributeTok{se =}\NormalTok{ F, }\FunctionTok{aes}\NormalTok{(}\AttributeTok{colour =}\NormalTok{ Treatment)) }\SpecialCharTok{+}
  \FunctionTok{labs}\NormalTok{(}\AttributeTok{x =} \StringTok{"Thorax length (mm)"}\NormalTok{, }\AttributeTok{y =} \StringTok{"Longevity (Days)"}\NormalTok{)}
\NormalTok{g1}
\end{Highlighting}
\end{Shaded}

\begin{verbatim}
`geom_smooth()` using formula = 'y ~ x'
\end{verbatim}

\begin{figure}[H]

{\centering \includegraphics{figure/minimal-gfly2-1} 

}

\caption{Demonstration of facet wrapping using the \texttt{fly.sex} dataset.}\label{fig:gfly2}
\end{figure}

On Line 4 I specify that different panels should be created for each treatment level using: \texttt{facet\_wrap(vars(Treatment))}. This could also be accomplished using: \texttt{facet\_wrap(\textasciitilde{}Treatment)}.
\end{example}

\(\blacksquare\)

\subsubsection{\texorpdfstring{\emph{cowplot} functions}{cowplot functions}}\label{cowplot-functions}

Multiple plots can also be assembled into a single graphical entity using functions from the \emph{cowplot} package. This requires creating separate plot objects and concatenating them in \texttt{cowplot::plot\_grid()}.

\begin{example}
\protect\hypertarget{exm:exm-cowplot}{}\label{exm:exm-cowplot}\(\text{}\)\\
Fig \ref{fig:cowplot} shows summaries of US per capita CO\(_2\) emissions and GDP since the start of the industrial revolution with two plots.

\begin{Shaded}
\begin{Highlighting}[numbers=left,,]
\FunctionTok{library}\NormalTok{(cowplot)}
\NormalTok{US }\OtherTok{\textless{}{-}}\NormalTok{ world.emissions }\SpecialCharTok{|\textgreater{}}
  \FunctionTok{filter}\NormalTok{(country }\SpecialCharTok{==} \StringTok{"United States"}\NormalTok{)}

\NormalTok{g2 }\OtherTok{\textless{}{-}} \FunctionTok{ggplot}\NormalTok{(US) }\SpecialCharTok{+}
  \FunctionTok{geom\_line}\NormalTok{(}\FunctionTok{aes}\NormalTok{(year, co2}\SpecialCharTok{/}\NormalTok{population), }\AttributeTok{col =} \StringTok{"dark red"}\NormalTok{) }\SpecialCharTok{+}
  \FunctionTok{theme\_classic}\NormalTok{() }\SpecialCharTok{+} \FunctionTok{margin\_theme}\NormalTok{() }\SpecialCharTok{+}
  \FunctionTok{theme}\NormalTok{(}\AttributeTok{axis.text.x =} \FunctionTok{element\_text}\NormalTok{(}\AttributeTok{angle =} \DecValTok{50}\NormalTok{, }\AttributeTok{hjust =} \DecValTok{1}\NormalTok{, }\AttributeTok{vjust =} \FloatTok{0.9}\NormalTok{)) }\SpecialCharTok{+}
  \FunctionTok{labs}\NormalTok{(}\AttributeTok{x =} \StringTok{"Year"}\NormalTok{,}
       \AttributeTok{y =} \FunctionTok{expression}\NormalTok{(}\FunctionTok{paste}\NormalTok{(}\StringTok{"Per capita "}\NormalTok{, CO[}\DecValTok{2}\NormalTok{],}
                            \StringTok{" emissions (tonnes x "}\NormalTok{, }\DecValTok{10}\SpecialCharTok{\^{}}\DecValTok{6}\NormalTok{, }\StringTok{")"}\NormalTok{)))}

\NormalTok{g3 }\OtherTok{\textless{}{-}} \FunctionTok{ggplot}\NormalTok{(US) }\SpecialCharTok{+}
  \FunctionTok{geom\_line}\NormalTok{(}\FunctionTok{aes}\NormalTok{(year, gdp}\SpecialCharTok{/}\NormalTok{population), }\AttributeTok{col =} \StringTok{"blue"}\NormalTok{) }\SpecialCharTok{+}
  \FunctionTok{theme\_classic}\NormalTok{() }\SpecialCharTok{+} \FunctionTok{margin\_theme}\NormalTok{() }\SpecialCharTok{+}
  \FunctionTok{theme}\NormalTok{(}\AttributeTok{axis.text.x =} \FunctionTok{element\_text}\NormalTok{(}\AttributeTok{angle =} \DecValTok{50}\NormalTok{, }\AttributeTok{hjust =} \DecValTok{1}\NormalTok{, }\AttributeTok{vjust =} \FloatTok{0.9}\NormalTok{)) }\SpecialCharTok{+}
  \FunctionTok{labs}\NormalTok{(}\AttributeTok{x =} \StringTok{"Year"}\NormalTok{, }\AttributeTok{y =} \FunctionTok{expression}\NormalTok{(}\FunctionTok{paste}\NormalTok{(}\StringTok{"Per capita GDP"}\NormalTok{)))}

\FunctionTok{plot\_grid}\NormalTok{(g2, g3)}
\end{Highlighting}
\end{Shaded}

\vspace{0.1in}
\begin{figure}[H]

{\centering \includegraphics{figure/minimal-cowplot-1} 

}

\caption{Two plots depicting US per capita trends in CO$_2$ emissions and GDP.}\label{fig:cowplot}
\end{figure}

The function \texttt{plot\_grid} is used on Line 17 to conjoin the ggplot objects \texttt{g2} and \texttt{g3}.
\end{example}

\(\blacksquare\)
\index[R]{\texttt{cowplot::plot\_grid}}
\index[terms]{R!package!\textit{cowplot}}

\subsection{Univariate Distributional Summaries}\label{univariate-distributional-summaries}

\index[terms]{R!graphics!histograms}
\index[terms]{R!graphics!frequency plots}
\index[terms]{R!graphics!dot plots}
A number of \emph{ggplot2} functions can be used to graphically summarize distributions of variables. These include \texttt{geom\_hist()} for histograms, \texttt{geom\_area()} for area plots, \texttt{geom\_freq()} for frequency plots, \texttt{geom\_dotplot()} for dot plots, and \texttt{geom\_density()} for density plots.

\begin{example}
\protect\hypertarget{exm:exm-univ}{}\label{exm:exm-univ}

\(\text{}\)\\
Fig \ref{fig:gdists} provides a multi-plot distributional summary of the US CO\(_2\) data using a histogram, an area plot, and a frequency plot. These are created as separate ggplot objects.

\vspace{0.1in}

\begin{Shaded}
\begin{Highlighting}[numbers=left,,]
\NormalTok{xlab }\OtherTok{\textless{}{-}} \FunctionTok{expression}\NormalTok{(}\FunctionTok{paste}\NormalTok{(CO[}\DecValTok{2}\NormalTok{], }\StringTok{" Emissions (metric tons x "}\NormalTok{, }\DecValTok{10}\SpecialCharTok{\^{}}\DecValTok{6}\NormalTok{, }\StringTok{")"}\NormalTok{))}
\NormalTok{Years }\OtherTok{\textless{}{-}} \FunctionTok{factor}\NormalTok{(}\FunctionTok{c}\NormalTok{(}\FunctionTok{rep}\NormalTok{(}\StringTok{"1800{-}1854"}\NormalTok{, }\DecValTok{55}\NormalTok{), }\FunctionTok{rep}\NormalTok{(}\StringTok{"1854{-}1908"}\NormalTok{, }\DecValTok{55}\NormalTok{),}
           \FunctionTok{rep}\NormalTok{(}\StringTok{"1908{-}1962"}\NormalTok{, }\DecValTok{55}\NormalTok{), }\FunctionTok{rep}\NormalTok{(}\StringTok{"1962{-}2019"}\NormalTok{, }\DecValTok{55}\NormalTok{)))}

\NormalTok{margin\_theme2 }\OtherTok{\textless{}{-}} \ControlFlowTok{function}\NormalTok{()\{}
  \FunctionTok{theme}\NormalTok{(}\AttributeTok{axis.title.y =} \FunctionTok{element\_text}\NormalTok{(}\AttributeTok{hjust=}\FloatTok{0.6}\NormalTok{, }\AttributeTok{vjust =} \FloatTok{2.8}\NormalTok{, }\AttributeTok{size =} \DecValTok{10}\NormalTok{),}
  \AttributeTok{plot.margin =} \FunctionTok{margin}\NormalTok{(}\AttributeTok{t =} \FloatTok{7.5}\NormalTok{, }\AttributeTok{r =} \FloatTok{7.5}\NormalTok{, }\AttributeTok{b =} \FloatTok{7.5}\NormalTok{, }\AttributeTok{l =} \DecValTok{15}\NormalTok{))}
\NormalTok{\}}

\NormalTok{histogram }\OtherTok{\textless{}{-}} \FunctionTok{ggplot}\NormalTok{(US, }\FunctionTok{aes}\NormalTok{(co2, }\AttributeTok{fill =}\NormalTok{ Years)) }\SpecialCharTok{+}
  \FunctionTok{geom\_histogram}\NormalTok{(}\AttributeTok{binwidth =} \DecValTok{500}\NormalTok{) }\SpecialCharTok{+}
  \FunctionTok{theme\_classic}\NormalTok{() }\SpecialCharTok{+}
  \FunctionTok{scale\_fill\_brewer}\NormalTok{(}\AttributeTok{palette =} \StringTok{"Blues"}\NormalTok{) }\SpecialCharTok{+}
  \FunctionTok{xlab}\NormalTok{(xlab) }\SpecialCharTok{+} \FunctionTok{ylab}\NormalTok{(}\StringTok{"Frequency"}\NormalTok{) }\SpecialCharTok{+}
  \FunctionTok{margin\_theme}\NormalTok{()}

\NormalTok{areaplot }\OtherTok{\textless{}{-}} \FunctionTok{ggplot}\NormalTok{(US, }\FunctionTok{aes}\NormalTok{(co2, }\AttributeTok{fill =}\NormalTok{ Years)) }\SpecialCharTok{+}
  \FunctionTok{geom\_area}\NormalTok{(}\AttributeTok{stat=}\StringTok{"bin"}\NormalTok{) }\SpecialCharTok{+}
  \FunctionTok{theme\_classic}\NormalTok{() }\SpecialCharTok{+}
  \FunctionTok{scale\_fill\_brewer}\NormalTok{(}\AttributeTok{palette =} \StringTok{"Spectral"}\NormalTok{) }\SpecialCharTok{+}
  \FunctionTok{xlab}\NormalTok{(}\StringTok{""}\NormalTok{) }\SpecialCharTok{+} \FunctionTok{ylab}\NormalTok{(}\StringTok{"Frequency"}\NormalTok{) }\SpecialCharTok{+}
  \FunctionTok{margin\_theme2}\NormalTok{()}

\NormalTok{freqplot }\OtherTok{\textless{}{-}} \FunctionTok{ggplot}\NormalTok{(US, }\FunctionTok{aes}\NormalTok{(co2, }\AttributeTok{colour =}\NormalTok{ Years)) }\SpecialCharTok{+}
  \FunctionTok{geom\_freqpoly}\NormalTok{() }\SpecialCharTok{+}
  \FunctionTok{theme\_classic}\NormalTok{() }\SpecialCharTok{+}
  \FunctionTok{scale\_fill\_brewer}\NormalTok{(}\AttributeTok{palette =} \StringTok{"Spectral"}\NormalTok{) }\SpecialCharTok{+}
  \FunctionTok{xlab}\NormalTok{(}\StringTok{""}\NormalTok{) }\SpecialCharTok{+} \FunctionTok{ylab}\NormalTok{(}\StringTok{""}\NormalTok{) }\SpecialCharTok{+}
  \FunctionTok{margin\_theme2}\NormalTok{()}
\end{Highlighting}
\end{Shaded}

The histogram, area plot, and frequency plot are created on Lines 10-15, 17-22, 24-29, respectively. Note the use of a second margin theme (Lines 5-8) and the use of \texttt{ggplot2::scale\_fill\_brewer()} to define specific \emph{RColorBrewer} color palettes.

The plots are conjoined, with the area plot and frequency plot splitting the first row, and the histogram occupying the entire second row of the graphical device using \texttt{cowplot::plot\_grid()}.

\vspace{0.1in}

\begin{Shaded}
\begin{Highlighting}[]
\NormalTok{title }\OtherTok{\textless{}{-}} \FunctionTok{ggdraw}\NormalTok{() }\SpecialCharTok{+} \FunctionTok{draw\_label}\NormalTok{(}\FunctionTok{expression}\NormalTok{(}\FunctionTok{paste}\NormalTok{(CO[}\DecValTok{2}\NormalTok{] , }\StringTok{" in the US"}\NormalTok{)), }
                               \AttributeTok{fontface=}\StringTok{\textquotesingle{}bold\textquotesingle{}}\NormalTok{)}
\NormalTok{top\_row }\OtherTok{\textless{}{-}} \FunctionTok{plot\_grid}\NormalTok{(areaplot, freqplot, }\AttributeTok{ncol =} \DecValTok{2}\NormalTok{, }\AttributeTok{labels =} \StringTok{"AUTO"}\NormalTok{)}

\FunctionTok{plot\_grid}\NormalTok{(title, top\_row, histogram, }\AttributeTok{rel\_heights =} \FunctionTok{c}\NormalTok{(}\FloatTok{0.2}\NormalTok{, }\DecValTok{1}\NormalTok{, }\FloatTok{1.2}\NormalTok{),}
          \AttributeTok{hjust =} \FunctionTok{c}\NormalTok{(}\DecValTok{0}\NormalTok{,}\DecValTok{0}\NormalTok{,}\SpecialCharTok{{-}}\FloatTok{0.6}\NormalTok{), }\AttributeTok{nrow =} \DecValTok{3}\NormalTok{, }\AttributeTok{labels =} \FunctionTok{c}\NormalTok{(}\StringTok{""}\NormalTok{, }\StringTok{""}\NormalTok{, }\StringTok{"C"}\NormalTok{))}
\end{Highlighting}
\end{Shaded}

\begin{figure}[H]

{\centering \includegraphics{figure/minimal-gdists-1} 

}

\caption{Distributional summaries of the US CO$_2$ data from \texttt{asbio::world.emissions}.}\label{fig:gdists}
\end{figure}

\end{example}

\(\blacksquare\)

\index[terms]{R!package!\textit{RColorBrewer}}
\index[R]{\texttt{ggplot2::scale\_fill\_brewer()}}
\index[R]{\texttt{ggplot2::geom\_hist()}}
\index[R]{\texttt{ggplot2::geom\_area()}}
\index[R]{\texttt{ggplot2::geom\_freq()}}
\index[R]{\texttt{ggplot2::geom\_dotplot()}}
\index[R]{\texttt{ggplot2::geom\_density()}}

\subsection{Barplots}\label{barplots-1}

\index[terms]{R!graphics!barplots}
Barplots are straightforward to create in \emph{ggplot2} using the function \texttt{geom\_bar()}.
\index[R]{\texttt{ggplot2::geom\_bar()}}

\begin{example}
\protect\hypertarget{exm:exm-bpp}{}\label{exm:exm-bpp}

\(\text{}\)\\
Consider the \texttt{asthma} dataframe from \emph{asbio}. We first convert the time series to a long table format using \texttt{reshape2::melt()}, and summarize it using \texttt{dplyr::summarise()}.

\vspace{0.1in}

\begin{Shaded}
\begin{Highlighting}[numbers=left,,]
\FunctionTok{library}\NormalTok{(reshape2); }\FunctionTok{data}\NormalTok{(asthma)}
\NormalTok{asthma.long }\OtherTok{\textless{}{-}}\NormalTok{ asthma }\SpecialCharTok{|\textgreater{}} \FunctionTok{melt}\NormalTok{(}\AttributeTok{id =} \FunctionTok{c}\NormalTok{(}\StringTok{"DRUG"}\NormalTok{, }\StringTok{"PATIENT"}\NormalTok{),}
                               \AttributeTok{value.name =} \StringTok{"FEV1"}\NormalTok{,}
                               \AttributeTok{variable.name =} \StringTok{"TIME"}\NormalTok{)}

\NormalTok{asthma.long}\SpecialCharTok{$}\NormalTok{TIME }\OtherTok{\textless{}{-}} \FunctionTok{factor}\NormalTok{(asthma.long}\SpecialCharTok{$}\NormalTok{TIME,}
                           \AttributeTok{labels =} \FunctionTok{c}\NormalTok{(}\StringTok{"BASE"}\NormalTok{,}
                                      \FunctionTok{paste}\NormalTok{(}\StringTok{"H"}\NormalTok{, }\DecValTok{11}\SpecialCharTok{:}\DecValTok{18}\NormalTok{, }\AttributeTok{sep =} \StringTok{""}\NormalTok{)))}

\NormalTok{summary.FEV }\OtherTok{\textless{}{-}}\NormalTok{ asthma.long }\SpecialCharTok{|\textgreater{}}
  \FunctionTok{group\_by}\NormalTok{(TIME, DRUG) }\SpecialCharTok{|\textgreater{}}
  \FunctionTok{summarise}\NormalTok{(}\AttributeTok{mean =} \FunctionTok{mean}\NormalTok{(FEV1),}
            \AttributeTok{se =} \FunctionTok{sd}\NormalTok{(FEV1)}\SpecialCharTok{/}\FunctionTok{sqrt}\NormalTok{(}\FunctionTok{length}\NormalTok{(FEV1)),}
            \AttributeTok{meanmse =}\NormalTok{ mean }\SpecialCharTok{{-}}\NormalTok{ se,}
            \AttributeTok{meanpse =}\NormalTok{ mean }\SpecialCharTok{+}\NormalTok{ se)}
\end{Highlighting}
\end{Shaded}

In the code for Fig \ref{fig:bpp} below, I group by drug treatments \texttt{group\ =\ DRUG} (line one) and plot bars using the mean values from \texttt{summary.FEV} using \texttt{ggplot2::geom\_bar()} (Line 6). The argument \texttt{stat\ =\ "identity"} allows bar heights to be represented by individual numbers, in this case means. Use of \texttt{stat\ =\ "identity"} is required here. The argument \texttt{position\ =\ "dodge"} creates side by side bar plots (Line 6).

\vspace{0.1in}

\begin{Shaded}
\begin{Highlighting}[numbers=left,,]
\NormalTok{g }\OtherTok{\textless{}{-}} \FunctionTok{ggplot}\NormalTok{(summary.FEV, }\FunctionTok{aes}\NormalTok{(}\AttributeTok{x =}\NormalTok{ TIME, }\AttributeTok{y =}\NormalTok{ mean, }\AttributeTok{group =}\NormalTok{ DRUG)) }\SpecialCharTok{+}
  \FunctionTok{margin\_theme}\NormalTok{() }\SpecialCharTok{+}
  \FunctionTok{labs}\NormalTok{(}\AttributeTok{y =} \StringTok{"Forced Expiratory Volume (1 min)"}\NormalTok{,}
       \AttributeTok{x =} \StringTok{"Time Period"}\NormalTok{)}

\NormalTok{g }\SpecialCharTok{+} \FunctionTok{geom\_bar}\NormalTok{(}\AttributeTok{stat =} \StringTok{"identity"}\NormalTok{, }\AttributeTok{position =} \StringTok{"dodge"}\NormalTok{, }\FunctionTok{aes}\NormalTok{(}\AttributeTok{fill =}\NormalTok{ DRUG))}
\end{Highlighting}
\end{Shaded}

\vspace{0.1in}
\begin{figure}[H]

{\centering \includegraphics{figure/minimal-bpp-1} 

}

\caption{Barplot of the asthma data.}\label{fig:bpp}
\end{figure}

\end{example}

\(\blacksquare\)

\index[R]{\texttt{reshape2::melt()}}
\index[R]{\texttt{dplyr::summarise()}}

\subsection{Interval Plots}\label{interval-plots-1}

\index[terms]{R!graphics!interval plots}
The \emph{ggplot2} package allows implementation of interval plots.

\begin{example}
\protect\hypertarget{exm:exm-giv}{}\label{exm:exm-giv}

\(\text{}\)\\
As an initial demonstration of interval plots, we continue use of barplots from Example \ref{exm:exm-bpp}. Overlaying errors on barplots requires the use of \texttt{stat\_summary()} (Line 1) in the code below. Outcomes from \texttt{meanmse} and \texttt{meanpse} in the \texttt{summary.FEV} dataset represent \(\bar{x}-SE\) and \(\bar{x}+SE\), respectively. These will define the lower and upper values of the error bars in interval plot. They are called in the arguments \texttt{ymin} and \texttt{ymax} in the aesthetics of \texttt{geom\_errorbar()} (Line 5). The background color of the plot is changed on Line 4. The final result is shown in Fig \ref{fig:ebbp}.

\vspace{0.1in}

\begin{Shaded}
\begin{Highlighting}[numbers=left,,]
\NormalTok{g }\SpecialCharTok{+} \FunctionTok{stat\_summary}\NormalTok{(}\AttributeTok{fun =} \StringTok{"identity"}\NormalTok{, }\AttributeTok{geom =} \StringTok{"bar"}\NormalTok{,}
                 \AttributeTok{position =} \FunctionTok{position\_dodge}\NormalTok{(}\AttributeTok{width =}\NormalTok{ .}\DecValTok{9}\NormalTok{),}
                 \FunctionTok{aes}\NormalTok{(}\AttributeTok{colour =}\NormalTok{ DRUG), }\AttributeTok{fill =} \StringTok{"white"}\NormalTok{) }\SpecialCharTok{+}
  \FunctionTok{theme}\NormalTok{(}\AttributeTok{panel.background =} \FunctionTok{element\_rect}\NormalTok{(}\AttributeTok{fill =} \FunctionTok{gray}\NormalTok{(}\FloatTok{0.8}\NormalTok{))) }\SpecialCharTok{+}
  \FunctionTok{geom\_errorbar}\NormalTok{(}\FunctionTok{aes}\NormalTok{(}\AttributeTok{ymin =}\NormalTok{ meanmse, }\AttributeTok{ymax =}\NormalTok{ meanpse, }\AttributeTok{colour =}\NormalTok{ DRUG),}
                \AttributeTok{width =} \FloatTok{0.2}\NormalTok{, }\AttributeTok{position =} \FunctionTok{position\_dodge}\NormalTok{(.}\DecValTok{9}\NormalTok{))}
\end{Highlighting}
\end{Shaded}

\begin{figure}[H]

{\centering \includegraphics{figure/minimal-ebbp-1} 

}

\caption{Error bars overlaid on a bar plot of the \texttt{asthma} data.}\label{fig:ebbp}
\end{figure}

\end{example}

\(\blacksquare\)

\index[R]{\texttt{ggplot2::stat\_summary()}}
\index[R]{\texttt{ggplot2::geom\_errorbar()}}

\begin{example}
\protect\hypertarget{exm:exm-giv2}{}\label{exm:exm-giv2}

\(\text{}\)\\
Next we consider overlaying intervals on a line plot Fig (\ref{fig:eb}). In the code below, lines connect points at treatment means (Lines 2-3).

\vspace{0.1in}

\begin{Shaded}
\begin{Highlighting}[numbers=left,,]
\NormalTok{g }\SpecialCharTok{+} \FunctionTok{geom\_point}\NormalTok{(}\AttributeTok{size =} \DecValTok{2}\NormalTok{, }\FunctionTok{aes}\NormalTok{(}\AttributeTok{colour =}\NormalTok{ DRUG)) }\SpecialCharTok{+}
  \FunctionTok{geom\_line}\NormalTok{(}\FunctionTok{aes}\NormalTok{(}\AttributeTok{lty =}\NormalTok{ DRUG, }\AttributeTok{colour =}\NormalTok{ DRUG)) }\SpecialCharTok{+}
  \FunctionTok{theme\_classic}\NormalTok{() }\SpecialCharTok{+}
  \FunctionTok{margin\_theme}\NormalTok{() }\SpecialCharTok{+}
  \FunctionTok{geom\_errorbar}\NormalTok{(}\FunctionTok{aes}\NormalTok{(}\AttributeTok{ymin =}\NormalTok{ meanmse, }\AttributeTok{ymax =}\NormalTok{ meanpse, }\AttributeTok{colour =}\NormalTok{ DRUG), }
                \AttributeTok{width =} \FloatTok{0.2}\NormalTok{)}
\end{Highlighting}
\end{Shaded}

\begin{figure}[H]

{\centering \includegraphics{figure/minimal-eb-1} 

}

\caption{Error bars overlaid on a line plot of the \texttt{asthma} data.}\label{fig:eb}
\end{figure}

\end{example}

\(\blacksquare\)
\index[R]{\texttt{ggplot2::stat\_summary()}}
\index[R]{\texttt{ggplot2::geom\_errorbar()}}

\begin{example}
\protect\hypertarget{exm:exm-giv3}{}\label{exm:exm-giv3}

\(\text{}\)\\
Other geoms can be used to create interval plots, including the \emph{ggplot2} function \texttt{geom\_crossbar()}. In Fig \ref{fig:cb} we show both raw data and summary standard error crossbars.

\vspace{0.1in}

\begin{Shaded}
\begin{Highlighting}[]
\NormalTok{g }\SpecialCharTok{+} \FunctionTok{geom\_crossbar}\NormalTok{(}\FunctionTok{aes}\NormalTok{(}\AttributeTok{ymin =}\NormalTok{ meanmse, }\AttributeTok{ymax =}\NormalTok{ meanpse, }\AttributeTok{colour =}\NormalTok{ DRUG,}
                      \AttributeTok{fill =}\NormalTok{ DRUG), }\AttributeTok{alpha =}\NormalTok{ .}\DecValTok{2}\NormalTok{) }\SpecialCharTok{+}
  \FunctionTok{geom\_point}\NormalTok{(}\AttributeTok{data =}\NormalTok{ asthma.long, }\FunctionTok{aes}\NormalTok{(}\AttributeTok{y =}\NormalTok{ FEV1, }\AttributeTok{x =}\NormalTok{ TIME, }\AttributeTok{colour =}\NormalTok{ DRUG))}
\end{Highlighting}
\end{Shaded}

\begin{figure}[H]

{\centering \includegraphics{figure/minimal-cb-1} 

}

\caption{Error cross bars overlaid on the \texttt{asthma} data.}\label{fig:cb}
\end{figure}

Note that individual data points are rather difficult to distinguish in Fig \ref{fig:cb}. As a solution we could plot points using using transparent colors, or jitter points with respect to the \(x\)-axis (Fig \ref{fig:cb1}).

\begin{Shaded}
\begin{Highlighting}[]
\NormalTok{g }\SpecialCharTok{+} \FunctionTok{geom\_crossbar}\NormalTok{(}\FunctionTok{aes}\NormalTok{(}\AttributeTok{ymin =}\NormalTok{ meanmse, }\AttributeTok{ymax =}\NormalTok{ meanpse, }\AttributeTok{colour =}\NormalTok{ DRUG,}
                      \AttributeTok{fill =}\NormalTok{ DRUG), }\AttributeTok{alpha =}\NormalTok{ .}\DecValTok{2}\NormalTok{) }\SpecialCharTok{+}
  \FunctionTok{geom\_jitter}\NormalTok{(}\AttributeTok{data =}\NormalTok{ asthma.long, }\FunctionTok{aes}\NormalTok{(}\AttributeTok{y =}\NormalTok{ FEV1, }\AttributeTok{x =}\NormalTok{ TIME, }\AttributeTok{colour =}\NormalTok{ DRUG), }
              \AttributeTok{alpha =}\NormalTok{ .}\DecValTok{4}\NormalTok{, }\AttributeTok{width =} \FloatTok{0.15}\NormalTok{) }\SpecialCharTok{+} 
  \FunctionTok{theme\_classic}\NormalTok{()}
\end{Highlighting}
\end{Shaded}

\begin{figure}[H]

{\centering \includegraphics{figure/minimal-cb1-1} 

}

\caption{Jitter and transparency added to points in Fig \ref{fig:cb}.}\label{fig:cb1}
\end{figure}

\end{example}

\(\blacksquare\)

\index[R]{\texttt{ggplot2::geom\_crossbar()}}

\subsubsection{Pairwise Comparisons}\label{pairwise-comparisons-1}

Results of statistical pairwise comparisons can be overlain on \texttt{ggplot2} rendered boxplots and interval plots using a number of approaches. The package \emph{ggpubr} \citep{kass} generates \emph{ggplot2}-based ``publication ready plots'\,', including interval plots showing pairwise comparisons. Examples given here largely follow those in the documentation for \emph{ggpubr}.
\index[terms]{R!package!\textit{ggpubr}}

\begin{example}
\protect\hypertarget{exm:exm-gpw}{}\label{exm:exm-gpw}

\(\text{}\)\\
\citet{crampton1947} measured the lengths of odontoblasts (cells responsible for tooth growth) in 60 guinea pigs with respect to three dosage levels of vitamin C (0.5, 1, and 2 mg/day), and two delivery methods, orange juice (OJ) or ascorbic acid (VC). The data are in the dataframe \texttt{ToothGrowth} from the package \emph{datasets}. In \texttt{ToothGrowth} \texttt{dose} contains dosages and \texttt{supp} contains delivery levels.

\vspace{0.1in}

\begin{Shaded}
\begin{Highlighting}[numbers=left,,]
\FunctionTok{library}\NormalTok{(ggpubr)}
\NormalTok{df }\OtherTok{\textless{}{-}}\NormalTok{ ToothGrowth}
\NormalTok{df}\SpecialCharTok{$}\NormalTok{dose }\OtherTok{\textless{}{-}} \FunctionTok{as.factor}\NormalTok{(df}\SpecialCharTok{$}\NormalTok{dose)}

\NormalTok{bxp }\OtherTok{\textless{}{-}} \FunctionTok{ggboxplot}\NormalTok{(}
\NormalTok{  df, }\AttributeTok{x =} \StringTok{"dose"}\NormalTok{, }\AttributeTok{y =} \StringTok{"len"}\NormalTok{,}
  \AttributeTok{color =} \StringTok{"supp"}\NormalTok{, }\AttributeTok{palette =} \FunctionTok{c}\NormalTok{(}\StringTok{"\#00AFBB"}\NormalTok{, }\StringTok{"\#E7B800"}\NormalTok{)) }\SpecialCharTok{+}
  \FunctionTok{ylab}\NormalTok{(}\FunctionTok{expression}\NormalTok{(}\FunctionTok{paste}\NormalTok{(}\StringTok{"Odontoblast length ("}\NormalTok{, mu,}\StringTok{"g)"}\NormalTok{), }\AttributeTok{sep =} \StringTok{""}\NormalTok{)) }\SpecialCharTok{+}
  \FunctionTok{xlab}\NormalTok{(}\StringTok{"Dosage (mg/day)"}\NormalTok{) }\SpecialCharTok{+}
  \FunctionTok{guides}\NormalTok{(}\AttributeTok{color=}\FunctionTok{guide\_legend}\NormalTok{(}\StringTok{"Delivery:"}\NormalTok{))  }\SpecialCharTok{+}
  \FunctionTok{scale\_y\_continuous}\NormalTok{(}\AttributeTok{expand =} \FunctionTok{expansion}\NormalTok{(}\AttributeTok{mult =} \FunctionTok{c}\NormalTok{(}\FloatTok{0.05}\NormalTok{, }\FloatTok{0.10}\NormalTok{)))}

\NormalTok{bxp }\SpecialCharTok{+} \FunctionTok{geom\_pwc}\NormalTok{(}
  \FunctionTok{aes}\NormalTok{(}\AttributeTok{group =}\NormalTok{ supp), }\AttributeTok{tip.length =} \DecValTok{0}\NormalTok{,}
  \AttributeTok{method =} \StringTok{"t\_test"}\NormalTok{, }\AttributeTok{label =} \StringTok{"\{p.adj.format\}\{p.adj.signif\}"}\NormalTok{,}
  \AttributeTok{p.adjust.method =} \StringTok{"bonferroni"}\NormalTok{, }\AttributeTok{p.adjust.by =} \StringTok{"panel"}\NormalTok{,}
  \AttributeTok{hide.ns =} \ConstantTok{TRUE}
\NormalTok{)}
\end{Highlighting}
\end{Shaded}

In the code above, we have the following important steps:

\begin{itemize}
\tightlist
\item
  On Lines 1-3, I bring in the \emph{ggpubr} package (Line 1), rename the dataframe \texttt{ToothGrowth} to be \texttt{df} and coerce the \texttt{dose} column to be a factor.
\item
  On Lines 5-7, I use the function \texttt{ggpubr::ggboxplot()} to define basic plot characteristics.
\item
  On Lines 8-11, nuances are added to the plot including customized axis labels (Lines 8-9), a customized legend title (Line 10), and an alteration to the axis scale (Line 11).
\item
  On Lines 13-18, annotations for pairwise comparisons of delivery methods (\texttt{OJ} and \texttt{VC}) within dosages are added to the graph using the function \texttt{ggpubr::geom\_pwc()}.
\item
  On Line 14, I specify that I want delivery methods (in \texttt{supp}) compared, and indicate that I don't want lines extending to the compared levels from the label lines (for comparison, see Fig \ref{fig:pw3}).
\item
  On Line 15, I indicate the type of test to be used in delivery method comparisons, and the labeling format. \texttt{"\{p.adj.format\}\{p.adj.signif\}"} indicates that both the adjusted \emph{p}-value and the significance level for the adjusted \emph{p}-value should be printed.
\item
  On Lines 16-17, I specify use of the Bonferroni correction for simultaneous inference for three tests, and to not print results that are non-significant.
\end{itemize}

The result is shown in Fig \ref{fig:pw1}.

\begin{figure}[H]

{\centering \includegraphics{figure/minimal-pw1-1} 

}

\caption{Boxplot showing pairwise comparison of delivery levels in dosage for the \texttt{Toothgrowth} dataframe.}\label{fig:pw1}
\end{figure}

Below we consider a more complex example that compares both delivery methods (\texttt{supp}) and dosage levels (\texttt{dose}). This is accomplished by applying \texttt{ggpubr::geom\_pwc()} twice (Lines 3-7 and Lines 11-15) and printing both results (Fig \ref{fig:pw2}).

\vspace{0.1in}

\begin{Shaded}
\begin{Highlighting}[numbers=left,,]
\CommentTok{\# 1. Add p{-}values of OJ vs VC in each dose group}
\NormalTok{bxp.complex }\OtherTok{\textless{}{-}}\NormalTok{ bxp }\SpecialCharTok{+}
  \FunctionTok{geom\_pwc}\NormalTok{(}
    \FunctionTok{aes}\NormalTok{(}\AttributeTok{group =}\NormalTok{ supp), }\AttributeTok{tip.length =} \DecValTok{0}\NormalTok{,}
    \AttributeTok{method =} \StringTok{"t\_test"}\NormalTok{, }\AttributeTok{label =} \StringTok{"p.adj.format"}\NormalTok{,}
    \AttributeTok{p.adjust.method =} \StringTok{"bonferroni"}\NormalTok{, }\AttributeTok{p.adjust.by =} \StringTok{"panel"}
\NormalTok{  )}
\CommentTok{\# 2. Add pairwise comparisons between dose levels}
\CommentTok{\# Nudge up the brackets by 20\% of the total height}
\NormalTok{bxp.complex }\OtherTok{\textless{}{-}}\NormalTok{ bxp.complex }\SpecialCharTok{+}
  \FunctionTok{geom\_pwc}\NormalTok{(}
    \AttributeTok{method =} \StringTok{"t\_test"}\NormalTok{, }\AttributeTok{label =} \StringTok{"p.adj.format"}\NormalTok{,}
    \AttributeTok{p.adjust.method =} \StringTok{"bonferroni"}\NormalTok{,}
    \AttributeTok{bracket.nudge.y =} \FloatTok{0.2}
\NormalTok{  )}
\CommentTok{\# 3. Display the plot}
\NormalTok{bxp.complex}
\end{Highlighting}
\end{Shaded}

\begin{figure}[H]

{\centering \includegraphics{figure/minimal-pw2-1} 

}

\caption{Boxplot showing pairwise comparison of delivery levels and delivery levels in dosage for the \texttt{Toothgrowth} dataframe.}\label{fig:pw2}
\end{figure}

In the code below, we create an interval plot with a barplot appearance using \texttt{ggpubr::ggbarplot()} (Fig \ref{fig:pw3}. Note that this requires a different approach for customizing the title of the legend (Line 7).

\vspace{0.1in}

\begin{Shaded}
\begin{Highlighting}[numbers=left,,]
\NormalTok{bp }\OtherTok{\textless{}{-}} \FunctionTok{ggbarplot}\NormalTok{(}
\NormalTok{  df, }\AttributeTok{x =} \StringTok{"supp"}\NormalTok{, }\AttributeTok{y =} \StringTok{"len"}\NormalTok{, }\AttributeTok{fill =} \StringTok{"dose"}\NormalTok{,}
  \AttributeTok{palette =} \StringTok{"npg"}\NormalTok{, }\AttributeTok{add =} \StringTok{"mean\_sd"}\NormalTok{,}
  \AttributeTok{position =} \FunctionTok{position\_dodge}\NormalTok{(}\FloatTok{0.8}\NormalTok{)) }\SpecialCharTok{+}
  \FunctionTok{ylab}\NormalTok{(}\FunctionTok{expression}\NormalTok{(}\FunctionTok{paste}\NormalTok{(}\StringTok{"Odontoblast length ("}\NormalTok{, mu,}\StringTok{"m)"}\NormalTok{), }\AttributeTok{sep =} \StringTok{""}\NormalTok{)) }\SpecialCharTok{+}
  \FunctionTok{xlab}\NormalTok{(}\StringTok{"Delivery method"}\NormalTok{) }\SpecialCharTok{+}
  \FunctionTok{scale\_fill\_discrete}\NormalTok{(}\AttributeTok{name =} \StringTok{"Dosage:"}\NormalTok{)}

\NormalTok{bp }\SpecialCharTok{+}
  \FunctionTok{geom\_pwc}\NormalTok{(}
    \FunctionTok{aes}\NormalTok{(}\AttributeTok{group =}\NormalTok{ dose), }\AttributeTok{tip.length =} \FloatTok{0.05}\NormalTok{,}
    \AttributeTok{method =} \StringTok{"t\_test"}\NormalTok{, }\AttributeTok{label =} \StringTok{"p.signif"}\NormalTok{,}
    \AttributeTok{bracket.nudge.y =} \SpecialCharTok{{-}}\FloatTok{0.08}
\NormalTok{  ) }\SpecialCharTok{+}
  \FunctionTok{scale\_y\_continuous}\NormalTok{(}\AttributeTok{expand =} \FunctionTok{expansion}\NormalTok{(}\AttributeTok{mult =} \FunctionTok{c}\NormalTok{(}\DecValTok{0}\NormalTok{, }\FloatTok{0.1}\NormalTok{)))}
\end{Highlighting}
\end{Shaded}

\begin{figure}[H]

{\centering \includegraphics{figure/minimal-pw3-1} 

}

\caption{Barplot showing pairwise comparison of dosage levels in delivery methods for the \texttt{Toothgrowth} dataframe.  Bar heights are means, errors are standard deviations.}\label{fig:pw3}
\end{figure}

\end{example}

\(\blacksquare\)

\index[R]{\texttt{ggpubr::ggboxplot()}}
\index[R]{\texttt{ggpubr::geom\_pwc()}}
\index[R]{\texttt{ggpubr::ggbarplot()}}

\begin{blackbox}

\begin{center}
\textbf{CAUTION!}

\end{center}

The function \texttt{ggpubr::geom\_pwc()} can be potentially misused, illustrating the need for clear explanations (or understanding) when applying statistical algorithms. The default \texttt{test} specification in \texttt{geom\_pwc()} is the Wilcox test, which will seldom be the most powerful method for comparing shifts in location for treatments (although it is strongly resistant to violations of normality). The argument \texttt{test\ =\ t\_test} (specified in Figs \ref{fig:pw1}-\ref{fig:pw3}) runs \(t\)-tests in isolation for each pairwise comparison, and thus will not utilize an omnibus ANOVA mean squared error, reducing power. The Bonferroni \(p\)-value adjustment method used in Figs \ref{fig:pw1}-\ref{fig:pw3} is also famous for its low power. Given this situation, it may be most prudent to use \texttt{ggpubr::geom\_pwc()} as a graphical framework into which summaries, including \(p\)-values can be inserted manually. This can be done with the function \texttt{ggpubr::stat\_pvalue\_manual()} whose usage is demonstrated \href{https://www.datanovia.com/en/blog/how-to-add-p-values-to-ggplot-facets/}{here}.

\end{blackbox}

\subsection{Trellis Plots with Faceting}\label{trellis-plots-with-faceting}

\index[terms]{R!graphics!trellis plots}
Like the package \emph{lattice}, \emph{ggplot2} contains functions for making trellis plots. We will use this approach to examine individual patient responses over time in the \texttt{asthma} dataset.

\vspace{0.1in}

\begin{Shaded}
\begin{Highlighting}[]
\CommentTok{\# subset data to allow readable plots}
\NormalTok{asthma.long.a }\OtherTok{\textless{}{-}}\NormalTok{ asthma.long }\SpecialCharTok{|\textgreater{}}
  \FunctionTok{filter}\NormalTok{(PATIENT }\SpecialCharTok{\%in\%} \DecValTok{201}\SpecialCharTok{:}\DecValTok{208}\NormalTok{)}
\end{Highlighting}
\end{Shaded}

Trellising can be enabled by using the \emph{ggplot2} functions \texttt{facet\_wrap()} and \texttt{facet\_grid()}. In Fig \ref{fig:tr} we define faceting within \texttt{facet\_grid()} using the \texttt{PATIENT} column in the data subset \texttt{asthma.long.a}. The function \texttt{ggplot2::vars()} in \texttt{facet\_grid()} is analogous to the use of \texttt{aes()} in geoms.

\vspace{0.1in}

\begin{Shaded}
\begin{Highlighting}[numbers=left,,]
\NormalTok{g }\OtherTok{\textless{}{-}} \FunctionTok{ggplot}\NormalTok{(asthma.long.a, }\FunctionTok{aes}\NormalTok{(}\AttributeTok{y =}\NormalTok{ FEV1, }\AttributeTok{x =}\NormalTok{ TIME, }\AttributeTok{colour =}\NormalTok{ DRUG,}
                                \AttributeTok{group =}\NormalTok{ DRUG)) }\SpecialCharTok{+}
  \FunctionTok{geom\_point}\NormalTok{() }\SpecialCharTok{+}
  \FunctionTok{geom\_line}\NormalTok{() }\SpecialCharTok{+}
  \FunctionTok{theme\_light}\NormalTok{() }\SpecialCharTok{+} \FunctionTok{margin\_theme}\NormalTok{() }\SpecialCharTok{+}
  \FunctionTok{theme}\NormalTok{(}\AttributeTok{axis.text.y =} \FunctionTok{element\_text}\NormalTok{(}\AttributeTok{size=}\FunctionTok{rel}\NormalTok{(}\FloatTok{0.7}\NormalTok{))) }\SpecialCharTok{+}
  \FunctionTok{facet\_grid}\NormalTok{(}\AttributeTok{rows =} \FunctionTok{vars}\NormalTok{(PATIENT)) }\SpecialCharTok{+}
  \FunctionTok{scale\_colour\_brewer}\NormalTok{(}\AttributeTok{palette =} \StringTok{"Dark2"}\NormalTok{) }\SpecialCharTok{+}
  \FunctionTok{ylab}\NormalTok{(}\StringTok{"Forced Expiratory Volume (1 min)"}\NormalTok{) }\SpecialCharTok{+}
  \FunctionTok{xlab}\NormalTok{(}\StringTok{"Time period"}\NormalTok{)}
\NormalTok{g}
\end{Highlighting}
\end{Shaded}

\begin{figure}[H]

{\centering \includegraphics{figure/minimal-tr-1} 

}

\caption{A trellis plot showing individual patient responses over time from the \texttt{asthma} dataset.}\label{fig:tr}
\end{figure}

\index[R]{\texttt{ggplot2::facet\_wrap()}}
\index[R]{\texttt{ggplot2::facet\_grid()}}

\subsection{Multivariate Distributional Summaries}\label{multivariate-distributional-summaries}

Bivariate summaries can be shown in many ways using a \emph{ggplot2} approach.
\index[terms]{Grob (graphical object)}

\begin{example}
\protect\hypertarget{exm:exm-gMV}{}\label{exm:exm-gMV}

\(\text{}\)\\
In Fig \ref{fig:bv}, I insert density grobs (graphical objects) on the margins of a scatterplot for five European countries using the \emph{cowplot} functions \texttt{axis\_canvas()}, \texttt{insert\_xaxis\_grob()}, \texttt{insert\_xaxis\_grob()}, and \texttt{gg\_draw()}. The right margin shows GDP distributions for each country, whereas the top margin shows CO\(_2\) emission distributions for each country. I also change symbol sizes with year in the main graph. Larger symbols indicate more recent years.

\vspace{0.1in}

\begin{Shaded}
\begin{Highlighting}[numbers=left,,]
\NormalTok{europe }\OtherTok{\textless{}{-}}\NormalTok{ world.emissions }\SpecialCharTok{|\textgreater{}}
  \FunctionTok{filter}\NormalTok{(country }\SpecialCharTok{==} \FunctionTok{c}\NormalTok{(}\StringTok{"France"}\NormalTok{, }\StringTok{"Italy"}\NormalTok{, }\StringTok{"Germany"}\NormalTok{, }\StringTok{"United Kingdom"}\NormalTok{)) }\SpecialCharTok{|\textgreater{}}
  \FunctionTok{filter}\NormalTok{(year }\SpecialCharTok{\textless{}=} \DecValTok{2019} \SpecialCharTok{\&}\NormalTok{ year }\SpecialCharTok{\textgreater{}} \DecValTok{1950}\NormalTok{) }\CommentTok{\# comparable data}

\NormalTok{pmain }\OtherTok{\textless{}{-}} \FunctionTok{ggplot}\NormalTok{(europe, }\FunctionTok{aes}\NormalTok{(}\AttributeTok{x=}\NormalTok{co2, }\AttributeTok{y=}\NormalTok{gdp, }\AttributeTok{color=}\NormalTok{ country)) }\SpecialCharTok{+}
  \FunctionTok{geom\_point}\NormalTok{(}\FunctionTok{aes}\NormalTok{(}\AttributeTok{size =}\NormalTok{ year), }\AttributeTok{alpha =}\NormalTok{ .}\DecValTok{6}\NormalTok{) }\SpecialCharTok{+}
  \FunctionTok{xlab}\NormalTok{(xlab) }\SpecialCharTok{+} \FunctionTok{ylab}\NormalTok{(}\StringTok{"GDP (International dollars)"}\NormalTok{) }\SpecialCharTok{+}
  \FunctionTok{margin\_theme}\NormalTok{() }\SpecialCharTok{+}
  \FunctionTok{theme\_classic}\NormalTok{()}

\NormalTok{xdens }\OtherTok{\textless{}{-}} \FunctionTok{axis\_canvas}\NormalTok{(pmain, }\AttributeTok{axis =} \StringTok{"x"}\NormalTok{) }\SpecialCharTok{+}
  \FunctionTok{geom\_density}\NormalTok{(}\AttributeTok{data =}\NormalTok{ europe, }\FunctionTok{aes}\NormalTok{(}\AttributeTok{x =}\NormalTok{ co2, }\AttributeTok{fill =}\NormalTok{ country), }\AttributeTok{alpha=}\FloatTok{0.6}\NormalTok{, }
               \AttributeTok{size=}\NormalTok{.}\DecValTok{2}\NormalTok{)}

\NormalTok{ydens }\OtherTok{\textless{}{-}} \FunctionTok{axis\_canvas}\NormalTok{(pmain, }\AttributeTok{axis =} \StringTok{"y"}\NormalTok{) }\SpecialCharTok{+}
  \FunctionTok{geom\_density}\NormalTok{(}\AttributeTok{data =}\NormalTok{ europe, }\FunctionTok{aes}\NormalTok{(}\AttributeTok{y =}\NormalTok{ gdp, }\AttributeTok{fill =}\NormalTok{ country), }\AttributeTok{alpha=}\FloatTok{0.6}\NormalTok{, }
               \AttributeTok{size=}\NormalTok{.}\DecValTok{2}\NormalTok{)}

\NormalTok{p1 }\OtherTok{\textless{}{-}} \FunctionTok{insert\_xaxis\_grob}\NormalTok{(pmain, xdens, grid}\SpecialCharTok{::}\FunctionTok{unit}\NormalTok{(.}\DecValTok{2}\NormalTok{, }\StringTok{"null"}\NormalTok{), }
                        \AttributeTok{position =} \StringTok{"top"}\NormalTok{)}
\NormalTok{p2 }\OtherTok{\textless{}{-}} \FunctionTok{insert\_yaxis\_grob}\NormalTok{(p1, ydens, grid}\SpecialCharTok{::}\FunctionTok{unit}\NormalTok{(.}\DecValTok{2}\NormalTok{, }\StringTok{"null"}\NormalTok{), }
                        \AttributeTok{position =} \StringTok{"right"}\NormalTok{)}
\FunctionTok{ggdraw}\NormalTok{(p2)}
\end{Highlighting}
\end{Shaded}

\begin{figure}[H]

{\centering \includegraphics{figure/minimal-bv-1} 

}

\caption{Bivariate summaries for European countries from the \texttt{asbio::world.emissions} dataset.}\label{fig:bv}
\end{figure}

\end{example}

\(\blacksquare\)

\index[R]{\texttt{cowplot::axis\_canvas}}
\index[R]{\texttt{cowplot::insert\_xaxis\_grob}}
\index[R]{\texttt{cowplot::insert\_yaxis\_grob}}
\index[R]{\texttt{cowplot::gg\_draw}}

\subsection{Maps}\label{maps}

\index[terms]{R!graphics!maps}
The \emph{ggplot2} ecosystem has some support for mapping, including import of ARC-GIS shape files, and creation of map polygons. The function \texttt{sf::st\_read()} allows loading of simple spatial features, including shapefiles, and the package \emph{ggspatial} provides a number for creating useful maps under a \emph{ggplot2} framework.
\index[R]{\texttt{sf::st\_read()}}
\index[terms]{R!package!\textit{sf}}

\begin{example}
\protect\hypertarget{exm:exm-map}{}\label{exm:exm-map}

\(\text{}\)\\
As an example we will create a map of a small stream network in southwest Idaho named Murphy Creek. Data concerning the creek, including shapefiles, is contained in the package \emph{streamDAG} \citep{aho2023non}.
\index[terms]{R!package!\textit{ggspatial}}
\vspace{0.1in}

\begin{Shaded}
\begin{Highlighting}[]
\FunctionTok{library}\NormalTok{(sf); }\FunctionTok{library}\NormalTok{(ggspatial); }\FunctionTok{library}\NormalTok{(streamDAG)}
\NormalTok{mur\_sf }\OtherTok{\textless{}{-}} \FunctionTok{st\_read}\NormalTok{(}\FunctionTok{system.file}\NormalTok{(}\StringTok{"shape/Murphy\_Creek.shp"}\NormalTok{, }
                              \AttributeTok{package=}\StringTok{"streamDAG"}\NormalTok{))}
\FunctionTok{data}\NormalTok{(mur\_coords)}
\NormalTok{coords }\OtherTok{\textless{}{-}}\NormalTok{ mur\_coords[,}\FunctionTok{c}\NormalTok{(}\DecValTok{2}\NormalTok{,}\DecValTok{3}\NormalTok{)]}
\end{Highlighting}
\end{Shaded}

\footnotesize

\begin{verbatim}
Reading layer `Murphy_Creek' from data source 
  `C:\Users\ahoken\AppData\Local\R\win-library\4.4\streamDAG\shape\Murphy_Creek.shp' 
  using driver `ESRI Shapefile'
Simple feature collection with 2 features and 2 fields
Geometry type: LINESTRING
Dimension:     XY
Bounding box:  xmin: 512860 ymin: 4789000 xmax: 514720 ymax: 4789300
Projected CRS: NAD83 / UTM zone 11N
\end{verbatim}

\normalsize

The function \texttt{ggplot2::geom\_sf()} (Line 2 below) can be used to draw different geometric objects depending on features present in the data, e.g., points, lines, or polygons. For the current case a line is generated. The function \texttt{ggplot2::expand\_limits()} (Line 6) is used to increase the spatial range of the \(y\)-axis which otherwise would be extremely narrow (since a single W to E trending line, representing the watershed, is being generated by \texttt{geom\_sf()}). The \emph{ggspatial} functions \texttt{annotation\_scale()} and \texttt{annotation\_north\_arrow()} provide spatially explicit scalebars and north-indicating arrows, respectively (Lines 8-9). The final product is shown in Fig \ref{fig:murmap}.

\vspace{0.1in}

\begin{Shaded}
\begin{Highlighting}[numbers=left,,]
\NormalTok{g }\OtherTok{\textless{}{-}} \FunctionTok{ggplot}\NormalTok{(mur\_sf) }\SpecialCharTok{+}
  \FunctionTok{geom\_sf}\NormalTok{(}\AttributeTok{colour =} \StringTok{"lightblue"}\NormalTok{, }\AttributeTok{lwd =} \DecValTok{2}\NormalTok{) }\SpecialCharTok{+}
  \FunctionTok{theme\_classic}\NormalTok{() }\SpecialCharTok{+}
  \FunctionTok{geom\_point}\NormalTok{(}\AttributeTok{data =}\NormalTok{ coords, }\FunctionTok{aes}\NormalTok{(}\AttributeTok{x =}\NormalTok{ E, }\AttributeTok{y =}\NormalTok{ N), }\AttributeTok{shape =} \DecValTok{21}\NormalTok{,}
             \AttributeTok{fill =} \StringTok{"orange"}\NormalTok{, }\AttributeTok{size =} \FloatTok{2.5}\NormalTok{) }\SpecialCharTok{+}
  \FunctionTok{expand\_limits}\NormalTok{(}\AttributeTok{y =} \FunctionTok{c}\NormalTok{(}\DecValTok{4788562}\NormalTok{,}\DecValTok{4789700}\NormalTok{)) }\SpecialCharTok{+}
  \FunctionTok{ylab}\NormalTok{(}\StringTok{""}\NormalTok{) }\SpecialCharTok{+} \FunctionTok{xlab}\NormalTok{(}\StringTok{""}\NormalTok{) }\SpecialCharTok{+}
  \FunctionTok{annotation\_scale}\NormalTok{() }\SpecialCharTok{+}
  \FunctionTok{annotation\_north\_arrow}\NormalTok{(}\AttributeTok{pad\_x =} \FunctionTok{unit}\NormalTok{(}\FloatTok{10.5}\NormalTok{, }\StringTok{"cm"}\NormalTok{), }\AttributeTok{pad\_y =} \FunctionTok{unit}\NormalTok{(}\FloatTok{6.6}\NormalTok{, }\StringTok{"cm"}\NormalTok{))}
\NormalTok{g}
\end{Highlighting}
\end{Shaded}

\begin{figure}[H]

{\centering \includegraphics{figure/minimal-murmap-1} 

}

\caption{Map of the Murphy Creek drainage system in southwest Idaho (outlet coordinates: 43.71839 $^\text{o}$N, 116.13747 $^\text{o}$W).}\label{fig:murmap}
\end{figure}

\end{example}

\(\blacksquare\)

\index[terms]{R!package!\textit{streamDAG}}
\index[R]{\texttt{ggplot2::geom\_sf()}}
\index[R]{\texttt{ggplot2::expand\_limits()}}
\index[R]{\texttt{ggspatial::annotation\_scale()}}
\index[R]{\texttt{ggspatial::annotation\_north\_arrow()}}

\subsection{Animation}\label{animation-1}

\index[terms]{R!graphics!animation}
\index[terms]{Loop (programming)}
Animations in \emph{ggplot2} can be created using looping strategies applied in Ch 6. Looping will be explicitly considered in the context of functions in Chapter 8.

\begin{example}
\protect\hypertarget{exm:exm-ganim1}{}\label{exm:exm-ganim1}

\(\text{}\)\\
As an initial demonstration, we reconsider the asthma data (Fig \ref{fig:bpp}). We first construct a function, \texttt{asthma.plot()} which will render a ggplot. The lone argument of \texttt{asthma.plot()}, \texttt{upper}, defines the upper time limit of under consideration in the longitudinal asthma drug study (Line 3). The \texttt{upper} argument is called in \texttt{geom\_line()} (Lines 5-6) to subset, if necessary, the underlying data. Vital to the animation is the \texttt{print.ggplot()} function (Line 12). Failure to include this code will create an empty animation.

\vspace{0.1in}

\begin{Shaded}
\begin{Highlighting}[numbers=left,,]
\NormalTok{summary.FEV}\SpecialCharTok{$}\NormalTok{time }\OtherTok{\textless{}{-}} \FunctionTok{rep}\NormalTok{(}\FunctionTok{c}\NormalTok{(}\DecValTok{0}\NormalTok{,}\DecValTok{11}\SpecialCharTok{:}\DecValTok{18}\NormalTok{), }\AttributeTok{each =} \DecValTok{3}\NormalTok{)}

\NormalTok{asthma.plot }\OtherTok{\textless{}{-}} \ControlFlowTok{function}\NormalTok{(upper)\{}
\NormalTok{  a }\OtherTok{\textless{}{-}} \FunctionTok{ggplot}\NormalTok{() }\SpecialCharTok{+}
    \FunctionTok{geom\_line}\NormalTok{(}\AttributeTok{data =}\NormalTok{ summary.FEV[summary.FEV}\SpecialCharTok{$}\NormalTok{time }\SpecialCharTok{\textgreater{}} \DecValTok{10} \SpecialCharTok{\&} 
\NormalTok{                                   summary.FEV}\SpecialCharTok{$}\NormalTok{time }\SpecialCharTok{\textless{}=}\NormalTok{ upper,],}
              \FunctionTok{aes}\NormalTok{(}\AttributeTok{x =}\NormalTok{ time, }\AttributeTok{y =}\NormalTok{ mean, }\AttributeTok{colour =}\NormalTok{ DRUG)) }\SpecialCharTok{+}
    \FunctionTok{ylim}\NormalTok{(}\FloatTok{2.6}\NormalTok{, }\DecValTok{4}\NormalTok{) }\SpecialCharTok{+}
    \FunctionTok{xlim}\NormalTok{(}\DecValTok{11}\NormalTok{, }\DecValTok{18}\NormalTok{) }\SpecialCharTok{+}
    \FunctionTok{margin\_theme}\NormalTok{() }\SpecialCharTok{+}
    \FunctionTok{labs}\NormalTok{(}\AttributeTok{y =} \StringTok{"Forced Expiratory Volume (1 min)"}\NormalTok{,}
         \AttributeTok{x =} \StringTok{"Time (Hrs)"}\NormalTok{)}
  \FunctionTok{print}\NormalTok{(a)}
\NormalTok{\}}
\end{Highlighting}
\end{Shaded}

Next, we create a function that runs \texttt{asthma.plot()} for a range of values for \texttt{upper}. The function consists of a loop run by \texttt{lapply()} (lines 2-3). The final lines of code (lines 8-9) allow the animation to be saved using the function \texttt{saveGIF()} from the package \emph{animation}\footnote{As noted in Ch 6, use of \texttt{animation::saveGIF} requires installation of open source software \href{http://www.imagemagick.org/script/convert.php}{ImageMagick} or \href{http://www.graphicsmagick.org}{GraphicsMagick} (see \texttt{?saveGIF}).}.
\index[R]{\texttt{animation::saveGIF()}}
\index[R]{\texttt{lapply()}}
\index[terms]{Loop (programming)}

\vspace{0.1in}

\begin{Shaded}
\begin{Highlighting}[numbers=left,,]
\NormalTok{asthma.animate }\OtherTok{\textless{}{-}} \ControlFlowTok{function}\NormalTok{() \{}
  \FunctionTok{lapply}\NormalTok{(}\DecValTok{12}\SpecialCharTok{:}\DecValTok{18}\NormalTok{, }\ControlFlowTok{function}\NormalTok{(i)\{}
    \FunctionTok{asthma.plot}\NormalTok{(i)}
\NormalTok{  \})}
\NormalTok{\}}

\CommentTok{\# run animation}
\FunctionTok{asthma.animate}\NormalTok{()}

\CommentTok{\# save frames into one GIF:}
\FunctionTok{library}\NormalTok{(animation)}
\FunctionTok{saveGIF}\NormalTok{(}\FunctionTok{asthma.animate}\NormalTok{(), }\AttributeTok{interval =} \DecValTok{1}\NormalTok{, }\AttributeTok{movie.name=}\StringTok{"asthma.gif"}\NormalTok{)}
\end{Highlighting}
\end{Shaded}

The animation result is shown in Fig \ref{fig:astanm1}.

\begin{figure}[H]

{\centering \animategraphics[,controls,loop]{10}{figure/minimal-astanm1-}{1}{7}

}

\caption[Animation demonstration using the \texttt{asthma} dataset]{Animation demonstration using the \texttt{asthma} dataset.}\label{fig:astanm1}
\end{figure}

\end{example}

\(\blacksquare\)

Amazing animations can be created with the package \emph{gganimate}. These are demonstrated using several examples.
\index[terms]{R!package!\textit{gganimate}}

\begin{example}
\protect\hypertarget{exm:exm-ganim2}{}\label{exm:exm-ganim2}

\(\text{}\)\\
In this example we create a scatterplot animation for the world emissions dataset. In the code below, steps particularly important to the animation occur on Lines 14-17.

\begin{itemize}
\tightlist
\item
  On Line 14 the plot title is modified as the animation progresses, allowing tracking of years. The code \texttt{title\ =\ \textquotesingle{}Year:\ \{frame\_time\}\textquotesingle{}} is a \emph{gganimate} convention for extracting corresponding time sequence values (in this case the column \texttt{world.emission\$year}) for a projection.
\item
  On line 16 The function \texttt{gganimate::transition\_time()} calls frame transitions between specific point in time in the column \texttt{year}. Usefully, the \emph{gganimate} sets the transition time between the states in \texttt{transition\_time()} to correspond to the actual time difference between them.
\item
  On line 17 \texttt{ease\_aes()} is used to linearly smooth the animation (in terms of coloration and the geometric positioning of features) between animation frames. The function is based on analogous functions from the package \emph{tweener}.
\end{itemize}

The package \emph{gapminder} contains rational color designations (i.e., variations on prime colors within continents) for 142 countries. Countries without a color designation are colored gray by \texttt{scale\_colour\_manual()}.

\vspace{0.1in}

\begin{Shaded}
\begin{Highlighting}[numbers=left,,]
\FunctionTok{library}\NormalTok{(gganimate)}
\FunctionTok{library}\NormalTok{(gapminder)}
\NormalTok{world.data.sub }\OtherTok{\textless{}{-}}\NormalTok{ world.emissions }\SpecialCharTok{|\textgreater{}}
  \FunctionTok{filter}\NormalTok{(continent }\SpecialCharTok{!=} \StringTok{"Redundant"}\NormalTok{) }\SpecialCharTok{|\textgreater{}}
  \FunctionTok{filter}\NormalTok{(year }\SpecialCharTok{\textgreater{}} \DecValTok{1950}\NormalTok{)}

\NormalTok{g }\OtherTok{\textless{}{-}} \FunctionTok{ggplot}\NormalTok{(world.data.sub, }\FunctionTok{aes}\NormalTok{(}\AttributeTok{x =}\NormalTok{ gdp, }\AttributeTok{y =}\NormalTok{ co2, }\AttributeTok{size =}\NormalTok{ population,}
                                \AttributeTok{colour =}\NormalTok{ country)) }\SpecialCharTok{+}
  \FunctionTok{geom\_point}\NormalTok{(}\AttributeTok{alpha =} \FloatTok{0.7}\NormalTok{, }\AttributeTok{show.legend =} \ConstantTok{FALSE}\NormalTok{) }\SpecialCharTok{+}
  \FunctionTok{scale\_colour\_manual}\NormalTok{(}\AttributeTok{values =}\NormalTok{ country\_colors) }\SpecialCharTok{+}
  \FunctionTok{scale\_size}\NormalTok{(}\AttributeTok{range =} \FunctionTok{c}\NormalTok{(}\DecValTok{2}\NormalTok{, }\DecValTok{12}\NormalTok{)) }\SpecialCharTok{+}
  \FunctionTok{scale\_x\_log10}\NormalTok{() }\SpecialCharTok{+} \FunctionTok{scale\_y\_log10}\NormalTok{() }\SpecialCharTok{+}
  \FunctionTok{facet\_wrap}\NormalTok{(}\SpecialCharTok{\textasciitilde{}}\NormalTok{continent) }\SpecialCharTok{+}
  \FunctionTok{labs}\NormalTok{(}\AttributeTok{title =} \StringTok{\textquotesingle{}Year: \{frame\_time\}\textquotesingle{}}\NormalTok{, }\AttributeTok{x =} \StringTok{\textquotesingle{}GDP\textquotesingle{}}\NormalTok{) }\SpecialCharTok{+}
  \FunctionTok{ylab}\NormalTok{(}\FunctionTok{bquote}\NormalTok{(CO[}\DecValTok{2}\NormalTok{])) }\SpecialCharTok{+}
  \FunctionTok{transition\_time}\NormalTok{(}\FunctionTok{as.integer}\NormalTok{(year)) }\SpecialCharTok{+}
  \FunctionTok{ease\_aes}\NormalTok{(}\StringTok{\textquotesingle{}linear\textquotesingle{}}\NormalTok{)}
  \FunctionTok{margin\_theme}\NormalTok{()}
\NormalTok{g}
\end{Highlighting}
\end{Shaded}

The final result is shown in Fig \ref{fig:weanim2}.

\begin{figure}[H]

{\centering \animategraphics[width=1\linewidth,,controls,loop]{5}{figure/minimal-weanim2-}{1}{100}

}

\caption[Animated scatterplot of CO$_2$ levels over time for countries within continents]{Animated scatterplot of CO$_2$ levels over time for countries within continents. Symbol size scaled by population size.}\label{fig:weanim2}
\end{figure}

\end{example}

\(\blacksquare\)

\index[terms]{R!package!\textit{sf}}
\index[terms]{R!package!\textit{gapminder}}
\index[terms]{R!package!\textit{tweenr}}
\index[R]{\texttt{ggnimate::ease\_aes}}
\index[R]{\texttt{ggnimate::transition\_time}}

\begin{example}
\protect\hypertarget{exm:exm-ganim3}{}\label{exm:exm-ganim3}

\(\text{}\)\\
The next example uses \emph{gganimate} to animate variation in CO\(_2\) levels over time within continents, using boxplots.

\vspace{0.1in}

\begin{Shaded}
\begin{Highlighting}[numbers=left,,]
\NormalTok{g }\OtherTok{\textless{}{-}} \FunctionTok{ggplot}\NormalTok{(world.data.sub, }\FunctionTok{aes}\NormalTok{(}\AttributeTok{x =}\NormalTok{ continent, }\AttributeTok{y =}\NormalTok{ co2,}
                           \AttributeTok{group =}\NormalTok{ continent)) }\SpecialCharTok{+}
  \FunctionTok{geom\_boxplot}\NormalTok{(}\FunctionTok{aes}\NormalTok{(}\AttributeTok{fill =}\NormalTok{ continent), }\AttributeTok{show.legend =} \ConstantTok{FALSE}\NormalTok{) }\SpecialCharTok{+}
  \FunctionTok{scale\_y\_log10}\NormalTok{() }\SpecialCharTok{+}
  \FunctionTok{labs}\NormalTok{(}\AttributeTok{title =} \StringTok{\textquotesingle{}Year: \{frame\_time\}\textquotesingle{}}\NormalTok{, }\AttributeTok{x =} \StringTok{\textquotesingle{}\textquotesingle{}}\NormalTok{,}
       \AttributeTok{y =} \StringTok{"CO\textbackslash{}U2082"}\NormalTok{) }\SpecialCharTok{+}
  \FunctionTok{theme}\NormalTok{(}\AttributeTok{axis.text.x =} \FunctionTok{element\_text}\NormalTok{(}\AttributeTok{angle =} \DecValTok{50}\NormalTok{, }\AttributeTok{hjust =} \DecValTok{1}\NormalTok{,}
                                   \AttributeTok{vjust =} \FloatTok{0.9}\NormalTok{, }\AttributeTok{size =} \DecValTok{12}\NormalTok{)) }\SpecialCharTok{+}
  \FunctionTok{transition\_time}\NormalTok{(}\FunctionTok{as.integer}\NormalTok{(year))}
\NormalTok{g}
\end{Highlighting}
\end{Shaded}

The final result is shown in Fig \ref{fig:bpanim2}.

\begin{figure}[H]

{\centering \animategraphics[,controls,loop]{5}{figure/minimal-bpanim2-}{1}{100}

}

\caption[Animated boxplot of CO$_2$ levels over time for countries within continents]{Animated boxplot of CO$_2$ levels over time for countries within continents.}\label{fig:bpanim2}
\end{figure}

\end{example}

\(\blacksquare\)

\begin{example}
\protect\hypertarget{exm:exm-ganim3a}{}\label{exm:exm-ganim3a}

\(\text{}\)\\
As a final (rather complex) example, I animate the non-perennial character of Murphy Creek (Fig \ref{fig:murmap}) over time.

To prepare for making the map animation, I first bring in a dataset that documents the presence/absence of surface water \(=\{0,1\}\) at 28 locations (i.e., nodes) over 1163 time steps, \texttt{mur\_node\_pres\_abs} (Line 1). The 27 stream sections bounded by the the 28 nodes are defined as stream segments. I select from these time designations, at even intervals, to create a data subset of 250 time steps (Lines 2-8).

\vspace{0.1in}

\begin{Shaded}
\begin{Highlighting}[numbers=left,,]
\FunctionTok{data}\NormalTok{(mur\_node\_pres\_abs)}
\NormalTok{u }\OtherTok{\textless{}{-}} \FunctionTok{unique}\NormalTok{(mur\_node\_pres\_abs}\SpecialCharTok{$}\NormalTok{Datetime)}
\NormalTok{n }\OtherTok{\textless{}{-}} \FunctionTok{length}\NormalTok{(u)}
\NormalTok{frames }\OtherTok{\textless{}{-}} \DecValTok{250}
\NormalTok{times.sub }\OtherTok{\textless{}{-}}\NormalTok{ u[}\FunctionTok{round}\NormalTok{(}\FunctionTok{seq}\NormalTok{(}\DecValTok{1}\NormalTok{, n, }\AttributeTok{length =}\NormalTok{ frames),}\DecValTok{0}\NormalTok{)]}

\NormalTok{w }\OtherTok{\textless{}{-}} \FunctionTok{which}\NormalTok{(mur\_node\_pres\_abs}\SpecialCharTok{$}\NormalTok{Datetime }\SpecialCharTok{\%in\%}\NormalTok{ times.sub)}
\NormalTok{mnpa.sub }\OtherTok{\textless{}{-}}\NormalTok{ mur\_node\_pres\_abs[w,]}
\end{Highlighting}
\end{Shaded}

In the code below, the function \texttt{sf::st\_coordinates()} is used to pull spatial coordinates from the Murphy Creek shapefile underlying the map in Fig \ref{fig:murmap}.
I also use several functions from the \emph{streamDAG} package, including \texttt{streamDAGs()}, which creates a graph-theoretic representation Murphy Creek (see \citet{aho2023non}), and thus defines how the stream flows from location to location. The function \texttt{streamDAG::STIC.RFimpute()} is a wrapper for the random forest algorithm \texttt{missForest::missForest()}, and allows imputation of missing stream presence/absence data from the dataset \texttt{mur\_node\_pres\_abs}. The function \texttt{arc.pa.from.nodes()} from \emph{streamDAG} creates stream segment surface water presence/absence outcomes based on data from the downstream bounding node of each segment (\texttt{approach\ =\ "dstream"}). The function \texttt{vector\_segments()} from \emph{streamDAG} is used to create the dataframe \texttt{vs} that contains arcs designations for shapefile coordinates in \texttt{sf.coords}, based on coordinates in the object \texttt{node.coords}, and the function \texttt{assign\_pa\_to\_segments()} adds surface water presence/absence designations to \texttt{vs} based on outcomes from the object \texttt{arc.pa}, whew.

\vspace{0.1in}

\begin{Shaded}
\begin{Highlighting}[numbers=left,,]
\NormalTok{mur\_graph }\OtherTok{\textless{}{-}} \FunctionTok{streamDAGs}\NormalTok{(}\StringTok{"mur\_full"}\NormalTok{)}
\CommentTok{\# impute missing presence/absence data}
\NormalTok{out }\OtherTok{\textless{}{-}} \FunctionTok{STIC.RFimpute}\NormalTok{(mnpa.sub[,}\SpecialCharTok{{-}}\DecValTok{1}\NormalTok{])}
\NormalTok{mur.pa.sub }\OtherTok{\textless{}{-}}\NormalTok{ out}\SpecialCharTok{$}\NormalTok{ximp}
\CommentTok{\# arcs from nodes}
\NormalTok{arc.pa }\OtherTok{\textless{}{-}} \FunctionTok{arc.pa.from.nodes}\NormalTok{(mur\_graph, mur.pa.sub, }\AttributeTok{approach =} \StringTok{"dstream"}\NormalTok{)}

\NormalTok{node.coords }\OtherTok{\textless{}{-}} \FunctionTok{data.frame}\NormalTok{(mur\_coords[,(}\DecValTok{2}\SpecialCharTok{:}\DecValTok{3}\NormalTok{)])}
\FunctionTok{row.names}\NormalTok{(node.coords) }\OtherTok{\textless{}{-}}\NormalTok{ mur\_coords[,}\DecValTok{1}\NormalTok{]}
\NormalTok{sf.coords }\OtherTok{\textless{}{-}} \FunctionTok{st\_coordinates}\NormalTok{(mur\_sf)[,}\SpecialCharTok{{-}}\DecValTok{3}\NormalTok{]}

\NormalTok{vs }\OtherTok{\textless{}{-}} \FunctionTok{vector\_segments}\NormalTok{(sf.coords, node.coords, }\AttributeTok{realign =} \ConstantTok{TRUE}\NormalTok{, }
                      \FunctionTok{colnames}\NormalTok{(arc.pa), }\AttributeTok{arc.symbol =} \StringTok{" {-}\textgreater{} "}\NormalTok{)}
\NormalTok{datetime }\OtherTok{\textless{}{-}}\NormalTok{ mnpa.sub}\SpecialCharTok{$}\NormalTok{Datetime}
\NormalTok{vsn }\OtherTok{\textless{}{-}} \FunctionTok{assign\_pa\_to\_segments}\NormalTok{(vs, frames, arc.pa, }\AttributeTok{datetime =}\NormalTok{ datetime)}
\end{Highlighting}
\end{Shaded}

Using the data summaries created from the steps above, I can finally create an animated ggplot map.

\vspace{0.1in}

\begin{Shaded}
\begin{Highlighting}[numbers=left,,]
\NormalTok{g }\OtherTok{\textless{}{-}} \FunctionTok{ggplot}\NormalTok{(mur\_sf) }\SpecialCharTok{+}
  \FunctionTok{geom\_sf}\NormalTok{(}\AttributeTok{colour =} \StringTok{"gray"}\NormalTok{, }\AttributeTok{lwd =} \FloatTok{1.8}\NormalTok{) }\SpecialCharTok{+}
  \FunctionTok{theme\_classic}\NormalTok{() }\SpecialCharTok{+}
  \FunctionTok{geom\_line}\NormalTok{(}\AttributeTok{data =}\NormalTok{ vsn, }\FunctionTok{aes}\NormalTok{(}\AttributeTok{x =}\NormalTok{ x, }\AttributeTok{y =}\NormalTok{ y, }\AttributeTok{group =}\NormalTok{ arc.label,}
                            \AttributeTok{colour =} \FunctionTok{as.factor}\NormalTok{(Presence)),}
              \AttributeTok{show.legend =} \ConstantTok{FALSE}\NormalTok{, }\AttributeTok{lwd =} \FloatTok{1.5}\NormalTok{) }\SpecialCharTok{+}
  \FunctionTok{scale\_colour\_manual}\NormalTok{(}\AttributeTok{values =} \FunctionTok{c}\NormalTok{(}\StringTok{"orange"}\NormalTok{,}\StringTok{"lightblue"}\NormalTok{)) }\SpecialCharTok{+}
  \FunctionTok{geom\_point}\NormalTok{(}\AttributeTok{data =}\NormalTok{ node.coords, }\FunctionTok{aes}\NormalTok{(}\AttributeTok{x =}\NormalTok{ E, }\AttributeTok{y =}\NormalTok{ N), }\AttributeTok{shape =} \DecValTok{21}\NormalTok{,}
             \AttributeTok{fill =} \StringTok{"white"}\NormalTok{, }\AttributeTok{size =} \FloatTok{1.4}\NormalTok{) }\SpecialCharTok{+}
  \FunctionTok{expand\_limits}\NormalTok{(}\AttributeTok{y =} \FunctionTok{c}\NormalTok{(}\DecValTok{4788562}\NormalTok{,}\DecValTok{4789700}\NormalTok{)) }\SpecialCharTok{+}
  \FunctionTok{annotation\_scale}\NormalTok{() }\SpecialCharTok{+}
  \FunctionTok{labs}\NormalTok{(}\AttributeTok{title =} \StringTok{"Date: \{frame\_time\}"}\NormalTok{, }\AttributeTok{x =} \StringTok{""}\NormalTok{, }\AttributeTok{y =} \StringTok{""}\NormalTok{) }\SpecialCharTok{+}
  \FunctionTok{annotation\_north\_arrow}\NormalTok{(}\AttributeTok{pad\_x =} \FunctionTok{unit}\NormalTok{(}\FloatTok{10.5}\NormalTok{, }\StringTok{"cm"}\NormalTok{), }
                         \AttributeTok{pad\_y =} \FunctionTok{unit}\NormalTok{(}\FloatTok{6.6}\NormalTok{, }\StringTok{"cm"}\NormalTok{)) }\SpecialCharTok{+}
  \FunctionTok{transition\_time}\NormalTok{(}\FunctionTok{as.Date}\NormalTok{(vsn}\SpecialCharTok{$}\NormalTok{Time))}
\end{Highlighting}
\end{Shaded}

The final result, Fig \ref{fig:muranim2}, shows changing patterns of surface water presence at the Murphy Creek network during the summer of 2019.

\begin{figure}[H]

{\centering \animategraphics[,controls,loop]{5}{figure/minimal-muranim2-}{1}{100}

}

\caption[Paterns of drying at Murphy Creek, Idaho shown with an animated map]{Paterns of drying at Murphy Creek, Idaho shown with an animated map. Blue segments indicate the presence of surface water. Gray segments indicate missing data.}\label{fig:muranim2}
\end{figure}

\end{example}

\(\blacksquare\)

\index[terms]{R!package!\textit{missForest}}
\index[R]{\texttt{streamDAG::STIC.RFimpute()}}
\index[R]{\texttt{missForest::missForest()}}
\index[R]{\texttt{streamDAG::assign\_pa\_to\_segments()}}
\index[R]{\texttt{sf::st\_coordinates())}}
\index[R]{\texttt{streamDAG::streamDAGs()}}
\index[R]{\texttt{streamDAG::arc.pa.from.nodes()}}
\index[R]{\texttt{streamDAG::vector\_segments()}}
\index[R]{\texttt{streamDAG::assign\_pa\_to\_segments()}}

\section*{Exercises}\label{exercises-6}
\addcontentsline{toc}{section}{Exercises}

\begin{enumerate}
\def\labelenumi{\arabic{enumi}.}
\tightlist
\item
  Complete the following data management steps on the \texttt{asbio::world.emissions} data.

  \begin{enumerate}
  \def\labelenumii{(\alph{enumii})}
  \tightlist
  \item
    Eliminate redundant rows using \texttt{continent\ !=\ \textquotesingle{}Redundant\textquotesingle{}} and \texttt{dplyr::filter}.
  \item
    Filter further to subset the data to the years 1955-2019.
  \item
    Filter further to subset the data to 8 countries of interest (your choice).
  \item
    Name the dataset \texttt{emissions.sub}.
  \item
    For the \texttt{emissions.sub} dataset, plot CO\(_2\) emissions as a function of year in a scatterplot. Save the ggplot as an object (e.g., \texttt{g}).
    \vspace{0.1in}
  \end{enumerate}
\item
  Continuing Question 1, overlay a linear regression model on \texttt{g} using \texttt{geom\_smooth(method\ =\ "lm")}.

  \begin{enumerate}
  \def\labelenumii{(\alph{enumii})}
  \tightlist
  \item
    Extract fitted model components using \texttt{g\ +\ stat\_poly\_eq(formula\ =\ y\ \textasciitilde{}\ x,\ geom\ =\ "debug")} from library \emph{gginnards}. What is the model slope?
  \item
    Interpret the meaning of the shaded envelope around the line.
  \item
    Annotate the model onto the graph using: \texttt{g\ +\ stat\_poly\_eq()} from library \emph{ggpmisc}.
    \vspace{0.1in}
  \end{enumerate}
\item
  Continuing Question 1, (1) color points in \texttt{g} by country (use transparency to allow viewing of points laying atop each other), and (2) vary point size by population size.
  \vspace{0.1in}
\item
  Continuing Question 1, add a label in \texttt{g} identifying US emissions in 2005.
  \vspace{0.1in}
\item
  Continuing Question 1, use \texttt{geom\_hline()} and/or \texttt{geom\_vline()} to add reference lines to \texttt{g} (your choice as to relevant \(x\) or \(y\)-axis location).
  \vspace{0.1in}
\item
  Continuing Question 1, alter the the \(y\)-axis limits in \texttt{g} (your choice of limits).
  \vspace{0.1in}
\item
  Continuing Question 1, create (1) a boxplot, and (2) an interval plot showing CO\(_2\) emissions as a function of country. Interpret the meaning of the hinges, centers, and whiskers of the boxplot and interpret the ``errors'' in the interval plot.
  \vspace{0.1in}
\item
  (Advanced) For the dataframe \texttt{npk} in the package \emph{datasets}, use functions in the the package \emph{ggpubr} to overlay results of pairwise comparisons of population means on interval plots. Specifically:

  \begin{enumerate}
  \def\labelenumii{(\alph{enumii})}
  \tightlist
  \item
    Use \texttt{ggbarplot()} to make a barplot showing mean Yields and standard errors as a function of nitrogen \texttt{N} and phosphate \texttt{P}. Vary bar colors using \texttt{P}. Create appropriate axis and legend labels.
  \item
    Overlay pairwise comparisons for both \texttt{N} and \texttt{P} levels on the barplot using the function \texttt{geom\_pwc()}. Specify \texttt{method\ =\ t\_test} since multiple tests for N will not occur within levels of P.
    \vspace{0.1in}
  \end{enumerate}
\item
  For the \texttt{asbio::goats} dataframe, use ggplot approaches to \(\dots\)

  \begin{enumerate}
  \def\labelenumii{(\alph{enumii})}
  \tightlist
  \item
    Make plots of the distribution of \texttt{NO3} using two of the following functions: \texttt{geom\_area()}, \texttt{geom\_freq()}, \texttt{geom\_dotplot()}, or \texttt{geom\_density()}.
  \item
    Create a scatterplot of \texttt{NO3} as a function of feces, Change symbol sizes to reflect the values in \texttt{organic.matter}.
  \item
    Plot \texttt{NO3} and \texttt{organic.matter} as simultaneous functions of \texttt{feces} by adding a second \emph{y}-axis.
    \vspace{0.1in}
  \end{enumerate}
\item
  Using \emph{gganimate}, and the \texttt{asbio::asthma} dataframe, track subject FEV1 levels over time with \texttt{geom\_point()}. Use faceting to distinguish drug levels.
\end{enumerate}

\chapter{Functions}\label{ch8}

\begin{quote}
\emph{``A computer will do what you tell it to do, but that may be much different from what you had in mind.''}
\end{quote}

- \textbf{Joseph Weizenbaum}, \emph{Important early software developer and AI ethisist}

\section{Introduction to Functions}\label{intro2functions}

In computer programming, a \emph{function} is a set of instructions for performing a specific task, or providing specific output. Essentially all processes in \textbf{R} are run via functions. For example, the command: \texttt{x\ \textless{}-\ 2}, assigns the label \texttt{x} to the numeric value 2. This is actually accomplished, however, via the function \texttt{\textasciigrave{}\textless{}-\textasciigrave{}}. That is, one could rewrite the expression \texttt{x\ \textless{}-\ 2} as:

\vspace{0.1in}

\begin{Shaded}
\begin{Highlighting}[]
\StringTok{\textasciigrave{}}\AttributeTok{\textless{}{-}}\StringTok{\textasciigrave{}}\NormalTok{(x, }\DecValTok{2}\NormalTok{)}
\NormalTok{x}
\end{Highlighting}
\end{Shaded}

\begin{verbatim}
[1] 2
\end{verbatim}

Similarly, summations are evaluated with the underlying function \texttt{\textasciigrave{}+\textasciigrave{}}.

\vspace{0.1in}

\begin{Shaded}
\begin{Highlighting}[]
\StringTok{\textasciigrave{}}\AttributeTok{+}\StringTok{\textasciigrave{}}\NormalTok{(}\DecValTok{2}\NormalTok{, }\DecValTok{2}\NormalTok{)}
\end{Highlighting}
\end{Shaded}

\begin{verbatim}
[1] 4
\end{verbatim}

Function call translations, for example, from \texttt{2\ +\ 2} to \texttt{\textasciigrave{}+\textasciigrave{}(2,\ 2)}, are made silently through the \textbf{R}-interpreter\footnote{\citet{chambers2008} describes function evaluation as a three step process: \emph{read} \(\rightarrow\) \emph{parse} \(\rightarrow\) \emph{evaluate}, and refers to the programmatic mechanisms underlying this process as the \textbf{R}-evaluator.}, which makes it unnecessary to compile \textbf{R} code into \emph{executable files} (see Ch \ref{ch9}).
\index[terms]{R!function}
\index[terms]{Function (computer algorithm)}
\index[R]{\texttt{function()}}
\index[terms]{R!interpreter}

\subsection{\texorpdfstring{\texttt{function()} and Function Base Types}{function() and Function Base Types}}\label{function-and-function-base-types}

Generally speaking, an \textbf{R} function --at the risk of sounding repetitive-- is a function defined by the function \texttt{function()} \large \smiley{} \normalsize. Arguments in \textbf{R} functions will be contained in a set of parentheses in the call to \texttt{function()} itself. The function contents follow, generally delineated by curly brackets. Thus, we have the form:

\texttt{function.name\ \textless{}-\ function(arg.1,\ arg.2,....,\ arg.n)\{function\ contents\}}

\index[terms]{R!classes!function}
Recall from Chapter \ref{ch2} that there are three \textbf{R} base types specific to functions: \texttt{closure}, \texttt{special}, and \texttt{builtin}. Functions of base types \texttt{special} and \texttt{builtin} are constrained to the \emph{base} package, and include \emph{primitive functions} built into the \textbf{R} system, and implemented in C. Types \texttt{builtin} and \texttt{special} can be distinguished as functions that \emph{do} and \emph{do not} evaluate their arguments, respectively \citep{team2024r}.

\begin{example}
\protect\hypertarget{exm:exm-fprim}{}\label{exm:exm-fprim}

The code below allows listing of \textbf{R} primitive functions \citep{chambers2008}. On Line 1, a character vector containing \emph{base} functions named \texttt{base.objs} is generated using the function \texttt{objects()}. Strings from \texttt{base.objs} are used to test if the functions are primitive by using \texttt{is.primitive()} as the \texttt{FUN} argument in \texttt{sapply()}. Boolean outcomes from the test are used to subset \texttt{base.objs} into the object \texttt{prim.objs}.\\
\(\text{}\)\\

\vspace{0.1in}

\begin{Shaded}
\begin{Highlighting}[numbers=left,,]
\NormalTok{base.objs }\OtherTok{\textless{}{-}} \FunctionTok{objects}\NormalTok{(}\StringTok{"package:base"}\NormalTok{, }\AttributeTok{all =} \ConstantTok{TRUE}\NormalTok{)}
\NormalTok{prim.objs }\OtherTok{\textless{}{-}}\NormalTok{ base.objs[}\FunctionTok{sapply}\NormalTok{(base.objs, }\ControlFlowTok{function}\NormalTok{(x) }\FunctionTok{is.primitive}\NormalTok{(}\FunctionTok{get}\NormalTok{(x)))]}
\end{Highlighting}
\end{Shaded}

The summarization is continued in the chunk below. On Line 1, the function \texttt{split()} used to split data in \texttt{prim.objs} into distinguishing base type categories using \texttt{typeof(get())} within \texttt{sapply()} (Lines 2 and 3). Numbers of items in these groups are tabulated using \texttt{sapply()} again (Line 5). There are currently 168 \texttt{builtin} and 42 \texttt{special} primitive functions in \textbf{R}.
\vspace{0.1in}

\begin{Shaded}
\begin{Highlighting}[numbers=left,,]
\NormalTok{base.types }\OtherTok{\textless{}{-}} \FunctionTok{split}\NormalTok{(prim.objs, }
                    \FunctionTok{sapply}\NormalTok{(prim.objs, }
                           \ControlFlowTok{function}\NormalTok{(x) }\FunctionTok{typeof}\NormalTok{(}\FunctionTok{get}\NormalTok{(x))))}

\FunctionTok{sapply}\NormalTok{(base.types, length)}
\end{Highlighting}
\end{Shaded}

\begin{verbatim}
builtin special 
    166      42 
\end{verbatim}

Here are the first 20 \texttt{special} primitive functions:
\vspace{0.1in}

\begin{Shaded}
\begin{Highlighting}[]
\NormalTok{base.types}\SpecialCharTok{$}\NormalTok{special[}\DecValTok{1}\SpecialCharTok{:}\DecValTok{20}\NormalTok{]}
\end{Highlighting}
\end{Shaded}

\begin{verbatim}
 [1] "$"         "$<-"       "&&"        ".Internal" "::"       
 [6] ":::"       "@"         "@<-"       "["         "[["       
[11] "[[<-"      "[<-"       "{"         "||"        "~"        
[16] "<-"        "<<-"       "="         "break"     "call"     
\end{verbatim}

Here are the first 20 \texttt{builtin} primitive functions:
\vspace{0.1in}

\begin{Shaded}
\begin{Highlighting}[]
\NormalTok{base.types}\SpecialCharTok{$}\NormalTok{builtin[}\DecValTok{1}\SpecialCharTok{:}\DecValTok{20}\NormalTok{]}
\end{Highlighting}
\end{Shaded}

\begin{verbatim}
 [1] "-"                  "!"                  "!="                
 [4] "%%"                 "%*%"                "%/%"               
 [7] "&"                  "("                  "*"                 
[10] "...elt"             "...length"          "...names"          
[13] ".C"                 ".cache_class"       ".Call"             
[16] ".Call.graphics"     ".class2"            ".External"         
[19] ".External.graphics" ".External2"        
\end{verbatim}

\vspace{0.1in}

Clearly, primitive functions of base type \texttt{special} and \texttt{builtin} include conventional operators (with bounding accent grave characters). For example, \texttt{\textasciigrave{}\$\textasciigrave{}} has base type \texttt{special},
\vspace{0.1in}

\begin{Shaded}
\begin{Highlighting}[]
\FunctionTok{typeof}\NormalTok{(}\StringTok{\textasciigrave{}}\AttributeTok{$}\StringTok{\textasciigrave{}}\NormalTok{)}
\end{Highlighting}
\end{Shaded}

\begin{verbatim}
[1] "special"
\end{verbatim}

\vspace{0.1in}

and \texttt{\textasciigrave{}+\textasciigrave{}} has base type \texttt{builtin}.\\
\vspace{0.1in}

\begin{Shaded}
\begin{Highlighting}[]
\FunctionTok{typeof}\NormalTok{(}\StringTok{\textasciigrave{}}\AttributeTok{+}\StringTok{\textasciigrave{}}\NormalTok{)}
\end{Highlighting}
\end{Shaded}

\begin{verbatim}
[1] "builtin"
\end{verbatim}

\end{example}

\(\blacksquare\)

Primitive functions generally make calls to the function \texttt{.Primitive()}, which identifies an underlying C routine used for evaluating the outer function. For example, we see that \texttt{\textasciigrave{}+\textasciigrave{}}, as codified in \textbf{R}, calls a C routine identified with \texttt{"+"}.
\vspace{0.1in}

\begin{Shaded}
\begin{Highlighting}[]
\StringTok{\textasciigrave{}}\AttributeTok{+}\StringTok{\textasciigrave{}}
\end{Highlighting}
\end{Shaded}

\begin{verbatim}
function (e1, e2)  .Primitive("+")
\end{verbatim}

\index[R]{\texttt{sapply()}}
\index[R]{\texttt{split()}}
\index[R]{\texttt{get()}}
\index[R]{\texttt{objects()}}
\index[R]{\texttt{is.primitive()}}
\index[R]{\texttt{.Primitive()}}
\index[terms]{R!base type!special}
\index[terms]{R!base type!builtin}

\subsection{\texorpdfstring{Base Type \texttt{closure}}{Base Type closure}}\label{base-type-closure}

\index[terms]{R!base type!closure}
Primitive functions (those of types \texttt{special} and \texttt{builtin}) cannot be created by users outside of the \textbf{R} development core team. Thus, base type \texttt{closure} represents the only kind of function \textbf{R}-users can actually create and easily modify. The name ``closure'' refers to the programming style underlying these functions, with each assigned to a particular environment with local internal objects (see Section 5.4 in \citet{chambers2008}). Consider the simple homemade function \texttt{square.me()}.

\vspace{0.1in}

\begin{Shaded}
\begin{Highlighting}[]
\NormalTok{square.me }\OtherTok{\textless{}{-}} \ControlFlowTok{function}\NormalTok{(x)\{}
\NormalTok{  x}\SpecialCharTok{\^{}}\DecValTok{2}
\NormalTok{\}}

\FunctionTok{square.me}\NormalTok{(}\DecValTok{4}\NormalTok{)}
\end{Highlighting}
\end{Shaded}

\begin{verbatim}
[1] 16
\end{verbatim}

\vspace{0.1in}

\begin{Shaded}
\begin{Highlighting}[]
\FunctionTok{typeof}\NormalTok{(square.me)}
\end{Highlighting}
\end{Shaded}

\begin{verbatim}
[1] "closure"
\end{verbatim}

Functions of base type \texttt{closure} will have three components.

\begin{itemize}
\tightlist
\item
  The \textbf{formal arguments} constitute the arguments that control the function. The formals can be accessed via the function \texttt{formals()}. For instance,
\end{itemize}

\index[R]{\texttt{formals()}}

\vspace{0.1in}

\begin{Shaded}
\begin{Highlighting}[]
\FunctionTok{formals}\NormalTok{(square.me)}
\end{Highlighting}
\end{Shaded}

\begin{verbatim}
$x
\end{verbatim}

The formals of a function will have a \texttt{pairlist} base type.
\index[terms]{R!base type!pairlist}

\vspace{0.1in}

\begin{Shaded}
\begin{Highlighting}[]
\FunctionTok{typeof}\NormalTok{(}\FunctionTok{formals}\NormalTok{(square.me))}
\end{Highlighting}
\end{Shaded}

\begin{verbatim}
[1] "pairlist"
\end{verbatim}

\begin{itemize}
\tightlist
\item
  The \textbf{body} constitutes the actual function code. The function \texttt{body()} returns the body of a function as an unevaluated expression.
\end{itemize}

\vspace{0.1in}

\begin{Shaded}
\begin{Highlighting}[]
\FunctionTok{body}\NormalTok{(square.me)}
\end{Highlighting}
\end{Shaded}

\begin{verbatim}
{
    x^2
}
\end{verbatim}

\index[R]{\texttt{body()}}

\begin{itemize}
\tightlist
\item
  The \textbf{environment} is a base type that defines the data structure the function requires for its computations. An environment is required for all \textbf{R} functions, whether they are \texttt{builtin}, \texttt{special}, or \texttt{closure}. When an object assignment (including the naming of a function) occurs at the ``top level'' in an \textbf{R} session (e.g., outside of the body of a function), its environment will be the global environment. The global environment is maintained throughout a session and can be saved across sessions using, for instance, the function \texttt{save.image()} (Section \ref{saveRobjects}). Environments of functions can be checked, created, or changed using the function \texttt{environment()}.
  \index[R]{\texttt{save.image()}}
  \index[terms]{R!base type!environment}
  \index[R]{\texttt{environment()}}
  The environment of \texttt{square.me()} is the global environment.
  \vspace{0.1in}
\end{itemize}

\begin{Shaded}
\begin{Highlighting}[]
\FunctionTok{environment}\NormalTok{(square.me)}
\end{Highlighting}
\end{Shaded}

\begin{verbatim}
<environment: R_GlobalEnv>
\end{verbatim}

As is the the environment of the session itself.
\vspace{0.1in}

\begin{Shaded}
\begin{Highlighting}[]
\FunctionTok{environment}\NormalTok{()}
\end{Highlighting}
\end{Shaded}

\begin{verbatim}
<environment: R_GlobalEnv>
\end{verbatim}

In contrast, the environment for the function \texttt{mean()} is the \emph{base} package.
\vspace{0.1in}

\begin{Shaded}
\begin{Highlighting}[]
\FunctionTok{environment}\NormalTok{(mean)}
\end{Highlighting}
\end{Shaded}

\begin{verbatim}
<environment: namespace:base>
\end{verbatim}

Functions in \emph{base}, including \texttt{mean()}, are accessible, because the \emph{base} package namespace is loaded automatically (along with most of the \textbf{R} distribution packages) upon opening \textbf{R}.

\begin{Shaded}
\begin{Highlighting}[]
\FunctionTok{isNamespaceLoaded}\NormalTok{(}\StringTok{"base"}\NormalTok{)}
\end{Highlighting}
\end{Shaded}

\begin{verbatim}
[1] TRUE
\end{verbatim}

\index[R]{\texttt{mean()}}

\begin{example}
\protect\hypertarget{exm:exm-f1}{}\label{exm:exm-f1}

\(\text{}\)\\
As a biological example, we will create a function for calculating predicted sizes of biological populations under geometric growth. The geometric growth equation (Eq. \eqref{eq:905}) is often used to represent population growth for a species with unlimited resources and non-overlapping generations:

\begin{equation} 
f(t) = N_0\lambda^t,
\label{eq:905}
\end{equation}

where: \(N_0 =\) initial number of individuals, \(\lambda =\) the geometric rate of increase, and \(t\) = the number of time intervals or generations. We have:

\vspace{0.1in}

\begin{Shaded}
\begin{Highlighting}[]
\NormalTok{Geo.growth }\OtherTok{\textless{}{-}} \ControlFlowTok{function}\NormalTok{(N}\FloatTok{.0}\NormalTok{, lambda, t)\{}
\NormalTok{  Nt }\OtherTok{\textless{}{-}}\NormalTok{ N}\FloatTok{.0} \SpecialCharTok{*}\NormalTok{ lambda}\SpecialCharTok{\^{}}\NormalTok{t}
\NormalTok{  Nt}
\NormalTok{\}}
\end{Highlighting}
\end{Shaded}

Note that the function has three arguments: \texttt{N.0}, \texttt{lambda}, and \texttt{t}.

A function-user must specify each of these. The first line of code in the function body solves \(N_0\lambda^t\). Importantly, the second (last) line of body code specifies the object we actually want returned, \texttt{Nt}. Without a ``return value'' nothing will be returned by the function. If one requires multiple return objects, then one can place them in single suitable container like a list.

To increase clarity, one should place the first curly bracket on same line as the arguments, and place last curly bracket on its own line. Readability can also be improved with the use of tabs and spaces. Note that I have indented lines containing related operations. This distinguishes those lines from the first (argument) line and the end (return) line. Note also that spaces are placed after commas, and before and after operators, including the assignment operator. This is also good general practice for code writing\footnote{A good \textbf{R} style guide can be found at: (\url{https://google.github.io/styleguide/Rguide.html}).}.

Below we run the function for different values of \texttt{N.0}, \texttt{lambda}, and \texttt{t}.

\vspace{0.1in}

\begin{Shaded}
\begin{Highlighting}[]
\FunctionTok{Geo.growth}\NormalTok{(}\AttributeTok{N.0 =} \DecValTok{100}\NormalTok{, }\AttributeTok{lambda =} \FloatTok{1.2}\NormalTok{, }\AttributeTok{t =} \DecValTok{20}\NormalTok{)}
\end{Highlighting}
\end{Shaded}

\begin{verbatim}
[1] 3833.8
\end{verbatim}

\vspace{0.1in}

\begin{Shaded}
\begin{Highlighting}[]
\FunctionTok{Geo.growth}\NormalTok{(}\AttributeTok{N.0 =} \DecValTok{30}\NormalTok{, }\AttributeTok{lambda =} \FloatTok{0.2}\NormalTok{, }\AttributeTok{t =} \DecValTok{3}\NormalTok{)}
\end{Highlighting}
\end{Shaded}

\begin{verbatim}
[1] 0.24
\end{verbatim}

\vspace{0.1in}

\begin{Shaded}
\begin{Highlighting}[]
\FunctionTok{Geo.growth}\NormalTok{(}\AttributeTok{N.0 =} \DecValTok{30}\NormalTok{, }\AttributeTok{lambda =} \DecValTok{1}\NormalTok{, }\AttributeTok{t =} \DecValTok{3}\NormalTok{)}
\end{Highlighting}
\end{Shaded}

\begin{verbatim}
[1] 30
\end{verbatim}

\end{example}

\(\blacksquare\)

\section{Global vs.~Local Variables}\label{global-vs.-local-variables}

As noted in Ch \ref{ch1}, objects in \textbf{R} are lexically scoped\index[terms]{Lexical scoping}, allowing distinctions of global and local variables. \emph{Global variables}\index[terms]{Global variable} are objects that exist within the global environment and consequently are broadly accessible, whereas \emph{local variables}\index[terms]{Local variable} are only accessible in particular settings. Objects defined within a function, including arguments, are (generally) local to that function, and thus are accessible only within the body of the function.

We see that the object \texttt{N.t}, which was defined in the last line of \texttt{Geo.growth()}, is local to that function, since it cannot be detected in the global environment\footnote{Commonly reported errors for functions using \emph{tidyverse} code are given at the \href{https://cran.r-project.org/web/packages/future/vignettes/future-4-issues.html}{A future for R} website.}.

\vspace{0.1in}

\begin{Shaded}
\begin{Highlighting}[]
\NormalTok{Nt}
\end{Highlighting}
\end{Shaded}

\begin{verbatim}
Error: object 'Nt' not found
\end{verbatim}

Global variables can be assigned in functions using the super-assignment operator, \texttt{\textless{}\textless{}-}, although I have found the need for this operator to be rare (but see Section \ref{asswidget}).
\index[R]{*aos@\texttt{<<-}}

\vspace{0.1in}

\begin{Shaded}
\begin{Highlighting}[]
\NormalTok{Geo.growth }\OtherTok{\textless{}{-}} \ControlFlowTok{function}\NormalTok{(N}\FloatTok{.0}\NormalTok{, lambda,t)\{}
\NormalTok{  Nt }\OtherTok{\textless{}\textless{}{-}}\NormalTok{ N}\FloatTok{.0} \SpecialCharTok{*}\NormalTok{ lambda}\SpecialCharTok{\^{}}\NormalTok{t}
\NormalTok{  Nt}
\NormalTok{\}}

\NormalTok{g }\OtherTok{\textless{}{-}} \FunctionTok{Geo.growth}\NormalTok{(}\AttributeTok{N.0 =} \DecValTok{30}\NormalTok{, }\AttributeTok{lambda =} \DecValTok{1}\NormalTok{, }\AttributeTok{t =} \DecValTok{3}\NormalTok{)}
\NormalTok{Nt}
\end{Highlighting}
\end{Shaded}

\begin{verbatim}
[1] 30
\end{verbatim}

\begin{example}
\protect\hypertarget{exm:exm-f2}{}\label{exm:exm-f2}

\(\text{}\)\\
The \texttt{apply} family of functions for data management, including \texttt{apply()}, \texttt{tapply()}, \texttt{sapply()} and \texttt{lapply()} (Section \ref{apply}) allow inclusion of user-defined functions (see Example \ref{exm:exm-fprim} from earlier in this chapter). The function \texttt{stan()} below centers and scales (standardizes) outcomes. That is, each element in the dataset is subtracted from its mean, and divided by its standard deviation). We can call \texttt{stan()} within \texttt{apply()}, using the latter function's third (\texttt{FUN}) argument.

\vspace{0.1in}

\begin{Shaded}
\begin{Highlighting}[]
\NormalTok{stan }\OtherTok{\textless{}{-}} \ControlFlowTok{function}\NormalTok{(x)\{}
\NormalTok{  (x }\SpecialCharTok{{-}} \FunctionTok{mean}\NormalTok{(x))}\SpecialCharTok{/}\FunctionTok{sd}\NormalTok{(x)}
\NormalTok{\}}

\NormalTok{out }\OtherTok{\textless{}{-}} \FunctionTok{apply}\NormalTok{(Loblolly[,}\DecValTok{1}\SpecialCharTok{:}\DecValTok{2}\NormalTok{], }\DecValTok{2}\NormalTok{, stan)}
\end{Highlighting}
\end{Shaded}

As a consequence of the transformation, columns in the object \texttt{out} will the same mean (zero), and the same variance (one)

\vspace{0.1in}

\begin{Shaded}
\begin{Highlighting}[]
\FunctionTok{apply}\NormalTok{(out, }\DecValTok{2}\NormalTok{, mean) }\CommentTok{\# zero with rounding error}
\end{Highlighting}
\end{Shaded}

\begin{verbatim}
    height        age 
1.6687e-16 1.8508e-17 
\end{verbatim}

\vspace{0.1in}

\begin{Shaded}
\begin{Highlighting}[]
\FunctionTok{apply}\NormalTok{(out, }\DecValTok{2}\NormalTok{, var)}
\end{Highlighting}
\end{Shaded}

\begin{verbatim}
height    age 
     1      1 
\end{verbatim}

\end{example}

\(\blacksquare\)

\begin{example}
\protect\hypertarget{exm:exm-f3}{}\label{exm:exm-f3}

\(\text{}\)\\
Below is a function called \texttt{stats()} that will simultaneously calculate a large number of distinct summary statistics.

\vspace{0.1in}

\begin{Shaded}
\begin{Highlighting}[numbers=left,,]
\NormalTok{stats }\OtherTok{\textless{}{-}} \ControlFlowTok{function}\NormalTok{(x, }\AttributeTok{digits =} \DecValTok{5}\NormalTok{)\{}
  \FunctionTok{require}\NormalTok{(asbio)}
\NormalTok{  ds }\OtherTok{\textless{}{-}} \FunctionTok{data.frame}\NormalTok{(}\AttributeTok{statistics =} \FunctionTok{round}\NormalTok{(}\FunctionTok{c}\NormalTok{(}\FunctionTok{length}\NormalTok{(x), }\FunctionTok{min}\NormalTok{(x), }\FunctionTok{max}\NormalTok{(x),}
                                        \FunctionTok{mean}\NormalTok{(x), }\FunctionTok{median}\NormalTok{(x), }\FunctionTok{sd}\NormalTok{(x), }\FunctionTok{var}\NormalTok{(x),}
                                        \FunctionTok{IQR}\NormalTok{(x), }\FunctionTok{sd}\NormalTok{(x)}\SpecialCharTok{/}\FunctionTok{sqrt}\NormalTok{(}\FunctionTok{length}\NormalTok{(x)),}
                                        \FunctionTok{kurt}\NormalTok{(x), }\FunctionTok{skew}\NormalTok{(x)), digits))}
  \FunctionTok{rownames}\NormalTok{(ds) }\OtherTok{\textless{}{-}} \FunctionTok{c}\NormalTok{(}\StringTok{"n"}\NormalTok{, }\StringTok{"min"}\NormalTok{, }\StringTok{"max"}\NormalTok{, }\StringTok{"mean"}\NormalTok{, }\StringTok{"median"}\NormalTok{, }\StringTok{"sd"}\NormalTok{,}
                    \StringTok{"var"}\NormalTok{, }\StringTok{"IQR"}\NormalTok{, }\StringTok{"SE"}\NormalTok{, }\StringTok{"kurtosis"}\NormalTok{, }\StringTok{"skew"}\NormalTok{)}
\NormalTok{  ds}
\NormalTok{\}}
\end{Highlighting}
\end{Shaded}

Note that the function contains two arguments (Line one): a call to a numeric data vector, \texttt{x}, and the number of significant digits to be used in printing the output. Because \texttt{digits} has been given a default value (\texttt{digits} = \texttt{5}), only the first argument needs to be specified by the user. The first line of code in the body of the function (Line 2) indicates that package \texttt{asbio} is required by the function (the package contains the functions \texttt{asbio::skew()} and \texttt{asbio::kurt()} for calculating the data skew and kurtosis, respectively.
\index[R]{\texttt{asbio::skew()}}
\index[R]{\texttt{asbio::kurt()}}
In Lines 3-8 a dataframe is created called \texttt{ds}. The dataframe has one
column called \texttt{statistics}, that will contain numeric entries for eleven statistical summaries of \texttt{x}. The summaries are rounded to the number of digits specified in \texttt{digits}.
Lines 7-8 define the row names of \texttt{ds}. These are the names of the statistics calculated by the function. The last line of code in the body (Line 9) prints \texttt{ds}.

We can readily apply \texttt{stats()} directly to a single numeric column.

\vspace{0.1in}

\begin{Shaded}
\begin{Highlighting}[]
\FunctionTok{stats}\NormalTok{(Loblolly[,}\DecValTok{1}\NormalTok{])}
\end{Highlighting}
\end{Shaded}

\begin{verbatim}
         statistics
n          84.00000
min         3.46000
max        64.10000
mean       32.36440
median     34.00000
sd         20.67360
var       427.39793
IQR        40.89500
SE          2.25568
kurtosis   -1.47347
skew       -0.06434
\end{verbatim}

Or apply the function to multiple columns, for instance, by calling \texttt{stats()} within \texttt{apply()}. For instance:

\vspace{0.1in}

\begin{Shaded}
\begin{Highlighting}[]
\FunctionTok{apply}\NormalTok{(Loblolly[,}\FunctionTok{c}\NormalTok{(}\DecValTok{1}\SpecialCharTok{:}\DecValTok{2}\NormalTok{)],}\DecValTok{2}\NormalTok{,stats)}
\end{Highlighting}
\end{Shaded}

\begin{verbatim}
$height
         statistics
n          84.00000
min         3.46000
max        64.10000
mean       32.36440
median     34.00000
sd         20.67360
var       427.39793
IQR        40.89500
SE          2.25568
kurtosis   -1.47347
skew       -0.06434

$age
         statistics
n          84.00000
min         3.00000
max        25.00000
mean       13.00000
median     12.50000
sd          7.89998
var        62.40964
IQR        15.00000
SE          0.86196
kurtosis   -1.37375
skew        0.18925
\end{verbatim}

\end{example}

\(\blacksquare\)

\section{Useful Functions for Writing Functions}\label{useful-functions-for-writing-functions}

\subsection{\texorpdfstring{\texttt{switch()}}{switch()}}\label{swtch}

A useful tool for function writing is \texttt{switch()}. It evaluates and switches among user-designated alternatives which can be defined in a function argument.

\begin{example}
\protect\hypertarget{exm:exm-f4}{}\label{exm:exm-f4}\(\text{}\)\\
The function below switches between five different estimators of location (i.e., estimators of a typical or central value from a sample). These are the sample mean, a trimmed mean (using 10\% trimming), the geometric mean, the median, and Huber's \(M\)-estimator. See Chapter 4 in \citet{aho2014} for details concerning these estimators.
\index[R]{\texttt{switch()}}
\vspace{0.1in}

\begin{Shaded}
\begin{Highlighting}[numbers=left,,]
\NormalTok{location }\OtherTok{\textless{}{-}} \ControlFlowTok{function}\NormalTok{(x, estimator)\{}
  \FunctionTok{require}\NormalTok{(asbio)}
  \ControlFlowTok{switch}\NormalTok{(estimator,}
    \AttributeTok{mean =} \FunctionTok{mean}\NormalTok{(x), }\CommentTok{\# arithmetic mean}
    \AttributeTok{trim =} \FunctionTok{mean}\NormalTok{(x, }\AttributeTok{trim =} \FloatTok{0.1}\NormalTok{), }\CommentTok{\# trimmed mean}
    \AttributeTok{geo =} \FunctionTok{exp}\NormalTok{(}\FunctionTok{mean}\NormalTok{(}\FunctionTok{log}\NormalTok{(x))), }\CommentTok{\# geometric mean (use for means of rates)}
    \AttributeTok{med =} \FunctionTok{median}\NormalTok{(x), }\CommentTok{\# median}
    \AttributeTok{huber =} \FunctionTok{huber.mu}\NormalTok{(x), }\CommentTok{\# Huber M{-}estimator}
    \FunctionTok{stop}\NormalTok{(}\StringTok{"Estimator not included"}\NormalTok{))}
\NormalTok{\}}
\end{Highlighting}
\end{Shaded}

\vspace{0.1in}

\begin{Shaded}
\begin{Highlighting}[]
\FunctionTok{location}\NormalTok{(Loblolly[,}\DecValTok{2}\NormalTok{], }\StringTok{"geo"}\NormalTok{)}
\end{Highlighting}
\end{Shaded}

\begin{verbatim}
[1] 10.198
\end{verbatim}

\vspace{0.1in}

\begin{Shaded}
\begin{Highlighting}[]
\FunctionTok{location}\NormalTok{(Loblolly[,}\DecValTok{2}\NormalTok{], }\StringTok{"trim"}\NormalTok{)}
\end{Highlighting}
\end{Shaded}

\begin{verbatim}
[1] 12.765
\end{verbatim}

Important to the function above is the pairing of the \texttt{estimator} argument in the overall function (Line 1) and a call to \texttt{estimator} in the first argument of \texttt{switch} (Line 3). As a final component of \texttt{switch} we address the contingency that a location estimator is specified that is not codified in the function. This is done using the \texttt{stop()} function with an appropriate message (Line 9).
\end{example}

\(\blacksquare\)
\index[R]{\texttt{stop()}}

\subsection{\texorpdfstring{\texttt{match.arg()}}{match.arg()}}\label{match.arg}

It is possible to specify partial matching for argument designations using the function \texttt{match.arg()}.

\begin{example}
\protect\hypertarget{exm:exm-f5}{}\label{exm:exm-f5}

\(\text{}\)\\
For instance, what if we knew (or only wanted to specify) the first couple letters in the location estimator options for the previous function, \texttt{location()}?
We could specify a step like the following.

\vspace{0.1in}

\begin{Shaded}
\begin{Highlighting}[]
\NormalTok{indices }\OtherTok{\textless{}{-}} \FunctionTok{c}\NormalTok{(}\StringTok{"mean"}\NormalTok{, }\StringTok{"trim"}\NormalTok{, }\StringTok{"geo"}\NormalTok{, }\StringTok{"median"}\NormalTok{, }\StringTok{"huber"}\NormalTok{)}
\NormalTok{method }\OtherTok{\textless{}{-}} \FunctionTok{match.arg}\NormalTok{(estimator, indices)}
\end{Highlighting}
\end{Shaded}

\index[R]{\texttt{match.arg()}}
This is incorporated into \texttt{location()} on Lines 3-4 below. Note also the change in the first argument of \texttt{switch} from \texttt{estimator} (Line 5), which may have incomplete spelling of a location estimator name, to \texttt{method}, which will contain the complete index names from \texttt{index}.

\vspace{0.1in}

\begin{Shaded}
\begin{Highlighting}[numbers=left,,]
\NormalTok{location }\OtherTok{\textless{}{-}} \ControlFlowTok{function}\NormalTok{(x, estimator)\{}
  \FunctionTok{require}\NormalTok{(asbio)}
\NormalTok{  indices }\OtherTok{\textless{}{-}} \FunctionTok{c}\NormalTok{(}\StringTok{"mean"}\NormalTok{, }\StringTok{"trim"}\NormalTok{, }\StringTok{"geo"}\NormalTok{, }\StringTok{"median"}\NormalTok{, }\StringTok{"huber"}\NormalTok{)}
\NormalTok{  method }\OtherTok{\textless{}{-}} \FunctionTok{match.arg}\NormalTok{(estimator, indices)}
  \ControlFlowTok{switch}\NormalTok{(method,}
    \AttributeTok{mean =} \FunctionTok{mean}\NormalTok{(x), }\CommentTok{\# arithmetic mean}
    \AttributeTok{trim =} \FunctionTok{mean}\NormalTok{(x, }\AttributeTok{trim =} \FloatTok{0.1}\NormalTok{), }\CommentTok{\# trimmed mean}
    \AttributeTok{geo =} \FunctionTok{exp}\NormalTok{(}\FunctionTok{mean}\NormalTok{(}\FunctionTok{log}\NormalTok{(x))), }\CommentTok{\# geometric mean (use for means of rates)}
    \AttributeTok{med =} \FunctionTok{median}\NormalTok{(x), }\CommentTok{\# median}
    \AttributeTok{huber =} \FunctionTok{huber.mu}\NormalTok{(x), }\CommentTok{\# Huber M{-}estimator}
    \FunctionTok{stop}\NormalTok{(}\StringTok{"Estimator not included"}\NormalTok{))}
\NormalTok{\}}
\end{Highlighting}
\end{Shaded}

Now we could do something like:

\vspace{0.1in}

\begin{Shaded}
\begin{Highlighting}[]
\FunctionTok{location}\NormalTok{(Loblolly[,}\DecValTok{2}\NormalTok{],}\StringTok{"t"}\NormalTok{)}
\end{Highlighting}
\end{Shaded}

\begin{verbatim}
[1] 12.765
\end{verbatim}

\begin{Shaded}
\begin{Highlighting}[]
\FunctionTok{location}\NormalTok{(Loblolly[,}\DecValTok{2}\NormalTok{],}\StringTok{"h"}\NormalTok{)}
\end{Highlighting}
\end{Shaded}

\begin{verbatim}
[1] 13
\end{verbatim}

\end{example}

\(\blacksquare\)
\index[R]{\texttt{stop()}}

\subsection{\texorpdfstring{\texttt{...}}{...}}\label{section}

The triple dot (\texttt{...}) operator\footnote{This operator is not the same as the C-internal \texttt{...} base type (Section \ref{bt}).} allows customization of existing functions within another function. Thus, it is useful for writing \emph{wrapper functions} (functions whose chief purpose is customization of an embedded function).

\index[terms]{Function (computer algorithm)!wrapper function}

\begin{example}
\protect\hypertarget{exm:exm-f6}{}\label{exm:exm-f6}

\(\text{}\)\\
Imagine you wished to create a wrapper for the function \texttt{plot()} that allowed simultaneous computation and customized plotting of a simple linear regression model. We could do something like:

\vspace{0.1in}

\begin{Shaded}
\begin{Highlighting}[numbers=left,,]
\NormalTok{plot.reg }\OtherTok{\textless{}{-}} \ControlFlowTok{function}\NormalTok{(x, y, ...)\{}
\NormalTok{  reg }\OtherTok{\textless{}{-}} \FunctionTok{lm}\NormalTok{(y }\SpecialCharTok{\textasciitilde{}}\NormalTok{ x)}
  \FunctionTok{plot}\NormalTok{(x, y, ...)}
  \FunctionTok{abline}\NormalTok{(reg)}
\NormalTok{\}}
\end{Highlighting}
\end{Shaded}

The first two formal arguments \texttt{x} and \texttt{y} on line Line 1, establish plotting coordinates of points, and define the outcomes for the explanatory and response variables, respectively. The third argument is the triple dot operator (Line 1). In the first line in the body of the function (Line 2) we create a general linear regression model using the function \texttt{lm()}. Line 3 creates a plot and, importantly, calls the triple dot operator from the arguments in \texttt{plot.reg()}. This allows specification of any possible \texttt{plot()} arguments, as arguments within \texttt{plot.reg()}. For instance, in the usage of \texttt{plot.reg()} below, I specify the \emph{x} and \emph{y} axis labels, a plotting character type, and symbols colors (Fig \ref{fig:trpdots}). The last line of code (Line 4) plots the regression line.

\vspace{0.1in}

\begin{Shaded}
\begin{Highlighting}[]
\FunctionTok{with}\NormalTok{(Loblolly, }\FunctionTok{plot.reg}\NormalTok{(age, height, }\AttributeTok{pch =} \DecValTok{19}\NormalTok{, }\AttributeTok{col =} \FunctionTok{as.numeric}\NormalTok{(Seed),}
         \AttributeTok{ylab =} \StringTok{"Height (ft)"}\NormalTok{, }\AttributeTok{xlab =} \StringTok{"Age (yrs)"}\NormalTok{))}
\end{Highlighting}
\end{Shaded}

\begin{figure}[H]

{\centering \includegraphics{figure/minimal-trpdots-1} 

}

\caption{Representation of loblolly pine tree height as a function of age.  Regression fit overlaid. Seed types are distinguished with colors.}\label{fig:trpdots}
\end{figure}

\end{example}

\(\blacksquare\)

\index[R]{\texttt{lm()}}
\index[R]{\texttt{...}}

\subsection{\texorpdfstring{\texttt{invisible()}}{invisible()}}\label{invisible}

The \texttt{invisible()} function can be useful when one wishes to have results computed and saved but not necessarily printed.

\begin{example}
\protect\hypertarget{exm:exm-f7}{}\label{exm:exm-f7}

\(\text{}\)\\
Assume that we want to retain \texttt{plot.reg()} as a plotting function, but wish to have potential access to actual statistical summaries from the regression model. We could rewrite \texttt{plot.reg()} as:

\vspace{0.1in}

\begin{Shaded}
\begin{Highlighting}[numbers=left,,]
\NormalTok{plot.reg }\OtherTok{\textless{}{-}} \ControlFlowTok{function}\NormalTok{(x, y, }\AttributeTok{plot =} \ConstantTok{TRUE}\NormalTok{, ...)\{}
\NormalTok{  reg }\OtherTok{\textless{}{-}} \FunctionTok{lm}\NormalTok{(y }\SpecialCharTok{\textasciitilde{}}\NormalTok{ x)}
  \ControlFlowTok{if}\NormalTok{(plot)\{ }\FunctionTok{plot}\NormalTok{(x, y, ...)}
  \FunctionTok{abline}\NormalTok{(}\FunctionTok{coef}\NormalTok{(reg))\}}
  \FunctionTok{invisible}\NormalTok{(}\FunctionTok{summary}\NormalTok{(reg))}
\NormalTok{\}}
\end{Highlighting}
\end{Shaded}

Note that I have added an argument, \texttt{plot} (Line 1), to control whether a plot is created via \texttt{if(plot)} (Line 3). By suppressing plotting I get no graphics (or text) output:

\vspace{0.1in}

\begin{Shaded}
\begin{Highlighting}[]
\FunctionTok{with}\NormalTok{(Loblolly, }\FunctionTok{plot.reg}\NormalTok{(age, height, }\AttributeTok{plot =} \ConstantTok{FALSE}\NormalTok{))}
\end{Highlighting}
\end{Shaded}

However, if I assign a name to the function's output, and print the assigned object, I get summary output for the regression model:

\vspace{0.1in}

\begin{Shaded}
\begin{Highlighting}[]
\NormalTok{lob.model }\OtherTok{\textless{}{-}} \FunctionTok{with}\NormalTok{(Loblolly, }\FunctionTok{plot.reg}\NormalTok{(age, height, }\AttributeTok{plot =} \ConstantTok{FALSE}\NormalTok{))}
\NormalTok{lob.model}
\end{Highlighting}
\end{Shaded}

\begin{verbatim}

Call:
lm(formula = y ~ x)

Residuals:
   Min     1Q Median     3Q    Max 
-7.021 -2.167 -0.439  2.054  6.855 

Coefficients:
            Estimate Std. Error t value Pr(>|t|)    
(Intercept)  -1.3124     0.6218   -2.11    0.038 *  
x             2.5905     0.0409   63.27   <2e-16 ***
---
Signif. codes:  0 '***' 0.001 '**' 0.01 '*' 0.05 '.' 0.1 ' ' 1

Residual standard error: 2.95 on 82 degrees of freedom
Multiple R-squared:  0.98,  Adjusted R-squared:  0.98 
F-statistic: 4e+03 on 1 and 82 DF,  p-value: <2e-16
\end{verbatim}

\end{example}

\(\blacksquare\)

\section{Loops}\label{loops}

Loop functions exist in some form in virtually all programming languages. A ``for loop'' in \textbf{R} is initiated using the function \texttt{for()}. The \texttt{for} construct requires a specification of three entities
\vspace{0.1in}

\begin{itemize}
\tightlist
\item
  An index variable, e.g., \texttt{i},
  \vspace{0.05in}
\item
  The statement \texttt{in}
  \vspace{0.05in}
\item
  A sequence that the index variable refers to as the loop commences.
\end{itemize}

Code defining the loop follows, generally delineated by curly brackets. In parallel to function writing it is good style to place the first curly bracket on the same line as the call to for, and to place the last curly bracket on its own line. Thus, we have the basic \texttt{for} loop format:

\vspace{0.1in}

\begin{Shaded}
\begin{Highlighting}[]
\ControlFlowTok{for}\NormalTok{(i }\ControlFlowTok{in}\NormalTok{ seq)\{}
\NormalTok{  loop contents}
\NormalTok{\}}
\end{Highlighting}
\end{Shaded}

\index[R]{\texttt{for()}}
\index[terms]{Loop (programming)}
In the loop, values of index are used directly or indirectly to specify the \(i\)th element of something (e.g., matrix column, vector entry, etc.) as the for loop sequence commences. The replacement/definition process takes place in the ``loop contents.'' For instance,

\vspace{0.1in}

\begin{Shaded}
\begin{Highlighting}[]
\ControlFlowTok{for}\NormalTok{(i }\ControlFlowTok{in} \DecValTok{1}\SpecialCharTok{:}\DecValTok{3}\NormalTok{)\{}
  \FunctionTok{print}\NormalTok{(i)}
\NormalTok{\}}
\end{Highlighting}
\end{Shaded}

\begin{verbatim}
[1] 1
[1] 2
[1] 3
\end{verbatim}

\subsection{Extending Scalar Arguments}\label{extending-scalar-arguments}

One application for a loop is to make functions with scalar input arguments amenable to multi-element vector, matrix or dataframe inputs.

\begin{example}
\protect\hypertarget{exm:exm-f8}{}\label{exm:exm-f8}

\(\text{}\)\\
A library I created, \emph{plant.ecol}, lives only on Github. We can access it with the code:
\vspace{0.1in}

\begin{Shaded}
\begin{Highlighting}[]
\FunctionTok{library}\NormalTok{(devtools)}
\FunctionTok{install\_github}\NormalTok{(}\StringTok{"moondog1969/plant.ecol"}\NormalTok{)}
\FunctionTok{library}\NormalTok{(plant.ecol)}
\end{Highlighting}
\end{Shaded}

The package \emph{devtools} facilities building \textbf{R} packages from source code, and contains functions, e.g., \texttt{install\_github()}, for downloading packages from unconventional repositories\footnote{Functions from the \emph{tidyverse} package \emph{usethis} will be useful for setting up passwords and tokens necessary for downloading from some repositories, including Github.}. The function \texttt{plant.ecol::radiation.heatl()} calculates annual incident solar radiation (MJ cm\(^{-2}\) yr\(^{-1}\)) and heatload, a northern hemisphere thermal index that acknowledges that highest levels of heat loading occur on southwest facing slopes north of the equator. The function requires three arguments for some location of interest: slope, aspect, and north latitude, all measured in degrees.
\index[terms]{R!package!\textit{usethis}}
\vspace{0.1in}

\begin{Shaded}
\begin{Highlighting}[]
\FunctionTok{formals}\NormalTok{(radiation.heatl)}
\end{Highlighting}
\end{Shaded}

\begin{verbatim}
$slope


$aspect


$lat
\end{verbatim}

The function is designed to accommodate scalar inputs (single values for slope, aspect and lat). We will create a for loop to allow calculation of multiple values from a matrix. For instance, here are potential values for five sites.

\vspace{0.1in}

\begin{Shaded}
\begin{Highlighting}[]
\NormalTok{envdata }\OtherTok{\textless{}{-}} \FunctionTok{data.frame}\NormalTok{(}\AttributeTok{slope =} \FunctionTok{c}\NormalTok{(}\DecValTok{10}\NormalTok{, }\DecValTok{12}\NormalTok{, }\DecValTok{15}\NormalTok{, }\DecValTok{20}\NormalTok{, }\DecValTok{3}\NormalTok{),}
                \AttributeTok{aspect =} \FunctionTok{c}\NormalTok{(}\DecValTok{148}\NormalTok{, }\DecValTok{110}\NormalTok{, }\DecValTok{0}\NormalTok{, }\DecValTok{30}\NormalTok{, }\DecValTok{130}\NormalTok{),}
                \AttributeTok{latitude =} \FunctionTok{c}\NormalTok{(}\DecValTok{40}\NormalTok{, }\DecValTok{50}\NormalTok{, }\DecValTok{20}\NormalTok{, }\DecValTok{25}\NormalTok{, }\DecValTok{45}\NormalTok{))}
\end{Highlighting}
\end{Shaded}

We first create storage containers for the radiation and heatload results.

\vspace{0.1in}

\begin{Shaded}
\begin{Highlighting}[]
\NormalTok{rad.out }\OtherTok{\textless{}{-}} \DecValTok{1}\SpecialCharTok{:}\FunctionTok{nrow}\NormalTok{(envdata) }\OtherTok{{-}\textgreater{}}\NormalTok{ hl.out}
\end{Highlighting}
\end{Shaded}

Here is a potential loop for our problem.

\vspace{0.1in}

\begin{Shaded}
\begin{Highlighting}[numbers=left,,]
\ControlFlowTok{for}\NormalTok{(i }\ControlFlowTok{in} \DecValTok{1}\SpecialCharTok{:}\FunctionTok{length}\NormalTok{(rad.out))\{}
\NormalTok{  temp }\OtherTok{\textless{}{-}} \FunctionTok{with}\NormalTok{(envdata, }\FunctionTok{radiation.heatl}\NormalTok{(slope[i],}
\NormalTok{                                         aspect[i],}
\NormalTok{                                         latitude[i]))}
\NormalTok{  rad.out[i] }\OtherTok{\textless{}{-}}\NormalTok{ temp}\SpecialCharTok{$}\NormalTok{radiation}
\NormalTok{  hl.out[i] }\OtherTok{\textless{}{-}}\NormalTok{ temp}\SpecialCharTok{$}\NormalTok{heat.load}
\NormalTok{\}}
\end{Highlighting}
\end{Shaded}

The function was forced to loop around on itself letting \texttt{i\ =\ 1} during the first loop, \texttt{i\ =\ 2}, during the second loop, up to \texttt{i\ =\ 5} on the final loop. Here are the results.

\vspace{0.1in}

\begin{Shaded}
\begin{Highlighting}[]
\FunctionTok{cbind}\NormalTok{(rad.out, hl.out)}
\end{Highlighting}
\end{Shaded}

\begin{verbatim}
     rad.out  hl.out
[1,] 0.99766 0.94551
[2,] 0.85008 0.77345
[3,] 0.93030 0.96684
[4,] 0.85604 0.83957
[5,] 0.91852 0.90011
\end{verbatim}

\end{example}

\(\blacksquare\)

\index[terms]{R!package!\textit{devtools}}
\index[terms]{R!package!\textit{plant.ecol}}
\index[R]{\texttt{plant.ecol::radiation.heatl()}}
\index[R]{\texttt{devtools::install\_github()}}

\subsection{Building on a Previous Result}\label{building-on-a-previous-result}

Another application of a loop is to iteratively build on the results of the previous step(s) in the loop.

\begin{example}
\protect\hypertarget{exm:exm-even}{}\label{exm:exm-even}

\(\text{}\)\\
Consider the following function that counts the number of even entries in a vector of integers.

\vspace{0.1in}

\begin{Shaded}
\begin{Highlighting}[numbers=left,,]
\NormalTok{evencount }\OtherTok{\textless{}{-}} \ControlFlowTok{function}\NormalTok{(x)\{}
\NormalTok{  res }\OtherTok{\textless{}{-}} \DecValTok{0}
  \ControlFlowTok{for}\NormalTok{(i }\ControlFlowTok{in} \DecValTok{1} \SpecialCharTok{:} \FunctionTok{length}\NormalTok{(x))\{}
    \ControlFlowTok{if}\NormalTok{(x[i] }\SpecialCharTok{\%\%} \DecValTok{2} \SpecialCharTok{==} \DecValTok{0}\NormalTok{) res }\OtherTok{\textless{}{-}}\NormalTok{ res }\SpecialCharTok{+} \DecValTok{1}
\NormalTok{    \}}
\NormalTok{  res}
\NormalTok{\}}
\end{Highlighting}
\end{Shaded}

Recall from Ch 2 that \texttt{\%\%} is the modulus operator in \textbf{R}. That is, it finds the remainder in division. By definition the remainder of any even integer divided by two will be zero. At each loop iteration the function adds one to the numeric object \texttt{res} if the current integer in the loop is even (if it has remainder zero if divided by two).

\vspace{0.1in}

\begin{Shaded}
\begin{Highlighting}[]
\FunctionTok{evencount}\NormalTok{(}\DecValTok{1}\SpecialCharTok{:}\DecValTok{3}\NormalTok{)}
\end{Highlighting}
\end{Shaded}

\begin{verbatim}
[1] 1
\end{verbatim}

\begin{Shaded}
\begin{Highlighting}[]
\FunctionTok{evencount}\NormalTok{(}\FunctionTok{c}\NormalTok{(}\DecValTok{1}\NormalTok{,}\DecValTok{2}\NormalTok{,}\DecValTok{3}\NormalTok{,}\DecValTok{4}\NormalTok{,}\DecValTok{10}\NormalTok{))}
\end{Highlighting}
\end{Shaded}

\begin{verbatim}
[1] 3
\end{verbatim}

\end{example}

\(\blacksquare\)

\index[R]{*f@\texttt{\%\%}}

\subsection{Summarizing Categorical Variables}\label{summarizing-categorical-variables}

A third loop application is the summarization of data with respect to levels in a categorical variable.

\begin{example}
\protect\hypertarget{exm:exm-f10}{}\label{exm:exm-f10}

\(\text{}\)\\
As an example we will create statistical summaries for \texttt{height} and \texttt{age} for each \texttt{Seed} type in the \texttt{Loblolly} dataset, using the \texttt{stats} function I created earlier. I first create an empty list to hold my result:

\begin{Shaded}
\begin{Highlighting}[]
\NormalTok{result }\OtherTok{\textless{}{-}} \FunctionTok{list}\NormalTok{()}
\end{Highlighting}
\end{Shaded}

\vspace{0.1in}

\begin{Shaded}
\begin{Highlighting}[numbers=left,,]
\ControlFlowTok{for}\NormalTok{(i }\ControlFlowTok{in} \FunctionTok{levels}\NormalTok{(Loblolly}\SpecialCharTok{$}\NormalTok{Seed))\{}
\NormalTok{    temp }\OtherTok{\textless{}{-}}\NormalTok{ Loblolly[,}\DecValTok{1}\SpecialCharTok{:}\DecValTok{2}\NormalTok{][Loblolly}\SpecialCharTok{$}\NormalTok{Seed }\SpecialCharTok{==}\NormalTok{i,]}
\NormalTok{    result[[i]] }\OtherTok{\textless{}{-}} \FunctionTok{as.data.frame}\NormalTok{(}\FunctionTok{apply}\NormalTok{(temp, }\DecValTok{2}\NormalTok{, stats))}
    \FunctionTok{names}\NormalTok{(result[[i]]) }\OtherTok{\textless{}{-}} \FunctionTok{c}\NormalTok{(}\StringTok{"Age"}\NormalTok{,}\StringTok{"Height"}\NormalTok{)}
\NormalTok{\}}
\end{Highlighting}
\end{Shaded}

\texttt{Loblolly\$Seed}. This is specified with: \texttt{for(i\ in\ levels(Loblolly\$Seed))}. Note that on Line 2, the first two columns of the \texttt{Loblolly} dataset are subset by levels in \texttt{Seed}. Here are the results for seed type 305.

\vspace{0.1in}

\begin{Shaded}
\begin{Highlighting}[]
\NormalTok{result}\SpecialCharTok{$}\StringTok{\textquotesingle{}305\textquotesingle{}} \CommentTok{\# "name" of one of the 14 Loblolly seed types}
\end{Highlighting}
\end{Shaded}

\begin{verbatim}
              Age   Height
n          6.0000  6.00000
min        4.7900  3.00000
max       64.1000 25.00000
mean      35.1150 13.00000
median    37.3050 12.50000
sd        23.9271  8.60233
var      572.5056 74.00000
IQR       36.8850 12.50000
SE         9.7682  3.51188
kurtosis  -1.8479 -1.47809
skew      -0.1613  0.25449
\end{verbatim}

\end{example}

\(\blacksquare\)

\subsection{\texorpdfstring{Looping Without \texttt{for()}}{Looping Without for()}}\label{looping-without-for}

Looping in \textbf{R} is also possible using other general styles of Algol-like\footnote{Algol (Algorithmic language) computer languages arose in the late 1950s from the language ALGOL 68. Important examples include Pascal, C, and Fortran.} languages (e.g., C, C++, Pascal, and Fortran). This is accomplished with the constructs \texttt{while()}, \texttt{repeat}, and \texttt{break}.
\index[R]{\texttt{while()}}
\index[R]{\texttt{repeat}}
\index[R]{\texttt{break}}

\begin{example}
\protect\hypertarget{exm:exm-f11}{}\label{exm:exm-f11}

\(\text{}\)\\
Consider an example in which 2 is added to a base number until the updated number becomes greater than or equal to 10: We have:

\vspace{0.1in}

\begin{Shaded}
\begin{Highlighting}[]
\NormalTok{i }\OtherTok{\textless{}{-}} \DecValTok{1}
\ControlFlowTok{while}\NormalTok{ (i }\SpecialCharTok{\textless{}} \DecValTok{10}\NormalTok{) i }\OtherTok{\textless{}{-}}\NormalTok{ i }\SpecialCharTok{+} \DecValTok{2}
\NormalTok{i}
\end{Highlighting}
\end{Shaded}

\begin{verbatim}
[1] 11
\end{verbatim}

Or, to explicitly track the loop, we could use:

\vspace{0.1in}

\begin{Shaded}
\begin{Highlighting}[]
\NormalTok{i }\OtherTok{\textless{}{-}} \DecValTok{1}\NormalTok{; out }\OtherTok{\textless{}{-}}\NormalTok{ i}

\ControlFlowTok{while}\NormalTok{(}\ConstantTok{TRUE}\NormalTok{)\{}
\NormalTok{    j }\OtherTok{\textless{}{-}}\NormalTok{ i }\SpecialCharTok{+} \DecValTok{2}
\NormalTok{    out }\OtherTok{\textless{}{-}} \FunctionTok{paste}\NormalTok{(out, j, }\AttributeTok{sep =} \StringTok{","}\NormalTok{)}
\NormalTok{    i }\OtherTok{\textless{}{-}}\NormalTok{ j}
    \ControlFlowTok{if}\NormalTok{ (i }\SpecialCharTok{\textgreater{}} \DecValTok{9}\NormalTok{) }\ControlFlowTok{break}
\NormalTok{\}}

\NormalTok{out}
\end{Highlighting}
\end{Shaded}

\begin{verbatim}
[1] "1,3,5,7,9,11"
\end{verbatim}

Or, more simply

\vspace{0.1in}

\begin{Shaded}
\begin{Highlighting}[]
\NormalTok{i }\OtherTok{\textless{}{-}} \DecValTok{1}\NormalTok{; out }\OtherTok{\textless{}{-}}\NormalTok{ i}

\ControlFlowTok{repeat}\NormalTok{\{}
\NormalTok{    j }\OtherTok{\textless{}{-}}\NormalTok{ i }\SpecialCharTok{+} \DecValTok{2}
\NormalTok{    out }\OtherTok{\textless{}{-}} \FunctionTok{paste}\NormalTok{(out, j, }\AttributeTok{sep =} \StringTok{","}\NormalTok{)}
\NormalTok{    i }\OtherTok{\textless{}{-}}\NormalTok{ j}
    \ControlFlowTok{if}\NormalTok{ (i }\SpecialCharTok{\textgreater{}} \DecValTok{9}\NormalTok{) }\ControlFlowTok{break}
\NormalTok{\}}

\NormalTok{out}
\end{Highlighting}
\end{Shaded}

\begin{verbatim}
[1] "1,3,5,7,9,11"
\end{verbatim}

Here \texttt{i} took on values 1, 3, 5, 7, 9, and 11 as the loop commenced (this information is accumulated in the object \texttt{out}). When \texttt{i} equaled 11, the condition for continuation of the loop failed and the loop was halted.

\begin{blackbox}

\begin{center}
\textbf{CAUTION!}

\end{center}

Some care should be exercised with \texttt{while()} and \texttt{repeat} since infinite loops will result if impossible breaks are specified.

\end{blackbox}

\end{example}

\(\blacksquare\)

\subsection{Final Looping Considerations}\label{final-looping-considerations}

Despite their potential usefulness, loops can run slowly in \textbf{R} because it is an interpreted language (see Ch \ref{ch9}). Loops can often be avoided altogether. For example, one could rewrite the earlier \texttt{evencount()} function as:

\vspace{0.1in}

\begin{Shaded}
\begin{Highlighting}[]
\NormalTok{evencounti }\OtherTok{\textless{}{-}} \ControlFlowTok{function}\NormalTok{(x)\{}
\NormalTok{  out }\OtherTok{\textless{}{-}} \FunctionTok{ifelse}\NormalTok{(x }\SpecialCharTok{\%\%} \DecValTok{2} \SpecialCharTok{==} \DecValTok{0}\NormalTok{, }\DecValTok{1}\NormalTok{, }\DecValTok{0}\NormalTok{)}
  \FunctionTok{sum}\NormalTok{(out)}
\NormalTok{\}}
\CommentTok{\# even outcomes from a random Poisson process}
\FunctionTok{evencounti}\NormalTok{(}\FunctionTok{rpois}\NormalTok{(}\DecValTok{1000}\NormalTok{, }\DecValTok{2}\NormalTok{))}
\end{Highlighting}
\end{Shaded}

\begin{verbatim}
[1] 503
\end{verbatim}

This increases efficiency, as documented by the function \texttt{system.time()}:

\vspace{0.1in}

\begin{Shaded}
\begin{Highlighting}[]
\FunctionTok{system.time}\NormalTok{(}\FunctionTok{evencount}\NormalTok{(}\DecValTok{1}\SpecialCharTok{:}\DecValTok{1000000}\NormalTok{))}
\end{Highlighting}
\end{Shaded}

\begin{verbatim}
   user  system elapsed 
   0.19    0.02    0.20 
\end{verbatim}

\vspace{0.1in}

\begin{Shaded}
\begin{Highlighting}[]
\FunctionTok{system.time}\NormalTok{(}\FunctionTok{evencounti}\NormalTok{(}\DecValTok{1}\SpecialCharTok{:}\DecValTok{1000000}\NormalTok{))}
\end{Highlighting}
\end{Shaded}

\begin{verbatim}
   user  system elapsed 
   0.04    0.01    0.04 
\end{verbatim}

Loops can often be run more efficiently using the \texttt{apply()} family of functions (see animation examples using \texttt{lapply()} in Chs \ref{ch6} and \ref{ch7}).
\index[R]{\texttt{system.time()}}
\index[R]{\texttt{lapply()}}

If loops are necessary, speed is an issue, and use of alternative approaches (e.g., \texttt{lapply()}) is awkward or suboptimal, one can call a compiled C, C++, or Fortran script from within \textbf{R} to run the loop. This topic is addressed further in Ch. \ref{ch9}.

\section{Functional Programming}\label{FunctionalProgramming}

\index[terms]{Functional programming}
\index[terms]{Binary operation}
\index[terms]{Infix operation}
In functional programming one uses a declarative programming style that applies ``pure'' (often argument-less) functions\footnote{Famous functional programming languages include \emph{Lisp}\index[terms]{Lisp (programming language)}, \emph{Scheme}\index[terms]{Scheme (programming language)}, \emph{F\#}\index[terms]{F\# (programming language)}, and \emph{Haskell}\index[terms]{Haskell (programming language)}.}. Binary or \emph{infix} operations require exactly two operands, and provide excellent examples of functional programming. The primitive functions \texttt{\textasciigrave{}+\textasciigrave{}}, \texttt{\textasciigrave{}-\textasciigrave{}}, \texttt{\textasciigrave{}*\textasciigrave{}} (Section \ref{intro2functions}) are binary operator functions. When more than two operands are supplied, the functions still work in pairs. Thus,
\vspace{0.1in}

\begin{Shaded}
\begin{Highlighting}[]
\StringTok{\textasciigrave{}}\AttributeTok{+}\StringTok{\textasciigrave{}}\NormalTok{(}\DecValTok{1}\NormalTok{,}\StringTok{\textasciigrave{}}\AttributeTok{+}\StringTok{\textasciigrave{}}\NormalTok{(}\DecValTok{2}\NormalTok{,}\DecValTok{3}\NormalTok{))}
\end{Highlighting}
\end{Shaded}

\begin{verbatim}
[1] 6
\end{verbatim}

is equivalent to
\vspace{0.1in}

\begin{Shaded}
\begin{Highlighting}[]
\DecValTok{1} \SpecialCharTok{+} \DecValTok{2} \SpecialCharTok{+} \DecValTok{3}
\end{Highlighting}
\end{Shaded}

\begin{verbatim}
[1] 6
\end{verbatim}

One can create personalized operator functions using the syntax: \texttt{\textasciigrave{}\%\ operator\ name\ \%\textasciigrave{}} or \texttt{"\%\ operator\ name\ \%"}.

\begin{example}
\protect\hypertarget{exm:exm-fp1L}{}\label{exm:exm-fp1L}

\(\text{}\)\\
It might be useful to have an operator-style function for computing cumulative sums of individual numbers (although \texttt{cumsum()} does this already for numerical or complex objects). We will call our new operator \texttt{\%+\%}.\\
\vspace{0.1in}

\begin{Shaded}
\begin{Highlighting}[]
\StringTok{\textasciigrave{}}\AttributeTok{\%+\%}\StringTok{\textasciigrave{}} \OtherTok{\textless{}{-}} \ControlFlowTok{function}\NormalTok{(a,b)\{}\FunctionTok{c}\NormalTok{(a, a }\SpecialCharTok{+} \FunctionTok{cumsum}\NormalTok{(b))\}}

\FloatTok{2.1} \SpecialCharTok{\%+\%} \FloatTok{7.4}
\end{Highlighting}
\end{Shaded}

\begin{verbatim}
[1] 2.1 9.5
\end{verbatim}

To make the operator work for more than two numbers, the second operand must be a multi-element numeric object.

\begin{Shaded}
\begin{Highlighting}[]
\FloatTok{2.1} \SpecialCharTok{\%+\%} \FunctionTok{c}\NormalTok{(}\FloatTok{2.6}\NormalTok{, }\FloatTok{1.5}\NormalTok{)}
\end{Highlighting}
\end{Shaded}

\begin{verbatim}
[1] 2.1 4.7 6.2
\end{verbatim}

\end{example}

\(\blacksquare\)

\index[R]{\texttt{cumsum()}}

Useful \textbf{R} functions for functional programming include \texttt{Reduce()}, \texttt{Filter()}, \texttt{Find()}, \texttt{Map()}, \texttt{Negate()}, and \texttt{Position()}.
\index[R]{\texttt{Reduce()}}
\index[R]{\texttt{Filter()}}
\index[R]{\texttt{Find()}}
\index[R]{\texttt{Map()}}
\index[R]{\texttt{Negate()}}
\index[R]{\texttt{Position()}}

\begin{example}
\protect\hypertarget{exm:exm-f11a}{}\label{exm:exm-f11a}

\(\text{}\)\\
As a more applied example, recall from Section \ref{whichin} that the \texttt{\%in\%} operator can be used to indicate if there is a match (or not) for its left operand. This does not clarify, however, how one might specify \emph{not} \texttt{\%in\%}. The operations \texttt{!\%in\%} and \texttt{\%!in\%}, for example, do not work. The code below creates a \texttt{\%!in\%} operator using the function \texttt{Negate()}.

\vspace{0.1in}

\begin{Shaded}
\begin{Highlighting}[]
\StringTok{\textasciigrave{}}\AttributeTok{\%!in\%}\StringTok{\textasciigrave{}} \OtherTok{\textless{}{-}} \FunctionTok{Negate}\NormalTok{(}\StringTok{\textasciigrave{}}\AttributeTok{\%in\%}\StringTok{\textasciigrave{}}\NormalTok{)}
\end{Highlighting}
\end{Shaded}

We apply \texttt{\textasciigrave{}\%!in\%\textasciigrave{}} below to subset bacterial phylum names.
\vspace{0.1in}

\begin{Shaded}
\begin{Highlighting}[]
\NormalTok{big }\OtherTok{\textless{}{-}} \FunctionTok{c}\NormalTok{(}\StringTok{"Abawacabacteria"}\NormalTok{, }\StringTok{"Absconditabacteria"}\NormalTok{, }\StringTok{"Acidobacteriota"}\NormalTok{, }
         \StringTok{"Actinomycetota"}\NormalTok{, }\StringTok{"Aminicenantes"}\NormalTok{, }\StringTok{"Atribacterota"}\NormalTok{, }
         \StringTok{"Aquificota"}\NormalTok{, }\StringTok{"Azambacteria"}\NormalTok{)}
\NormalTok{small }\OtherTok{\textless{}{-}} \FunctionTok{c}\NormalTok{(}\StringTok{"Acidobacteriota"}\NormalTok{, }\StringTok{"Actinomycetota"}\NormalTok{)}

\NormalTok{w }\OtherTok{\textless{}{-}} \FunctionTok{which}\NormalTok{(big }\SpecialCharTok{\%!in\%}\NormalTok{ small) }
\NormalTok{big[w]}
\end{Highlighting}
\end{Shaded}

\begin{verbatim}
[1] "Abawacabacteria"    "Absconditabacteria" "Aminicenantes"     
[4] "Atribacterota"      "Aquificota"         "Azambacteria"      
\end{verbatim}

\end{example}

\(\blacksquare\)

\index[R]{*in@\texttt{\%in\%}}
\index[R]{\texttt{grep()}}

\section{Functions with Classes and Methods}\label{functions-with-classes-and-methods}

\textbf{R} object classes can have particular \textbf{methods} for plotting, printing, and summarization. Following a relatively simple series of steps, these methods can be implemented using generic function names, i.e., \texttt{plot()}, \texttt{print()}, \texttt{summary()}. For example, the function \texttt{lm()} creates objects of class \texttt{lm}.

\vspace{0.1in}

\begin{Shaded}
\begin{Highlighting}[]
\NormalTok{model }\OtherTok{\textless{}{-}} \FunctionTok{lm}\NormalTok{(height }\SpecialCharTok{\textasciitilde{}}\NormalTok{ age, }\AttributeTok{data =}\NormalTok{ Loblolly)}
\FunctionTok{class}\NormalTok{(model)}
\end{Highlighting}
\end{Shaded}

\begin{verbatim}
[1] "lm"
\end{verbatim}

There are specific summary, print, and plot methods for an object of class \texttt{lm}. Code for these methods can be viewed by typing \texttt{stats:::summary.lm}, \texttt{stats:::print.lm}, and \texttt{stats:::plot.lm}, respectively. The \texttt{stats:::print.lm} will be called automatically to print an object of class \texttt{lm}. For instance,

\vspace{0.1in}

\begin{Shaded}
\begin{Highlighting}[]
\FunctionTok{print}\NormalTok{(model)}
\end{Highlighting}
\end{Shaded}

\begin{verbatim}

Call:
lm(formula = height ~ age, data = Loblolly)

Coefficients:
(Intercept)          age  
      -1.31         2.59  
\end{verbatim}

or, more simply,

\vspace{0.1in}

\begin{Shaded}
\begin{Highlighting}[]
\NormalTok{model}
\end{Highlighting}
\end{Shaded}

\begin{verbatim}

Call:
lm(formula = height ~ age, data = Loblolly)

Coefficients:
(Intercept)          age  
      -1.31         2.59  
\end{verbatim}

Here are 20 out of the more than 500 functions on my workstation that can be called with \texttt{print}, depending on the class of the object that is being printed.

\vspace{0.1in}

\begin{Shaded}
\begin{Highlighting}[]
\FunctionTok{methods}\NormalTok{(print)[}\DecValTok{1}\SpecialCharTok{:}\DecValTok{20}\NormalTok{]}
\end{Highlighting}
\end{Shaded}

\begin{verbatim}
 [1] "print,ANY-method"            "print,diagonalMatrix-method"
 [3] "print,modelMatrix-method"    "print,sparseMatrix-method"  
 [5] "print.aareg"                 "print.abbrev"               
 [7] "print.abuocc"                "print.acf"                  
 [9] "print.activeConcordance"     "print.addtest"              
[11] "print.AES"                   "print.agnes"                
[13] "print.all_vars"              "print.allPerms"             
[15] "print.anosim"                "print.anova"                
[17] "print.Anova"                 "print.anova.gam"            
[19] "print.anova.lme"             "print.anova.loglm"          
\end{verbatim}

Importantly, we are not limited to the pre-existing object classes in \textbf{R} (e.g., \texttt{lm}, \texttt{numeric}, \texttt{factor}, etc.). Instead, we can create user-defined classes for function output. These classes can also have methods for plotting, printing, and summarization.

\subsection{S3 and S4}\label{s3-and-s4}

\textbf{R} has two-main approaches for developing OOP classes: \textbf{S3}, and \textbf{S4}\footnote{S1 and S2 OOP classes do not exist. S3 and S4 were named according to the versions of S that they accompanied. S versions 1 and 2 didn't have an OOP framework.} \footnote{\citet{wickham2019advanced} discusses two other less widely-used approaches: RC and R6. RC, or Reference Classes, are generated using the base function \texttt{setRefClass()}. Approaches for generating R6 objects and classes are provided by the R6 package, and are more similar to RC than to S3 and S4.} \citet{wickham2019advanced} notes:

\begin{quote}
\emph{``S3 allows your functions to return rich results with user-friendly display and programmer-friendly internals''}
\end{quote}

and

\begin{quote}
\emph{``S4 is a rigorous system that forces you to think carefully about program design. It's particularly well-suited for building large systems that evolve over time and will receive contributions from many programmers.''}
\end{quote}

S3 methods tend to be easier to develop than S4 methods, and this approach is recommended for most applications in \textbf{R}. The amenability of S4 for interfacing with multiple programmers explains why this approach is required for contribution to the highly collaborative Bioconductor project. S4 OOP classes and their associated methods are implemented via the \textbf{R}-distribution package \emph{methods}. I focus on S3 methods here, but briefly consider S4 methods.

\index[R]{\texttt{setRefClass()}}
\index[terms]{Bioconductor}
\index[terms]{R!S3 (object type)}
\index[terms]{R!S4 (object type)}

\subsubsection{S3}\label{s3}

S3 classes are created using the function \texttt{class()}.

\vspace{0.1in}

\begin{Shaded}
\begin{Highlighting}[]
\NormalTok{ISU }\OtherTok{\textless{}{-}} \FunctionTok{list}\NormalTok{(}\AttributeTok{name =} \StringTok{"Idaho State University"}\NormalTok{, }\AttributeTok{n.students =} \DecValTok{12000}\NormalTok{,  }
            \AttributeTok{founded =} \DecValTok{1905}\NormalTok{)}
\FunctionTok{class}\NormalTok{(ISU) }\OtherTok{\textless{}{-}} \StringTok{"univ"}
\NormalTok{ISU}
\end{Highlighting}
\end{Shaded}

\begin{verbatim}
$name
[1] "Idaho State University"

$n.students
[1] 12000

$founded
[1] 1905

attr(,"class")
[1] "univ"
\end{verbatim}

Note that the object \texttt{ISU} has the class attribute \texttt{"univ"}. The \emph{sloop} package contains functions to help distinguish OOP class frameworks. The function \texttt{sloop::otype()} can be used to determine if an object is S3, S4, RC, or R6.

\vspace{0.1in}

\begin{Shaded}
\begin{Highlighting}[]
\FunctionTok{library}\NormalTok{(sloop)}
\FunctionTok{otype}\NormalTok{(ISU)}
\end{Highlighting}
\end{Shaded}

\begin{verbatim}
[1] "S3"
\end{verbatim}

\index[terms]{R!package!\textit{sloop}}
\index[R]{\texttt{sloop::otype()}}

The object \texttt{ISU} is \texttt{S3}.

An S3 (or S4 object) is fairly useless without associated methods. Here is a simple print method for an object of class \texttt{univ}, i.e., a list with components: \texttt{name}, \texttt{founded}, and \texttt{n.students}.

\vspace{0.1in}

\begin{Shaded}
\begin{Highlighting}[]
\NormalTok{print.univ }\OtherTok{\textless{}{-}} \ControlFlowTok{function}\NormalTok{(x)\{}
  \FunctionTok{cat}\NormalTok{(x}\SpecialCharTok{$}\NormalTok{name, }\StringTok{" was founded in "}\NormalTok{, x}\SpecialCharTok{$}\NormalTok{founded, }\StringTok{",}\SpecialCharTok{\textbackslash{}n}\StringTok{and has an enrollment of "}\NormalTok{,}
\NormalTok{      x}\SpecialCharTok{$}\NormalTok{n.students, }\StringTok{" students."}\NormalTok{, }\AttributeTok{sep =} \StringTok{""}\NormalTok{)}
\NormalTok{\}}
\end{Highlighting}
\end{Shaded}

This dramatically changes the way the object \texttt{ISU} is printed.

\vspace{0.1in}

\begin{Shaded}
\begin{Highlighting}[]
\NormalTok{ISU}
\end{Highlighting}
\end{Shaded}

\begin{verbatim}
Idaho State University was founded in 1905,
and has an enrollment of 12000 students.
\end{verbatim}

Functions useful in creating print methods include \texttt{cat()} (used above) and \texttt{structure()}. The function \texttt{cat()} concatenates text into a single character vector, and prints the results. As a simple example, in the code below we bind the string ``\texttt{iteration\ =\ \textquotesingle{}\textquotesingle{}}, to a random integer generated from a Poisson distribution \texttt{rpois(1,\ 10)}, and apply a double line break \texttt{"\textbackslash n\textbackslash n"}.

\vspace{0.1in}

\begin{Shaded}
\begin{Highlighting}[]
\FunctionTok{cat}\NormalTok{(}\StringTok{"iteration = "}\NormalTok{, }\FunctionTok{rpois}\NormalTok{(}\DecValTok{1}\NormalTok{, }\DecValTok{10}\NormalTok{), }\StringTok{"}\SpecialCharTok{\textbackslash{}n\textbackslash{}n}\StringTok{"}\NormalTok{, }\AttributeTok{sep =} \StringTok{""}\NormalTok{)}
\end{Highlighting}
\end{Shaded}

\begin{verbatim}
iteration = 6
\end{verbatim}

The function \texttt{structure()} allows one to assign an \texttt{attribute} set to data.

\vspace{0.1in}

\begin{Shaded}
\begin{Highlighting}[]
\FunctionTok{structure}\NormalTok{(}\AttributeTok{.Data =} \DecValTok{1}\SpecialCharTok{:}\DecValTok{6}\NormalTok{, }\AttributeTok{dim =} \DecValTok{2}\SpecialCharTok{:}\DecValTok{3}\NormalTok{)}
\end{Highlighting}
\end{Shaded}

\begin{verbatim}
     [,1] [,2] [,3]
[1,]    1    3    5
[2,]    2    4    6
\end{verbatim}

\vspace{0.1in}

\begin{Shaded}
\begin{Highlighting}[]
\FunctionTok{structure}\NormalTok{(}\AttributeTok{.Data =} \DecValTok{1}\SpecialCharTok{:}\DecValTok{6}\NormalTok{, }\AttributeTok{names =}\NormalTok{ LETTERS[}\DecValTok{1}\SpecialCharTok{:}\DecValTok{6}\NormalTok{])}
\end{Highlighting}
\end{Shaded}

\begin{verbatim}
A B C D E F 
1 2 3 4 5 6 
\end{verbatim}

\index[R]{\texttt{cat()}}
\index[R]{\texttt{structure()}}

We can identify methods specific to some class using the function \texttt{sloop::ftype}.

\vspace{0.1in}

\begin{Shaded}
\begin{Highlighting}[]
\FunctionTok{ftype}\NormalTok{(print.univ)}
\end{Highlighting}
\end{Shaded}

\begin{verbatim}
[1] "S3"     "method"
\end{verbatim}

\begin{example}
\protect\hypertarget{exm:exm-f12}{}\label{exm:exm-f12}

\(\text{}\)\\
Here is a more complex example in which an output object from a \emph{function} has an S3 class. The function \texttt{asbio::pairw.anova} is used for adjusting \(p\)-values resulting from multiple pairwise comparisons following an omnibus ANOVA (ANalysis Of VAriance). Objects generated by the function have class \texttt{pairw} and are S3.

\vspace{0.1in}

\begin{Shaded}
\begin{Highlighting}[]
\NormalTok{eggs }\OtherTok{\textless{}{-}} \FunctionTok{c}\NormalTok{(}\DecValTok{11}\NormalTok{,}\DecValTok{17}\NormalTok{,}\DecValTok{16}\NormalTok{,}\DecValTok{14}\NormalTok{,}\DecValTok{15}\NormalTok{,}\DecValTok{12}\NormalTok{,}\DecValTok{10}\NormalTok{,}\DecValTok{15}\NormalTok{,}\DecValTok{19}\NormalTok{,}\DecValTok{11}\NormalTok{,}\DecValTok{23}\NormalTok{,}\DecValTok{20}\NormalTok{,}\DecValTok{18}\NormalTok{,}\DecValTok{17}\NormalTok{,}\DecValTok{27}\NormalTok{,}\DecValTok{33}\NormalTok{,}\DecValTok{22}\NormalTok{,}\DecValTok{26}\NormalTok{,}\DecValTok{28}\NormalTok{)}
\NormalTok{trt }\OtherTok{\textless{}{-}} \FunctionTok{factor}\NormalTok{(}\FunctionTok{rep}\NormalTok{(}\DecValTok{1}\SpecialCharTok{:}\DecValTok{4}\NormalTok{, }\FunctionTok{c}\NormalTok{(}\DecValTok{5}\NormalTok{,}\DecValTok{5}\NormalTok{,}\DecValTok{4}\NormalTok{,}\DecValTok{5}\NormalTok{)))}

\NormalTok{tukey }\OtherTok{\textless{}{-}} \FunctionTok{pairw.anova}\NormalTok{(}\AttributeTok{y =}\NormalTok{ eggs, }\AttributeTok{x =}\NormalTok{ trt)}
\FunctionTok{class}\NormalTok{(tukey)}
\end{Highlighting}
\end{Shaded}

\begin{verbatim}
[1] "pairw"
\end{verbatim}

\begin{Shaded}
\begin{Highlighting}[]
\FunctionTok{otype}\NormalTok{(tukey)}
\end{Highlighting}
\end{Shaded}

\begin{verbatim}
[1] "S3"
\end{verbatim}

Objects of class \texttt{pairw} have both print and plot methods.

\vspace{0.1in}

\begin{Shaded}
\begin{Highlighting}[]
\FunctionTok{ftype}\NormalTok{(print.pairw) }\CommentTok{\# print method for class pairw}
\end{Highlighting}
\end{Shaded}

\begin{verbatim}
[1] "S3"     "method"
\end{verbatim}

\begin{Shaded}
\begin{Highlighting}[]
\FunctionTok{ftype}\NormalTok{(plot.pairw) }\CommentTok{\# plot method for class pairw}
\end{Highlighting}
\end{Shaded}

\begin{verbatim}
[1] "S3"     "method"
\end{verbatim}

The correct S3 method for a generic function call, e.g., \texttt{print()}, is identified using a process called method dispatch. Method dispatch steps in \textbf{R} are identifiable using the function \texttt{sloop::s3\_dispatch}. Here is the process of identifying the correct printing method for an object of class \texttt{pairw}.

\vspace{0.1in}

\begin{Shaded}
\begin{Highlighting}[]
\FunctionTok{s3\_dispatch}\NormalTok{(}\FunctionTok{print}\NormalTok{(tukey))}
\end{Highlighting}
\end{Shaded}

\begin{verbatim}
=> print.pairw
 * print.default
\end{verbatim}

\index[R]{\texttt{sloop::s3\_dispatch()}}

Here is the actual print method's output:

\vspace{0.1in}

\begin{Shaded}
\begin{Highlighting}[]
\NormalTok{tukey}
\end{Highlighting}
\end{Shaded}

\begin{verbatim}

95% Tukey-Kramer confidence intervals 

         Diff     Lower    Upper  Decision Adj. p-value
mu1-mu2   1.2  -4.71976  7.11976    FTR H0     0.935298
mu1-mu3  -4.9 -11.17885  1.37885    FTR H0      0.15489
mu2-mu3  -6.1 -12.37885  0.17885    FTR H0     0.058287
mu1-mu4 -12.6 -18.51976 -6.68024 Reject H0     0.000101
mu2-mu4 -13.8 -19.71976 -7.88024 Reject H0      3.7e-05
mu3-mu4  -7.7 -13.97885 -1.42115 Reject H0     0.014218
\end{verbatim}

This same method is used for other objects from \emph{asbio} whose output has class \texttt{pairw}. These include objects from functions providing pairwise comparisons of factor levels following: an omnibus Friedman's test\footnote{Friedman's test is a non-parametric alternative to an ANOVA with a blocking variable.}, \texttt{pairw.fried()}, and an omnibus Welch's test\footnote{Welch's test, implemented using \texttt{oneway.test}, allows heteroscedasticity among factor levels when comparing factor level means.}, \texttt{pairw.oneway()}.

\vspace{0.1in}

\begin{Shaded}
\begin{Highlighting}[]
\NormalTok{welch }\OtherTok{\textless{}{-}} \FunctionTok{pairw.oneway}\NormalTok{(}\AttributeTok{y =}\NormalTok{ eggs, }\AttributeTok{x =}\NormalTok{ trt)}
\NormalTok{welch}
\end{Highlighting}
\end{Shaded}

\begin{verbatim}

95% Welch adjusted confidence intervals 

         Diff     Lower    Upper  Decision Adj. p-value
mu1-mu2   1.2  -3.39503  5.79503    FTR H0      0.55425
mu1-mu3  -4.9    -8.991   -0.809    FTR H0     0.068803
mu2-mu3  -6.1 -11.06986 -1.13014    FTR H0     0.068803
mu1-mu4 -12.6 -17.53547 -7.66453 Reject H0    0.0032699
mu2-mu4 -13.8 -19.36022 -8.23978 Reject H0    0.0027003
mu3-mu4  -7.7 -12.95069 -2.44931 Reject H0      0.04229
\end{verbatim}

\end{example}

\(\blacksquare\)

\index[R]{\texttt{sloop::ftype()}}
\index[R]{\texttt{asbio::pairw.anova()}}
\index[terms]{R!package!\textit{asbio}}
\index[R]{\texttt{asbio::pairw.oneway()}}
\index[R]{\texttt{asbio::pairw.fried()}}

Viewing of code for S3 methods functions may require use of the double colon operator, e.g., \texttt{asbio::plotpairw}, or the triple colon operator, \texttt{:::}.

\begin{example}
\protect\hypertarget{exm:exm-f13}{}\label{exm:exm-f13}

\(\text{}\)\\
In this extended exercise we will fashion an advanced function, with an S3 class and create associated methods, using a number of approaches discussed so far in this chapter.

In ecological studies, \(\alpha\)-diversity measures the level of species evenness and richness within individual plots in a dataset. The most widely used alpha diversity indices are Simpson's index, \(D_1\), and the Shannon-Weiner index, \(H'\).

\begin{equation}
D_1 = 1 - \sum_{i=1}^S p_i^2
\label{eq:91}
\end{equation}

\begin{equation}
H' = -\sum_{i=1}^S p_i  \ln p_i
\label{eq:92}
\end{equation}

where \(S\) denotes the number of species, and \(p_i\) is the proportional abundance of the \(i\)th species, \(i = 1,2,\ldots,S\).

Here are features I want my advanced \(\alpha\)-diversity function to have:

\begin{itemize}
\tightlist
\item
  Arguments specifying (1) a dataset for analysis, and (2) the type of \(\alpha\)-diversity we want calculated. So, two arguments.
\item
  A function capable of handing summaries of communities for a single site, whose data will be a single numeric vector, and dataframes describing abundances of taxa at multiple sites.
\item
  Assignment of correct names of sites (if any) to results.
\item
  Partial matching of diversity method names using \texttt{arg.match}.
\item
  An S3 class.
\item
  Invisible components, appropriate for class print and plot methods.
\end{itemize}

\vspace{0.1in}

\begin{Shaded}
\begin{Highlighting}[numbers=left,,]
\NormalTok{alpha.div }\OtherTok{\textless{}{-}} \ControlFlowTok{function}\NormalTok{(x, }\AttributeTok{method =} \StringTok{"simpson"}\NormalTok{)\{}
  \ControlFlowTok{if}\NormalTok{(}\FunctionTok{is.data.frame}\NormalTok{(x)) rn }\OtherTok{\textless{}{-}} \FunctionTok{rownames}\NormalTok{(x) }\ControlFlowTok{else}\NormalTok{ \{}
    \ControlFlowTok{if}\NormalTok{(}\FunctionTok{ncol}\NormalTok{(}\FunctionTok{as.matrix}\NormalTok{(x) }\SpecialCharTok{==} \DecValTok{1}\NormalTok{)) rn }\OtherTok{=} \FunctionTok{noquote}\NormalTok{(}\StringTok{""}\NormalTok{) }\ControlFlowTok{else} 
\NormalTok{      rn }\OtherTok{=} \DecValTok{1}\SpecialCharTok{:}\FunctionTok{nrow}\NormalTok{(}\FunctionTok{as.matrix}\NormalTok{(x))}
\NormalTok{  \}}

\NormalTok{  indices }\OtherTok{\textless{}{-}} \FunctionTok{c}\NormalTok{(}\StringTok{"simpson"}\NormalTok{, }\StringTok{"shannon"}\NormalTok{); method }\OtherTok{\textless{}{-}} \FunctionTok{match.arg}\NormalTok{(method, indices)}

\NormalTok{  x }\OtherTok{\textless{}{-}} \FunctionTok{as.matrix}\NormalTok{(x)}

\NormalTok{  prop }\OtherTok{\textless{}{-}} \ControlFlowTok{function}\NormalTok{(x)\{}
      \ControlFlowTok{if}\NormalTok{(}\FunctionTok{ncol}\NormalTok{(x) }\SpecialCharTok{==} \DecValTok{1}\NormalTok{) out }\OtherTok{\textless{}{-}}\NormalTok{ x}\SpecialCharTok{/}\FunctionTok{sum}\NormalTok{(x)}
      \ControlFlowTok{else}
\NormalTok{      out }\OtherTok{\textless{}{-}} \FunctionTok{apply}\NormalTok{(x, }\DecValTok{1}\NormalTok{, }\ControlFlowTok{function}\NormalTok{(x) x}\SpecialCharTok{/}\FunctionTok{sum}\NormalTok{(x))}
\NormalTok{      out}
\NormalTok{    \}}

\NormalTok{    p.i }\OtherTok{\textless{}{-}} \FunctionTok{prop}\NormalTok{(x)}

\NormalTok{  simp }\OtherTok{\textless{}{-}} \ControlFlowTok{function}\NormalTok{(x, p.i)\{}
      \ControlFlowTok{if}\NormalTok{(}\FunctionTok{ncol}\NormalTok{(x) }\SpecialCharTok{==} \DecValTok{1}\NormalTok{) D }\OtherTok{\textless{}{-}} \DecValTok{1} \SpecialCharTok{{-}} \FunctionTok{sum}\NormalTok{(p.i}\SpecialCharTok{\^{}}\DecValTok{2}\NormalTok{)}
      \ControlFlowTok{else}
\NormalTok{      D }\OtherTok{\textless{}{-}} \DecValTok{1} \SpecialCharTok{{-}} \FunctionTok{apply}\NormalTok{(p.i}\SpecialCharTok{\^{}}\DecValTok{2}\NormalTok{, }\DecValTok{2}\NormalTok{, sum)}
\NormalTok{      D}
\NormalTok{    \}}

\NormalTok{  shan }\OtherTok{\textless{}{-}} \ControlFlowTok{function}\NormalTok{(x, p.i)\{}
      \ControlFlowTok{if}\NormalTok{(}\FunctionTok{ncol}\NormalTok{(x) }\SpecialCharTok{==} \DecValTok{1}\NormalTok{) H }\OtherTok{\textless{}{-}} \SpecialCharTok{{-}}\FunctionTok{sum}\NormalTok{(p.i[p.i }\SpecialCharTok{\textgreater{}} \DecValTok{0}\NormalTok{] }\SpecialCharTok{*} \FunctionTok{log}\NormalTok{(p.i[p.i }\SpecialCharTok{\textgreater{}} \DecValTok{0}\NormalTok{]))}
      \ControlFlowTok{else}
\NormalTok{      H }\OtherTok{\textless{}{-}} \FunctionTok{apply}\NormalTok{(p.i, }\DecValTok{2}\NormalTok{, }\ControlFlowTok{function}\NormalTok{(x)}\SpecialCharTok{{-}}\FunctionTok{sum}\NormalTok{(x[x }\SpecialCharTok{!=} \DecValTok{0}\NormalTok{] }\SpecialCharTok{*} \FunctionTok{log}\NormalTok{(x[x }\SpecialCharTok{!=} \DecValTok{0}\NormalTok{])))}
\NormalTok{      H}
\NormalTok{    \}}

\NormalTok{  div }\OtherTok{\textless{}{-}} \ControlFlowTok{switch}\NormalTok{(method,}
                \AttributeTok{simpson =} \FunctionTok{simp}\NormalTok{(x, p.i),}
                \AttributeTok{shannon =} \FunctionTok{shan}\NormalTok{(x, p.i))}

\NormalTok{    out }\OtherTok{\textless{}{-}} \FunctionTok{list}\NormalTok{(}\AttributeTok{p.i =}\NormalTok{ p.i, }\AttributeTok{rn =}\NormalTok{ rn, }\AttributeTok{method =}\NormalTok{ method, }\AttributeTok{div =}\NormalTok{ div)}
  \FunctionTok{class}\NormalTok{(out) }\OtherTok{\textless{}{-}} \StringTok{"a\_div"}\NormalTok{; }\FunctionTok{invisible}\NormalTok{(out)}
\NormalTok{\}}
\end{Highlighting}
\end{Shaded}

Below is a breakdown of important components of the function \texttt{alpha.div()} above.

\begin{itemize}
\tightlist
\item
  In the arguments (Line 1), \texttt{x} is assumed to be either 1) a dataframe of taxa abundances at sites, with sites in rows (identified by row names) and taxa in columns, or 2) a numeric vector containing abundances of distinct taxa at a single site.
\item
  In the first lines of code in the function \emph{body} (Lines 2-4), the function attempts to obtain site names from \texttt{x}. If \texttt{x} is a dataframe, this is done using \texttt{rn\ \textless{}-\ rownames(x)}. If \texttt{x} is a vector describing a single site, distinguishing site names is probably not important, hence the code \texttt{rn\ =\ noquote("")}.
\item
  The \texttt{alpha.div()} function contains three sub-functions: \texttt{prop()} (Lines 11-16), which allows computation of \(p_i\), and is used to create the object \texttt{p.i}, \texttt{simp()} (Lines 20-25), which calculates Simpson's diversities, and \texttt{shan()} (Lines 27-32), which calculates Shannon-Weiner diversities. The latter function contains exception handling steps for taxa abundances of zero which will be undefined in Eq \eqref{eq:92}. For instance, \texttt{p.i{[}p.i\ \textgreater{}\ 0{]}} on Line 28, and \texttt{x{[}x\ !=\ 0{]}} on Line 30.
\item
  Partial matching of diversity method names (i.e., \texttt{"simpson"} and \texttt{"shannon"}) is facilitated through the function \texttt{match.arg()}.
\item
  Switching of diversity methods is done via \texttt{switch()} (Lines 34-36).
\item
  The function output is a list named \texttt{out}, which contain four objects: the proportional abundances of taxa, the rownames of \texttt{x} (i.e, the site names), the diversity method used, and the actual calculated diversities (Line 38).
\item
  In the last line of body code, \texttt{out} is assigned to the user-defined class \texttt{a\_div} and made invisible.
\end{itemize}

Here we apply the function to the dataset \texttt{varespec} from the library \emph{vegan}.

\vspace{0.1in}

\begin{Shaded}
\begin{Highlighting}[]
\FunctionTok{library}\NormalTok{(vegan)}
\FunctionTok{data}\NormalTok{(varespec)}
\NormalTok{v.div }\OtherTok{\textless{}{-}} \FunctionTok{alpha.div}\NormalTok{(varespec)}
\FunctionTok{class}\NormalTok{(v.div)}
\end{Highlighting}
\end{Shaded}

\begin{verbatim}
[1] "a_div"
\end{verbatim}

\vspace{0.1in}

\begin{Shaded}
\begin{Highlighting}[]
\FunctionTok{otype}\NormalTok{(v.div)}
\end{Highlighting}
\end{Shaded}

\begin{verbatim}
[1] "S3"
\end{verbatim}

Printing the output object \texttt{v.div} results in a rather messy rendering of a list, prompting the creation of an \texttt{a\_div} print method. Our \texttt{print.a\_div()} function will succinctly and effectively summarize results from \texttt{alpha.div} while allowing access to additional (invisible) information.

Our function for printing can be relatively simple.

\vspace{0.1in}

\begin{Shaded}
\begin{Highlighting}[numbers=left,,]
\NormalTok{print.a\_div }\OtherTok{\textless{}{-}} \ControlFlowTok{function}\NormalTok{(x, }\AttributeTok{digits =} \DecValTok{5}\NormalTok{)\{}
\NormalTok{  method }\OtherTok{\textless{}{-}} \FunctionTok{ifelse}\NormalTok{(x}\SpecialCharTok{$}\NormalTok{method }\SpecialCharTok{==} \StringTok{"simpson"}\NormalTok{, }\StringTok{"Simpson"}\NormalTok{,}
                   \StringTok{"Shannon{-}Weiner"}\NormalTok{)}
  \FunctionTok{cat}\NormalTok{(method, }\StringTok{" diversity:"}\NormalTok{, }\StringTok{"}\SpecialCharTok{\textbackslash{}n}\StringTok{"}\NormalTok{, }\AttributeTok{sep =} \StringTok{""}\NormalTok{)}
\NormalTok{  rq }\OtherTok{\textless{}{-}} \FunctionTok{structure}\NormalTok{(x}\SpecialCharTok{$}\NormalTok{div,  }\AttributeTok{names =}\NormalTok{ x}\SpecialCharTok{$}\NormalTok{rn)}
  \FunctionTok{print}\NormalTok{(rq, }\AttributeTok{digits =}\NormalTok{ digits)}
  \FunctionTok{invisible}\NormalTok{(x)}
\NormalTok{\}}
\end{Highlighting}
\end{Shaded}

\begin{itemize}
\tightlist
\item
  The required argument, \texttt{x} in \texttt{print.a\_div} (Line 1), will be an object of class \texttt{a\_div}, created by the function \texttt{alpha.div()}, e.g., the object \texttt{v.div}. Recall that this is a list containing multiple objects.
\item
  The object \texttt{x\$method} is used create a tidy text summary of the diversity method used (Lines 2-3). This string is combined with a another string and printed with a line break in:
  \texttt{cat(method,\ "\ diversity:",\ "\textbackslash{}n",\ sep\ =\ "")}\}` (Line 4).
\item
  The actual diversities are printed with the help of the function \texttt{structure()} on Lines 5-6.
  \vspace{0.1in}
\end{itemize}

\begin{Shaded}
\begin{Highlighting}[]
\FunctionTok{print}\NormalTok{(v.div)}
\end{Highlighting}
\end{Shaded}

\begin{verbatim}
Simpson diversity:
     18      15      24      27      23      19      22      16      28 
0.82171 0.76276 0.78101 0.74414 0.84108 0.81819 0.80310 0.82477 0.55996 
     13      14      20      25       7       5       6       3       4 
0.81828 0.82994 0.84615 0.83991 0.70115 0.56149 0.73888 0.64181 0.78261 
      2       9      12      10      11      21 
0.55011 0.49614 0.67568 0.50261 0.80463 0.85896 
\end{verbatim}

Output from \texttt{alpha.div()} can also be used for plotting. Here is a plot function for objects of class \texttt{a\_div}.

\vspace{0.1in}

\begin{Shaded}
\begin{Highlighting}[numbers=left,,]
\NormalTok{plot.a\_div }\OtherTok{\textless{}{-}} \ControlFlowTok{function}\NormalTok{(x, }\AttributeTok{plot.RAC =} \ConstantTok{FALSE}\NormalTok{)\{}
  \FunctionTok{require}\NormalTok{(ggplot2)}
\NormalTok{  margin\_theme }\OtherTok{\textless{}{-}} \ControlFlowTok{function}\NormalTok{()\{}
  \FunctionTok{theme}\NormalTok{(}\AttributeTok{axis.title.x =} \FunctionTok{element\_text}\NormalTok{(}\AttributeTok{vjust=}\SpecialCharTok{{-}}\DecValTok{5}\NormalTok{),}
        \AttributeTok{axis.title.y =} \FunctionTok{element\_text}\NormalTok{(}\AttributeTok{vjust=}\DecValTok{5}\NormalTok{),}
        \AttributeTok{plot.margin =} \FunctionTok{margin}\NormalTok{(}\AttributeTok{t =} \FloatTok{7.5}\NormalTok{, }\AttributeTok{r =} \FloatTok{7.5}\NormalTok{,}
                             \AttributeTok{b =} \DecValTok{20}\NormalTok{, }\AttributeTok{l =} \DecValTok{15}\NormalTok{))}
\NormalTok{  \}}

\NormalTok{  ptype1 }\OtherTok{\textless{}{-}} \ControlFlowTok{function}\NormalTok{()\{}
\NormalTok{  spi }\OtherTok{\textless{}{-}} \FunctionTok{apply}\NormalTok{(x}\SpecialCharTok{$}\NormalTok{p.i, }\DecValTok{2}\NormalTok{, }\ControlFlowTok{function}\NormalTok{(x)}\FunctionTok{sort}\NormalTok{(x, }\AttributeTok{decreasing =} \ConstantTok{TRUE}\NormalTok{))}
\NormalTok{  sspi }\OtherTok{\textless{}{-}} \FunctionTok{data.frame}\NormalTok{(}\AttributeTok{p.i =} \FunctionTok{stack}\NormalTok{(}\FunctionTok{as.data.frame}\NormalTok{(spi))[,}\DecValTok{1}\NormalTok{])}
\NormalTok{  sspi}\SpecialCharTok{$}\NormalTok{Rank }\OtherTok{\textless{}{-}} \FunctionTok{rep}\NormalTok{(}\DecValTok{1}\SpecialCharTok{:}\FunctionTok{nrow}\NormalTok{(x}\SpecialCharTok{$}\NormalTok{p.i), }\FunctionTok{ncol}\NormalTok{(x}\SpecialCharTok{$}\NormalTok{p.i))}
\NormalTok{  sspi}\SpecialCharTok{$}\NormalTok{Site }\OtherTok{\textless{}{-}} \FunctionTok{rep}\NormalTok{(x}\SpecialCharTok{$}\NormalTok{rn,  }\AttributeTok{each =} \FunctionTok{nrow}\NormalTok{(x}\SpecialCharTok{$}\NormalTok{p.i))}
    \FunctionTok{ggplot}\NormalTok{(sspi, }\FunctionTok{aes}\NormalTok{(}\AttributeTok{y =}\NormalTok{ p.i, }\AttributeTok{x =}\NormalTok{ Rank, }\AttributeTok{group =}\NormalTok{ Site)) }\SpecialCharTok{+}
      \FunctionTok{geom\_line}\NormalTok{(}\FunctionTok{aes}\NormalTok{(}\AttributeTok{y =}\NormalTok{ p.i, }\AttributeTok{x =}\NormalTok{ Rank, }\AttributeTok{colour =}\NormalTok{ Site), }\AttributeTok{alpha =}\NormalTok{ .}\DecValTok{4}\NormalTok{) }\SpecialCharTok{+}
      \FunctionTok{ylab}\NormalTok{(}\FunctionTok{expression}\NormalTok{(}\FunctionTok{italic}\NormalTok{(p[i]))) }\SpecialCharTok{+}
      \FunctionTok{theme\_classic}\NormalTok{() }\SpecialCharTok{+} \FunctionTok{margin\_theme}\NormalTok{()}
\NormalTok{  \}}

\NormalTok{  ptype2 }\OtherTok{\textless{}{-}} \ControlFlowTok{function}\NormalTok{()\{}
\NormalTok{      diversity }\OtherTok{\textless{}{-}} \FunctionTok{data.frame}\NormalTok{(}\AttributeTok{div =}\NormalTok{  x}\SpecialCharTok{$}\NormalTok{div, }\AttributeTok{Site =} \FunctionTok{factor}\NormalTok{(x}\SpecialCharTok{$}\NormalTok{rn))}
\NormalTok{      method }\OtherTok{\textless{}{-}} \FunctionTok{ifelse}\NormalTok{(x}\SpecialCharTok{$}\NormalTok{method }\SpecialCharTok{==} \StringTok{"simpson"}\NormalTok{, }\StringTok{"Simpson diveristy"}\NormalTok{,}
                   \StringTok{"Shannon{-}Weiner diveristy"}\NormalTok{)}
      \FunctionTok{ggplot}\NormalTok{(diversity) }\SpecialCharTok{+}
      \FunctionTok{geom\_bar}\NormalTok{(}\FunctionTok{aes}\NormalTok{(}\AttributeTok{y =}\NormalTok{ div, }\AttributeTok{x =}\NormalTok{ Site, }\AttributeTok{fill =}\NormalTok{ div), }\AttributeTok{show.legend =} \ConstantTok{FALSE}\NormalTok{,}
               \AttributeTok{stat =} \StringTok{"identity"}\NormalTok{) }\SpecialCharTok{+}
      \FunctionTok{theme\_classic}\NormalTok{() }\SpecialCharTok{+}
      \FunctionTok{margin\_theme}\NormalTok{() }\SpecialCharTok{+}
      \FunctionTok{ylab}\NormalTok{(method) }\SpecialCharTok{+} \FunctionTok{xlab}\NormalTok{(}\StringTok{"Site"}\NormalTok{)}
\NormalTok{  \}}
  \ControlFlowTok{if}\NormalTok{(plot.RAC)  }\FunctionTok{ptype1}\NormalTok{() }\ControlFlowTok{else} \FunctionTok{ptype2}\NormalTok{()}
\NormalTok{\}}
\end{Highlighting}
\end{Shaded}

\begin{itemize}
\tightlist
\item
  The plot method allows the creation of two distinct types of ggplots by calling distinct sub-functions, \texttt{ptype1()} and \texttt{ptype2()} via the argument \texttt{plot.RAC} (Line one).
\item
  Barplots of site diversities are produced by using the default \texttt{plot.RAC\ =\ FALSE} which will run the function \texttt{ptype2()} on Lines 21-31 (Fig \ref{fig:diva})
\item
  Rank abundance curves (RACs) are created by specifying \texttt{plot.RAC\ =\ TRUE} which runs the function \texttt{ptype1()} on Lines 10-19 (Fig \ref{fig:divb}). RAC plots allow graphical expressions of both taxa richness and evenness, and may even provide insights regarding resource exploitation in community \citep{magurran1988}.
\end{itemize}

\vspace{0.1in}

\begin{Shaded}
\begin{Highlighting}[]
\FunctionTok{plot}\NormalTok{(v.div)}
\end{Highlighting}
\end{Shaded}

\begin{figure}[H]

{\centering \includegraphics{figure/minimal-diva-1} 

}

\caption{Barplot of site diversities from the \texttt{vegan::varespec} data. Note that bar colors are varied using the diversities themselves, i.e., \texttt{fill = div}.}\label{fig:diva}
\end{figure}

\vspace{0.1in}

\vspace{0.1in}

\begin{Shaded}
\begin{Highlighting}[]
\FunctionTok{plot}\NormalTok{(v.div, }\AttributeTok{plot.RAC =} \ConstantTok{TRUE}\NormalTok{)}
\end{Highlighting}
\end{Shaded}

\begin{figure}[H]

{\centering \includegraphics{figure/minimal-divb-1} 

}

\caption{Rank abundance curves of the \texttt{vegan::varespec} dataset.  Line lengths indicate species richness.  Larger negative slopes indicate less species evenness.}\label{fig:divb}
\end{figure}

\end{example}

\(\blacksquare\)

\index[R]{\texttt{rownames()}}
\index[R]{\texttt{noquote()}}
\index[R]{\texttt{match.arg()}}
\index[R]{\texttt{switch()}}

\subsubsection{S4}\label{s4}

An S4 class is defined using the function \texttt{setClass()}. Unlike S3 objects and classes, S4 class components, i.e., \emph{slots}, must be defined in \texttt{setClass()}, along with the sub-classes of those components. Here I create an S4 class called \texttt{univ} for comparison to the S3 class \texttt{univ} created in the previous section.

\vspace{0.1in}

\begin{Shaded}
\begin{Highlighting}[]
\FunctionTok{setClass}\NormalTok{(}\StringTok{"univ"}\NormalTok{, }\AttributeTok{slots =} \FunctionTok{list}\NormalTok{(}\AttributeTok{name =} \StringTok{"character"}\NormalTok{, }\AttributeTok{n.students =} \StringTok{"numeric"}\NormalTok{,}
                              \AttributeTok{founded =} \StringTok{"numeric"}\NormalTok{))}
\end{Highlighting}
\end{Shaded}

The S4 class \texttt{univ} will have three slots: \texttt{name}, \texttt{n.students}, and \texttt{founded}. S4 objects are created using the \texttt{new()} function.
\index[R]{\texttt{setClass()}}
\index[R]{\texttt{new()}}

\vspace{0.1in}

\begin{Shaded}
\begin{Highlighting}[]
\NormalTok{ISU }\OtherTok{\textless{}{-}} \FunctionTok{new}\NormalTok{(}\AttributeTok{Class =} \StringTok{"univ"}\NormalTok{, }\AttributeTok{name =} \StringTok{"Idaho State University"}\NormalTok{, }
           \AttributeTok{n.students =} \DecValTok{12000}\NormalTok{, }\AttributeTok{founded =} \DecValTok{1905}\NormalTok{)}
\end{Highlighting}
\end{Shaded}

The object \texttt{ISU} has the S4 class \texttt{univ}.

\vspace{0.1in}

\begin{Shaded}
\begin{Highlighting}[]
\FunctionTok{class}\NormalTok{(ISU)}
\end{Highlighting}
\end{Shaded}

\begin{verbatim}
[1] "univ"
attr(,"package")
[1] ".GlobalEnv"
\end{verbatim}

\begin{Shaded}
\begin{Highlighting}[]
\FunctionTok{typeof}\NormalTok{(ISU)}
\end{Highlighting}
\end{Shaded}

\begin{verbatim}
[1] "S4"
\end{verbatim}

\index[terms]{R!base type!S4}

\begin{Shaded}
\begin{Highlighting}[]
\FunctionTok{otype}\NormalTok{(ISU)}
\end{Highlighting}
\end{Shaded}

\begin{verbatim}
[1] "S4"
\end{verbatim}

Here is the structure of the object:

\vspace{0.1in}

\begin{Shaded}
\begin{Highlighting}[]
\FunctionTok{str}\NormalTok{(ISU)}
\end{Highlighting}
\end{Shaded}

\begin{verbatim}
Formal class 'univ' [package ".GlobalEnv"] with 3 slots
  ..@ name      : chr "Idaho State University"
  ..@ n.students: num 12000
  ..@ founded   : num 1905
\end{verbatim}

Just as components of a list are accessed using \texttt{\$}, the slots of an S4 object are accessed using \texttt{@}.

\vspace{0.1in}

\begin{Shaded}
\begin{Highlighting}[]
\NormalTok{ISU}\SpecialCharTok{@}\NormalTok{founded}
\end{Highlighting}
\end{Shaded}

\begin{verbatim}
[1] 1905
\end{verbatim}

Or with the function \texttt{slot()}.

\vspace{0.1in}

\begin{Shaded}
\begin{Highlighting}[]
\FunctionTok{slot}\NormalTok{(ISU, }\StringTok{"founded"}\NormalTok{)}
\end{Highlighting}
\end{Shaded}

\begin{verbatim}
[1] 1905
\end{verbatim}

We set S4 methods using the function \texttt{setMethod()}. Here is an S4 \texttt{show()} method (analogous to S3 \texttt{print()}) for objects of class \texttt{univ}.
\index[R]{\texttt{show()}}
\index[R]{\texttt{setMethod()}}

\vspace{0.1in}

\begin{Shaded}
\begin{Highlighting}[]
\FunctionTok{setMethod}\NormalTok{(}\StringTok{"show"}\NormalTok{,}
\StringTok{"univ"}\NormalTok{,}
\ControlFlowTok{function}\NormalTok{(object) \{}
 \FunctionTok{cat}\NormalTok{(object}\SpecialCharTok{@}\NormalTok{name, }\StringTok{"was founded in"}\NormalTok{, object}\SpecialCharTok{@}\NormalTok{founded, }
     \StringTok{"and has an enrollment of"}\NormalTok{,}
\NormalTok{     object}\SpecialCharTok{@}\NormalTok{n.students, }\StringTok{"students."}\NormalTok{, }\AttributeTok{sep =} \StringTok{" "}\NormalTok{)}
\NormalTok{\}}
\NormalTok{)}

\NormalTok{ISU}
\end{Highlighting}
\end{Shaded}

\begin{verbatim}
Idaho State University was founded in 1905 and has an enrollment of 12000 students.
\end{verbatim}

\begin{example}
\protect\hypertarget{exm:exm-f14}{}\label{exm:exm-f14}

\(\text{}\)\\
As a real-world S4 example, the function \texttt{stats4::mle()} estimates parameters for probability density functions using the method of maximum likelihood. Below we estimate the rate parameter for a Poisson distribution, \(\lambda\), based on a sample of count data.

\vspace{0.1in}

\begin{Shaded}
\begin{Highlighting}[]
\CommentTok{\# count data}
\NormalTok{y }\OtherTok{\textless{}{-}} \FunctionTok{c}\NormalTok{(}\DecValTok{26}\NormalTok{, }\DecValTok{17}\NormalTok{, }\DecValTok{13}\NormalTok{, }\DecValTok{12}\NormalTok{, }\DecValTok{20}\NormalTok{, }\DecValTok{5}\NormalTok{, }\DecValTok{9}\NormalTok{, }\DecValTok{8}\NormalTok{, }\DecValTok{5}\NormalTok{, }\DecValTok{4}\NormalTok{, }\DecValTok{8}\NormalTok{)}

\CommentTok{\# MLE for lambda in a Poisson distribution given data in y}
\NormalTok{nLL }\OtherTok{\textless{}{-}} \ControlFlowTok{function}\NormalTok{(lambda) }\SpecialCharTok{{-}} \FunctionTok{sum}\NormalTok{(}\FunctionTok{dpois}\NormalTok{(y, lambda, }\AttributeTok{log =}\NormalTok{ T))}
\CommentTok{\# mle finds negative log{-}likelihoods}
\NormalTok{fit0 }\OtherTok{\textless{}{-}}\NormalTok{ stats4}\SpecialCharTok{::}\FunctionTok{mle}\NormalTok{(nLL, }\AttributeTok{start =} \FunctionTok{list}\NormalTok{(}\AttributeTok{lambda =} \DecValTok{5}\NormalTok{),}
                    \AttributeTok{nobs =} \FunctionTok{length}\NormalTok{(y))}
\end{Highlighting}
\end{Shaded}

The MLE for the Poisson rate parameter, from data outcomes in \texttt{y}, is approximately 11.545. Notably, this is equal to the sample mean of \texttt{y}.

\vspace{0.1in}

\begin{Shaded}
\begin{Highlighting}[]
\NormalTok{fit0}
\end{Highlighting}
\end{Shaded}

\begin{verbatim}

Call:
stats4::mle(minuslogl = nLL, start = list(lambda = 5), nobs = length(y))

Coefficients:
lambda 
11.545 
\end{verbatim}

\vspace{0.1in}

\begin{Shaded}
\begin{Highlighting}[]
\FunctionTok{mean}\NormalTok{(y)}
\end{Highlighting}
\end{Shaded}

\begin{verbatim}
[1] 11.545
\end{verbatim}

The class of \texttt{fit0} is \texttt{mle}. The class has an S4 designation.

\vspace{0.1in}

\begin{Shaded}
\begin{Highlighting}[]
\FunctionTok{class}\NormalTok{(fit0)}
\end{Highlighting}
\end{Shaded}

\begin{verbatim}
[1] "mle"
attr(,"package")
[1] "stats4"
\end{verbatim}

\vspace{0.1in}

\begin{Shaded}
\begin{Highlighting}[]
\FunctionTok{otype}\NormalTok{(fit0)}
\end{Highlighting}
\end{Shaded}

\begin{verbatim}
[1] "S4"
\end{verbatim}

\index[terms]{R!package!\textit{stats4}}
\index[terms]{R!package!\textit{sloop}}
\index[R]{\texttt{stats4::mle()}}
\index[R]{\texttt{sloop::otype()}}

The slot structure of an object from class \texttt{mle} is complex:

\vspace{0.1in}

\begin{Shaded}
\begin{Highlighting}[]
\FunctionTok{str}\NormalTok{(fit0)}
\end{Highlighting}
\end{Shaded}

\scriptsize

\begin{verbatim}
Formal class 'mle' [package "stats4"] with 10 slots
  ..@ call     : language stats4::mle(minuslogl = nLL, start = list(lambda = 5), nobs = length(y))
  ..@ coef     : Named num 11.5
  .. ..- attr(*, "names")= chr "lambda"
  ..@ fullcoef : Named num 11.5
  .. ..- attr(*, "names")= chr "lambda"
  ..@ fixed    : Named num NA
  .. ..- attr(*, "names")= chr "lambda"
  ..@ vcov     : num [1, 1] 1.05
  .. ..- attr(*, "dimnames")=List of 2
  .. .. ..$ : chr "lambda"
  .. .. ..$ : chr "lambda"
  ..@ min      : num 42.7
  ..@ details  :List of 6
  .. ..$ par        : Named num 11.5
  .. .. ..- attr(*, "names")= chr "lambda"
  .. ..$ value      : num 42.7
  .. ..$ counts     : Named int [1:2] 14 8
  .. .. ..- attr(*, "names")= chr [1:2] "function" "gradient"
  .. ..$ convergence: int 0
  .. ..$ message    : NULL
  .. ..$ hessian    : num [1, 1] 0.953
  .. .. ..- attr(*, "dimnames")=List of 2
  .. .. .. ..$ : chr "lambda"
  .. .. .. ..$ : chr "lambda"
  ..@ minuslogl:function (lambda)  
  ..@ nobs     : int 11
  ..@ method   : chr "BFGS"
\end{verbatim}

\normalsize

\end{example}

\(\blacksquare\)

\section{Advanced Biometric Examples}\label{advanced-biometric-examples}

Customized \textbf{R} functions can be used to provide straightforward solutions of complex mathematical procedures associated with biological research.

\begin{example}
\protect\hypertarget{exm:exm-f15}{}\label{exm:exm-f15}

\(\text{}\)\\
Biologists often need to solve systems of dependent differential equations in models describing the propagation of electrochemical signals via action action potentials in neurons \citep{hodgkin1952}, or models involving species interactions (e.g., competition or predation). For instance, the Lotka-Volterra competition model has the form:

\begin{equation}
\begin{split}
\frac{dN_1}{dt} &= r_{max1}N_1\frac{K_1-N_1-\alpha_{12}N_2}{K_1}\\
\frac{dN_2}{dt} &= r_{max2}N_2\frac{K_2-N_2-\alpha_{21}N_1}{K_2}
\end{split}
\label{eq:lv}
\end{equation}

where \(t\) denotes time, \(r_{max1}\) is the maximum per capita rate of increase (empirical rate) for species 1, and \(r_{max2}\) is the empirical rate for species 2, \(N_1\) and \(N_2\) are the number of individuals from species 1 and 2, respectively, \(K_1\) = the carrying capacity for species 1, i.e., the maximum population size for that species, \(K_2\) = the carrying capacity for species 2, \(\alpha_{12}\) is the competitive effect of species 2 on the growth of species 1, and \(\alpha_{21}\) is the competitive effect of species 1 on the growth rate of species 2.

We first bring in the package \emph{deSolve}, which contains functions for solving ordinary differential equations (ODEs).

\index[terms]{R!package!\textit{deSolve}}

\vspace{0.1in}

\begin{Shaded}
\begin{Highlighting}[]
\FunctionTok{library}\NormalTok{(deSolve)}
\end{Highlighting}
\end{Shaded}

We then define starting values for \(N_1\) and \(N_2\) and model parameters.

\vspace{0.1in}

\begin{Shaded}
\begin{Highlighting}[]
\NormalTok{xstart }\OtherTok{\textless{}{-}} \FunctionTok{c}\NormalTok{(}\AttributeTok{N1 =} \DecValTok{10}\NormalTok{, }\AttributeTok{N2 =} \DecValTok{10}\NormalTok{)}
\NormalTok{pars }\OtherTok{\textless{}{-}} \FunctionTok{c}\NormalTok{(}\AttributeTok{r1 =} \FloatTok{0.5}\NormalTok{, }\AttributeTok{r2 =} \FloatTok{0.4}\NormalTok{, }\AttributeTok{K1 =} \DecValTok{400}\NormalTok{, }\AttributeTok{K2 =} \DecValTok{300}\NormalTok{, }\AttributeTok{a2.1 =} \FloatTok{0.4}\NormalTok{,}
             \AttributeTok{a1.2 =} \FloatTok{1.1}\NormalTok{)}
\end{Highlighting}
\end{Shaded}

Next, we specify the Lotka-Volterra equations as a function. We will include the argument \texttt{time\ =\ time} even though it is not explicitly used in the function. This is required by ODE evaluators from \emph{deSolve}.

\vspace{0.1in}

\begin{Shaded}
\begin{Highlighting}[numbers=left,,]
\NormalTok{LV }\OtherTok{\textless{}{-}} \ControlFlowTok{function}\NormalTok{(}\AttributeTok{time =}\NormalTok{ time, }\AttributeTok{xstart =}\NormalTok{ xstart, }\AttributeTok{pars =}\NormalTok{ pars)\{}
\NormalTok{  N1 }\OtherTok{\textless{}{-}}\NormalTok{ xstart[}\DecValTok{1}\NormalTok{]}
\NormalTok{  N2 }\OtherTok{\textless{}{-}}\NormalTok{ xstart[}\DecValTok{2}\NormalTok{]}
    \FunctionTok{with}\NormalTok{(}\FunctionTok{as.list}\NormalTok{(pars),\{}
\NormalTok{    dn1 }\OtherTok{\textless{}{-}}\NormalTok{ r1 }\SpecialCharTok{*}\NormalTok{ N1 }\SpecialCharTok{*}\NormalTok{ ((K1 }\SpecialCharTok{{-}}\NormalTok{ N1 }\SpecialCharTok{{-}}\NormalTok{ (a1}\FloatTok{.2} \SpecialCharTok{*}\NormalTok{ N2))}\SpecialCharTok{/}\NormalTok{K1)}
\NormalTok{    dn2 }\OtherTok{\textless{}{-}}\NormalTok{ r2 }\SpecialCharTok{*}\NormalTok{ N2 }\SpecialCharTok{*}\NormalTok{ ((K2 }\SpecialCharTok{{-}}\NormalTok{ N2 }\SpecialCharTok{{-}}\NormalTok{ (a2}\FloatTok{.1} \SpecialCharTok{*}\NormalTok{ N1))}\SpecialCharTok{/}\NormalTok{K2)}
\NormalTok{    res }\OtherTok{\textless{}{-}} \FunctionTok{list}\NormalTok{(}\FunctionTok{c}\NormalTok{(dn1, dn2))}
\NormalTok{    \})}
\NormalTok{\}}
\end{Highlighting}
\end{Shaded}

The most complex part of the function occurs on Lines 4-8 where the system of ODEs in Eq \eqref{eq:lv} is solved. Note the use of \texttt{with()} to make the components of the object \texttt{pars} accessible between braces on Lines 4 and 8.\\

The \texttt{deSolve::rk4()} function solves simultaneous differential equations using classical Runge-Kutta 4th order integration \citep{butcher1987}\footnote{The method of Euler (the simplest method to find approximate solutions to first order equations) can be specified with the function \texttt{deSolve::euler()}.}. The arguments for \texttt{rk4()} are, in order, the initial population numbers from species 1 and 2, the time frames to be considered, the function to be evaluated, and the parameter values.

\vspace{0.1in}

\begin{Shaded}
\begin{Highlighting}[]
\NormalTok{out }\OtherTok{\textless{}{-}} \FunctionTok{as.data.frame}\NormalTok{(}\FunctionTok{rk4}\NormalTok{(xstart, }\AttributeTok{time =} \DecValTok{1}\SpecialCharTok{:}\DecValTok{200}\NormalTok{, LV, pars))}
\end{Highlighting}
\end{Shaded}

The object out contains the number of individuals in species 1 and 2 (\(N_1\) and \(N_2\)) for time frames 1-200 (Fig \ref{fig:lv}).

\vspace{0.1in}

\begin{Shaded}
\begin{Highlighting}[]
\FunctionTok{plot}\NormalTok{(out}\SpecialCharTok{$}\NormalTok{time, out}\SpecialCharTok{$}\NormalTok{N2, }\AttributeTok{xlab =} \StringTok{"Time"}\NormalTok{, }\AttributeTok{ylab =} \StringTok{"Number of individuals"}\NormalTok{,}
     \AttributeTok{type =} \StringTok{"l"}\NormalTok{)}
\FunctionTok{lines}\NormalTok{(out}\SpecialCharTok{$}\NormalTok{time, out}\SpecialCharTok{$}\NormalTok{N1, }\AttributeTok{type =} \StringTok{"l"}\NormalTok{, }\AttributeTok{col =} \StringTok{"red"}\NormalTok{, }\AttributeTok{lty =} \DecValTok{2}\NormalTok{)}
\FunctionTok{legend}\NormalTok{(}\StringTok{"bottomright"}\NormalTok{, }\AttributeTok{lty =} \FunctionTok{c}\NormalTok{(}\DecValTok{1}\NormalTok{, }\DecValTok{2}\NormalTok{), }\AttributeTok{legend =} \FunctionTok{c}\NormalTok{(}\StringTok{"Species 2"}\NormalTok{, }\StringTok{"Species 1"}\NormalTok{),}
\AttributeTok{col =} \FunctionTok{c}\NormalTok{(}\DecValTok{1}\NormalTok{, }\DecValTok{2}\NormalTok{))}
\end{Highlighting}
\end{Shaded}

\begin{figure}[H]

{\centering \includegraphics{figure/minimal-lv-1} 

}

\caption{Solutions from Lotka Volterra ODEs for $t = \{1, 2, ..., 200\}$. Species 1 and 2 coexist, but at levels appreciably below their carrying capacities as a result of interspecific competition.}\label{fig:lv}
\end{figure}

\end{example}

\(\blacksquare\)

\index[R]{\texttt{with()}}
\index[R]{\texttt{deSolve::rk4()}}
\index[R]{\texttt{deSolve::euler()}}

\begin{example}
\protect\hypertarget{exm:exm-f16}{}\label{exm:exm-f16}\(\text{}\)\\
Parameter estimation in biostatistics often requires optimization of mathematical functions (finding function minima and maxima). A useful function for this application is \texttt{uniroot()}, which searches an interval for the zero root of a function. For instance, many location estimators (those which estimate ``central''
or ``typical'' values, e.g., estimators of the true underlying mean) will be the zero root the function:
\begin{equation}
\sum_{i = 1}^n(x_i - \hat{\mu})
\label{eq:ss}
\end{equation}

where \(x_i\) is the \(i\)th observation from a dataset, and \(\hat{\mu}\) is an estimator of a true location value. We will use \texttt{uniroot()} to find a location estimate that provides a zero root for this function.

As data we will use tree heights from the dataframe \texttt{loblolly}.

\vspace{0.1in}

\begin{Shaded}
\begin{Highlighting}[]
\NormalTok{data }\OtherTok{\textless{}{-}}\NormalTok{ Loblolly}\SpecialCharTok{$}\NormalTok{height}
\NormalTok{f }\OtherTok{\textless{}{-}} \ControlFlowTok{function}\NormalTok{(x) }\FunctionTok{sum}\NormalTok{(data }\SpecialCharTok{{-}}\NormalTok{ x)}
\FunctionTok{uniroot}\NormalTok{(f, }\FunctionTok{c}\NormalTok{(}\FunctionTok{min}\NormalTok{(data), }\FunctionTok{max}\NormalTok{(data)))}\SpecialCharTok{$}\NormalTok{root}
\end{Highlighting}
\end{Shaded}

\begin{verbatim}
[1] 32.364
\end{verbatim}

\index[R]{\texttt{uniroot()}}
This value is identical to the sample mean of the tree heights.

\vspace{0.1in}

\begin{Shaded}
\begin{Highlighting}[]
\FunctionTok{mean}\NormalTok{(data)}
\end{Highlighting}
\end{Shaded}

\begin{verbatim}
[1] 32.364
\end{verbatim}

Indeed, the sample mean will be always be the zero root of Eq. \eqref{eq:ss}.
Normally the differences of the data points and the location estimate are squared, preceding summation. Minimizing this function is the process of ordinary least squares.
\end{example}

\(\blacksquare\)

\section{The Process of Function Evaluation in R}\label{howfunctionswork}

Under construction

\subsection{Promises}\label{promises}

\index[terms]{R!base type!promise}
\index[terms]{R!base type!symbol}

\section*{Exercises}\label{exercises-7}
\addcontentsline{toc}{section}{Exercises}

\begin{enumerate}
\def\labelenumi{\arabic{enumi}.}
\item
  Divide the plant height and soil N values from the dataset from Q. 3 in the Exercises for Chapter \ref{ch3} (the first two columns of the dataset) by their respective column sums by specifying an appropriate function as the 3rd argument for \texttt{apply()}.
  \vspace{0.1in}
\item
  Below is McIntosh's index of site biodiversity \citep{mcintosh1967}:
  \[U = \sqrt{\sum_{i = 1}^s n_i^2}\]
  where \(s\) is the total number of species from a particular site, and \(n_i\) is the number of individuals from the \(i\)th species, \(i = 1,2,3,\ldots s\), from that site.

  \begin{enumerate}
  \def\labelenumii{(\alph{enumii})}
  \tightlist
  \item
    Write a function to calculate the index.
  \item
    Check it by running it on the following collection of \(n_is\) obtained from a single site: \texttt{ni\ \textless{}-\ c(5,4,5,3,2)}.
    \vspace{0.1in}
  \end{enumerate}
\item
  Below is the Satterthwaite formula for approximating degrees of freedom for the \(t\) distribution, under heteroscedasticity.
  \[\frac{\left(\frac{S_x^2}{n_x} + \frac{S_y^2}{n_y}\right)^2}{\frac{\left(S^2_x/n_x\right)^2}{n_x-1}+\frac{\left(S^2_y/n_y\right)^2}{n_y-1}}\]
  where \(S^2_x\) is the sample variance for variable \(x\), \(S^2_y\) is the sample variance for variable \(y\), \(n_x\) is the sample size for \(x\), and \(n_y\) is the sample size for \(y\).

  \begin{enumerate}
  \def\labelenumii{(\alph{enumii})}
  \tightlist
  \item
    Write a function for this equation that has the variables \(x\) and \(y\) as arguments.\\
  \item
    Test the function for \texttt{x\ \textless{}-\ c(1,2,3,2,4,5)} and \texttt{y\ \textless{}-\ c(2,3,7,8,9,10,11)}.
    \vspace{0.1in}
  \end{enumerate}
\item
  Create a function, implementing \texttt{switch()}, that can calculate the first or second derivative of a mathematical expression with respect to \texttt{"x"}. Test it on \texttt{x\^{}3}.
  \vspace{0.1in}
\item
  Create a function that calculates trimmed means for columns in a quantitative matrix or dataframe. Within the function, use the triple dot operator as an argument in \texttt{mean()}, to allow: 1) user-defined trimming (\texttt{trim} is an argument in the function \texttt{mean()}), and 2) user-defined handling of \texttt{NA} outcomes (\texttt{na.rm} is also an argument in \texttt{mean()}). Your function should have two arguments: one for input data, and one for the triple dot operator, \texttt{...} . Test the function on the first two columns of \texttt{asbio::cliff.sp}. In your test specify both 10\% trimming, and the removal of missing values.
  \vspace{0.1in}
\item
  Here are some classic computer science loop applications.

  \begin{enumerate}
  \def\labelenumii{(\alph{enumii})}
  \tightlist
  \item
    A sequence of Fibonacci numbers is based on the function:
    \begin{align*}
    f(n) &= f(n-1) + f(n-2) \text{ for } n > 2\\
    f(1) &= f(2) = 1
    \end{align*}
    where \(n\) represents the \(n\)th step in the sequence. Using a loop, create the first 100 numbers in the sequence, i.e., find \(f(1)\) to \(f(100)\). As a check, the first five numbers in the sequence should be: 1, 1, 2, 3, 5.
  \item
    An interesting chaotic recursive sequence has the function:
    \begin{align*}
    f(n) &= f(n- f(n - 1)) + f(n - f(n-2)) \text{ for } n > 2\\
    f(1) &= f(2) = 1
    \end{align*}
    Using a loop, create the first 100 numbers in the sequence, i.e., find \(f(1)\) to \(f(100)\). As a check, the first five numbers in the sequence should be: 1, 1, 2, 3, 3.
    \vspace{0.1in}
  \end{enumerate}
\item
  Create an \textbf{R} animation by creating a for loop 360 steps long that changes the font-size, color, and string rotation of your name (as a character string) in an otherwise empty plot. Assuming the index \texttt{i} is used, your loop should include something resembling the code:
  \vspace{0.1in}

\begin{Shaded}
\begin{Highlighting}[]
\FunctionTok{plot}\NormalTok{(}\DecValTok{1}\SpecialCharTok{:}\DecValTok{10}\NormalTok{, }\AttributeTok{type =} \StringTok{"n"}\NormalTok{)}
\FunctionTok{text}\NormalTok{(}\FloatTok{5.5}\NormalTok{, }\StringTok{"your name"}\NormalTok{, }\AttributeTok{cex =}\NormalTok{ i}\SpecialCharTok{/}\DecValTok{36}\NormalTok{, }\AttributeTok{srt =}\NormalTok{ i, }\AttributeTok{col =}\NormalTok{ i)}
\FunctionTok{Sys.sleep}\NormalTok{(}\FloatTok{0.1}\NormalTok{)}
\end{Highlighting}
\end{Shaded}

  \vspace{0.1in}
\item
  The Stirling number of the second kind (or the Stirling partition number) counts the number of ways a set of \(n\) objects can be partitioned into \(k\) groups. This is generally denoted \(S(n,k)\) or \(\left\{n \atop k \right\}\), and is calculated as:
\end{enumerate}

\[\left\{n \atop k \right\} = \sum_{i=0}^k \frac{(-1)^{k-i} i^n}{(k-i)!i!}.\]

~~~~~~~~~Write a function that calculates \(\left\{n \atop k \right\}\) without using a for loop.\\
\strut ~~~~~~~~~Use the form: \texttt{stirling2\ \textless{}-\ function(n,\ k)\{function\ contents\}}.

\vspace{0.1in}

\begin{enumerate}
\def\labelenumi{\arabic{enumi}.}
\setcounter{enumi}{8}
\tightlist
\item
  The Bell number, \(B_n\), counts the number of ways a set with \(n\) elements can be partitioned \citep{bell1938}. That is, \(B_n\) will be the sum of Stirling numbers for a particular set, for \(k = \{1,2,\ldots,n\}\):
\end{enumerate}

\[B_n = \sum_{k=1}^n \left\{{n\atop k}\right\}.\]

~~~~~~~~~Write a function for calculating \(B_n\) that uses \texttt{stirling2}, from Q 8, in a for loop.\\
\strut ~~~~~~~~~Use the form: \texttt{belln\ \textless{}-\ function(n)\{function\ contents\}}.

\vspace{0.1in}

\begin{enumerate}
\def\labelenumi{\arabic{enumi}.}
\setcounter{enumi}{9}
\item
  The exercise below concerns the speed of loops in \textbf{R}. Use \texttt{tapply()} to simultaneously find the mean estimates of \texttt{Loblolly\$height} for each level in \texttt{Loblolly\$seed}, with and without a loop, and find the run times of the operations using \texttt{system.time()}. That is, run the following chunks:
  \vspace{0.1in}

\begin{Shaded}
\begin{Highlighting}[]
\FunctionTok{system.time}\NormalTok{(}\FunctionTok{tapply}\NormalTok{(Loblolly}\SpecialCharTok{$}\NormalTok{height, Loblolly}\SpecialCharTok{$}\NormalTok{Seed, mean))}
\end{Highlighting}
\end{Shaded}

  \vspace{0.1in}

\begin{Shaded}
\begin{Highlighting}[]
\NormalTok{out }\OtherTok{\textless{}{-}} \DecValTok{1}\SpecialCharTok{:}\FunctionTok{nlevels}\NormalTok{(Loblolly}\SpecialCharTok{$}\NormalTok{Seed)}
\FunctionTok{system.time}\NormalTok{(}\ControlFlowTok{for}\NormalTok{(i }\ControlFlowTok{in} \FunctionTok{levels}\NormalTok{(Loblolly}\SpecialCharTok{$}\NormalTok{Seed))\{}
\NormalTok{  temp }\OtherTok{\textless{}{-}}\NormalTok{ Loblolly[Loblolly}\SpecialCharTok{$}\NormalTok{Seed }\SpecialCharTok{==}\NormalTok{ i,]}
\NormalTok{  out[i] }\OtherTok{\textless{}{-}} \FunctionTok{mean}\NormalTok{(temp}\SpecialCharTok{$}\NormalTok{height)}
\NormalTok{\})}
\end{Highlighting}
\end{Shaded}

  Describe and discuss your results.

  \vspace{0.1in}
\item
  Write a function to solve the systems of ODEs below
  \begin{align*}
  \frac{dx}{dt} &= ax + by\\
  \frac{dy}{dt} &= cx + dy
  \end{align*}
  To test the function, let \(a = 3\), \(b = 4\), \(c = 5\), \(d = 6\), and solve for for \(t = 1,2,\ldots,20\), using classical Runge-Kutta 4\(^\text{th}\) order integration, as implemented by the function \texttt{deSolve::rk4}. Initial values for \(x\) and \(y\) can be anything but \(\{0,0\}\).
  \vspace{0.1in}
\item
  Make output from the function in previous question have an S3 class, and create a plotting method for objects of this class.
\end{enumerate}

\chapter{R Interfaces}\label{ch9}

\begin{quote}
\emph{``You should try things; \textbf{R} won't break."}
\end{quote}

- \textbf{Duncan Murdoch}, \emph{from \textbf{R}-help (May 2016)}

Code from other languages can be interfaced to \textbf{R} at the command line prompt and within \textbf{R} scripts. For instance, we have already considered the use of Perl regex calls for managing character strings in Ch \ref{ch4} (Section \ref{regex}). Other examples include code interfaces from C, C++ (via package \emph{Rcpp}, \citet{edRcpp}, \citet{eddbook}, \citet{eddTAS}), Fortran, MATLAB (via package \emph{R.matlab}, \citet{bengtsson}), Python (via package \emph{reticulate}, \citet{ushey}), Java (via package \emph{rJava}, \citet{urbanak}). \textbf{R} can also be called \emph{from} a number of different languages including C and C++ (see package \emph{RInside}, \citet{edd}), Java (via the Java package \emph{RCaller}, \citet{satman2014}), or Python (via the Python package, \href{https://github.com/rpy2}{\emph{rpy2}}). For instance, the \textbf{R} package \emph{RCytoscape}, from the \href{https://bioconductor.org/}{Bioconductor project}, allows cross-communication between the popular Java-driven software for molecular networks \href{https://cytoscape.org/}{Cytoscape}, and \textbf{R}.

There are costs and benefits to interfacing with \textbf{R} to and from other languages \citep{chambers2008}. Costs include:

\begin{itemize}
\tightlist
\item
  Non interpreted languages (see Section \ref{interp} immediately below) will require compilation. Therefore it may be wise to limit such code to package-development applications (Ch \ref{ch10}) since \textbf{R} built-in procedures can facilitate this process during package building.
\item
  Interfacing with older low level languages (e.g., Fortran and C (Section \ref{FC})) increases the possibility for programming errors, often with serious consequences, including memory faults. That is, \emph{bugs bite}!
\item
  Interfacing with some languages may increase the possibility for programs being limited to specific platforms.
\item
  \textbf{R} programs can often be written more succinctly. For instance, \citet{morandat2012} found that \textbf{R} programs are about 40\% smaller than analogous programs written in C.
\end{itemize}

Despite these issues, there are a number of strong potential benefits. These include:

\begin{itemize}
\tightlist
\item
  A huge number of useful, well-tested, algorithms have been written in other languages, and it is often straightforward to interface these procedures through \textbf{R}.\\
\item
  The system speed of other languages may be much better than \textbf{R} for many tasks. For instance, algorithms written in non-interpreted languages, are often much faster than corresponding procedures written in \textbf{R}.
\item
  Non-OOP languages may be more efficient than \textbf{R} with respect to memory usage.
\end{itemize}

This chapter considers interfaces with several important programming languages, including Fortran, C, C++, and particularly, Python. Interfaces with languages used primarily for GUI generation and web-based applications, for example, Tcl/Tk, JavaScript, JavaScript Object Notation (JSON), HTML, and Cascading Style Sheets (CSS), are considered in Ch \ref{ch11}.

\index[terms]{R!package!\textit{Rcpp}}
\index[terms]{R!package!\textit{RCytoscape}}
\index[terms]{R!package!\textit{rJava}}
\index[terms]{R!package!\textit{reticulate}}
\index[terms]{R!package!\textit{RInside}}
\index[terms]{Python!package!\textit{rpy2}}
\index[terms]{Java!package!\textit{RCaller}}
\index[terms]{Cytoscape}

\section{Interpreted versus Compiled Languages}\label{interp}

Along with many other useful languages (e.g., Python, JavaScript), \textbf{R} is generally applied as an interpreted language. Interpreted code must be translated into binary before it can be executed. This process can slow down function run times, particularly if the function includes iterative procedures like loops. Non-interpreted (compiled) languages include C, Fortran, and Java. For these languages, a compiler (a translation program) is used to transform the source code into a target ``object'' language, which is generally binary (Ch \ref{ch11}). The end product of the compilation is called an executable file (Figure \ref{fig:compiled}). Executables from other languages can be called from within \textbf{R} to run \textbf{R} functions and procedures.

\index[terms]{Interpreted programming language}
\index[terms]{Executable file}
\index[terms]{Compiler}
\index[terms]{Compiled programming language}

\begin{figure}[H]

{\centering \includegraphics[width=0.96\linewidth,]{figs9/compile} 

}

\caption{Creating an executable file in a compiled language.}\label{fig:compiled}
\end{figure}

\begin{Shaded}
\begin{Highlighting}[]
\NormalTok{knitr}\SpecialCharTok{::}\NormalTok{opts\_knit}\SpecialCharTok{$}\FunctionTok{get}\NormalTok{(}\StringTok{\textquotesingle{}rmarkdown.pandoc.to\textquotesingle{}}\NormalTok{)}
\end{Highlighting}
\end{Shaded}

\begin{verbatim}
[1] "latex"
\end{verbatim}

\section{Interfacing with R Markdown/RStudio}\label{markint}

Language and program interfacing with \textbf{R} is often greatly facilitated through the use of \textbf{R} Markdown chunks. This is because many languages other than \textbf{R} are supported by \textbf{R} Markdown, via \emph{knitr}. The language definition for a particular \textbf{R} Markdown chunk is given by the first term in that chunk. For instance, \texttt{\textasciigrave{}\textasciigrave{}\textasciigrave{}\{r\}\ \textasciigrave{}\textasciigrave{}\textasciigrave{}} initiates a conventional \textbf{R} code chunk, whereas \texttt{\textasciigrave{}\textasciigrave{}\textasciigrave{}\{python\}\textasciigrave{}\textasciigrave{}\textasciigrave{}} initiates a Python code chunk. Here are the current \textbf{R} Markdown language engines (note that items 52-64 are not explicit computer languages).

\vspace{0.1in}

\begin{Shaded}
\begin{Highlighting}[]
\FunctionTok{names}\NormalTok{(knitr}\SpecialCharTok{::}\NormalTok{knit\_engines}\SpecialCharTok{$}\FunctionTok{get}\NormalTok{())}
\end{Highlighting}
\end{Shaded}

\begin{verbatim}
 [1] "awk"         "bash"        "coffee"      "gawk"        "groovy"     
 [6] "haskell"     "lein"        "mysql"       "node"        "octave"     
[11] "perl"        "php"         "psql"        "Rscript"     "ruby"       
[16] "sas"         "scala"       "sed"         "sh"          "stata"      
[21] "zsh"         "asis"        "asy"         "block"       "block2"     
[26] "bslib"       "c"           "cat"         "cc"          "comment"    
[31] "css"         "ditaa"       "dot"         "embed"       "eviews"     
[36] "exec"        "fortran"     "fortran95"   "go"          "highlight"  
[41] "js"          "julia"       "python"      "R"           "Rcpp"       
[46] "sass"        "scss"        "sql"         "stan"        "targets"    
[51] "tikz"        "verbatim"    "theorem"     "lemma"       "corollary"  
[56] "proposition" "conjecture"  "definition"  "example"     "exercise"   
[61] "hypothesis"  "proof"       "remark"      "solution"    "glue"       
[66] "glue_sql"    "gluesql"    
\end{verbatim}

As evident in the output above, \textbf{R} Markdown engines extend to compiled languages including Fortran (engine = \texttt{fortran}) and C (engine = \texttt{c}). Ideally, this is accomplished by compiling source code in a chunk in an on-the-fly, automated step, using native \textbf{R} compilers, and automatically loading the resulting compiled executable for potential calls in \textbf{R} chunks \citep{xie2020}. This process may be hampered, however, by a number of factors including non-administrator permissions and environmental path definitions, particularly on Windows computers. As a result, I present more complex, but more dependable method for code compilation and execution here.

\section{Fortran and C}\label{FC}

\index[terms]{C (programming language)}
\index[terms]{Fortran (programming language)}
\textbf{S}, the progenitor of \textbf{R}, was created at a time when Fortran routines dominated numerical programming, and \textbf{R} arose when C was approaching its peak in popularity. As a result, strong connections to those languages, particularly C, remain in \textbf{R}\footnote{As noted in Ch 1, Fortran is one of the oldest programming languages still in active use. Specifically, although Fortran's development followed an IBM proposal for an alternative to assembly language in 1953 \citep{backus}, and the first correctly compiled version of Fortran occurred in 1958, Fortran remains among the top programming languages in the \href{@https://www.tiobe.com/tiobe-index/}{TIOBE index} \citep{wikifor}. C is a widely-used general programming language developed during the 1970s \citep{ritchie1993develC}. Recall that early iterations of \textbf{S} were strongly dependent on Fortran procedures (Section \ref{Rhist}). Further, because \textbf{R} is largely written in C, it is not surprising that the most direct language for interfacing with \textbf{R} is C \citep{chambers2008}.}. \textbf{R} contains specific base functions for interfacing with both C and Fortran executables: \texttt{.C()} and \texttt{.Fortran()}\footnote{The \href{https://cran.r-project.org/doc/manuals/R-exts.html}{Writing \textbf{R} Extensions} guide details several potential problems with \texttt{.Fortran()}, including issues with passing character strings, and the fact that \texttt{.Fortran()} was primarily intended for Fortran 77 code, which precludes any support for `modern' Fortran. The guide notes that a better way to interface modern Fortran code is using \texttt{.C()}, and writing a C interface using \emph{use iso\_c\_binding}, a standard Fortran 2003 module that defines named constants, types, and procedures for C interoperability.
  \index[terms]{Fortran (programming language)!Fortran 2003}
  \index[terms]{Fortran (programming language)!Fortran 2003!modules!\textit{iso\_c\_binding}}}, respectively. More flexible general interface functions, which were introduced in later versions of \textbf{R}, and may have better performance characteristics, include \texttt{.Call()} and \texttt{.External()}.\\
\index[R]{\texttt{.Fortran()}}
\index[R]{\texttt{.C()}}
\index[R]{\texttt{.Call()}}
\index[R]{\texttt{.External()}}

Recall that an \textbf{R} object of class numeric will be automatically assigned to base type \texttt{double}, although it can be coerced to base type \texttt{integer} (this will potentially result in information loss through the elimination of its ``decimal'' component). Fortran, C, and many other languages require explicit user-assignments for underlying base types.

If one is interfacing \textbf{R} with Fortran or C, only a limited number of base types are possible (Table \ref{tab:CRF}), and one will need to use appropriate coercion functions for \textbf{R} objects if one wishes to use those objects in Fortran or C scripts\footnote{Specifically, the functions \texttt{.C()} and \texttt{.Call} expect that \textbf{R}-dependent code will be used. The functions \texttt{.C()} and \texttt{.Fortran()}, assume that \textbf{R} objects will not be directly used in algorithms.}. Interfaced C script arguments must be pointers\footnote{A pointer is a variable used to store the memory address of another variable as its value.} and arguments in Fortran scripts must be arrays for the types given in Table \ref{tab:CRF}.
\index[terms]{R!base type!double}
\index[terms]{R!base type!integer}
\index[terms]{C (programming language)!Pointer}
\index[terms]{Fortran (programming language)!array}

\begin{table}[ht]
\centering
\caption{Correpsonding types for \textbf{R}, C, and Fortran.  Table adapted from \citet{chambers2008}.} 
\label{tab:CRF}
\begin{tabular}{llll}
  \toprule
**R** base type & **R** coercion function & C type & Fortran type \\ 
  \midrule
\texttt{logical} & \texttt{as.integer()} & \texttt{int *} & \texttt{integer} \\ 
  \texttt{integer} & \texttt{as.integer()} & \texttt{int *} & \texttt{integer} \\ 
  \texttt{double} & \texttt{as.double()} & \texttt{double *} & \texttt{double precision} \\ 
  \texttt{complex} & \texttt{as.complex()} & \texttt{Rcomplex *} & \texttt{double complex} \\ 
  \texttt{charater} & \texttt{as.character()} & \texttt{char **} & \texttt{character*255} \\ 
  \texttt{raw} & \texttt{as.character()} & \texttt{char *} & none \\ 
   \bottomrule
\end{tabular}
\end{table}

\subsection{Compiling Executables}\label{compile}

Raw Fortran source code is generally saved as an .f, or (.f90 or .f95; modern Fortran) file, whereas C source code is saved as an .c file. One can create a file with the correct file type extension by using \texttt{file.create()}. For example, below I create a file called \texttt{foo.f90} that I can open (from my working directory) in a text editor (e.g., Notepad, RStudio) to build a Fortran script.

\vspace{0.1in}

\begin{Shaded}
\begin{Highlighting}[]
\FunctionTok{file.create}\NormalTok{(}\StringTok{"foo.f90"}\NormalTok{)}
\end{Highlighting}
\end{Shaded}

\index[R]{\texttt{file.create()}}

Windows executable files compiled from source code will generally have an .exe or .cmd extension, whereas Mac OS executable files generally have have .app extension. For use in \textbf{R}, however, these files must be shared library executables (see below), with .dll and .so extensions for Windows and Unix-alike (e.g., Mac-OS) operating systems, respectively. Shared library objects are different from conventional executables in that they cannot be evaluated directly. In this case, \textbf{R} will be required as the executable entry point\footnote{The extension .dll identifies a Windows \emph{dynamic-link library} (DLL) file, whereas the extension .so stands for \emph{shared object} or \emph{shared library} file.}.
\index[terms]{Dynamic-link library (DLL)}
\index[terms]{Shared library}

\textbf{R} contains shared library compilers for Fortran and C within the \texttt{Rcmd} executable, which is located in the \textbf{R} bin directory, along with several other important executables. \texttt{Rcmd} is typically invoked from a shell command line (in Windows one can access this editor by typing \texttt{cmd} in the Search bar) using the format: \texttt{R\ CMD\ command\ args}. Here \texttt{command} is currently one of \texttt{INSTALL}, \texttt{REMOVE}, \texttt{SHLIB}, \texttt{BATCH}, \texttt{build}, \texttt{check}, \texttt{Rprof}, \texttt{Rdconfig}, \texttt{Rdiff}, \texttt{Rd2pdf}, \texttt{Stangle}, \texttt{Sweave}, \texttt{config}, \texttt{open}, and \texttt{texify}, and \texttt{args} defines arguments specific to the \texttt{R\ CMD} command. The shell script:
\vspace{0.1in}

\begin{Shaded}
\begin{Highlighting}[]
\NormalTok{R CMD SHLIB foo}
\end{Highlighting}
\end{Shaded}

\index[terms]{R!Rcmd.exe}
\index[R]{\texttt{R CMD INSTALL}}
\index[R]{\texttt{R CMD REMOVE}}
\index[R]{\texttt{R CMD SHLIB}}
\index[R]{\texttt{R CMD BATCH}}
\index[R]{\texttt{R CMD build}}
\index[R]{\texttt{R CMD check}}
\index[R]{\texttt{R CMD Rprof}}
\index[R]{\texttt{R CMD Rconfig}}
\index[R]{\texttt{R CMD Rdiff}}
\index[R]{\texttt{R CMD Rd2pdf}}
\index[R]{\texttt{R CMD Stangle}}
\index[R]{\texttt{R CMD Sweave}}
\index[R]{\texttt{R CMD config}}
\index[R]{\texttt{R CMD open}}
\index[R]{\texttt{R CMD texify}}
will prompt the building of a shared library object from the user-defined script \texttt{foo}, which is generally Fortran or C source code\footnote{\texttt{SHLIB} stands for \emph{shared library}.}. The compilation of shared libraries will be facilitated by the installation of the toolbox bundle \href{https://cran.r-project.org/bin/windows/Rtools/rtools44/rtools.html}{Rtools} along an accessible environmental path.

Notably, the \texttt{SHLIB} compilers will only work for Fortran code written as a subroutine\footnote{A Fortran subroutine is invoked with a \texttt{CALL} statement. Unlike a Fortran function, which returns a single value, a subroutine can return many (or no) values.} and C code written in void formats\footnote{Void functions in C are used for their side effects, such as performing a task or writing to output.}. As a result, neither code type will return a value directly\footnote{Several other \texttt{R\ CMD} commands are addressed in Ch \ref{ch10}, including \texttt{INSTALL}, \texttt{check}, \texttt{BATCH}, \texttt{build}, AND \texttt{Rd2pdf}}.

\begin{example}
\protect\hypertarget{exm:exm-c91}{}\label{exm:exm-c91}

\(\text{}\)\\
Here is a simple example for calling Fortran and C compiled executables, from \textbf{R}, to speed up looping. The content follows \href{http://users.stat.umn.edu/~geyer/rc/}{class notes} created by Charles Geyer at the University of Minnesota. Clearly, the example could also be run without looping. Equation \eqref{eq:ftoc} shows the simple formula for converting temperature measurements in degrees Fahrenheit to degrees Celsius.

\begin{equation}
C = 5/9(F - 32)
\label{eq:ftoc}
\end{equation}

where \(C\) and \(F\) denote temperatures in Celsius and Fahrenheit, respectively.

Here is a Fortan subroutine for calculating Celsius temperatures from a dataset of Fahrenheit measures, using a loop.
\vspace{0.1in}

\begin{Shaded}
\begin{Highlighting}[numbers=left,,]
\KeywordTok{subroutine}\NormalTok{ FtoC(n, x)}
\DataTypeTok{integer}\NormalTok{ n}
\DataTypeTok{double precision}\NormalTok{ x(n)}
\DataTypeTok{integer}\NormalTok{ i}
\KeywordTok{do} \DecValTok{100}\NormalTok{ i }\KeywordTok{=} \DecValTok{1}\NormalTok{, n}
\NormalTok{x(i) }\KeywordTok{=}\NormalTok{ (x(i)}\KeywordTok{{-}}\DecValTok{32}\NormalTok{)}\KeywordTok{*}\NormalTok{(}\FloatTok{5.}\KeywordTok{/}\FloatTok{9.}\NormalTok{)}
\DecValTok{100} \KeywordTok{continue}
\KeywordTok{end}
\end{Highlighting}
\end{Shaded}

The Fortran code above consists of the following steps:

\begin{itemize}
\tightlist
\item
  On Line 1 a subroutine is invoked using the Fortran function \texttt{subroutine}. The subroutine is named \texttt{FtoC}, and has arguments \texttt{x} (the Fahrenheit temperatures) and \texttt{n} (the number of temperatures)
\item
  On Line 2 the entry given for \texttt{n} is defined to be an integer (Table \ref{tab:CRF}).
\item
  On Line 3 we define \texttt{x} to be a double precision numeric vector of length \texttt{n}.
\item
  On Line 4 we define that the looping index to be used, \texttt{i}, will be an integer.
\item
  On Lines 5-7 we proceed with a Fortran \texttt{do} loop. The code \texttt{do\ 100\ i\ =\ 1,\ n} means that the loop will 1) run initially up to 100 times, 2) has a lower limit of 1, and 3) has an upper limit of \texttt{n}. The code: \texttt{x(i)\ =\ (x(i)-32)*(5./9.)} calculates Eq. \eqref{eq:ftoc}. The code \texttt{5./9.} is used because the result of the division can be a non-integer. The code \texttt{100\ continue} allows the loop to continue to \texttt{n}.
\item
  On Line 8 the subroutine ends. All Fortran scripts must end with \texttt{end}.
\end{itemize}

I save the code under the filename \texttt{FtoC.f90}, and transfer it to an appropriate directory (I use \texttt{C:/Users/ahoken/Documents/Amalgam/Amalgam\_Bookdown/scripts/}). I then open a shell editor (the Windows command shell can be accessed by typing \texttt{cmd} in the Search bar), and navigate to the \textbf{R} bin\textbackslash x64 directory. For my Windows machine, the address will be: \texttt{C:\textbackslash{}Program\ Files\textbackslash{}R-4.4.1\textbackslash{}bin\textbackslash{}x64}.

The shell language for Windows is somewhat similar to the POSIX (Portable Operating System Interface) compliant shell language generally used by Unix-like systems (guidance can be found \href{https://learn.microsoft.com/en-us/windows-server/administration/windows-commands/windows-commands}{here}). For instance, the command \texttt{cd} changes directories, and the command \texttt{cd\ ..} navigates \emph{up} (toward the root directory).

\index[terms]{POSIX (Portable Operating System Interface)}

I compile \texttt{FtoC.f90} using the script \texttt{R\ CMD\ SHLIB\ FtoC.f90}. Thus, at the shell command line I enter:

\vspace{0.1in}
\small

\begin{Shaded}
\begin{Highlighting}[]
\NormalTok{cd C:\textbackslash{}Program Files\textbackslash{}R\textbackslash{}R{-}4.4.2\textbackslash{}R\textbackslash{}bin\textbackslash{}x64  }
\NormalTok{R CMD SHLIB C:/Users/ahoken/Documents/Amalgam/Amalgam\_Bookdown/scripts/FtoC.f90}
\end{Highlighting}
\end{Shaded}

\normalsize

Note the change from back slashes to (Unix-style) forward slashes when specifying addresses for \texttt{SHLIB}.
The command above creates the compiled Fortran executable \texttt{FtoC.dll}. Specifically, the Fortran compiler, GNU Fortran (GCC), is used to create a Unix-style shared library \texttt{FtoC.o} (GCC is short for GNU compiler collection). This file is then converted to a .dll file, aided by the RTools GCC 10/MinGW-w64 compiler toolchain. By default, the .dll is saved the directory that contained the source code. The compilation process can be followed (with difficulty) in the cryptic shell output below:
\index[terms]{GNU compiler collection (GCC)}

\vspace{0.1in}

\begin{center}\includegraphics[width=1\linewidth,]{figs9/cl2_2n} \end{center}

\hfill\break
Here is an analogous C loop function for converting Fahrenheit to Celsius.
\vspace{0.1in}

\begin{Shaded}
\begin{Highlighting}[numbers=left,,]

\DataTypeTok{void}\NormalTok{ ftocc}\OperatorTok{(}\DataTypeTok{int} \OperatorTok{*}\NormalTok{nin}\OperatorTok{,} \DataTypeTok{double} \OperatorTok{*}\NormalTok{x}\OperatorTok{)}
\OperatorTok{\{}
  \DataTypeTok{int}\NormalTok{ n }\OperatorTok{=}\NormalTok{ nin}\OperatorTok{[}\DecValTok{0}\OperatorTok{];}
  \DataTypeTok{int}\NormalTok{ i}\OperatorTok{;}
  \ControlFlowTok{for} \OperatorTok{(}\NormalTok{i}\OperatorTok{=}\DecValTok{0}\OperatorTok{;}\NormalTok{ i}\OperatorTok{\textless{}}\NormalTok{n}\OperatorTok{;}\NormalTok{ i}\OperatorTok{++)}
\NormalTok{    x}\OperatorTok{[}\NormalTok{i}\OperatorTok{]} \OperatorTok{=} \OperatorTok{(}\NormalTok{x}\OperatorTok{[}\NormalTok{i}\OperatorTok{]} \OperatorTok{{-}} \DecValTok{32}\OperatorTok{)} \OperatorTok{*} \DecValTok{5} \OperatorTok{/} \DecValTok{9}\OperatorTok{;}
\OperatorTok{\}}
\end{Highlighting}
\end{Shaded}

The C code above consists of the following steps.

\begin{itemize}
\tightlist
\item
  Line 1 is a line break. This currently appears to be required for the compilation of C void functions in \texttt{SHLIB}.
\item
  On Line 2 a \texttt{void} function is initialized with two arguments. The code \texttt{int\ *nin} means ``access the value that \texttt{nin} points to and define it as an integer.'' The code \texttt{double\ *x} means: ``access the value that \texttt{x} points to and define it as double precision.''
\item
  Lines 8-9 define the C \texttt{for} loop. These loops have the general format: \texttt{for\ (\ init;\ condition;\ increment\ )\ \{statement(s);\ \}}. The \texttt{init} step is executed first and only once. Next the \texttt{condition} is evaluated. If true, the loop is executed. The syntax \texttt{i++} literally means: \texttt{i\ =\ i\ +\ 1}.
\end{itemize}

One again, I save the source code, \texttt{FtoCc.c}, within an appropriate directory. I compile the code using the command \texttt{R\ CMD\ SHLIB\ FtoCc.c}. Thus, at the shell command line I enter:
\vspace{0.1in}
\small

\begin{Shaded}
\begin{Highlighting}[]
\NormalTok{cd C:\textbackslash{}Program Files\textbackslash{}R\textbackslash{}R{-}4.4.2\textbackslash{}R\textbackslash{}bin\textbackslash{}x64  }
\NormalTok{R CMD SHLIB C:/Users/ahoken/Documents/Amalgam/Amalgam\_Bookdown/scripts/FtoCc.c }
\end{Highlighting}
\end{Shaded}

\normalsize

This creates the shared library executable \texttt{FtoCc.dll}.\\
\vspace{0.1in}

\begin{center}\includegraphics[width=1\linewidth,]{figs9/cl3_2n} \end{center}

Below is an \textbf{R}-wrapper that can call the Fortran executable, \texttt{call\ =\ "Fortran"}, the C executable, \texttt{call\ =\ "C"}, or use \textbf{R} looping, \texttt{call\ =\ "R"}. Several new functions are used. On Line 10 the function \texttt{dyn.load()} is used to load the shared Fortran library file \texttt{FtoC.dll}, while on Lines 14-15 \texttt{dyn.load()} loads the shared C library file \texttt{FtoCc.dll}. Note that the variable \texttt{nin} is pointed toward \texttt{n}, and \texttt{x} is included as an argument in \texttt{dyn.load()} on Line 15. On Line 11 the function \texttt{.Fortran()} is used to execute \texttt{FtoC.dll}, and on Line 16 \texttt{.C()} is used to execute \texttt{FtoCc.dll}.
\vspace{0.1in}
\footnotesize

\begin{Shaded}
\begin{Highlighting}[numbers=left,,]
\NormalTok{F2C }\OtherTok{\textless{}{-}} \ControlFlowTok{function}\NormalTok{(x, }\AttributeTok{call =} \StringTok{"R"}\NormalTok{)\{}
\NormalTok{  n }\OtherTok{\textless{}{-}} \FunctionTok{length}\NormalTok{(x)}
  \ControlFlowTok{if}\NormalTok{(call }\SpecialCharTok{==} \StringTok{"R"}\NormalTok{)\{}
\NormalTok{    out }\OtherTok{\textless{}{-}} \DecValTok{1}\SpecialCharTok{:}\NormalTok{n}
    \ControlFlowTok{for}\NormalTok{(i }\ControlFlowTok{in} \DecValTok{1}\SpecialCharTok{:}\NormalTok{n)\{}
\NormalTok{    out[i] }\OtherTok{\textless{}{-}}\NormalTok{ (x[i] }\SpecialCharTok{{-}} \DecValTok{32}\NormalTok{) }\SpecialCharTok{*}\NormalTok{ (}\DecValTok{5}\SpecialCharTok{/}\DecValTok{9}\NormalTok{)}
\NormalTok{    \}}
\NormalTok{  \}}
  \ControlFlowTok{if}\NormalTok{(call }\SpecialCharTok{==} \StringTok{"Fortran"}\NormalTok{)\{}
    \FunctionTok{dyn.load}\NormalTok{(}\StringTok{"C:/Users/ahoken/Documents/Amalgam/Amalgam\_Bookdown/scripts/FtoC.dll"}\NormalTok{)}
\NormalTok{    out }\OtherTok{\textless{}{-}} \FunctionTok{.Fortran}\NormalTok{(}\StringTok{"ftoc"}\NormalTok{, }\AttributeTok{n =} \FunctionTok{as.integer}\NormalTok{(n), }\AttributeTok{x =} \FunctionTok{as.double}\NormalTok{(x))}
\NormalTok{  \}}
  \ControlFlowTok{if}\NormalTok{(call }\SpecialCharTok{==} \StringTok{"C"}\NormalTok{)\{}
    \FunctionTok{dyn.load}\NormalTok{(}\StringTok{"C:/Users/ahoken/Documents/Amalgam/Amalgam\_Bookdown/scripts/FtoCc.dll"}\NormalTok{,}
             \AttributeTok{nin =}\NormalTok{ n, x)}
\NormalTok{    out }\OtherTok{\textless{}{-}} \FunctionTok{.C}\NormalTok{(}\StringTok{"ftocc"}\NormalTok{, }\AttributeTok{n =} \FunctionTok{as.integer}\NormalTok{(n), }\AttributeTok{x =} \FunctionTok{as.double}\NormalTok{(x))}
\NormalTok{  \}}
\NormalTok{  out}
\NormalTok{\}}
\end{Highlighting}
\end{Shaded}

\normalsize

Here I create \(10^8\) potential Fahrenheit temperatures that will be converted to Celsius using (unnecessary) looping.
\vspace{0.1in}

\begin{Shaded}
\begin{Highlighting}[]
\NormalTok{x }\OtherTok{\textless{}{-}} \FunctionTok{runif}\NormalTok{(}\DecValTok{100000000}\NormalTok{, }\DecValTok{0}\NormalTok{, }\DecValTok{100}\NormalTok{)}
\FunctionTok{head}\NormalTok{(x)}
\end{Highlighting}
\end{Shaded}

\begin{verbatim}
[1] 76.391 68.578 12.061 71.886 66.711 72.958
\end{verbatim}

Note first that the Fortran, C, and \textbf{R} loops provide identical temperature transformations. Here are first 6 transformations:
\vspace{0.1in}

\begin{Shaded}
\begin{Highlighting}[]
\FunctionTok{head}\NormalTok{(}\FunctionTok{F2C}\NormalTok{(x[}\DecValTok{1}\SpecialCharTok{:}\DecValTok{10}\NormalTok{], }\StringTok{"Fortran"}\NormalTok{)}\SpecialCharTok{$}\NormalTok{x)}
\end{Highlighting}
\end{Shaded}

\begin{verbatim}
[1]  24.661  20.321 -11.077  22.159  19.284  22.755
\end{verbatim}

\vspace{0.1in}

\begin{Shaded}
\begin{Highlighting}[]
\FunctionTok{head}\NormalTok{(}\FunctionTok{F2C}\NormalTok{(x[}\DecValTok{1}\SpecialCharTok{:}\DecValTok{10}\NormalTok{], }\StringTok{"C"}\NormalTok{)}\SpecialCharTok{$}\NormalTok{x)}
\end{Highlighting}
\end{Shaded}

\begin{verbatim}
[1]  24.661  20.321 -11.077  22.159  19.284  22.755
\end{verbatim}

\vspace{0.1in}

\begin{Shaded}
\begin{Highlighting}[]
\FunctionTok{head}\NormalTok{(}\FunctionTok{F2C}\NormalTok{(x[}\DecValTok{1}\SpecialCharTok{:}\DecValTok{10}\NormalTok{], }\StringTok{"R"}\NormalTok{))}
\end{Highlighting}
\end{Shaded}

\begin{verbatim}
[1]  24.661  20.321 -11.077  22.159  19.284  22.755
\end{verbatim}

However, the run times are dramatically different\footnote{Run on an Intel Core processor with a clock speed of 3 GHz, and 32 GB of RAM.}. The C executable is much faster than \textbf{R}, and the venerable Fortran executable is even faster than C!
\vspace{0.1in}

\begin{Shaded}
\begin{Highlighting}[]
\FunctionTok{system.time}\NormalTok{(}\FunctionTok{F2C}\NormalTok{(x, }\StringTok{"Fortran"}\NormalTok{))}
\end{Highlighting}
\end{Shaded}

\begin{verbatim}
   user  system elapsed 
   0.51    0.24    0.75 
\end{verbatim}

\vspace{0.1in}

\begin{Shaded}
\begin{Highlighting}[]
\FunctionTok{system.time}\NormalTok{(}\FunctionTok{F2C}\NormalTok{(x, }\StringTok{"C"}\NormalTok{))}
\end{Highlighting}
\end{Shaded}

\begin{verbatim}
   user  system elapsed 
   0.56    0.33    0.89 
\end{verbatim}

\vspace{0.1in}

\begin{Shaded}
\begin{Highlighting}[]
\FunctionTok{system.time}\NormalTok{(}\FunctionTok{F2C}\NormalTok{(x, }\StringTok{"R"}\NormalTok{))}
\end{Highlighting}
\end{Shaded}

\begin{verbatim}
   user  system elapsed 
   5.43    0.33    5.77 
\end{verbatim}

\end{example}

\(\blacksquare\)

\index[R]{\texttt{R CMD SHLIB}}
\index[R]{\texttt{system.time()}}
\index[R]{\texttt{dyn.load()}}
\index[R]{\texttt{.Fortran()}}
\index[R]{\texttt{.C()}}

\section{C++}\label{c}

\section{Python}\label{python}

\textbf{Python}, whose image logo is shown in Fig \ref{fig:pysym}, is similar to \textbf{R} in several respects. Python was formally introduced in the early 90s, is an open source OOP language that is rapidly gaining popularity, and its user code is evaluated in an on-the-fly manner. That is Python, like \textbf{R}, is an interpreted language.

\begin{figure}[H]

{\centering \includegraphics[width=0.6\linewidth,]{figs9/python} 

}

\caption{The symbol for Python, a high-level, general-purpose, programming language.}\label{fig:pysym}
\end{figure}

Like \textbf{R}, comments in Python are made using the metacharacter \#\footnote{A Python comment spanning multiple lines can be implemented by enclosing the comment in triple quotes (\texttt{"""} or \texttt{\textquotesingle{}\textquotesingle{}\textquotesingle{}}).}. Boolean operators are similar, although, while the unary operator for ``not'' in \textbf{R} is \texttt{!}, in Python it actually \emph{is} \texttt{not}, and Python uses \texttt{True} and \texttt{False} instead of \texttt{TRUE} and \texttt{FALSE}.

There are, however, several fundamental differences. These include the fact that while white spaces in \textbf{R} code (including tabs) simply reflect coding style preferences --for example, to increase code clarity-- Python indentations denote code blocks\footnote{In computer science this is called \emph{significant indentation}, or the \emph{off-side} rule.}\index[terms]{Significant indentation}. That is, Python indentations serve the same purpose as \textbf{R} curly braces. Another important difference is that \textbf{R} object names can contain a \texttt{.} (dot), while in Python \texttt{.} means: ``attribute in a namespace.'' Useful guidance for converting \textbf{R} code to analogous Python code can be found \href{https://www.mit.edu/~amidi/teaching/data-science-tools/conversion-guide/r-python-data-manipulation/}{here}.

Python can be downloaded for free from (\url{https://www.python.org/downloads/}), and can be activated from the Windows command line using the command \texttt{py}, and activated from the Mac and Unix/Linux command line using the command \texttt{python} (Fig \ref{fig:py1}). As with previous sections on Fortran, C, and C++, this short section is not meant to be a thorough introduction to Python. General guidance for the Python language can be found at (\url{https://docs.python.org/}) and many other sources including \href{https://wiki.python.org/moin/PythonBooks}{these books}.

\begin{figure}[H]

{\centering \includegraphics[width=1\linewidth,]{figs9/py_cl1} 

}

\caption{The Python command line interface in Windows.}\label{fig:py1}
\end{figure}

Note that the standard command line prompt for the Python shell is \texttt{\textgreater{}\textgreater{}\textgreater{}}. We can exit Python from the command line by typing \texttt{quit()}.
\index[terms]{Python!functions/operators!\texttt{>>>}}
\index[terms]{Python!functions/operators!\texttt{quit()}}

\subsection{\texorpdfstring{\emph{reticulate}}{reticulate}}\label{reticulate}

\index[terms]{R!package!\textit{reticulate}}
Because our primary interest is interfacing Python and \textbf{R}, and not Python itself, we will use \textbf{R} as our base of operations. This will require the \textbf{R} package \emph{reticulate}.\\
\vspace{0.1in}

\begin{Shaded}
\begin{Highlighting}[]
\CommentTok{\# install.packages("reticulate")}
\FunctionTok{library}\NormalTok{(reticulate)}
\end{Highlighting}
\end{Shaded}

RStudio (via \emph{reticulate}) can be used as an IDE for Python\footnote{Many IDEs have been developed specifically for Python, although quite a few are proprietary. Free IDEs include a primitive Python-bundled interface called IDLE. IDLE can be opened from the Windows command line using: \texttt{Path\ to\ python.exe\textbackslash{}python.exe\ -m\ idlelib}, \href{https://jupyter.org/}{Jupyter Notebook}, a web-based IDE, with many useful features, including support for \textbf{R} and Markdown-driven workflow documentation, \href{https://www.spyder-ide.org/}{Spyder}, a widely used IDE, (e.g., \citet{pine2019}), \href{https://pythontoolkit.sourceforge.net/}{Python Toolkit}, which hasn't been updated for a while, and \href{https://www.jetbrains.com/pycharm/}{pycharm} (which also has a commercial version).
  \index[terms]{Python!Spyder IDE}
  \index[terms]{Python!Python Toolkit IDE}
  \index[terms]{Python!pycharm IDE}
  \index[terms]{Jupyter notebook}}. In this capacity RStudio will:

\begin{itemize}
\tightlist
\item
  Generate a Python specific environment (to allow separate listing of Python and \textbf{R} objects).
\item
  Call separate \textbf{R} and Python environments, depending on which language is currently used in a chunk. Python code can be run in \textbf{R} Markdown (via RStudio) by defining \texttt{python} (instead of \texttt{r}) as the first option in an \textbf{R} Markdown chunk.
\end{itemize}

We can specify a path to the Python system (allowing us to use different versions of Python) with \texttt{reticulate::use\_python}. This is important because specific versions of Python may dramatically affect the usability of basic Python functions. The code below specifies use of the current version of Python, as accessed with \texttt{Sys.which()}, which finds full paths to program executables.
\vspace{0.1in}

\begin{Shaded}
\begin{Highlighting}[]
\FunctionTok{use\_python}\NormalTok{(}\FunctionTok{Sys.which}\NormalTok{(}\StringTok{"python"}\NormalTok{))}
\end{Highlighting}
\end{Shaded}

\index[R]{\texttt{reticulate::use\_python()}}
\index[R]{\texttt{Sys.which()}}
A Python command line interface can also be called directly in \textbf{R} using:
\vspace{0.1in}

\begin{Shaded}
\begin{Highlighting}[]
\FunctionTok{repl\_python}\NormalTok{()}
\end{Highlighting}
\end{Shaded}

Python can be closed from the resulting interface (returning one to \textbf{R}) by typing:
\vspace{0.1in}

\begin{Shaded}
\begin{Highlighting}[]
\NormalTok{exit}
\end{Highlighting}
\end{Shaded}

One can obtain information about the version of Python currently being used by \emph{reticulate} by running the function \texttt{retuclate::py\_config} (in \textbf{R}).

\vspace{0.1in}

\begin{Shaded}
\begin{Highlighting}[]
\NormalTok{reticulate}\SpecialCharTok{::}\FunctionTok{py\_config}\NormalTok{()}
\end{Highlighting}
\end{Shaded}

\scriptsize

\begin{verbatim}
python:         C:/Users/ahoken/AppData/Local/Programs/Python/Python311/python.exe
libpython:      C:/Users/ahoken/AppData/Local/Programs/Python/Python311/python311.dll
pythonhome:     C:/Users/ahoken/AppData/Local/Programs/Python/Python311
version:        3.11.3 (tags/v3.11.3:f3909b8, Apr  4 2023, 23:49:59) [MSC v.1934 64 bit (AMD64)]
Architecture:   64bit
numpy:          C:/Users/ahoken/AppData/Local/Programs/Python/Python311/Lib/site-packages/numpy
numpy_version:  1.24.2

NOTE: Python version was forced by RETICULATE_PYTHON_FALLBACK
\end{verbatim}

\normalsize

\begin{example}
\protect\hypertarget{exm:exm-c92}{}\label{exm:exm-c92}\(\text{}\)\\

The following are Python operations, run directly from RStudio.

\vspace{0.1in}

\begin{Shaded}
\begin{Highlighting}[]
\DecValTok{2} \OperatorTok{+} \DecValTok{2}
\end{Highlighting}
\end{Shaded}

\begin{verbatim}
4
\end{verbatim}

The Python assignment operator is \texttt{=}.
\vspace{0.1in}

\begin{Shaded}
\begin{Highlighting}[]
\NormalTok{x }\OperatorTok{=} \DecValTok{2}
\NormalTok{x }\OperatorTok{+}\NormalTok{ x}
\end{Highlighting}
\end{Shaded}

\begin{verbatim}
4
\end{verbatim}

Here we see the aforementioned importance of indentation.
\vspace{0.1in}

\begin{Shaded}
\begin{Highlighting}[]
\ControlFlowTok{if}\NormalTok{ x }\OperatorTok{\textless{}} \DecValTok{0}\NormalTok{:}
  \BuiltInTok{print}\NormalTok{(}\StringTok{"negative"}\NormalTok{)}
\ControlFlowTok{else}\NormalTok{:}
  \BuiltInTok{print}\NormalTok{(}\StringTok{"positive"}\NormalTok{)}
\end{Highlighting}
\end{Shaded}

\begin{verbatim}
positive
\end{verbatim}

Lack of an indented ``block'' following \texttt{if} will produce an error. Indentations in code can be made flexibly (e.g., one space, two space, tab, etc.) but they should be used consistently.
\end{example}

\(\blacksquare\)

\index[terms]{Python!functions/operators!\texttt{print()}}
\index[terms]{Python!functions/operators!\texttt{if()}}
\index[terms]{Python!indentation}

\subsection{Packages}\label{packages}

Like \textbf{R}, Python consists of a core language, a set of built-in functions, modules, and libraries (i.e., the \href{https://docs.python.org/3/library/index.html}{\emph{Python standard library}})\index[terms]{Python!standard library}, and a vast collection (\href{https://www.activestate.com/blog/top-10-must-have-python-packages/}{\(>200,000\)}) of supplemental libraries. Imported libraries are extremely important in Python because its distributed version has limited functional capabilities (compared to \textbf{R}). A number of important Python supplemental libraries, each of which contain multiple packages, are shown in Table \ref{tab:pypack}.

\begin{table}[ht]
\centering
\caption{Important supplemental Python libraries. For more information use hyperlinks.} 
\label{tab:pypack}
\begin{tabular}{ll}
  \toprule
Library & Purpose \\ 
  \midrule
\href{https://docs.scipy.org/doc/numpy/reference/index.html}{\textit{sumpy}} &  Fundamental package for scientific computing \\ 
  \href{https://docs.scipy.org/doc/scipy/reference}{\textit{scipy}} & Mathematical functions and routines \\ 
  \href{https//matplotlib.sourceforge.net}{\textit{matplotlib}} &  2- and 3-dimensional plots \\ 
  \href{https://pandas.pydata.org}{\textit{pandas}} &  Data manipulation and analysis \\ 
  \href{https://www.sympy.org/en/index.html}{\textit{sympy}} & Symbolic mathematics \\ 
  \href{https://docs.bokeh.org/en/latest/index.html}{\textit{bokeh}} & Interactive data visualizations  \\ 
   \bottomrule
\end{tabular}
\end{table}

\index[terms]{Python!package!\textit{pandas}}
\index[terms]{Python!package!\textit{sympy}}
\index[terms]{Python!package!\textit{bokeh}}
\index[terms]{Python!package!\textit{matplotlib}}
\index[terms]{Python!package!\textit{NumPy}}
\index[terms]{Python!package!\textit{SciPy}}

We can install Python packages and libraries using the \emph{pip} package manager for Python\footnote{The name ``pip'' is recursive acronym for ``Pip Installs Packages.''}. Installation only needs to occur once on a workstation (similar to \texttt{install.packages()} in \textbf{R}). Following installation, one can load a package for a particular work session using the Python function \texttt{import} (analogous to \texttt{library()} in \textbf{R})\footnote{Loading Python libraries (aside from \emph{numpy}) in \texttt{reticulate} will produce an error if one specifies a Python location in \texttt{use\_python()} that does not contain the installed libraries.}.

Installation of a Python package, \emph{foo}, with \emph{reticulate}, via pip, can be accomplished using the function \texttt{reticulate::py\_install} (in \textbf{R})\footnote{This approach generally works well. If problems occur loading libraries \texttt{reticulate::py\_install} one can download libraries from the command line using using \texttt{pip\ install} \emph{foo}.}.
\index[terms]{Python!pip}
\vspace{0.1in}

\begin{Shaded}
\begin{Highlighting}[]
\FunctionTok{py\_install}\NormalTok{(}\StringTok{"foo"}\NormalTok{, }\AttributeTok{pip =} \ConstantTok{TRUE}\NormalTok{)}
\end{Highlighting}
\end{Shaded}

\index[R]{\texttt{reticulate::py\_install()}}

For example, to install the \emph{scipy} library I use the command:
\vspace{0.1in}

\begin{Shaded}
\begin{Highlighting}[]
\FunctionTok{py\_install}\NormalTok{(}\StringTok{"scipy"}\NormalTok{, }\AttributeTok{pip =} \ConstantTok{TRUE}\NormalTok{) }\CommentTok{\# Run in R, if scipy has not been installed}
\end{Highlighting}
\end{Shaded}

To load the \emph{scipy} library I could use the Python function \texttt{import()}:
\vspace{0.1in}

\begin{Shaded}
\begin{Highlighting}[]
\ImportTok{import}\NormalTok{ scipy }
\end{Highlighting}
\end{Shaded}

\index[terms]{Python!functions/operators!\texttt{import()}}

\subsection{Functions in Packages}\label{functions-in-packages}

Functions within Python packages are obtained using a \texttt{package.function} syntax.
Here I import \emph{numpy} and run the function \texttt{pi} (which is contained in \emph{numpy}).
\vspace{0.1in}

\begin{Shaded}
\begin{Highlighting}[]
\ImportTok{import}\NormalTok{ numpy}
\NormalTok{numpy.pi}
\end{Highlighting}
\end{Shaded}

\begin{verbatim}
3.141592653589793
\end{verbatim}

If we are writing a lot of \emph{mumpy} functions, Python will allow you to define a simplified library prefix. For instance, here I created a shortcut for \emph{numpy} called \texttt{np} and use this shortcut to access the \emph{numpy} functions \texttt{pi()} and \texttt{sin()}.
\vspace{0.1in}

\begin{Shaded}
\begin{Highlighting}[]
\ImportTok{import}\NormalTok{ numpy }\ImportTok{as}\NormalTok{ np}
\NormalTok{np.sin(}\DecValTok{20} \OperatorTok{*}\NormalTok{ np.pi}\OperatorTok{/}\DecValTok{180}\NormalTok{) }\CommentTok{\# sin(20 degrees)}
\end{Highlighting}
\end{Shaded}

\begin{verbatim}
0.3420201433256687
\end{verbatim}

\index[terms]{Python!functions/operators!\texttt{numpy.pi()}}
\index[terms]{Python!functions/operators!\texttt{numpy.sin()}}

Use of the command \texttt{from\ numpy\ import\ *} would cause names of functions from \emph{NumPy} to overwrite functions with the same name from other packages. That is, we could run \texttt{numpy.pi} simply using \texttt{pi}.
\index[terms]{Python!functions/operators!\texttt{*}}

Here we import the package \emph{pyplot} from the library \emph{matplotlib}, rename the package \texttt{plt}, and create a plot using the function \texttt{pyplot.plot()} (as \texttt{plt.plot()}) by calling:
\vspace{0.1in}

\begin{Shaded}
\begin{Highlighting}[numbers=left,,]
\ImportTok{import}\NormalTok{ matplotlib.pyplot }\ImportTok{as}\NormalTok{ plt}
\NormalTok{plt.plot(}\BuiltInTok{range}\NormalTok{(}\DecValTok{10}\NormalTok{), }\StringTok{\textquotesingle{}bo\textquotesingle{}}\NormalTok{)}
\end{Highlighting}
\end{Shaded}

\includegraphics{figure/minimal-unnamed-chunk-732-1}

In Line 2, the command \texttt{range(10)} creates a sequence of integers from zero to ten. This is used as the first argument of \texttt{plt.plot()}, which specifies the plot \(x\)-coordinates. If \(y\) coordinates are not specified in the second argument, \(x\)-coordinates will be reused as \(y\) coordinates. The command \texttt{\textquotesingle{}bo\textquotesingle{}} places blue filled circles at \(x\),\(y\) coordinates. Documentation for \texttt{matplotlib.pyplot.plot()} can be found \href{https://matplotlib.org/stable/api/_as_gen/matplotlib.pyplot.plot.html}{here}.

\index[terms]{Python!functions/operators!\texttt{matplotlib.pyplot.plot()}}
\index[terms]{Python!functions/operators!\texttt{range()}}

\subsection{Dataset Storage Types}\label{dataset-storage-types}

There are four different built-in dataset storage types in Python: lists, tuples, sets, and dictionaries (Table \ref{tab:dso}). Data storage types of Python objects can be identified with the Python function \texttt{type()}.
\index[terms]{Python!list}
\index[terms]{Python!tuple}
\index[terms]{Python!set}
\index[terms]{Python!dictionary}
\index[terms]{Python!functions/operators!\texttt{type()}}
\index[terms]{Python!functions/operators!\texttt{type()}}

\begin{landscape}
\begin{table}[ht]
\centering
\caption{The four Python dataset storage types.} 
\label{tab:dso}
\begin{tabular}{lll}
  \toprule
Storage type & Example & Entry characteristics \\ 
  \midrule
List    &  \texttt{["hot","cold"]}   & Ordered entries,  Changeable,  Mult. data storage types,  Duplicates OK. \\ 
  Tuple    &  \texttt{("hot","cold")}  & Ordered entries,  Unchangeable, Mult. data storage types,  Duplicates OK. \\ 
  Set    &  \texttt{\{"hot","cold"\}}  & Ordered entries,  Unchangeable,  Mult. data storage types,  Duplicates not OK. \\ 
  Dictionary   &  \texttt{\{"temp":["hot", cold"]\}}  & Ordered entries,  Changeable, Mult. data storage types,  Duplicates not OK. \\ 
   \bottomrule
\end{tabular}
\end{table}
\end{landscape}

We can make a \emph{Python} list, which can contain both text and numeric data, using square brackets or the function \texttt{list()}.
\index[terms]{Python!functions/operators!\texttt{list()}}
\vspace{0.1in}

\begin{Shaded}
\begin{Highlighting}[]
\NormalTok{a }\OperatorTok{=}\NormalTok{ [}\DecValTok{20}\NormalTok{, }\DecValTok{7}\NormalTok{, }\StringTok{"Hi"}\NormalTok{, }\DecValTok{7}\NormalTok{, }\StringTok{"end"}\NormalTok{]}
\end{Highlighting}
\end{Shaded}

An empty list can be specified as \texttt{{[}{]}}

\begin{Shaded}
\begin{Highlighting}[]
\NormalTok{empty }\OperatorTok{=}\NormalTok{ []}
\NormalTok{empty}
\end{Highlighting}
\end{Shaded}

\begin{verbatim}
[]
\end{verbatim}

Like \textbf{R}, we can index list elements using square brackets. Importantly, \texttt{a{[}0{]}} refers to the first element of the list \texttt{a}.
\vspace{0.1in}

\begin{Shaded}
\begin{Highlighting}[]
\NormalTok{a[}\DecValTok{0}\NormalTok{]}
\end{Highlighting}
\end{Shaded}

\begin{verbatim}
20
\end{verbatim}

And the third element would be:
\vspace{0.1in}

\begin{Shaded}
\begin{Highlighting}[]
\NormalTok{a[}\DecValTok{3}\NormalTok{]}
\end{Highlighting}
\end{Shaded}

\begin{verbatim}
7
\end{verbatim}

Square brackets can also be used to reassign list values
\vspace{0.1in}

\begin{Shaded}
\begin{Highlighting}[]
\NormalTok{a[}\DecValTok{3}\NormalTok{] }\OperatorTok{=} \DecValTok{10}
\NormalTok{a}
\end{Highlighting}
\end{Shaded}

\begin{verbatim}
[20, 7, 'Hi', 10, 'end']
\end{verbatim}

We can use the function \texttt{.append()} to append entries to the end of list. For instance, to append the number 9 to the object \texttt{a} in the previous example, I could type:
\vspace{0.1in}

\begin{Shaded}
\begin{Highlighting}[]
\NormalTok{a.append(}\DecValTok{9}\NormalTok{)}
\NormalTok{a}
\end{Highlighting}
\end{Shaded}

\begin{verbatim}
[20, 7, 'Hi', 10, 'end', 9]
\end{verbatim}

\index[terms]{Python!functions/operators!\texttt{.append()}}

Unlike a Python list, a data object called a tuple, which is designated using parentheses, contains elements that cannot be changed:
\vspace{0.1in}

\begin{Shaded}
\begin{Highlighting}[]
\NormalTok{b }\OperatorTok{=}\NormalTok{ (}\DecValTok{1}\NormalTok{,}\DecValTok{2}\NormalTok{,}\DecValTok{3}\NormalTok{,}\DecValTok{4}\NormalTok{,}\DecValTok{5}\NormalTok{)}
\NormalTok{b[}\DecValTok{0}\NormalTok{]}
\end{Highlighting}
\end{Shaded}

\begin{verbatim}
1
\end{verbatim}

\vspace{0.1in}

\begin{Shaded}
\begin{Highlighting}[]
\NormalTok{b[}\DecValTok{0}\NormalTok{] }\OperatorTok{=} \DecValTok{10} \CommentTok{\# produces error}
\end{Highlighting}
\end{Shaded}

Multidimensional numerical arrays, including matrices, can be created using functions from \emph{numpy}. Here we define:

\[\boldsymbol{B} =
\begin{bmatrix}
1 & 4 & 5 \\
9 & 7.2 & 4
\end{bmatrix}
\]

and find \(\boldsymbol{B} - 5\).
\vspace{0.1in}

\begin{Shaded}
\begin{Highlighting}[]
\NormalTok{B }\OperatorTok{=}\NormalTok{ np.array([[}\DecValTok{1}\NormalTok{, }\DecValTok{4}\NormalTok{, }\DecValTok{5}\NormalTok{], [}\DecValTok{9}\NormalTok{, }\FloatTok{7.2}\NormalTok{, }\DecValTok{4}\NormalTok{]])}
\NormalTok{B}
\end{Highlighting}
\end{Shaded}

\begin{verbatim}
array([[1. , 4. , 5. ],
       [9. , 7.2, 4. ]])
\end{verbatim}

\vspace{0.1in}

\begin{Shaded}
\begin{Highlighting}[]
\NormalTok{B }\OperatorTok{{-}} \DecValTok{5}
\end{Highlighting}
\end{Shaded}

\begin{verbatim}
array([[-4. , -1. ,  0. ],
       [ 4. ,  2.2, -1. ]])
\end{verbatim}

\index[terms]{Python!functions/operators!\texttt{numpy.array()}}

Extensive linear algebra tools are contained in the libraries \emph{numpy} and \emph{scipy}.

\subsection{Mathematical Operations}\label{mathematical-operations}

Basic Python mathematical operators are generally (but not always) identical to \textbf{R}. For instance, note that for exponentiation \texttt{**} is used instead of \texttt{\^{}} (Table \ref{tab:pymath}).

\begin{table}[ht]
\centering
\caption{Basic Python mathematical functions and operators.} 
\label{tab:pymath}
\begin{tabular}{llll}
  \toprule
Operator & Operation & To find & We type \\ 
  \midrule
\texttt{+} & addition & $2 + 2$ & \texttt{2 + 2} \\ 
  \texttt{-} & subtraction & $2 - 2$ & \texttt{2 - 2} \\ 
  \texttt{*} & multiplication & $2 \times 2$ & \texttt{2 * 2} \\ 
  \texttt{/} & division & $\frac{2}{3}$ & \texttt{2/3} \\ 
  \texttt{**} & exponentiation & $2^3$ & \texttt{2**3} \\ 
  \texttt{sqrt(x)} & $\sqrt{x}$ & $\sqrt{2}$ & \texttt{numpy.sqrt(2)} \\ 
  \texttt{factorial(x)} & $x!$ & $5!$ & \texttt{numpy.math.factorial(5)} \\ 
  \texttt{pi} & $\pi = 3.141593 \dots$ & $\pi$ & \texttt{numpy.pi} \\ 
  \texttt{log} & $\log_e$ & $\log_e(3)$ & \texttt{numpy.log} \\ 
   \bottomrule
\end{tabular}
\end{table}

\index[terms]{Python!functions/operators!\texttt{**} (exponentiation)}
\index[terms]{Python!functions/operators!\texttt{exp()}}

Symbolic derivative solutions to functions can be obtained using functions from the library \emph{sympy}. Results from the package functions can be printed in LaTeX for pretty mathematics.
\vspace{0.1in}

\begin{Shaded}
\begin{Highlighting}[]
\FunctionTok{py\_install}\NormalTok{(}\StringTok{"sympy"}\NormalTok{, }\AttributeTok{pip =} \ConstantTok{TRUE}\NormalTok{) }\CommentTok{\# run in R if sympy hasn\textquotesingle{}t been installed}
\end{Highlighting}
\end{Shaded}

Here we solve:
\[\frac{d}{dx} 3e^{-x^2}\]
\vspace{0.1in}

\begin{Shaded}
\begin{Highlighting}[numbers=left,,]
\ImportTok{from}\NormalTok{ sympy }\ImportTok{import} \OperatorTok{*}
\NormalTok{x }\OperatorTok{=}\NormalTok{ symbols (}\StringTok{\textquotesingle{}x\textquotesingle{}}\NormalTok{)}
\NormalTok{fx }\OperatorTok{=} \DecValTok{3} \OperatorTok{*}\NormalTok{ exp(}\OperatorTok{{-}}\NormalTok{x }\OperatorTok{**} \DecValTok{2}\NormalTok{)}
\BuiltInTok{print}\NormalTok{(diff(fx))}
\end{Highlighting}
\end{Shaded}

\[- 6 x e^{- x^{2}}\]
In Line 2, \texttt{x} is defined symbolically using the \texttt{sympy.symbols()} function. The variable \texttt{x} is used as a term in the expression \texttt{fx} in Line 3. The function \texttt{fx} is differentiated in Line 4 using the function \texttt{sympy.diff()}.

\index[terms]{Python!functions/operators!\texttt{sympy.diff()}}
\index[terms]{Python!functions/operators!\texttt{sympy.symbols()}}

Integration in Python can be handled with the function \texttt{quad()} in \emph{scipy}. Here we find:
\[\int_0^1 3e^{-x^2} dx\]

To perform integration we must install the \emph{scipy.integrate} library using \emph{pip} and bring in the function \texttt{quad()}.
\index[terms]{Python!functions/operators!\texttt{scipy.integrate.quad()}}
\index[terms]{Python!functions/operators!\texttt{scipy.integrate.def()}}

\vspace{0.1in}

\begin{Shaded}
\begin{Highlighting}[]
\ImportTok{from}\NormalTok{ scipy.integrate }\ImportTok{import}\NormalTok{ quad}
\end{Highlighting}
\end{Shaded}

We then define the integrand as a Python function using the function \texttt{def()}. That is, \texttt{def()} is analogous to \texttt{function()} in \textbf{R}.
\vspace{0.1in}

\begin{Shaded}
\begin{Highlighting}[]
\KeywordTok{def}\NormalTok{ f(x):}
  \ControlFlowTok{return} \DecValTok{3} \OperatorTok{*}\NormalTok{ np.exp(}\OperatorTok{{-}}\NormalTok{x}\OperatorTok{**}\DecValTok{2}\NormalTok{)}
\end{Highlighting}
\end{Shaded}

We now run \texttt{quad()} on the user function \texttt{f} with the defined bounds of integration.
\vspace{0.1in}

\begin{Shaded}
\begin{Highlighting}[]
\NormalTok{quad(f, }\DecValTok{0}\NormalTok{, }\DecValTok{1}\NormalTok{)}
\end{Highlighting}
\end{Shaded}

\begin{verbatim}
(2.240472398437281, 2.487424042782217e-14)
\end{verbatim}

The first number is the value of the definite integral (in this case, the area under the function \texttt{f} from 0 to 1). The second is a measure of the absolute error in the numerical approximation.
\index[terms]{Python!functions/operators!\texttt{def()}}

\subsection{Reading in Data}\label{reading-in-data}

Data in delimited files, including .csv files, can be read into Python using the \emph{mumpy} function \texttt{loadtxt()}.

\begin{example}
\protect\hypertarget{exm:exm-c93}{}\label{exm:exm-c93}

\(\text{}\)\\
Assume that we have a comma separated dataset, named \texttt{ffall.csv}, located in the Python working directory, describing the free fall properties of some object over six seconds, with columns for observation number, time (in seconds), altitude (in mm) and uncertainty (in mm). The Python working directory (which need not be the same as the \textbf{R} working directory in RStudio) can be identified using the function \texttt{getcwd()} from the library \emph{os}.
\vspace{0.1in}

\begin{Shaded}
\begin{Highlighting}[]
\ImportTok{import}\NormalTok{ os}
\NormalTok{os.getcwd()}
\end{Highlighting}
\end{Shaded}

\begin{verbatim}
'C:\\Users\\ahoken\\Documents\\GitHub\\Amalgam'
\end{verbatim}

We can load \texttt{freefall.csv} using:
\vspace{0.1in}

\begin{Shaded}
\begin{Highlighting}[]
\NormalTok{obs, time, height, error }\OperatorTok{=}\NormalTok{ np.loadtxt(}\StringTok{"ffall.csv"}\NormalTok{, }
\NormalTok{delimiter }\OperatorTok{=} \StringTok{","}\NormalTok{, skiprows }\OperatorTok{=} \DecValTok{1}\NormalTok{, unpack }\OperatorTok{=} \VariableTok{True}\NormalTok{)}
\end{Highlighting}
\end{Shaded}

The first row was skipped (using \texttt{skiprows\ =\ 1}) because it contained column names and those were re-assigned when I brought in the data.
Note that, unlike \textbf{R}, columns in the dataset are automatically attached to the global environment upon loading, and will overwrite objects with the same name.
\vspace{0.1in}

\begin{Shaded}
\begin{Highlighting}[]
\NormalTok{height}\OperatorTok{/}\DecValTok{1000} \CommentTok{\# height in meters}
\end{Highlighting}
\end{Shaded}

\begin{verbatim}
array([0.18 , 0.182, 0.178, 0.165, 0.16 , 0.148, 0.136, 0.12 , 0.099,
       0.083, 0.055, 0.035, 0.005])
\end{verbatim}

File readers in \emph{pandas} are less clunky (and more similar to \textbf{R}).
We can bring in \texttt{freefall.csv} using the function \texttt{pandas.read\_csv()}:
\vspace{0.1in}

\begin{Shaded}
\begin{Highlighting}[]
\FunctionTok{py\_install}\NormalTok{(}\StringTok{"pandas"}\NormalTok{)  }\CommentTok{\# Run if pandas is not installed }
\end{Highlighting}
\end{Shaded}

\vspace{0.1in}

\begin{Shaded}
\begin{Highlighting}[]
\ImportTok{import}\NormalTok{ pandas }\ImportTok{as}\NormalTok{ pd }\CommentTok{\# run in a Python chunk}
\NormalTok{ffall }\OperatorTok{=}\NormalTok{ pd.read\_csv(}\StringTok{\textquotesingle{}ffall.csv\textquotesingle{}}\NormalTok{)}
\NormalTok{ffall}
\end{Highlighting}
\end{Shaded}

\begin{verbatim}
    obs  time  height  error
0     1   0.0     180   3.50
1     2   0.5     182   4.50
2     3   1.0     178   4.00
3     4   1.5     165   5.50
4     5   2.0     160   2.50
5     6   2.5     148   3.00
6     7   3.0     136   2.50
7     8   3.5     120   3.00
8     9   4.0      99   4.00
9    10   4.5      83   2.50
10   11   5.0      55   3.60
11   12   5.5      35   1.75
12   13   6.0       5   0.75
\end{verbatim}

The object \texttt{ffall} is a \emph{Pandas} \emph{DataFrame}, which is different in several respects, from an \textbf{R} dataframe. Column arrays in \texttt{ffall} can be called using the syntax: \texttt{ffall.}, or by using braces. For instance:
\vspace{0.1in}

\begin{Shaded}
\begin{Highlighting}[]
\NormalTok{ffall.height}
\end{Highlighting}
\end{Shaded}

\begin{verbatim}
0     180
1     182
2     178
3     165
4     160
5     148
6     136
7     120
8      99
9      83
10     55
11     35
12      5
Name: height, dtype: int64
\end{verbatim}

\begin{Shaded}
\begin{Highlighting}[]
\NormalTok{ffall[}\StringTok{"height"}\NormalTok{]}
\end{Highlighting}
\end{Shaded}

\begin{verbatim}
0     180
1     182
2     178
3     165
4     160
5     148
6     136
7     120
8      99
9      83
10     55
11     35
12      5
Name: height, dtype: int64
\end{verbatim}

\end{example}

\(\blacksquare\)

\index[terms]{Python!functions/operators!\texttt{os.getcwd()}}

In RStudio, \textbf{R} and Python (\texttt{reticulate}) sessions are considered separately. In this process, when accessing Python from \textbf{R}, \textbf{R} data types are automatically converted to their equivalent Python types. Conversely, When values are returned from Python to \textbf{R} they are converted back to \textbf{R} types. It is possible, however, to access each from the others' session.

The \emph{reticulate} operator \texttt{py} allows one to interact with a Python session directly from the \textbf{R} console. Here I convert the \emph{pandas} DataFrame \texttt{ffall} into a recognizable \textbf{R} dataframe, within \textbf{R}.\\
\vspace{0.1in}

\begin{Shaded}
\begin{Highlighting}[]
\NormalTok{ffallR }\OtherTok{\textless{}{-}}\NormalTok{ py}\SpecialCharTok{$}\NormalTok{ffall}
\end{Highlighting}
\end{Shaded}

Which allows me to examine it with \textbf{R} functions.
\vspace{0.1in}

\begin{Shaded}
\begin{Highlighting}[]
\FunctionTok{colMeans}\NormalTok{(ffallR)}
\end{Highlighting}
\end{Shaded}

\begin{verbatim}
     obs     time   height    error 
  7.0000   3.0000 118.9231   3.1615 
\end{verbatim}

On Lines 1 and 2 in the chunk below, I bring in the Python library \emph{pandas} into \textbf{R} with the function \texttt{reticulate:import()}. The code \texttt{pd\ \textless{}-\ import("pandas",\ convert\ =\ FALSE)} is the Python equivalent to: \texttt{import\ pandas\ as\ pd}.
\index[R]{\texttt{reticulate::import()}}
\index[R]{\texttt{reticulate::py}}
\vspace{0.1in}

\begin{Shaded}
\begin{Highlighting}[]
\NormalTok{pd }\OtherTok{\textless{}{-}} \FunctionTok{import}\NormalTok{(}\StringTok{"pandas"}\NormalTok{, }\AttributeTok{convert =} \ConstantTok{FALSE}\NormalTok{)}
\end{Highlighting}
\end{Shaded}

As expected, the column names constitute the names attribute of the dataframe \texttt{ffallR}.
\vspace{0.1in}

\begin{Shaded}
\begin{Highlighting}[]
\FunctionTok{names}\NormalTok{(ffallR)}
\end{Highlighting}
\end{Shaded}

\begin{verbatim}
[1] "obs"    "time"   "height" "error" 
\end{verbatim}

The \texttt{ffall} dataset, however, has different characteristics as a Python object. Note that in the code below the \emph{pandas} function \texttt{read\_csv()} is accessed using \texttt{pd\$read\_csv()} instead of \texttt{pd.read\_csv()} because an \textbf{R} chunk is being used.
\vspace{0.1in}

\begin{Shaded}
\begin{Highlighting}[]
\NormalTok{ffallP }\OtherTok{\textless{}{-}}\NormalTok{ pd}\SpecialCharTok{$}\FunctionTok{read\_csv}\NormalTok{(}\StringTok{"ffall.csv"}\NormalTok{)}
\end{Highlighting}
\end{Shaded}

The names attribute of the \emph{pandas} DataFrame \texttt{ffallP}, as perceived by \textbf{R}, contains over 200 entities, many of which are provided by the built-in Python module \emph{statistics}. Here are the first 20.
\vspace{0.1in}

\begin{Shaded}
\begin{Highlighting}[]
\FunctionTok{head}\NormalTok{(}\FunctionTok{names}\NormalTok{(ffallP), }\DecValTok{20}\NormalTok{)}
\end{Highlighting}
\end{Shaded}

\begin{verbatim}
 [1] "abs"        "add"        "add_prefix" "add_suffix" "agg"       
 [6] "aggregate"  "align"      "all"        "any"        "apply"     
[11] "applymap"   "asfreq"     "asof"       "assign"     "astype"    
[16] "at"         "at_time"    "attrs"      "axes"       "backfill"  
\end{verbatim}

I can use these entities to obtain statistical summaries of each column array, revealing an approach for \textbf{R}/Python syntheses.
\vspace{0.1in}

\begin{Shaded}
\begin{Highlighting}[]
\NormalTok{ffallP}\SpecialCharTok{$}\FunctionTok{mean}\NormalTok{()}
\end{Highlighting}
\end{Shaded}

\begin{verbatim}
obs         7.000000
time        3.000000
height    118.923077
error       3.161538
dtype: float64
\end{verbatim}

\vspace{0.1in}

\begin{Shaded}
\begin{Highlighting}[]
\NormalTok{ffallP}\SpecialCharTok{$}\FunctionTok{var}\NormalTok{()}
\end{Highlighting}
\end{Shaded}

\begin{verbatim}
obs         15.166667
time         3.791667
height    3495.243590
error        1.512147
dtype: float64
\end{verbatim}

\vspace{0.1in}

\begin{Shaded}
\begin{Highlighting}[]
\NormalTok{ffallP}\SpecialCharTok{$}\FunctionTok{kurt}\NormalTok{()}
\end{Highlighting}
\end{Shaded}

\begin{verbatim}
obs      -1.200000
time     -1.200000
height   -0.692166
error     0.445443
dtype: float64
\end{verbatim}

For further analysis in \textbf{R} these expression will need to be explicitly converted to \textbf{R} objects using the function \texttt{py\_to\_r()}.
\vspace{0.1in}

\begin{Shaded}
\begin{Highlighting}[]
\NormalTok{trans }\OtherTok{\textless{}{-}}\NormalTok{ ffallP}\SpecialCharTok{$}\FunctionTok{transpose}\NormalTok{() }\CommentTok{\# transpose matrix}
\NormalTok{transR }\OtherTok{\textless{}{-}} \FunctionTok{py\_to\_r}\NormalTok{(trans)}

\FunctionTok{apply}\NormalTok{(transR, }\DecValTok{1}\NormalTok{, mean)}
\end{Highlighting}
\end{Shaded}

\begin{verbatim}
     obs     time   height    error 
  7.0000   3.0000 118.9231   3.1615 
\end{verbatim}

\subsection{Python versus R}\label{pvR}

\textbf{R} allows much greater flexibility than Python for explicit statistical analyses and graphical summaries. For example, the Python statistics library \emph{Pymer4} actually uses generalized linear mixed effect model (see \citet{aho2014}) functions from the \textbf{R} package \emph{lme4} to complete computations. Additionally, Python tends to be \emph{less} efficient than \textbf{R} for pseudo-random number generation\footnote{A \emph{pseudo-random number generator} (PRNG) is a deterministic algorithm for generating numbers whose properties approximate those of random numbers \citep{wikrandom}. A PRNG sequence is dependent on an initial \emph{seed} value provided to the generator. By default, \textbf{R} PRNG seeds are generated from the current session time. Details are provided in \texttt{?RNG}\index[terms]{Pseudo-random number}. Numbers from distinct probability distributions can be generated from an underlying pseudo-random continuous uniform sequence by obtaining the corresponding inverse CDF outcomes for the same lower-tailed probability. \textbf{R} can employ a large number algorithmic approaches for generating pseudo-random numbers, including, by default, the ``Mersenne-Twister'' \citep{matsumoto1998}.}, since it requires looping to generate multiple pseudo-random outcomes (see \citet{python3}).

\begin{example}
\protect\hypertarget{exm:exm-Rpy}{}\label{exm:exm-Rpy}\(\text{}\)\\
Here I generate \(10^8\) pseudo-random outcomes from a continuous uniform distribution (processor details footnoted in Example \ref{exm:exm-c91}).\\
\textbf{R}:
\vspace{0.1in}

\begin{Shaded}
\begin{Highlighting}[]
\FunctionTok{system.time}\NormalTok{(ranR }\OtherTok{\textless{}{-}} \FunctionTok{runif}\NormalTok{(}\FloatTok{1e8}\NormalTok{))}
\end{Highlighting}
\end{Shaded}

\begin{verbatim}
   user  system elapsed 
   2.39    0.22    2.60 
\end{verbatim}

Python:
\vspace{0.1in}

\begin{Shaded}
\begin{Highlighting}[numbers=left,,]
\ImportTok{import}\NormalTok{ time}
\ImportTok{import}\NormalTok{ random}
\NormalTok{ranP }\OperatorTok{=}\NormalTok{ []}

\NormalTok{start\_time }\OperatorTok{=}\NormalTok{ time.time()}
\ControlFlowTok{for}\NormalTok{ i }\KeywordTok{in} \BuiltInTok{range}\NormalTok{(}\DecValTok{0}\NormalTok{,}\DecValTok{9999999}\NormalTok{):}
\NormalTok{  n }\OperatorTok{=}\NormalTok{ random.random()}
\NormalTok{  ranP.append(n)}
\NormalTok{time.time() }\OperatorTok{{-}}\NormalTok{ start\_time}
\end{Highlighting}
\end{Shaded}

\begin{verbatim}
5.708770990371704
\end{verbatim}

The operation takes much longer for Python than \textbf{R}.

The Python code above requires some explanation. On Lines 1 and 2, the Python modules \(time\) and \(random\) are loaded from the Python standard library, and on Line 3 an empty list \texttt{ranP} is created that will be filled as the loop commences. On Line 5, the start time for the operation is recorded using the function \texttt{time()} from the module \emph{time}. On Line 6 a sequence of length \(10^8\) is defined as a reference for the index variable \texttt{i} as the for loop commences. On Lines 7 and 8 a random number is generated using the function \texttt{random()} from the module \emph{random} and this number is appended to \texttt{ranP}. Note that Lines 7 and 8 are indented to indicate that they reside in the loop. Finally, on Line 9 the start time is subtracted from the end time to get the system time for the operation.
\end{example}

\(\blacksquare\)

\index[terms]{Python!package!\textit{time}}
\index[terms]{Python!package!\textit{random}}
\index[terms]{Python!functions/operators!\texttt{for()}}
\index[terms]{Python!functions/operators!\texttt{.append()}}
\index[terms]{Python!functions/operators!\texttt{random.random()}}
\index[terms]{Python!functions/operators!\texttt{time.time()}}
\index[R]{\texttt{runif()}}

On the other hand, the system time efficiency of Python may be better than \textbf{R} for many applications, including the management of large datasets \citep{morandat2012}.

\begin{example}
\protect\hypertarget{exm:exm-Rpy2}{}\label{exm:exm-Rpy2}\(\text{}\)\\
Here I add the randomly generated dataset to itself in \textbf{R}:
\vspace{0.1in}

\begin{Shaded}
\begin{Highlighting}[]
\FunctionTok{system.time}\NormalTok{(ranR }\SpecialCharTok{+}\NormalTok{ ranR)}
\end{Highlighting}
\end{Shaded}

\begin{verbatim}
   user  system elapsed 
   0.10    0.20    0.32 
\end{verbatim}

and Python:
\vspace{0.1in}

\begin{Shaded}
\begin{Highlighting}[]
\NormalTok{start\_time }\OperatorTok{=}\NormalTok{ time.time()}
\NormalTok{diff }\OperatorTok{=}\NormalTok{ ranP }\OperatorTok{+}\NormalTok{ ranP}
\NormalTok{time.time() }\OperatorTok{{-}}\NormalTok{ start\_time}
\end{Highlighting}
\end{Shaded}

\begin{verbatim}
0.10865020751953125
\end{verbatim}

For this operation, Python is faster.
\end{example}

\(\blacksquare\)

Of course, IDEs like RStudio allow, through the package \emph{reticulate}, simultaneous use of both \textbf{R} and Python systems, allowing one to draw on the strengths of each language.

\section*{Exercises}\label{exercises-8}
\addcontentsline{toc}{section}{Exercises}

\begin{enumerate}
\def\labelenumi{\arabic{enumi}.}
\tightlist
\item
  The Fortran script below calculates the circumference of the earth (in km) for a given latitude (measured in radians). For additional information, see Question 6 from the Exercise in \ref{ch2}. Explain what is happening in each line of code below.
  \texttt{hi.temps\$day2} using \texttt{do.call()}.
  \vspace{0.1in}
  \vspace{0.1in}
\end{enumerate}

\begin{Shaded}
\begin{Highlighting}[numbers=left,,]
\KeywordTok{subroutine}\NormalTok{ circumf(x, n)}
\DataTypeTok{double precision}\NormalTok{ x(n)}
\DataTypeTok{integer}\NormalTok{ n}
\NormalTok{x }\KeywordTok{=} \BuiltInTok{cos}\NormalTok{(x)}\KeywordTok{*}\FloatTok{40075.017}
\KeywordTok{end}
\end{Highlighting}
\end{Shaded}

\begin{enumerate}
\def\labelenumi{\arabic{enumi}.}
\setcounter{enumi}{1}
\item
  Create a file \texttt{circumf.f90} containing the code and save it to an appropriate directory. Take a screen shot of the directory.
\item
  Compile \texttt{circumf.f90} to create \texttt{circumf.dll}. In Windows this will require the shell script:
  \vspace{0.1in}
\end{enumerate}

\begin{Shaded}
\begin{Highlighting}[]
\NormalTok{cd Root part of address\textbackslash{}bin\textbackslash{}x64  }
\NormalTok{R CMD SHLIB Appropriate directory/circumf.f90}
\end{Highlighting}
\end{Shaded}

~~~~~~You will have to supply your own \texttt{Root\ part\ of\ address}, and \texttt{Approriate\ directory} will be the directory containing \texttt{circumf.f90}.\\
\strut ~~~~~~Take a screenshot to show you have created \texttt{circumf.dll}. Running the shell code may require that you use the shell as an Administrator.)

\begin{enumerate}
\def\labelenumi{\arabic{enumi}.}
\setcounter{enumi}{3}
\tightlist
\item
  Here is a wrapper\footnote{Unix-alikes should replace \texttt{circumf.dll} with \texttt{circumf.o}.} for \texttt{circumf.dll}. Again, you will have to supply \texttt{Approriate\ directory}. Explain what is happening on Lines 2, 4, and 5. And, finally, run: \texttt{cearthf(0:90)}.
  \texttt{hi.temps\$day2} using \texttt{do.call()}.
  \vspace{0.1in}
\end{enumerate}

\begin{Shaded}
\begin{Highlighting}[numbers=left,,]
\NormalTok{cearthf }\OtherTok{\textless{}{-}} \ControlFlowTok{function}\NormalTok{(latdeg)\{}
\NormalTok{  x }\OtherTok{\textless{}{-}}\NormalTok{ latdeg }\SpecialCharTok{*}\NormalTok{ pi}\SpecialCharTok{/}\DecValTok{180}
\NormalTok{  n }\OtherTok{\textless{}{-}} \FunctionTok{length}\NormalTok{(x)}
  \FunctionTok{dyn.load}\NormalTok{(}\StringTok{"Appropriate directory/circumf.dll"}\NormalTok{)}
\NormalTok{  out }\OtherTok{\textless{}{-}} \FunctionTok{.Fortran}\NormalTok{(}\StringTok{"circumf"}\NormalTok{, }\AttributeTok{x =} \FunctionTok{as.double}\NormalTok{(x), }\AttributeTok{n =} \FunctionTok{as.integer}\NormalTok{(n))}
\NormalTok{  out}
\NormalTok{\}}
\end{Highlighting}
\end{Shaded}

\begin{enumerate}
\def\labelenumi{\arabic{enumi}.}
\setcounter{enumi}{4}
\tightlist
\item
  Here is a C script that is identical in functionality to the Fortran script in Q. 1. The code: \texttt{\#include\ \textless{}math.h\textgreater{}} allows access to C mathematical functions, including \texttt{cos()}. Describe what is happening on Lines 7-10.
  \vspace{0.1in}
\end{enumerate}

\begin{Shaded}
\begin{Highlighting}[numbers=left,,]

\PreprocessorTok{\#include }\ImportTok{\textless{}math.h\textgreater{}}


\DataTypeTok{void}\NormalTok{ circumc}\OperatorTok{(}\DataTypeTok{int} \OperatorTok{*}\NormalTok{nin}\OperatorTok{,} \DataTypeTok{double} \OperatorTok{*}\NormalTok{x}\OperatorTok{)}
\OperatorTok{\{}
  \DataTypeTok{int}\NormalTok{ n }\OperatorTok{=}\NormalTok{ nin}\OperatorTok{[}\DecValTok{0}\OperatorTok{];}
  \DataTypeTok{int}\NormalTok{ i}\OperatorTok{;}
  \ControlFlowTok{for} \OperatorTok{(}\NormalTok{i}\OperatorTok{=}\DecValTok{0}\OperatorTok{;}\NormalTok{ i}\OperatorTok{\textless{}}\NormalTok{n}\OperatorTok{;}\NormalTok{ i}\OperatorTok{++)}
\NormalTok{    x}\OperatorTok{[}\NormalTok{i}\OperatorTok{]} \OperatorTok{=} \OperatorTok{(}\NormalTok{cos}\OperatorTok{(}\NormalTok{x}\OperatorTok{[}\NormalTok{i}\OperatorTok{])} \OperatorTok{*} \FloatTok{40075.017}\OperatorTok{);}
\OperatorTok{\}}
\end{Highlighting}
\end{Shaded}

\begin{enumerate}
\def\labelenumi{\arabic{enumi}.}
\setcounter{enumi}{5}
\item
  Repeat Qs, 2 and 3 for the C subroutine \texttt{circumc}.
\item
  Here is an \textbf{R} wrapper for \texttt{circumc.dll}. Explain what is happening on Lines 4-6 and run: \texttt{cearthc(0:90)}.
\end{enumerate}

\begin{Shaded}
\begin{Highlighting}[numbers=left,,]
\NormalTok{cearthc }\OtherTok{\textless{}{-}} \ControlFlowTok{function}\NormalTok{(latdeg)\{}
\NormalTok{  x }\OtherTok{\textless{}{-}}\NormalTok{ latdeg }\SpecialCharTok{*}\NormalTok{ pi}\SpecialCharTok{/}\DecValTok{180}
\NormalTok{  n }\OtherTok{\textless{}{-}} \FunctionTok{length}\NormalTok{(x)}
  \FunctionTok{dyn.load}\NormalTok{(}\StringTok{"Appropriate directory/circumc.dll"}\NormalTok{,}
           \AttributeTok{nin =}\NormalTok{ n, x)}
\NormalTok{  out }\OtherTok{\textless{}{-}} \FunctionTok{.C}\NormalTok{(}\StringTok{"circumc"}\NormalTok{, }\AttributeTok{n =} \FunctionTok{as.integer}\NormalTok{(n), }\AttributeTok{x =} \FunctionTok{as.double}\NormalTok{(x))}
\NormalTok{  out}
\NormalTok{\}}
\end{Highlighting}
\end{Shaded}

\begin{enumerate}
\def\labelenumi{\arabic{enumi}.}
\setcounter{enumi}{4}
\tightlist
\item
  Make a Python list with elements \texttt{"pear"}, \texttt{"banana"}, and \texttt{"cherry"}.

  \begin{enumerate}
  \def\labelenumii{(\alph{enumii})}
  \tightlist
  \item
    Extract the second item in the list.
  \item
    Replace the first item in the list with \texttt{"melon"}.
  \item
    Append the number 3 to the list.
    \vspace{0.1in}
  \end{enumerate}
\item
  Make a Python tuple with elements \texttt{"pear"}, \texttt{"banana"}, and \texttt{"cherry"}.

  \begin{enumerate}
  \def\labelenumii{(\alph{enumii})}
  \tightlist
  \item
    Extract the second item in the tuple.
  \item
    Replace the first item in the tuple with \texttt{"melon"}. Was there an issue?
  \item
    Append the number 3 to the tuple. Was there an issue?
    \vspace{0.1in}
  \end{enumerate}
\item
  Using \texttt{def()}, write a Python function that will square any value \texttt{x}, and adds a constant \texttt{c} to the squared value of \texttt{x}.
  \vspace{0.1in}
\item
  From the Exercises in Ch \ref{ch2}, use Python (or call Python from \textbf{R}) to complete Problem 4 (a-h). Document your work in \textbf{R} Markdown (note how much better Python is at simplifying derivatives).
\end{enumerate}

\chapter{Building R Packages}\label{ch10}

\begin{quote}
\emph{``There are two ways of constructing a software design: One way is to make it so simple that there are obviously no deficiencies, and the other way is to make it so complicated that there are no obvious deficiencies. The first method is far more difficult.''}
\end{quote}

- \textbf{Tony Hoare}, \emph{Pioneering British computer scientist}

\section{Introduction}\label{introduction-1}

One of strengths of \textbf{R} is its capacity to format and share user-designed software as packages. Clearly it is possible to apply \textbf{R} for one's entire scientific career without creating an \textbf{R} package. However, development of a package, even if it is not distributed to a formal repository, ensures that your software is trustworthy and portable. Importantly, this chapter only provides a overview of basic topics in package development. The most thorough guide to package creation is the document \href{https://cran.r-project.org/doc/manuals/R-exts.html}{Writing R Extensions}, which is maintained by the the \textbf{R} development core team.

\section{Package Components}\label{package-components}

An \textbf{R} package is a directory of files, generally with nested subdirectories. Specifically,

\begin{itemize}
\tightlist
\item
  \texttt{DESCRIPTION} and \texttt{NAMESPACE} files define fundamental characteristics of the package, e.g., the author(s), the maintainer, the package version, the dependency on other packages, etc.
\item
  Subdirectories, and their nested files, contain the package contents. The following subdirectories are possible, although not all need to exist within a package.
\item
  The \texttt{R} subdirectory contains the package \textbf{R} code, stored as .r files, and will almost always exist.
\item
  The \texttt{data} subdirectory contains package datasets, usually stored as .rda files, which can be created using \texttt{save()}.
\item
  The \texttt{man} subdirectory contains the package documentation, stored as .rd files, for functions (in the \texttt{R} directory) and data (in \texttt{data}), and almost always exists.
\item
  The (optional) \texttt{src} subdirectory contains raw source code requiring compilation (C, C++, Fortran). When building a package \textbf{R} will call \texttt{R\ CMD\ SHLIB} (see Section \ref{compile}) to create appropriate binary shared library files.\\
\item
  Other potential subdirectories include: \texttt{demo}, \texttt{exec}, \texttt{inst}, \texttt{po}, \texttt{tests}, \texttt{tools}, and \texttt{vignettes}.
\end{itemize}

\index[R]{\texttt{save()}}

Fig \ref{fig:sdag} shows the contents of the \emph{streamDAG} package. These directories, and their files, are contained within a parent directory called \texttt{streamDAG}.
\index[terms]{R!package!\textit{streamDAG}}

\begin{figure}[H]

{\centering \includegraphics[width=0.6\linewidth,]{figs10/sddirs} 

}

\caption{Subdirectory level components of the \textit{streamDAG} package.}\label{fig:sdag}
\end{figure}

\begin{example}
\protect\hypertarget{exm:exm-c101}{}\label{exm:exm-c101}

\(\text{}\)\\

Creation of package components can be facilitated with the function \texttt{package.skeleton()} \index[R]{\texttt{package.skeleton()}}. From the \texttt{package.skeleton()} documentation Examples (see \texttt{?package.skeleton}), assume that we want to build a package that contains two silly functions: (\texttt{f} and \texttt{g}) and two silly datasets: (\texttt{d} and \texttt{e}).
\vspace{0.1in}

\begin{Shaded}
\begin{Highlighting}[]
\NormalTok{f }\OtherTok{\textless{}{-}} \ControlFlowTok{function}\NormalTok{(x, y) x }\SpecialCharTok{+}\NormalTok{ y}
\NormalTok{g }\OtherTok{\textless{}{-}} \ControlFlowTok{function}\NormalTok{(x, y) x }\SpecialCharTok{{-}}\NormalTok{ y}
\NormalTok{d }\OtherTok{\textless{}{-}} \FunctionTok{data.frame}\NormalTok{(}\AttributeTok{a =} \DecValTok{1}\NormalTok{, }\AttributeTok{b =} \DecValTok{2}\NormalTok{)}
\NormalTok{e }\OtherTok{\textless{}{-}} \FunctionTok{rnorm}\NormalTok{(}\DecValTok{1000}\NormalTok{)}
\end{Highlighting}
\end{Shaded}

We specify these as the \texttt{list} argument in \texttt{package.skeleton()} and give the package the name \emph{mypkg}.
\vspace{0.1in}

\begin{Shaded}
\begin{Highlighting}[]
\FunctionTok{package.skeleton}\NormalTok{(}\AttributeTok{list =} \FunctionTok{c}\NormalTok{(}\StringTok{"f"}\NormalTok{,}\StringTok{"g"}\NormalTok{,}\StringTok{"d"}\NormalTok{,}\StringTok{"e"}\NormalTok{), }\AttributeTok{name =} \StringTok{"mypkg"}\NormalTok{)}
\end{Highlighting}
\end{Shaded}

Running this code will cause a package skeleton for \emph{mypkg} to be sent to the working directory. Note that the skeleton contains the subdirectories: \texttt{data}, \texttt{r}, and \texttt{man} (Fig \ref{fig:mypkg1}). The datasets \texttt{d} and \texttt{e} were converted to .rda files by \texttt{package.skeleton()} and were placed in the \texttt{data} subdirectory. The functions \texttt{f} and \texttt{g} were converted to .r files and placed in the \texttt{r} subdirectory. Documentation skeletons for both functions and both datasets, as .rd files, were placed in the \texttt{man} subdirectory. Package \texttt{DESCRIPTION}, \texttt{NAMESPACE} files, and a throw-away (\texttt{Read-and-delete-me}) file were also created (Fig \ref{fig:mypkg1}).

\begin{figure}[H]

{\centering \includegraphics[width=0.6\linewidth,]{figs10/mypkg1} 

}

\caption{Subdirectory level components of the toy \textit{mypkg} package.}\label{fig:mypkg1}
\end{figure}

\end{example}

\(\blacksquare\)

\section{\texorpdfstring{Datasets (the \texttt{data} Subdirectory)}{Datasets (the data Subdirectory)}}\label{datasets-the-data-subdirectory}

Datasets in \textbf{R} are stored in the \texttt{data} subdirectory. Three data formats are possible:

\begin{itemize}
\tightlist
\item
  Raw .r code
\item
  Tabular data (e.g., .txt, .csv files)
\item
  Data `\texttt{images\textquotesingle{}\textquotesingle{}\ created\ using\ the\ function}save()`, e.g., .rda or .Rdata files. This approach is generally recommended, particularly for large datasets. Here we create a simple .rda dataset, and send it to the working directory.
  \vspace{0.1in}
\end{itemize}

\begin{Shaded}
\begin{Highlighting}[]
\NormalTok{x }\OtherTok{\textless{}{-}} \FunctionTok{rnorm}\NormalTok{(}\DecValTok{5}\NormalTok{)}
\FunctionTok{save}\NormalTok{(x, }\AttributeTok{file =} \StringTok{"x.rda"}\NormalTok{)}
\end{Highlighting}
\end{Shaded}

\index[terms]{R!.rda file}
\index[terms]{R!.Rdata file}
Data from packages will either be accessible via lazy loading (which allows increased accessibility) or with the \texttt{data()} function. Under the former approach, package data objects will not be loaded upon loading of their package environment, however promises are created, requiring the object to be loaded when its name is entered in a session. Lazy loading always occurs for package \textbf{R} code but is optional for package data. Lazy loading of data can be specified in a \texttt{‘LazyData’} field from a package's \texttt{DESCRIPTION} file (see below). Examples of lazy loaded data include objects from the package \emph{datasets}. Note that these do not require \texttt{data()} for loading:
\vspace{0.1in}

\begin{Shaded}
\begin{Highlighting}[]
\NormalTok{datasets}\SpecialCharTok{::}\NormalTok{BOD }\CommentTok{\# data describing Biochemical Oxygen Demand}
\end{Highlighting}
\end{Shaded}

\begin{verbatim}
  Time demand
1    1    8.3
2    2   10.3
3    3   19.0
4    4   16.0
5    5   15.6
6    7   19.8
\end{verbatim}

Under the latter, more common approach, \texttt{data(*foo*)} must be called to allow availability of the dataset \emph{foo}.
\vspace{0.1in}

\begin{Shaded}
\begin{Highlighting}[]
\FunctionTok{library}\NormalTok{(asbio)}
\FunctionTok{data}\NormalTok{(bighorn.sel) }\CommentTok{\# bighorn sheep resource use and availability}
\NormalTok{bighorn.sel}
\end{Highlighting}
\end{Shaded}

\begin{verbatim}
          resources avail  y1  n1
1          Riparian  0.06   0 445
2           Conifer  0.13   6 445
3       Mt. Shrub 1  0.16   9 445
4             Aspen  0.15  18 445
5      Rock outcrop  0.06  14 445
6  Sage/Bitterbrush  0.17  63 445
7  Windblown ridges  0.12  46 445
8        Mt shrub 2  0.04  62 445
9  Prescribed burns  0.09 178 445
10         Clearcut  0.02  49 445
\end{verbatim}

\index[terms]{Lazy loading}

\section{\texorpdfstring{R Code (the \texttt{r} Subdirectory)}{R Code (the r Subdirectory)}}\label{r-code-the-r-subdirectory}

Code for functions is generally stored in the \texttt{r} directory, as .r files. IDEs like RStudio, which contain options for the generation of .r scripts, e.g., \textbf{File \(>\) New File \(>\) R script}, can greatly aid in this process. Single .r files can contain multiple functions, although a one function per file approach may be easier to manage.
\index[terms]{R!.r file}

\section{\texorpdfstring{Documentation (the \texttt{man} Subdirectory)}{Documentation (the man Subdirectory)}}\label{documentation-the-man-subdirectory}

As functions become complex, it may become difficult to keep track of the meaning of function arguments, and the characteristics of function output, using a simple notes-to-self approach, e.g., \texttt{\#}. \textbf{R} documentation (.rd) files\index[terms]{R!.rd file} provide a framework for documenting, \textbf{R} functions, methods, and datasets. The \texttt{prompt()} family of functions can greatly facilitate the creation of .rd files. In Example \ref{exm:exm-c101}, the function \texttt{package.skelton()} used the functions \texttt{prompt()} and \texttt{promptData} to build documentation skeletons for functions and datasets, respectively. For instance, the code below was applied to create documentation for the function \texttt{f()}.
\vspace{0.1in}

\begin{Shaded}
\begin{Highlighting}[]
\NormalTok{f }\OtherTok{\textless{}{-}} \ControlFlowTok{function}\NormalTok{(x, y) x }\SpecialCharTok{+}\NormalTok{ y}
\FunctionTok{prompt}\NormalTok{(f, }\AttributeTok{filename =} \StringTok{"f"}\NormalTok{)}
\end{Highlighting}
\end{Shaded}

\begin{verbatim}
Created file named 'f'.
Edit the file and move it to the appropriate directory.
\end{verbatim}

This code causes the file \texttt{f.rd} to be generated, and sent to the working directory for further editing (Fig \ref{fig:fdoc}).

\begin{figure}[H]

{\centering \includegraphics[width=0.6\linewidth,]{figs10/frd} 

}

\caption{Documentation file skeleton for the toy function   exttt{f()}}\label{fig:fdoc}
\end{figure}

Some guidance for completing .rd files is provided by notes in the skeleton generated by \texttt{prompt()}. I have removed these notes in Fig \ref{fig:fdoc} to save space. As before, the authoritative resource for documentation building is \href{https://cran.r-project.org/doc/manuals/R-exts.html}{Writing R Extensions}.

Package documentation files can be placed into a \texttt{man} directory and compiled into a single documentation entity as the package is compiled\footnote{Following package compilation, installation, and loading, this allows access to documentation via \texttt{help(documented\ topic)} or \texttt{?documented\ topic} (Section \ref{help})}, or compiled singly for \textbf{R} objects that a user deems worthy of documentation. The latter approach is facilitated with the \textbf{Preview} widget in RStudio, which is available upon opening an .rd file. Running \textbf{Preview} on the file \texttt{f.rd} resulting in the .html preview shown in Fig \ref{fig:fpre}.

\begin{figure}[H]

{\centering \includegraphics[width=0.8\linewidth,]{figs10/fpre} 

}

\caption{Preview of the .html generated from the code shown Fig \ref{fig:fdoc}.}\label{fig:fpre}
\end{figure}

An .rd file can be converted to legible documentation in .html, .pdf or other formats by depositing the file in the \textbf{R} directory containing \texttt{R\ CMD} routines (e.g., bin/x64), and running the appropriate \texttt{R\ CMD} algorithms from the command line. In Windows this requires first navigating to the directory containing the \texttt{R\ CMD} routines using the Windows shell command line editor (see Ch 9). Important \texttt{R\ CMD} documentation rendering algorithms include:

\begin{itemize}
\tightlist
\item
  \texttt{R\ CMD\ Rd2pdf} \emph{foo.rd}`, can be used to compile the documentation file \emph{foo.rd} into a .pdf document.
\item
  \texttt{R\ CMD\ Rd2txt} \emph{foo.rd}`, can be used to compile the documentation file \emph{foo.rd} into a pretty text format.
\item
  \texttt{R\ CMD\ Rdconv} \emph{foo.rd}`, can be used to compile the documentation file \emph{foo.rd} into a variety formats including plain text, HTML, or LaTeX.
\end{itemize}

\index[R]{\texttt{R CMD Rd2pdf}}
\index[R]{\texttt{R CMD Rd2txt}}
\index[R]{\texttt{R CMD Rdconv}}

\section{The DESCRIPTION File}\label{the-description-file}

The \texttt{DESCRIPTION} file contains basic information about a package. The \texttt{DESCRIPTION} file skeleton for the \emph{mypkg} package, created by \texttt{package.skeleton()} in Example \ref{exm:exm-c101}, is shown in Fig \ref{fig:desc}.

\begin{figure}[H]

{\centering \includegraphics[width=0.8\linewidth,]{figs10/desc} 

}

\caption{\texttt{DESCRIPTION} file of the toy \textit{mypkg} package.}\label{fig:desc}
\end{figure}

The \texttt{DESCRIPTION} file will have a Debian control file format (see \texttt{?read.dcf}. Specifically, fields in \texttt{DESCRIPTION} must start with the field name, comprised of ASCII (Ch 12) printable characters, followed by a colon. The value for the field is given after the colon and an additional space (Fig \ref{fig:desc}). If allowed, field values longer than one line must use a space or a tab to start a new line. Specification of \texttt{‘Package’}, \texttt{‘Version’}, \texttt{‘License’}, \texttt{‘Description’}, \texttt{‘Title’}, \texttt{‘Author’}, and \texttt{‘Maintainer’} fields, shown in Fig \ref{fig:desc}, are mandatory.

\begin{itemize}
\tightlist
\item
  The \texttt{‘Package’} field gives the name of the package.
\item
  The \texttt{‘Version’} field gives a user-specified package version. It should be a sequence of at least two non-negative integers separated by single usages \texttt{‘.’} and/or \texttt{‘-’} characters.
\item
  The \texttt{‘Title’} field should provide a descriptive title for the package. It should use title case (capitals for principal words), and not have any continuation lines.
\item
  The \texttt{‘Author’} field describes who wrote the package. Note that if your package contains wrappers of the work of others, which are included in the \texttt{src} directory, then you are not the sole author.
\item
  The \texttt{‘Maintainer’} field provides a single name followed by a valid email address in angle brackets (Fig \ref{fig:desc}).
\item
  The \texttt{‘Description’} field should provide a comprehensive description of what the package does. Several (complete) sentences, complete, although these should limited to one paragraph. The field value should not to start with the package name, or \texttt{‘This\ package...’}.
\item
  The \texttt{‘License’} field provides standard open source
  license information for the package. Failure to specify license information may prevent others from legally using, or distributing your package. Standard licenses available from (\url{https://www.R-project.org/Licenses/}) include GPL-2, GPL-3, LGPL-2, LGPL-2.1, LGPL-3, AGPL-3, Artistic-2.0, BSD\_2\_clause, and BSD\_3\_clause MIT. See \href{https://cran.r-project.org/doc/manuals/R-exts.html}{Writing R Extensions} for more information.
\item
  Other optional fields include: \texttt{‘Copyright’}, \texttt{‘Date’}, \texttt{‘Depends’}, \texttt{‘Imports’}, \texttt{‘Suggests’}, \texttt{‘Enhances’}, \texttt{‘LinkingTo’}, \texttt{‘Additional\_repositories’}, \texttt{‘SystemRequirements’}, \texttt{‘URL’}, \texttt{‘BugReports’}, \texttt{‘Collate’}, \texttt{‘LazyData’}, \texttt{‘KeepSource’}, \texttt{‘ByteCompile’}, \texttt{‘UseLTO’}, \texttt{‘StagedInstall’}, \texttt{‘Biarch’}, \texttt{‘BuildVignettes’}, \texttt{‘VignetteBuilder’}, \texttt{‘NeedsCompilation’}, \texttt{‘OS\_type’}, and \texttt{‘Type’}. See \href{https://cran.r-project.org/doc/manuals/R-exts.html}{Writing R Extensions} for more information on these fields.
  \index[R]{\texttt{ead.dcf()}}
  \index[terms]{Debian control file}
  \index[terms]{R!DESCRIPTION file}
\end{itemize}

\section{The NAMESPACE File}\label{the-namespace-file}

The \textbf{R} namespace management system allows package authors to specify which variables in the package can be exported to package users, and which variables should be imported from other packages. The mandatory \texttt{NAMESPACE} file for the toy \emph{mypkg} package is extremely simple (Fig \ref{fig:namespace}). It indicates that all four objects contained in the package, and their associated names, can be exported. If one wishes to export all objects and names for a large package, it is simpler to specify: \texttt{exportPattern(.)}.

\begin{figure}[H]

{\centering \includegraphics[width=0.4\linewidth,]{figs10/namespace} 

}

\caption{\texttt{NAMESPACE} file of the toy \textit{mypkg} package.}\label{fig:namespace}
\end{figure}

Import of exported variables from other packages requires specification of \texttt{import} and \texttt{importFrom}. The \texttt{import} directive imports all exported variables from specified package(s). Thus, \texttt{import(foo)}
imports all exported variables in the package \emph{foo}. If a package requires some of the exported variables from a package, then \texttt{importFrom} can be used. The \texttt{NAMESPACE} directive \texttt{importFrom(foo,\ f,\ g)} indicates that \texttt{f} and \texttt{g} from package \emph{foo} should be imported.

To ensure that S3 methods for package classes are available, one must register the methods in the \texttt{NAMESPACE} file. For instance, if a package has a function \texttt{print.foo()} that serves as a print method for class \texttt{foo}, then one should include \texttt{S3method(print,\ foo)} as a line in \texttt{NAMESPACE}.

\index[terms]{R!NAMESPACE file}
\#\# Package Compilation
As with compilation of C and Fortran files (Ch 9), and the conversion of individual .rd files, the building and installation of a user-designed package requires depositing the package contents in the \textbf{R} directory containing the \texttt{R\ CMD} routines.\textsuperscript{{[}Or providing a navigation address to the package for \texttt{R\ CMD}{]}\(^{,}\)}{[}Probably the only \texttt{R\ CMD} routine isn't clearly tied to the development of \textbf{R} packages is \texttt{Rcmd\ BATCH}, which is used for running \textbf{R} scripts from the command line.{]} As before, one must run \texttt{R\ CMD} routines from the command line, requiring (in Windows) that a user navigate to the directory containing the \texttt{R\ CMD} routines at the Windows shell command line. This is unnecessary in Unix-like operating system (including MacOS), as these algorithms can be called directly from the computer's command line. \texttt{R\ CMD} routines for package building include:

\begin{itemize}
\tightlist
\item
  \texttt{R\ CMD\ build} \emph{foo}, which would build the package \emph{foo}.
\item
  \texttt{R\ CMD\ check} \emph{foo.tar.gz}, which would check the tarballed package \emph{foo.tar.gz}, created by \texttt{R\ CMD\ build}.
\item
  \texttt{R\ CMD\ INSTALL} \emph{foo.tar.gz} can be used to install the package \emph{foo}.
\end{itemize}

\index[R]{\texttt{R CMD build}}
\index[R]{\texttt{R CMD check}}
\index[R]{\texttt{R CMD INSTALL}}
\index[R]{\texttt{R CMD BATCH}}

\begin{example}
\protect\hypertarget{exm:exm-c102}{}\label{exm:exm-c102}

\(\text{}\)\\
Continuing from Example \ref{exm:exm-c101}, I complete the following steps for package building/compression, checking, and installation.

\begin{itemize}
\tightlist
\item
  Here I Build a tarballed version of the \emph{mypkg} package using: \texttt{R\ CMD\ build\ mypkg}.
\end{itemize}

\begin{center}\includegraphics[width=0.8\linewidth,]{figs10/RCMDbuild} \end{center}

\begin{itemize}
\tightlist
\item
  Here I check the tarballed version of the package using: \texttt{R\ CMD\ check\ mypkg\_0.1.tar.gz}.
\end{itemize}

\begin{center}\includegraphics[width=0.8\linewidth,]{figs10/RCMDcheck} \end{center}

Note that the checks from \texttt{R\ CMD\ check} can be extensive (the output above is just an excerpt). Checks are even more taxing if one uses the option \texttt{-\/-as-cran} which performs assessments one must pass for submission to CRAN.

\begin{itemize}
\tightlist
\item
  Finally, I Install the \emph{mypkg} package into my workstation using: \texttt{R\ CMD\ INSTALL\ mypkg\_0.1.tar.gz}.
\end{itemize}

\begin{center}\includegraphics[width=0.9\linewidth,]{figs10/RCMDINSTALL} \end{center}

\end{example}

\(\blacksquare\)

\section*{Exercises}\label{exercises-9}
\addcontentsline{toc}{section}{Exercises}

\begin{enumerate}
\def\labelenumi{\arabic{enumi}.}
\tightlist
\item
  Create an .rd documentation file for the function for McIntosh's index of site biodiversity from Exercise 2 in \ref{ch8}. Make a .pdf or .html from the .rd file using the appropriate \texttt{R\ CMD} routines.
  \vspace{0.1in}
\item
  Create an \textbf{R} package consisting of at least one function. Specifically,

  \begin{enumerate}
  \def\labelenumii{(\alph{enumii})}
  \tightlist
  \item
    Create a skeleton of the package using \texttt{package.skeleton()}.
  \item
    Finish the .rd file(s) in \texttt{man}.
  \item
    Complete the \texttt{DESCRIPTION} file.
  \item
    Complete the \texttt{NAMESPACE} file.
  \item
    Build the package using \texttt{R\ CMD\ build}.
  \item
    Check the package using \texttt{R\ CMD\ check}. Modify the package (if necessary) until no more \texttt{ERRORS} or \texttt{WARNINGS} occur.
  \end{enumerate}
\end{enumerate}

\chapter{Interactive and Web Applications}\label{ch11}

\begin{quote}
\emph{``A user interface is like a joke. If you have to explain it, it's not that good.''}
\end{quote}

- \textbf{Martin LeBlanc}, \emph{Iconfinder cofounder}

\section{Introduction to GUIs}\label{introduction-to-guis}

\emph{GUIs} (Graphical User Interfaces) allow users to interact with software using graphical icons and point-and-click specifications. The interactive control components of a GUI are called \emph{widgets}. The following are some commonly used GUI widgets:
\index[terms]{Widget (GUI controller)}

\begin{itemize}
\tightlist
\item
  \emph{button}: typically a GUI controller for binary (e.g., on/off or run/don't run) operations.
\item
  \emph{radiobutton}: A controller allowing selection from a group of mutually-exclusive options which are linked to specific operations.
\item
  \emph{checkbutton}: A controller that allows selection binary of multiple mutually-exclusive options, and is often linked to a second variable, allowing flexible rendering of a secondary set of widgets.
\item
  \emph{spinbox}: Provides a ``spin-able'' set of mutually exclusive options that can be selected and linked to operations.
\item
  \emph{combobox}: A text field with a popdown selection list.
\item
  \emph{slider}: A sliding controller that defines the numeric value of a linked variable that changes uniformly over some range.
\item
  \emph{message box}: A message window that typically prompts a user response and a corresponding linked operation.
\item
  \emph{scrollbar}: A modifiable viewport for a scrollable object (e.g., text that can be examined line by line).
\item
  \emph{pulldown menus}: Interactive menus with pulldown tabs and \emph{menubuttons} that specify operations, potentially including links to other GUIs.
\end{itemize}

\textbf{R} allows building of GUIs to run functions and interact with graphics using a number of methods and language frameworks.
\index[terms]{GUI (Graphical User Interface)}
\index[terms]{GUI (Graphical User Interface)!widget}

It should be noted that \textbf{R} GUIs are are a mixed bag. On the plus side, \textbf{R} GUIs: 1) increase user-friendliness by allowing point and click operations, 2) allow rapid visual assessment of alteration to function arguments via widgets, 3) are often very amenable to graphics manipulations, and 4) are often very useful for data exploration or heuristic demonstrations. On the other hand, \textbf{R} GUIs: 1) often result in a loss of flexibility in controlling functions, 2) may contain a visually confusing mish-mash of widgets, and 3) constitute mysterious black boxes, which is contrary to the ``mission statement'' of \textbf{R} \citep{chambers2008}. Further, command line (non-GUI) code entry allows an exact record of characteristics given to objects, and specifications provided to functions. This allows straightforward tracking, dissemination, and repeatability of computational analyses.

I will explore three methods for generating GUIs, named for the principal \textbf{R} package allowing their implementation: \emph{tcltk}, \emph{plotly}, and \emph{shiny}.

\section{\texorpdfstring{\emph{tcltk}}{tcltk}}\label{tcltk}

The \textbf{R} distribution package \emph{tcltk} (pronounced: \emph{tickle tee kay}) allows building of GUIs by providing a binding wrapper for the Tcl language, via the Tk toolkit, under a configuration called Tcl/Tk \citep{ousterhout1991}\footnote{Tcl, an acronym for ``tool command language,'' is an interpreted programming language that is often embedded into C applications. Tk is a Tcl package for GUI building. Tk, when implemented in Tcl, is termed Tcl/Tk.}. In programming, \emph{binding} refers to an \emph{application programming interface} (API) that provides glue code to allow a programming language to implement a foreign language or software package \citep{wikbinding}. Python bindings for Tcl/Tk are provided by the Python library \href{https://docs.python.org/3/library/tkinter.html}{\emph{tkinter}} which is included in the \href{https://docs.python.org/3/library/index.html}{Python standard library} of packages. Much better support exists for \emph{tkinter} than \emph{tcltk}. Additionally, \href{https://wiki.python.org/moin/GuiProgramming}{many non-Tcl/Tk approaches} exist for GUI building in Python, although they are not included in the Python standard library.

Lack of guidance for the \emph{tcltk} package is likely due to the absence of a large user group. Assistance for the creation of \emph{tcltk} GUIs can be found in several older articles from \textbf{R} News (which has since been replaced by the \textbf{R} Journal)\index[terms]{R Journal (the)} \citep{dalgaard2001, dalgaard2002, fox2007}, the book ``Programming GUIs in \textbf{R}'' \citep{lawrence2018programming}, and in the GUI code for a number of newer \textbf{R} packages, including \emph{Rcmdr} \citep{foxRcmdrArticle, foxRcmdrManual} and \emph{asbio} \citep{aho2023}. Despite resources, however, it is expected that users refer to the \href{https://www.tcl.tk/man/tcl8.6/UserCmd/contents.html}{Tcl/Tk package manual} for argument lists and descriptions of \emph{tcltk} functions. Arguments in \emph{tcltk} functions (generally) have the same names and functionality as their Tcl/tk equivalents, although some experimentation may be necessary.

Tcl/Tk itself is cross platform, and uses facilities particular to the underlying OS. These are \emph{Xlib} (X11) (a windowing system, written in C, for bitmap displays) for Unix/Linux, \emph{Cocoa} for Mac OS, and the \emph{Graphics Device Interface} (GDI) for Windows.

So-called \emph{Themed Tk} (Ttk) GUIs often have advantages over older Tk GUIs, including anti-aliased font rendering, and have been a part of the Tk distribution since Tcl version 8.5. Naming conventions in \emph{tcltk} indicate whether functions are binding for Tk or Ttk operations. The former function names start with \texttt{tk}, while the latter start with \texttt{ttk}.

Notably, \emph{tcltk} GUIs that use or manipulate \textbf{R} graphics devices, particularly those with slider widgets, may work poorly with the native RStudio graphics device: \texttt{RStudioGD}. Thus, to run these sorts of GUIs in RStudio, one should open a non-\texttt{RStudioGD} device using:
\vspace{0.1in}

\begin{Shaded}
\begin{Highlighting}[]
\FunctionTok{dev.new}\NormalTok{(}\AttributeTok{noRStudioGD =} \ConstantTok{TRUE}\NormalTok{)}
\end{Highlighting}
\end{Shaded}

\index[terms]{R!package!\textit{asbio}}
\index[terms]{R!package!\textit{Rcmdr}}
\index[terms]{R!package!\textit{gWidgets2tcltk}}
\index[terms]{RStudioGD}
\index[terms]{Binding (software)}
\index[terms]{API (application programming interface)}
\index[terms]{Tcl (programming language)}
\index[terms]{Tcl/Tk}
\index[terms]{R!package!\textit{tcltk}}
\index[terms]{Python!package!\textit{tkinter}}
\index[terms]{Mac (operating system)!cocoa}
\index[terms]{Windows (operating system)!DPI}
\index[terms]{Linux/Unix (operating system)!Xlib (X11)}
\index[R]{\texttt{dev.new()}}

\begin{example}
\protect\hypertarget{exm:exm-c11tcltkex}{}\label{exm:exm-c11tcltkex}\(\text{}\)\\
The binding mechanisms of \emph{tcltk} can be viewed by examining underlying code from some its seminal functions. The \emph{tcltk} function \texttt{tcl()} provides a generic interface for calling any Tk or Tcl command\footnote{The code: \texttt{tcl("label",\ tt,\ text\ =\ "Hello\ world",\ bg\ =\ "red")} is equivalent to \texttt{tklabel(tt,\ text\ =\ "Hello\ world",\ bg\ =\ "red")}, where \texttt{tt} is the top-level GUI object. Both prompt the Tcl/tk code: \texttt{label\ tp\ -text\ "Hello\ world"\ \ -bg\ red}, where \texttt{tp} is the path name of the Tcl/Tk label object.}. Indeed, many other \emph{tcltk} commands are simply calls to \texttt{tcl()}.

We see that \texttt{tcl()} calls \texttt{.Tcl.objv()} whose arguments, in turn, are formatted by \texttt{.Tcl.args.objv()}.

\vspace{0.1in}

\begin{Shaded}
\begin{Highlighting}[]
\FunctionTok{require}\NormalTok{(tcltk)}
\NormalTok{tcl}
\end{Highlighting}
\end{Shaded}

\begin{verbatim}
function (...) 
.Tcl.objv(.Tcl.args.objv(...))
<bytecode: 0x0000026e3f227050>
<environment: namespace:tcltk>
\end{verbatim}

The function \texttt{.Tcl.objv()} calls an underlying C algorithm, \texttt{.C\_dotTclObjv()}, using \texttt{.External()} that ostensibly binds \texttt{tcl()} to Tcl/Tk.

\vspace{0.1in}

\begin{Shaded}
\begin{Highlighting}[]
\NormalTok{.Tcl.objv}
\end{Highlighting}
\end{Shaded}

\begin{verbatim}
function (objv) 
structure(.External(.C_dotTclObjv, objv), class = "tclObj")
<bytecode: 0x0000026e3f226640>
<environment: namespace:tcltk>
\end{verbatim}

The compiled C executable (Section \ref{interp}), \texttt{tcltk.dll}, is housed in the \emph{tlctk} package \texttt{libs/x64} directory (Ch \ref{ch10}).

\vspace{0.1in}

\begin{Shaded}
\begin{Highlighting}[]
\NormalTok{tcltk}\SpecialCharTok{:::}\NormalTok{.C\_dotTclObjv}\SpecialCharTok{$}\NormalTok{dll}
\end{Highlighting}
\end{Shaded}

\begin{verbatim}
DLL name: tcltk
Filename: C:/Program
       Files/R/R-4.4.2/library/tcltk/libs/x64/tcltk.dll
Dynamic lookup: FALSE
\end{verbatim}

\index[R]{\texttt{tcltk::.Tcl.objv()}}
\index[R]{\texttt{tcltk::tcl()}}
\index[R]{\texttt{.External()}}
\end{example}

\(\blacksquare\)

\begin{example}
\protect\hypertarget{exm:exm-c111}{}\label{exm:exm-c111}\(\text{}\)~As an initial foray into \emph{tcltk} GUI-building we will create a button interface whose only purpose is to provide a message, and a means for its own destruction.

\vspace{0.1in}

\begin{Shaded}
\begin{Highlighting}[numbers=left,,]
\NormalTok{tt }\OtherTok{\textless{}{-}} \FunctionTok{tktoplevel}\NormalTok{()}
\NormalTok{hello }\OtherTok{\textless{}{-}} \FunctionTok{tkmessage}\NormalTok{(tt, }\AttributeTok{text =} \StringTok{"Hello world!"}\NormalTok{)}
\NormalTok{spacer }\OtherTok{=} \FunctionTok{tklabel}\NormalTok{(tt, }\AttributeTok{padx =} \DecValTok{20}\NormalTok{)}
\NormalTok{DM.but }\OtherTok{\textless{}{-}} \FunctionTok{tkbutton}\NormalTok{(tt, }\AttributeTok{text =} \StringTok{"Exit"}\NormalTok{, }\AttributeTok{foreground =} \StringTok{"red"}\NormalTok{,}
                   \AttributeTok{background =} \StringTok{"lightgreen"}\NormalTok{, }\AttributeTok{padx =} \DecValTok{10}\NormalTok{,}
                   \AttributeTok{command =} \ControlFlowTok{function}\NormalTok{() }\FunctionTok{tkdestroy}\NormalTok{(tt))}
\FunctionTok{tkpack}\NormalTok{(hello, spacer, DM.but)}
\end{Highlighting}
\end{Shaded}

\begin{itemize}
\tightlist
\item
  On Line 1, I load the \emph{tcltk} package.
\item
  On Line 2, I use use \texttt{tktoplevel()} to hierarchically define the ``top level'' widget as the object \texttt{tt}.
\item
  On Lines 3-4, I create a text message object, \texttt{hello}, and a spacer object, \texttt{spacer}. That latter is used to make room between the message and a button created in the next two lines of code.
\item
  On Lines 5-6 the button widget object \texttt{DM.but} is created, using the function \texttt{tkbutton()}. The first argument is name of parent widget, \texttt{tt}. The \texttt{text} argument provides a text label for the button. The arguments \texttt{foreground}, \texttt{background}, and \texttt{padx} are used to define the foreground color (the color of the button text label), the background color of the button, and to make the button wider, respectively. The \texttt{command} argument defines the function that the button initiates. In this case, the function \texttt{tkdestroy()}, which destroys the GUI.
\item
  On Line 8, \texttt{tkpack()} is used to place the button on the parent widget.
\end{itemize}

The GUI itself is shown in Fig \ref{fig:tcltk1}.

\begin{figure}[H]

{\centering \includegraphics[width=0.5\linewidth,]{figs11/tcltk1} 

}

\caption{A simple \textit{tcltk} GUI. }\label{fig:tcltk1}
\end{figure}

\index[R]{\texttt{tcltk::tkbutton()}}
\index[R]{\texttt{tcltk::tkdestroy()}}
\index[R]{\texttt{tcltk::tktoplevel()}}
\index[R]{\texttt{tcltk::tkbutton()}}
\index[R]{\texttt{tcltk::tkpack()}}
\index[R]{\texttt{tcltk::tklabel()}}
\index[R]{\texttt{tcltk::tkdestroy()}}
\end{example}

\(\blacksquare\)

\subsection{Assigning and Manipulating Widget Values}\label{asswidget}

It is often useful to ``remember'' object characteristics and assignment values over the course of a GUI's usage. For example, it may be necessary to count the number of times a button is pressed, or display a particular message based on a previous response. Because GUI actions will be carried out by local variables in functions, modifications to those variables will be lost when the function exits. While not usually good practice, one can use the super assignment operator \texttt{\textless{}\textless{}-} inside a function to create global variables. These will retain their values after the function exits.
\index[R]{*aos@\texttt{<<-}}

The \texttt{tclVar()} function can be used to create an empty or specific values, which can then be used in call to other functions in the \emph{tcltk} package. For example, to specify an empty \texttt{tclVar()} value, one could use:
\vspace{0.1in}

\begin{Shaded}
\begin{Highlighting}[]
\NormalTok{myvar }\OtherTok{\textless{}{-}} \FunctionTok{tclVar}\NormalTok{(}\StringTok{\textquotesingle{}\textquotesingle{}}\NormalTok{)}
\end{Highlighting}
\end{Shaded}

To access \texttt{myvar} information in widgets with \textbf{R} one could then use:
\vspace{0.1in}

\begin{Shaded}
\begin{Highlighting}[]
\NormalTok{rmyvar }\OtherTok{\textless{}{-}} \FunctionTok{tclvalue}\NormalTok{(myvar)}
\end{Highlighting}
\end{Shaded}

Conjoined use of the super-assignment operator with \texttt{tclVar()} and \texttt{tcl()} is often very important when altering object parameters within \emph{tcltk} functions.
\index[R]{\texttt{tcltk::tclVar()}}
\index[R]{\texttt{tcltk::tclvalue()}}
\index[R]{\texttt{tcltk::tkmessage()}}

\subsection{\texorpdfstring{User functions and \emph{tcltk} GUIs}{User functions and tcltk GUIs}}\label{user-functions-and-tcltk-guis}

Callbacks are functions that are linked to GUI events. In \emph{tclk} these functions can be user-defined, although they should not have arguments. Callbacks, user defined or otherwise, are generally executed using the \texttt{command} argument in a widget function. Recall, for example, use of \texttt{tkbutton(tt,\ command\ =\ functiton()\ tkdestroy)} in Example \ref{exm:exm-c111}. In general, a callback function \texttt{foo()} is called using \texttt{command\ =\ foo()} or \texttt{command\ =\ substitute(foo())}. Use of \texttt{substitute(foo())} allows substitution of variable values in \texttt{foo()}. Calling a function \texttt{bar()} from within the callback \texttt{foo()} may require coding similar to \texttt{foo\ \textless{}-\ function()\{substitute(bar())\}}. See, for instance, \texttt{asbio::anm.ci.tck()}.
\index[R]{\texttt{asbio::anm.ci.tck()}}

\begin{example}
\protect\hypertarget{exm:exm-c112}{}\label{exm:exm-c112}

\(\text{}\)\\
Consider the following silly example for finding the sum of two numbers.
\vspace{0.1in}

\begin{Shaded}
\begin{Highlighting}[numbers=left,,]
\NormalTok{tt }\OtherTok{\textless{}{-}} \FunctionTok{tktoplevel}\NormalTok{()}
\NormalTok{tw1 }\OtherTok{\textless{}{-}} \FunctionTok{tclVar}\NormalTok{(}\StringTok{\textquotesingle{}\textquotesingle{}}\NormalTok{); tw2 }\OtherTok{\textless{}{-}} \FunctionTok{tclVar}\NormalTok{(}\StringTok{\textquotesingle{}\textquotesingle{}}\NormalTok{)}
\NormalTok{tke1 }\OtherTok{\textless{}{-}} \FunctionTok{tkentry}\NormalTok{(tt, }\AttributeTok{width =} \DecValTok{6}\NormalTok{, }\AttributeTok{textvariable =}\NormalTok{ tw1, }\AttributeTok{justify =} \StringTok{"center"}\NormalTok{)}
\NormalTok{tke2 }\OtherTok{\textless{}{-}} \FunctionTok{tkentry}\NormalTok{(tt, }\AttributeTok{width =} \DecValTok{6}\NormalTok{, }\AttributeTok{textvariable =}\NormalTok{ tw2, }\AttributeTok{justify =} \StringTok{"center"}\NormalTok{)}

\NormalTok{sumf }\OtherTok{\textless{}{-}} \ControlFlowTok{function}\NormalTok{()\{}
\NormalTok{  temp }\OtherTok{\textless{}{-}} \FunctionTok{as.numeric}\NormalTok{(}\FunctionTok{tclvalue}\NormalTok{(tw1)) }\SpecialCharTok{+} \FunctionTok{as.numeric}\NormalTok{(}\FunctionTok{tclvalue}\NormalTok{(tw2))}
  \FunctionTok{tkconfigure}\NormalTok{(ans, }\AttributeTok{text =} \FunctionTok{paste}\NormalTok{(temp))}
\NormalTok{\}}

\NormalTok{ans }\OtherTok{\textless{}{-}} \FunctionTok{tklabel}\NormalTok{(tt, }\AttributeTok{text =} \StringTok{\textquotesingle{}\textquotesingle{}}\NormalTok{, }\AttributeTok{background=}\StringTok{"white"}\NormalTok{, }\AttributeTok{relief =} \StringTok{"sunken"}\NormalTok{, }\AttributeTok{padx =} \DecValTok{20}\NormalTok{)}
\FunctionTok{tkgrid}\NormalTok{(tke1, }\FunctionTok{tklabel}\NormalTok{(tt, }\AttributeTok{text =} \StringTok{\textquotesingle{}+\textquotesingle{}}\NormalTok{), tke2, }\FunctionTok{tklabel}\NormalTok{(tt, }\AttributeTok{text =} \StringTok{\textquotesingle{}=\textquotesingle{}}\NormalTok{), ans)}
\FunctionTok{tkgrid}\NormalTok{(}\FunctionTok{tklabel}\NormalTok{(tt, }\AttributeTok{text =} \StringTok{\textquotesingle{}\textquotesingle{}}\NormalTok{), }\AttributeTok{columnspan =} \DecValTok{5}\NormalTok{)}
\FunctionTok{tkgrid}\NormalTok{(}\FunctionTok{tkbutton}\NormalTok{(tt, }\AttributeTok{text =} \StringTok{\textquotesingle{}Get Sum!\textquotesingle{}}\NormalTok{, }\AttributeTok{foreground =} \StringTok{"red"}\NormalTok{,}
                   \AttributeTok{background =} \StringTok{"lightgreen"}\NormalTok{, }\AttributeTok{command =}\NormalTok{ sumf),}
       \AttributeTok{columnspan =} \DecValTok{5}\NormalTok{, }\AttributeTok{sticky =} \StringTok{"e"}\NormalTok{)}
\end{Highlighting}
\end{Shaded}

\begin{itemize}
\tightlist
\item
  On Line 1, I use \texttt{tktoplevel()} to define the ``top level'' widget.
\item
  On Line 2, I specify empty initial values for the variables \texttt{tw1} and \texttt{tw2} using \texttt{tclVar()}. These values will be editable by users via \texttt{tkentry()} widgets.
\item
  On Lines 3-4, I use the function \texttt{tkentry()}\index[R]{\texttt{tcltk::tkentry()}} to provide widgets for users to enter numbers to be summed.
\item
  On Lines 6-9, I create the function \texttt{sumf}. The function \texttt{tclvalue()}, used to compute the summation object \texttt{temp}, allows Tcl variables from \texttt{tclVar()} to be evaluated in \textbf{R}. These variables, however, will have class \texttt{character}, and will require \texttt{as.numeric()}, as shown, for mathematical evaluation. One Line 8 (the final line of code in \texttt{sumf}, \texttt{tkconfigure()} is used to potentially change \texttt{ans}, a \texttt{tkentry()} object defined on Line 11.
\item
  On Line 11, the \texttt{tkentry()} object \texttt{ans} is created and an initial empty value is assigned.
\item
  On Lines 12-16, widgets are placed in the GUI using \texttt{tkgrid()}. The use of grid geometry approaches including \texttt{tkgrid()} is elaborated next. The \texttt{tkbutton} widget in the final (bottom) grid of the GUI calls the \texttt{sumf} using either \texttt{command\ =\ sumf}, as shown, or \texttt{command\ =\ substitute(sumf())}.
\end{itemize}

\index[R]{\texttt{tcltk::tkgrid()}}
The resulting GUI is shown in Fig \ref{fig:getsum}.

\begin{figure}[H]

{\centering \includegraphics[width=0.5\linewidth,]{figs11/getsum} 

}

\caption{A simple \textit{tcltk} GUI, demonstrating the use of \texttt{tclVar()} and \texttt{tclvalue()}.}\label{fig:getsum}
\end{figure}

\end{example}

\(\blacksquare\)

\subsection{GUI Geometry}\label{gui-geometry}

An important consideration in GUI development is \emph{geometry management}\index[terms]{GUI (Graphical User Interface)!geometry management}, e.g., the dimensions of the GUI and the organization of widgets. By default, Tcl/Tk GUI windows are autosized to hold widgets as they are added. Widgets may be reorganized as the sizes of windows are altered. If a Window becomes too small to contain widgets, the last widget added will be the first removed.

The initial size of GUIs can be specified using the function \texttt{tkcanvas()}. The result of the code below is shown in Fig \ref{fig:wide}
\vspace{0.1in}

\begin{Shaded}
\begin{Highlighting}[]
\NormalTok{tt }\OtherTok{\textless{}{-}} \FunctionTok{tktoplevel}\NormalTok{()}
\FunctionTok{tktitle}\NormalTok{(tt) }\OtherTok{=} \StringTok{"Wide GUI"}
\NormalTok{dim }\OtherTok{\textless{}{-}} \FunctionTok{tkcanvas}\NormalTok{(tt, }\AttributeTok{height =} \DecValTok{30}\NormalTok{, }\AttributeTok{width =} \DecValTok{500}\NormalTok{)}
\FunctionTok{tkgrid}\NormalTok{(dim)}
\end{Highlighting}
\end{Shaded}

\begin{figure}[H]

{\centering \includegraphics[width=0.95\linewidth,]{figs11/wide} 

}

\caption{A \textit{tcltk} GUI whose initial width was specified using \texttt{tkcanvas()}.}\label{fig:wide}
\end{figure}

Three different geometry managers are available in Tcl/Tk for inserting widgets in GUIs. These are called \emph{placer}, \emph{packer}, and \emph{grid manager}. The placer tool is seldom used in GUI creation \citep{dalgaard2001}. Thus, we will concentrate on GUI construction using packer and grid manager approaches. Only one of these approaches is generally used in the creation of a GUI. The initial Examples \ref{exm:exm-c111} and \ref{exm:exm-c112} use simple applications of packing and grid management, respectively.

\subsubsection{Packing}\label{packing}

The function \texttt{tkpack()} packs widgets around the edges of a conceptual cavity. Control of this process is provided by the \texttt{side}, which has options: \texttt{"left"}, \texttt{"right"}, \texttt{"top"} or \texttt{"bottom"}.
\index[R]{\texttt{tcltk::tkpack()}}

\begin{example}
\protect\hypertarget{exm:exm-c113}{}\label{exm:exm-c113}\(\text{}\)\\
Note the result of the code below (Fig \ref{fig:pack}).
\vspace{0.1in}

\begin{Shaded}
\begin{Highlighting}[numbers=left,,]
\NormalTok{tt }\OtherTok{\textless{}{-}} \FunctionTok{tktoplevel}\NormalTok{()}
\NormalTok{edge }\OtherTok{\textless{}{-}} \FunctionTok{c}\NormalTok{(}\StringTok{"top"}\NormalTok{,}\StringTok{"right"}\NormalTok{,}\StringTok{"bottom"}\NormalTok{,}\StringTok{"left"}\NormalTok{)}
\NormalTok{buttons }\OtherTok{\textless{}{-}} \FunctionTok{lapply}\NormalTok{(}\DecValTok{1}\SpecialCharTok{:}\DecValTok{4}\NormalTok{, }
                  \ControlFlowTok{function}\NormalTok{(i) }\FunctionTok{tkbutton}\NormalTok{(tt, }\AttributeTok{text =}\NormalTok{ edge[i],}
                     \AttributeTok{background =} \StringTok{"lightgreen"}\NormalTok{, }\AttributeTok{foreground =} \StringTok{"red"}\NormalTok{))}
\ControlFlowTok{for}\NormalTok{ (i }\ControlFlowTok{in} \DecValTok{1}\SpecialCharTok{:}\DecValTok{4}\NormalTok{)}
\FunctionTok{tkpack}\NormalTok{(buttons[[i]], }\AttributeTok{side=}\NormalTok{edge[i], }\AttributeTok{fill =} \StringTok{"both"}\NormalTok{)}
\end{Highlighting}
\end{Shaded}

\begin{itemize}
\tightlist
\item
  On Line 1, the top level widget is designated.
\item
  On Line 2, a character vector is created, containing all the possible \texttt{side} options for the function \texttt{tkpack()}.
\item
  On Lines 3-5, a four item list is generated containing four \texttt{tkbutton} widgets.
\item
  In Lines 6-7, buttons are accessed from the \texttt{buttons} list and packed, in order, at the specified locations \texttt{"top"}, \texttt{"right"}, \texttt{"bottom"}, and \texttt{"left"}. The argument \texttt{fill\ =\ "both"} ensures that the buttons will occupy all of their allocated \emph{parcels} with respect to the \texttt{tkpack()} conceptual central cavity. Because the \texttt{top} button was specified first, it takes up the entire top of the GUI. The \texttt{right} button, codified next, occupies the entire right-side of the GUI, except for the area now occupied by \texttt{top}, and so on. If an object does not fill its parcel it can be anchored to a GUI location using the \texttt{tkpack()} argument \texttt{anchor}. This is accomplished by specifying compass-style values like \texttt{"n"} or \texttt{"sw"} which place a widget the middle top, and bottom left of the parcel, respectively. The default option is \texttt{anchor\ =\ "center"}.
\end{itemize}

\begin{figure}[H]

{\centering \includegraphics[width=0.4\linewidth,]{figs11/pack} 

}

\caption{A demonstration of the result of packing using \texttt{tkpack()}. Code follows \citep{dalgaard2001}.}\label{fig:pack}
\end{figure}

\index[R]{\texttt{tcltk::tkcanvas()}}
\end{example}

\(\blacksquare\)

\begin{example}
\protect\hypertarget{exm:exm-c114}{}\label{exm:exm-c114}

\(\text{}\)\\
Calculator construction is often used as a pedagogic exercise in computer programming. As an extended example of packing using \texttt{tkframe()}, we will build a \emph{tcltk} calculator GUI. For this example I am indebted to lecture notes for a 2011 \href{https://www.stat.berkeley.edu/~s133/}{statistical programming course} at UC Berkeley.

The most important coding concept used here is the pairing of the base \textbf{R} functions \texttt{parse()} (which converts a string to an expression) and \texttt{eval()} (which evaluates an expression). This combination allows the mathematical evaluation of a character string. Consider the string \texttt{"9\ *\ 3"}. The mathematical solution can be obtained using:
\vspace{0.1in}

\begin{Shaded}
\begin{Highlighting}[]
\NormalTok{txt }\OtherTok{=} \StringTok{"9 * 3"}
\FunctionTok{eval}\NormalTok{(}\FunctionTok{parse}\NormalTok{(}\AttributeTok{text =}\NormalTok{ txt))}
\end{Highlighting}
\end{Shaded}

\begin{verbatim}
[1] 27
\end{verbatim}

\index[R]{\texttt{parse()}}
\index[R]{\texttt{eval()}}

Our calculator GUI will require three functions: \texttt{mkput()}, \texttt{clearit()}, and \texttt{docalc()}. Each of the functions creates a global variable, \texttt{calcinp}, using the super assignment operator, \texttt{\textless{}\textless{}-}, that provides input to the calculator. Further, in all three functions, \texttt{tkconfigure()} is used to change the calculator's display, based on input from the GUI calculator keys.
\vspace{0.1in}

\begin{Shaded}
\begin{Highlighting}[numbers=left,,]
\NormalTok{calcinp }\OtherTok{\textless{}{-}} \StringTok{\textquotesingle{}\textquotesingle{}}

\NormalTok{mkput }\OtherTok{\textless{}{-}} \ControlFlowTok{function}\NormalTok{(sym)\{}
       \ControlFlowTok{function}\NormalTok{()\{}
\NormalTok{          calcinp }\OtherTok{\textless{}\textless{}{-}} \FunctionTok{paste}\NormalTok{(calcinp, sym, }\AttributeTok{sep=}\StringTok{\textquotesingle{}\textquotesingle{}}\NormalTok{)}
            \FunctionTok{tkconfigure}\NormalTok{(display, }\AttributeTok{text =}\NormalTok{ calcinp)}
\NormalTok{          \}}
\NormalTok{\}}
\end{Highlighting}
\end{Shaded}

\begin{itemize}
\item
  On Line 1 in the chunk above, \texttt{calcinp} is initially set to be an empty character string, i.e., \texttt{calcinp\ \textless{}-\ \textquotesingle{}\textquotesingle{}}.
\item
  On Lines 3-8, the callback function \texttt{mkput} is defined. Note that \texttt{mkput} itself contains an argument-less function. This allows \texttt{mkput} to have an argument, \texttt{sym}, while satisfying the \emph{tcltk} requirement for argument-less callbacks. The code on Line 5, \texttt{calcinp\ \textless{}\textless{}-\ paste(calcinp,\ sym,\ sep=\textquotesingle{}\textquotesingle{})}, generates a global, updated form of \texttt{calcinp}, that combines an older \texttt{calcinp} value with a new calculator key specification, \texttt{sym}. The resulting string is placed in the display using \texttt{tkconfigure()}.
\item
  The callback function \texttt{clearit} below, clears the display (Line 10), and redefines \texttt{calcinp} as an empty string (Line 11).
  \vspace{0.1in}
\end{itemize}

\begin{Shaded}
\begin{Highlighting}[numbers=left,,firstnumber=9,]
\NormalTok{clearit }\OtherTok{\textless{}{-}} \ControlFlowTok{function}\NormalTok{()\{}
      \FunctionTok{tkconfigure}\NormalTok{(display, }\AttributeTok{text =} \StringTok{\textquotesingle{}\textquotesingle{}}\NormalTok{)}
\NormalTok{      calcinp }\OtherTok{\textless{}\textless{}{-}} \StringTok{\textquotesingle{}\textquotesingle{}}
\NormalTok{      \}}
\end{Highlighting}
\end{Shaded}

\begin{itemize}
\tightlist
\item
  The callback function \texttt{docalc} below, evaluates the general \texttt{eval(parse(text\ =\ calcinp))} framework created by key entry, and provides exception handling in the case of key stroke errors by using \texttt{if(class(result)\ ==\ \textquotesingle{}try-error\textquotesingle{});\ calcinp\ \textless{}\textless{}-\ \textquotesingle{}Error\textquotesingle{}} on Lines 15-16. Importantly, the function \texttt{try()} (Line 14) will assign the class \texttt{try-error} to an expression that fails.
  \index[terms]{R!classes!try-error}
  \vspace{0.1in}
\end{itemize}

\begin{Shaded}
\begin{Highlighting}[numbers=left,,firstnumber=13,]
\NormalTok{docalc }\OtherTok{\textless{}{-}} \ControlFlowTok{function}\NormalTok{()\{}
\NormalTok{      result }\OtherTok{=} \FunctionTok{try}\NormalTok{(}\FunctionTok{eval}\NormalTok{(}\FunctionTok{parse}\NormalTok{(}\AttributeTok{text =}\NormalTok{ calcinp)))}
      \ControlFlowTok{if}\NormalTok{(}\FunctionTok{class}\NormalTok{(result) }\SpecialCharTok{==} \StringTok{\textquotesingle{}try{-}error\textquotesingle{}}\NormalTok{)}
\NormalTok{        calcinp }\OtherTok{\textless{}\textless{}{-}} \StringTok{\textquotesingle{}Error\textquotesingle{}}
      \ControlFlowTok{else}\NormalTok{ calcinp }\OtherTok{\textless{}\textless{}{-}}\NormalTok{ result}
      \FunctionTok{tkconfigure}\NormalTok{(display, }\AttributeTok{text =}\NormalTok{ calcinp)}
\NormalTok{      calcinp }\OtherTok{\textless{}\textless{}{-}} \StringTok{\textquotesingle{}\textquotesingle{}}
\NormalTok{      \}}
\end{Highlighting}
\end{Shaded}

We call these three functions in the GUI itself, which is generated in the code below (Lines 21-60).

\begin{itemize}
\tightlist
\item
  The largest calculator code component defines the form and arrangement of calculator key (27-60). Note that buttons are packed, by row, using \texttt{tkpack()} within \texttt{tkframe()} objects. All button widgets use \texttt{command\ =\ mkput} except for the clear key, which uses \texttt{command\ =\ clearit}, and the equals key, which uses \texttt{command\ =\ docalc}.
  \vspace{0.1in}
\end{itemize}

\begin{Shaded}
\begin{Highlighting}[numbers=left,,firstnumber=21,]
\NormalTok{base }\OtherTok{\textless{}{-}} \FunctionTok{tktoplevel}\NormalTok{()}
\FunctionTok{tkwm.title}\NormalTok{(base,}\StringTok{\textquotesingle{}Calculator\textquotesingle{}}\NormalTok{)}

\NormalTok{display }\OtherTok{\textless{}{-}} \FunctionTok{tklabel}\NormalTok{(base,}\AttributeTok{justify=}\StringTok{\textquotesingle{}right\textquotesingle{}}\NormalTok{,}\AttributeTok{background=}\StringTok{"white"}\NormalTok{,}
                   \AttributeTok{relief=}\StringTok{"sunken"}\NormalTok{, }\AttributeTok{padx =} \DecValTok{50}\NormalTok{)}
\FunctionTok{tkpack}\NormalTok{(display,}\AttributeTok{side=}\StringTok{\textquotesingle{}top\textquotesingle{}}\NormalTok{)}
\NormalTok{row1 }\OtherTok{\textless{}{-}} \FunctionTok{tkframe}\NormalTok{(base)}
\FunctionTok{tkpack}\NormalTok{(}\FunctionTok{tkbutton}\NormalTok{(row1,}\AttributeTok{text=}\StringTok{\textquotesingle{}7\textquotesingle{}}\NormalTok{,}\AttributeTok{command=}\FunctionTok{mkput}\NormalTok{(}\StringTok{\textquotesingle{}7\textquotesingle{}}\NormalTok{),}\AttributeTok{width=}\DecValTok{3}\NormalTok{),}\AttributeTok{side=}\StringTok{\textquotesingle{}left\textquotesingle{}}\NormalTok{)}
\FunctionTok{tkpack}\NormalTok{(}\FunctionTok{tkbutton}\NormalTok{(row1,}\AttributeTok{text=}\StringTok{\textquotesingle{}8\textquotesingle{}}\NormalTok{,}\AttributeTok{command=}\FunctionTok{mkput}\NormalTok{(}\StringTok{\textquotesingle{}8\textquotesingle{}}\NormalTok{),}\AttributeTok{width=}\DecValTok{3}\NormalTok{),}\AttributeTok{side=}\StringTok{\textquotesingle{}left\textquotesingle{}}\NormalTok{)}
\FunctionTok{tkpack}\NormalTok{(}\FunctionTok{tkbutton}\NormalTok{(row1,}\AttributeTok{text=}\StringTok{\textquotesingle{}9\textquotesingle{}}\NormalTok{,}\AttributeTok{command=}\FunctionTok{mkput}\NormalTok{(}\StringTok{\textquotesingle{}9\textquotesingle{}}\NormalTok{),}\AttributeTok{width=}\DecValTok{3}\NormalTok{),}\AttributeTok{side=}\StringTok{\textquotesingle{}left\textquotesingle{}}\NormalTok{)}
\FunctionTok{tkpack}\NormalTok{(}\FunctionTok{tkbutton}\NormalTok{(row1,}\AttributeTok{text=}\StringTok{\textquotesingle{}+\textquotesingle{}}\NormalTok{,}\AttributeTok{command=}\FunctionTok{mkput}\NormalTok{(}\StringTok{\textquotesingle{}+\textquotesingle{}}\NormalTok{),}\AttributeTok{width=}\DecValTok{3}\NormalTok{),}\AttributeTok{side=}\StringTok{\textquotesingle{}left\textquotesingle{}}\NormalTok{)}
\FunctionTok{tkpack}\NormalTok{(row1,}\AttributeTok{side=}\StringTok{\textquotesingle{}top\textquotesingle{}}\NormalTok{)}

\NormalTok{row2 }\OtherTok{\textless{}{-}} \FunctionTok{tkframe}\NormalTok{(base)}
\FunctionTok{tkpack}\NormalTok{(}\FunctionTok{tkbutton}\NormalTok{(row2,}\AttributeTok{text=}\StringTok{\textquotesingle{}4\textquotesingle{}}\NormalTok{,}\AttributeTok{command=}\FunctionTok{mkput}\NormalTok{(}\StringTok{\textquotesingle{}4\textquotesingle{}}\NormalTok{),}\AttributeTok{width=}\DecValTok{3}\NormalTok{),}\AttributeTok{side=}\StringTok{\textquotesingle{}left\textquotesingle{}}\NormalTok{)}
\FunctionTok{tkpack}\NormalTok{(}\FunctionTok{tkbutton}\NormalTok{(row2,}\AttributeTok{text=}\StringTok{\textquotesingle{}5\textquotesingle{}}\NormalTok{,}\AttributeTok{command=}\FunctionTok{mkput}\NormalTok{(}\StringTok{\textquotesingle{}5\textquotesingle{}}\NormalTok{),}\AttributeTok{width=}\DecValTok{3}\NormalTok{),}\AttributeTok{side=}\StringTok{\textquotesingle{}left\textquotesingle{}}\NormalTok{)}
\FunctionTok{tkpack}\NormalTok{(}\FunctionTok{tkbutton}\NormalTok{(row2,}\AttributeTok{text=}\StringTok{\textquotesingle{}6\textquotesingle{}}\NormalTok{,}\AttributeTok{command=}\FunctionTok{mkput}\NormalTok{(}\StringTok{\textquotesingle{}6\textquotesingle{}}\NormalTok{),}\AttributeTok{width=}\DecValTok{3}\NormalTok{),}\AttributeTok{side=}\StringTok{\textquotesingle{}left\textquotesingle{}}\NormalTok{)}
\FunctionTok{tkpack}\NormalTok{(}\FunctionTok{tkbutton}\NormalTok{(row2,}\AttributeTok{text=}\StringTok{\textquotesingle{}{-}\textquotesingle{}}\NormalTok{,}\AttributeTok{command=}\FunctionTok{mkput}\NormalTok{(}\StringTok{\textquotesingle{}{-}\textquotesingle{}}\NormalTok{),}\AttributeTok{width=}\DecValTok{3}\NormalTok{),}\AttributeTok{side=}\StringTok{\textquotesingle{}left\textquotesingle{}}\NormalTok{)}
\FunctionTok{tkpack}\NormalTok{(row2,}\AttributeTok{side=}\StringTok{\textquotesingle{}top\textquotesingle{}}\NormalTok{)}

\NormalTok{row3 }\OtherTok{\textless{}{-}} \FunctionTok{tkframe}\NormalTok{(base)}
\FunctionTok{tkpack}\NormalTok{(}\FunctionTok{tkbutton}\NormalTok{(row3,}\AttributeTok{text=}\StringTok{\textquotesingle{}1\textquotesingle{}}\NormalTok{,}\AttributeTok{command=}\FunctionTok{mkput}\NormalTok{(}\StringTok{\textquotesingle{}1\textquotesingle{}}\NormalTok{),}\AttributeTok{width=}\DecValTok{3}\NormalTok{),}\AttributeTok{side=}\StringTok{\textquotesingle{}left\textquotesingle{}}\NormalTok{)}
\FunctionTok{tkpack}\NormalTok{(}\FunctionTok{tkbutton}\NormalTok{(row3,}\AttributeTok{text=}\StringTok{\textquotesingle{}2\textquotesingle{}}\NormalTok{,}\AttributeTok{command=}\FunctionTok{mkput}\NormalTok{(}\StringTok{\textquotesingle{}2\textquotesingle{}}\NormalTok{),}\AttributeTok{width=}\DecValTok{3}\NormalTok{),}\AttributeTok{side=}\StringTok{\textquotesingle{}left\textquotesingle{}}\NormalTok{)}
\FunctionTok{tkpack}\NormalTok{(}\FunctionTok{tkbutton}\NormalTok{(row3,}\AttributeTok{text=}\StringTok{\textquotesingle{}3\textquotesingle{}}\NormalTok{,}\AttributeTok{command=}\FunctionTok{mkput}\NormalTok{(}\StringTok{\textquotesingle{}3\textquotesingle{}}\NormalTok{),}\AttributeTok{width=}\DecValTok{3}\NormalTok{),}\AttributeTok{side=}\StringTok{\textquotesingle{}left\textquotesingle{}}\NormalTok{)}
\FunctionTok{tkpack}\NormalTok{(}\FunctionTok{tkbutton}\NormalTok{(row3,}\AttributeTok{text=}\StringTok{\textquotesingle{}*\textquotesingle{}}\NormalTok{,}\AttributeTok{command=}\FunctionTok{mkput}\NormalTok{(}\StringTok{\textquotesingle{}*\textquotesingle{}}\NormalTok{),}\AttributeTok{width=}\DecValTok{3}\NormalTok{),}\AttributeTok{side=}\StringTok{\textquotesingle{}left\textquotesingle{}}\NormalTok{)}
\FunctionTok{tkpack}\NormalTok{(row3,}\AttributeTok{side=}\StringTok{\textquotesingle{}top\textquotesingle{}}\NormalTok{)}

\NormalTok{row4 }\OtherTok{\textless{}{-}} \FunctionTok{tkframe}\NormalTok{(base)}
\FunctionTok{tkpack}\NormalTok{(}\FunctionTok{tkbutton}\NormalTok{(row4,}\AttributeTok{text=}\StringTok{\textquotesingle{}0\textquotesingle{}}\NormalTok{,}\AttributeTok{command=}\FunctionTok{mkput}\NormalTok{(}\StringTok{\textquotesingle{}0\textquotesingle{}}\NormalTok{),}\AttributeTok{width=}\DecValTok{3}\NormalTok{),}\AttributeTok{side=}\StringTok{\textquotesingle{}left\textquotesingle{}}\NormalTok{)}
\FunctionTok{tkpack}\NormalTok{(}\FunctionTok{tkbutton}\NormalTok{(row4,}\AttributeTok{text=}\StringTok{\textquotesingle{}(\textquotesingle{}}\NormalTok{,}\AttributeTok{command=}\FunctionTok{mkput}\NormalTok{(}\StringTok{\textquotesingle{}(\textquotesingle{}}\NormalTok{),}\AttributeTok{width=}\DecValTok{3}\NormalTok{),}\AttributeTok{side=}\StringTok{\textquotesingle{}left\textquotesingle{}}\NormalTok{)}
\FunctionTok{tkpack}\NormalTok{(}\FunctionTok{tkbutton}\NormalTok{(row4,}\AttributeTok{text=}\StringTok{\textquotesingle{})\textquotesingle{}}\NormalTok{,}\AttributeTok{command=}\FunctionTok{mkput}\NormalTok{(}\StringTok{\textquotesingle{})\textquotesingle{}}\NormalTok{),}\AttributeTok{width=}\DecValTok{3}\NormalTok{),}\AttributeTok{side=}\StringTok{\textquotesingle{}left\textquotesingle{}}\NormalTok{)}
\FunctionTok{tkpack}\NormalTok{(}\FunctionTok{tkbutton}\NormalTok{(row4,}\AttributeTok{text=}\StringTok{\textquotesingle{}/\textquotesingle{}}\NormalTok{,}\AttributeTok{command=}\FunctionTok{mkput}\NormalTok{(}\StringTok{\textquotesingle{}/\textquotesingle{}}\NormalTok{),}\AttributeTok{width=}\DecValTok{3}\NormalTok{),}\AttributeTok{side=}\StringTok{\textquotesingle{}left\textquotesingle{}}\NormalTok{)}
\FunctionTok{tkpack}\NormalTok{(row4,}\AttributeTok{side=}\StringTok{\textquotesingle{}top\textquotesingle{}}\NormalTok{)}

\NormalTok{row5 }\OtherTok{\textless{}{-}} \FunctionTok{tkframe}\NormalTok{(base)}
\FunctionTok{tkpack}\NormalTok{(}\FunctionTok{tkbutton}\NormalTok{(row5,}\AttributeTok{text=}\StringTok{\textquotesingle{}.\textquotesingle{}}\NormalTok{,}\AttributeTok{command=}\FunctionTok{mkput}\NormalTok{(}\StringTok{\textquotesingle{}.\textquotesingle{}}\NormalTok{),}\AttributeTok{width=}\DecValTok{3}\NormalTok{),}\AttributeTok{side=}\StringTok{\textquotesingle{}left\textquotesingle{}}\NormalTok{)}
\FunctionTok{tkpack}\NormalTok{(}\FunctionTok{tkbutton}\NormalTok{(row5,}\AttributeTok{text=}\StringTok{\textquotesingle{}\^{}\textquotesingle{}}\NormalTok{,}\AttributeTok{command=}\FunctionTok{mkput}\NormalTok{(}\StringTok{\textquotesingle{}\^{}\textquotesingle{}}\NormalTok{),}\AttributeTok{width=}\DecValTok{3}\NormalTok{),}\AttributeTok{side=}\StringTok{\textquotesingle{}left\textquotesingle{}}\NormalTok{)}
\FunctionTok{tkpack}\NormalTok{(}\FunctionTok{tkbutton}\NormalTok{(row5,}\AttributeTok{text=}\StringTok{\textquotesingle{}C\textquotesingle{}}\NormalTok{,}\AttributeTok{command=}\NormalTok{clearit,}\AttributeTok{width=}\DecValTok{3}\NormalTok{),}\AttributeTok{side=}\StringTok{\textquotesingle{}left\textquotesingle{}}\NormalTok{)}
\FunctionTok{tkpack}\NormalTok{(}\FunctionTok{tkbutton}\NormalTok{(row5,}\AttributeTok{text=}\StringTok{\textquotesingle{}=\textquotesingle{}}\NormalTok{,}\AttributeTok{command=}\NormalTok{docalc,}\AttributeTok{width=}\DecValTok{3}\NormalTok{),}\AttributeTok{side=}\StringTok{\textquotesingle{}left\textquotesingle{}}\NormalTok{)}
\FunctionTok{tkpack}\NormalTok{(row5,}\AttributeTok{side=}\StringTok{\textquotesingle{}top\textquotesingle{}}\NormalTok{)}
\end{Highlighting}
\end{Shaded}

A slightly modified form of the GUI (with colored buttons)\footnote{Code for Fig \ref{fig:tcltkcalc} can be obtained using \texttt{source(url("http://www2.cose.isu.edu/\textasciitilde{}ahoken/book/calctcltk.R"))}.} is shown in Fig \ref{fig:tcltkcalc}.

\begin{figure}[H]

{\centering \includegraphics[width=0.45\linewidth,]{figs11/tcltk_calc} 

}

\caption{A Tcl/Tk calculator GUI generated using the \textbf{R} package \textit{tcltk}.}\label{fig:tcltkcalc}
\end{figure}

The Python library \emph{tkinter} provides a well supported binding resource for Tcl/Tk. As we know (Ch \ref{ch9}) Python code can be run in \textbf{R}, using the package \emph{reticulate}. Fig \ref{fig:tkintercalc} shows an analogous calculator the one shown in Fig \ref{fig:tcltkcalc}, generated in \textbf{R} via the Python script \texttt{calc.py}, which is contained at the book website. It is important to note that while the resulting GUI is generated below in an RStudio \textbf{R} chunk, via \emph{reticulate}, the code and engines for running the GUI are Python, and thus, do not actually require \textbf{R}.

\begin{Shaded}
\begin{Highlighting}[]
\FunctionTok{library}\NormalTok{(reticulate)}
\FunctionTok{py\_run\_file}\NormalTok{(}\FunctionTok{source\_python}\NormalTok{(}\StringTok{"http://www2.cose.isu.edu/\textasciitilde{}ahoken/book/calc.py"}\NormalTok{))}
\end{Highlighting}
\end{Shaded}

The function \texttt{reticulate::source\_python()} allows one to access Python source code.
\index[R]{\texttt{reticulate::source\_python()}}

\begin{figure}[H]

{\centering \includegraphics[width=0.5\linewidth,]{figs11/tcltk_tkinter} 

}

\caption{A Tcl/Tk calculator GUI generated using Python code via the Python binding library \textit{tkinter}. Code follows a Python demo at the the \href{https://www.geeksforgeeks.org/python-simple-calculator-using-tkinter/}{geeksforgeeks} website.}\label{fig:tkintercalc}
\end{figure}

\end{example}

\(\blacksquare\)

\index[terms]{Python!package!\textit{tkinter}}
\index[terms]{R!package!\textit{reticulate}}
\index[R]{\texttt{reticulate::py\_run\_file}}

\subsubsection{Grid Manager}\label{grid-manager}

Use of \texttt{tkpack()} and \texttt{tkframe()} provides a great deal of flexibility for creating GUI layouts. They are, however, insufficient for handling a number of issues including lining widgets up vertically and horizontally. The grid manager function \texttt{tkgrid()} can be used to lay out widgets in rows and columns using the arguments \texttt{column} and \texttt{row}. Importantly, \texttt{column\ =\ 0} and \texttt{row\ =\ 0} equate to the first column and first row, respectively, in a GUI or container widget. Additional important
\texttt{tkgrid()} arguments include \texttt{columnspan}, \texttt{rowspan}, and \texttt{sticky}. The latter argument is analogous to \texttt{side} in \texttt{tkpack()}.

\index[R]{\texttt{tcltk::tkgrid()}}

\begin{example}
\protect\hypertarget{exm:exm-c115a}{}\label{exm:exm-c115a}

\(\text{}\)\\
The callback function below creates a single large blue dot in an \textbf{R} graphics device whose vertical position can be altered with a slider widget.
\vspace{0.1in}

\begin{Shaded}
\begin{Highlighting}[numbers=left,,]
\NormalTok{plot.me }\OtherTok{\textless{}{-}} \ControlFlowTok{function}\NormalTok{()\{}
\NormalTok{  y }\OtherTok{\textless{}{-}} \FunctionTok{evalq}\NormalTok{(}\FunctionTok{tclvalue}\NormalTok{(SliderValue)) }\CommentTok{\# Evaluate the expression}
  \FunctionTok{plot}\NormalTok{(}\DecValTok{1}\NormalTok{, }\FunctionTok{as.numeric}\NormalTok{(y),}\AttributeTok{xlab =} \StringTok{""}\NormalTok{, }\AttributeTok{ylab =} \StringTok{"\%"}\NormalTok{, }\AttributeTok{xaxt =} \StringTok{"n"}\NormalTok{, }\AttributeTok{ylim =} \FunctionTok{c}\NormalTok{(}\DecValTok{0}\NormalTok{,}\DecValTok{100}\NormalTok{),}
       \AttributeTok{cex =} \DecValTok{4}\NormalTok{, }\AttributeTok{col =} \DecValTok{4}\NormalTok{, }\AttributeTok{pch =} \DecValTok{19}\NormalTok{)}
\NormalTok{\}}
\end{Highlighting}
\end{Shaded}

The operation \texttt{evalq(foo)}, (Line 2) above, is equivalent to \texttt{eval(quote(foo))}. The operation \texttt{quote(foo)} simply returns the argument \texttt{foo} as an object of class ``call''.
\index[R]{\texttt{evalq()}}
\index[R]{\texttt{quote()}}
\index[terms]{R!classes!call}\}

The GUI code below uses grid geometry to place widgets in specified GUI rows and columns.
\vspace{0.1in}

\begin{Shaded}
\begin{Highlighting}[numbers=left,,firstnumber=6,]
\ControlFlowTok{if}\NormalTok{(}\FunctionTok{names}\NormalTok{(}\FunctionTok{dev.cur}\NormalTok{()) }\SpecialCharTok{==} \StringTok{"RStudioGD"}\NormalTok{) }\FunctionTok{dev.new}\NormalTok{(}\AttributeTok{noRStudioGD =} \ConstantTok{TRUE}\NormalTok{)}

\NormalTok{slider.env }\OtherTok{\textless{}\textless{}{-}} \FunctionTok{new.env}\NormalTok{()}
\NormalTok{tt }\OtherTok{\textless{}{-}} \FunctionTok{tktoplevel}\NormalTok{(); }\FunctionTok{tkwm.title}\NormalTok{(tt, }\StringTok{"Slider demo"}\NormalTok{)}
\NormalTok{SliderValue }\OtherTok{\textless{}{-}} \FunctionTok{tclVar}\NormalTok{(}\StringTok{"50"}\NormalTok{)}
\NormalTok{SliderValueLabel }\OtherTok{\textless{}{-}} \FunctionTok{tklabel}\NormalTok{(tt, }\AttributeTok{text =} \FunctionTok{as.character}\NormalTok{(}\FunctionTok{tclvalue}\NormalTok{(SliderValue)))}
\FunctionTok{tkgrid}\NormalTok{(}\FunctionTok{tklabel}\NormalTok{(tt, }\AttributeTok{text =} \StringTok{"Slider Value: "}\NormalTok{),}
\NormalTok{       SliderValueLabel, }\FunctionTok{tklabel}\NormalTok{(tt, }\AttributeTok{text =} \StringTok{"\%"}\NormalTok{))}
\FunctionTok{tkconfigure}\NormalTok{(SliderValueLabel, }\AttributeTok{textvariable =}\NormalTok{ SliderValue)}

\NormalTok{slider }\OtherTok{\textless{}{-}} \FunctionTok{tkscale}\NormalTok{(tt, }\AttributeTok{from =} \DecValTok{100}\NormalTok{, }\AttributeTok{to =} \DecValTok{0}\NormalTok{, }\AttributeTok{showvalue =}\NormalTok{ F,}
                  \AttributeTok{variable =}\NormalTok{ SliderValue, }\AttributeTok{resolution =} \DecValTok{1}\NormalTok{,}
                  \AttributeTok{orient =} \StringTok{"vertical"}\NormalTok{, }\AttributeTok{command =} \FunctionTok{substitute}\NormalTok{(}\FunctionTok{plot.me}\NormalTok{()))}

\FunctionTok{tkgrid}\NormalTok{(slider, }\AttributeTok{column =} \DecValTok{0}\NormalTok{, }\AttributeTok{row =} \DecValTok{1}\NormalTok{, }\AttributeTok{columnspan =} \DecValTok{2}\NormalTok{)}
\NormalTok{message }\OtherTok{=} \FunctionTok{tkmessage}\NormalTok{(tt, }\AttributeTok{text =} \StringTok{"Move the slider to see changes in the plot"}\NormalTok{)}
\FunctionTok{tkgrid}\NormalTok{(message, }\AttributeTok{column =} \DecValTok{3}\NormalTok{, }\AttributeTok{row =} \DecValTok{1}\NormalTok{)}
\end{Highlighting}
\end{Shaded}

\begin{itemize}
\tightlist
\item
  On Line 6, I insure that the interactive will work in the RStudio system by creating a non-RStudio graphics device if the default device is \texttt{"RStudioGD"}. The code should work inside and outside of RStudio.
\item
  On Line 8, I create an environment for the slider widget using \texttt{new.env()}.
\item
  On Line 9, I use use \texttt{tktoplevel()} to hierarchically define the ``top level'' widget as the object \texttt{tt}, and make a title.
\item
  On Line 10, I define 50 as the initial value for the slider widget that will be created.
\item
  On Line 11, a Tk label is created based on the slider output. Note the pairing of \texttt{tclVar()} input and \texttt{tclvalue()} output. In this process, the \texttt{SliderValueLabel} label object is configured to make its value equal to the \texttt{SliderValue} object.
\item
  On Lines 12-13, the \texttt{SliderValueLabel} is inserted between two text strings in a grid geometry.
\item
  On Lines 16-18, the slider is parameterized using the function \texttt{tkscale()}. The use of \texttt{substitute()} allows substitution of values for variables bound in the \texttt{plot.me()} function.
\item
  On Line 20, the slider is placed into the GUI at column 0 and row 1 (the first column and second row).
\item
  On Lines 21-22, a message is created and placed on the GUI at column 3 and row 1 (the fourth column and second row).
  \index[R]{\texttt{new.env()}}
  \index[R]{\texttt{tcltk::tkscale()}}
  \index[R]{\texttt{substitute()}}
\end{itemize}

The form of the final GUI is shown in Figure \ref{fig:tcltk2}.

\begin{figure}[H]

{\centering \includegraphics[width=0.4\linewidth,]{figs11/tcltk2} 

}

\caption{A \textit{tcltk} GUI for manipulating an \textbf{R} plot.}\label{fig:tcltk2}
\end{figure}

\end{example}

\(\blacksquare\)

\begin{example}
\protect\hypertarget{exm:exm-c11pack2}{}\label{exm:exm-c11pack2}

\(\text{}\)\\
Here is another grid manager example with a radiobutton GUI that allows selection of a bacterial phylum and printing of \texttt{"Correct"}, \texttt{"Incorrect"} text in the console, based on the button selection. It also embeds a photo.
\vspace{0.1in}

\begin{Shaded}
\begin{Highlighting}[numbers=left,,]
\NormalTok{tt }\OtherTok{\textless{}{-}} \FunctionTok{tktoplevel}\NormalTok{()}
\FunctionTok{tkwm.title}\NormalTok{(tt, }\StringTok{"Bacterial phyla"}\NormalTok{)}
\FunctionTok{tkgrid}\NormalTok{(}\FunctionTok{tklabel}\NormalTok{(tt, }\AttributeTok{text =} \StringTok{"Which phylum is shown?"}\NormalTok{, }\AttributeTok{padx =} \DecValTok{5}\NormalTok{,}
               \AttributeTok{pady =} \DecValTok{5}\NormalTok{, }\AttributeTok{font =} \StringTok{"bold"}\NormalTok{), }\AttributeTok{column =} \DecValTok{1}\NormalTok{, }\AttributeTok{row =} \DecValTok{1}\NormalTok{)}

\NormalTok{values }\OtherTok{\textless{}{-}} \FunctionTok{c}\NormalTok{(}\StringTok{"Acidobacteriota"}\NormalTok{, }\StringTok{"Armatimonadota"}\NormalTok{,}
            \StringTok{"Caldisericota"}\NormalTok{, }\StringTok{"Cyanobacteriota"}\NormalTok{,}
            \StringTok{"Elusimicrobiota"}\NormalTok{, }\StringTok{"Spirochaetota"}\NormalTok{,}
            \StringTok{"Thermomicrobia"}\NormalTok{)}

\NormalTok{var }\OtherTok{\textless{}{-}} \FunctionTok{tclVar}\NormalTok{(values[}\DecValTok{0}\NormalTok{]) }\CommentTok{\# initially, no phyla selected}

\FunctionTok{tkimage.create}\NormalTok{(}\StringTok{"photo"}\NormalTok{, }\StringTok{"cyano"}\NormalTok{, }\AttributeTok{file =} \StringTok{"figs11/cyano.gif"}\NormalTok{)}

\NormalTok{callback }\OtherTok{\textless{}{-}} \ControlFlowTok{function}\NormalTok{() }\FunctionTok{ifelse}\NormalTok{(}\FunctionTok{tclvalue}\NormalTok{(var) }\SpecialCharTok{==} \StringTok{"Cyanobacteriota"}\NormalTok{,}
                              \FunctionTok{print}\NormalTok{(}\StringTok{"Correct"}\NormalTok{),}
                              \FunctionTok{print}\NormalTok{(}\StringTok{"Incorrect"}\NormalTok{))}

\NormalTok{lf }\OtherTok{\textless{}{-}} \FunctionTok{ttkframe}\NormalTok{(tt)}
\FunctionTok{sapply}\NormalTok{(values, }\ControlFlowTok{function}\NormalTok{(x) \{}
\NormalTok{  radio\_button }\OtherTok{\textless{}{-}} \FunctionTok{ttkradiobutton}\NormalTok{(tt, }\AttributeTok{variable =}\NormalTok{ var,}
                                 \AttributeTok{text =}\NormalTok{ x, }\AttributeTok{value =}\NormalTok{ x,}
                                 \AttributeTok{command =}\NormalTok{ callback)}

 \FunctionTok{tkgrid}\NormalTok{(radio\_button, }\AttributeTok{pady =} \DecValTok{0}\NormalTok{, }\AttributeTok{padx =} \DecValTok{5}\NormalTok{, }\AttributeTok{sticky =} \StringTok{"nw"}\NormalTok{,}
        \AttributeTok{column =} \DecValTok{1}\NormalTok{, }\AttributeTok{rowspan =} \DecValTok{1}\NormalTok{)}
\NormalTok{ \})}

\FunctionTok{tkgrid}\NormalTok{(}\FunctionTok{tklabel}\NormalTok{(tt, }\AttributeTok{text =} \StringTok{""}\NormalTok{), }\AttributeTok{sticky =} \StringTok{"n"}\NormalTok{, }\AttributeTok{column =} \DecValTok{1}\NormalTok{,}
       \AttributeTok{row =} \DecValTok{9}\NormalTok{, }\AttributeTok{columnspan =} \DecValTok{1}\NormalTok{)}
\NormalTok{l }\OtherTok{\textless{}{-}} \FunctionTok{ttklabel}\NormalTok{(tt, }\AttributeTok{image =} \StringTok{"cyano"}\NormalTok{, }\AttributeTok{relief =} \StringTok{"ridge"}\NormalTok{)}
\FunctionTok{tkgrid}\NormalTok{(l, }\AttributeTok{sticky =} \StringTok{"nw"}\NormalTok{, }\AttributeTok{rowspan =} \DecValTok{10}\NormalTok{, }\AttributeTok{column =} \DecValTok{2}\NormalTok{,}
       \AttributeTok{row =} \DecValTok{0}\NormalTok{, }\AttributeTok{pady =} \DecValTok{25}\NormalTok{, }\AttributeTok{padx =}  \DecValTok{10}\NormalTok{)}
\end{Highlighting}
\end{Shaded}

\begin{itemize}
\tightlist
\item
  On Line 1, the top level widget, \texttt{tt}, is designated.
\item
  On Line 2, a GUI title is created.
\item
  On Lines 3-4, the text ``Which phylum is shown?'' is placed in the column 1, row 1 position, using the grid manager function \texttt{tkgrid()}.
\item
  On Lines 5-9, a character vector of bacterial phylum names is created for use in the GUI.
\item
  On Line 11 the initial phylum selection is specified. The use of \texttt{tclVar(values{[}0{]})} means that \emph{no} selection will be initially designated.
\item
  On Line 13, the function \texttt{tkimage.create()} is used to import a photo with .gif formatting (currently the only accepted format). The first argument \texttt{"photo"} indicates that an image will be created from a photo. The second argument creates an object name for the import, \texttt{"cyano"} that will called in later code.
\item
  On Lines 15-17, a callback function, \texttt{callback()} is created to print the text \texttt{"correct"} if Cyanobacteriota is selected, and print \texttt{"incorrect"} if some other selection is made.
\item
  On Line 19, an embedded frame is created to hold the radiobuttons.
\item
  On Lines 20-27, \texttt{sapply()} is used to run a user-defined function that embeds radiobuttons for each level in the character vector \texttt{values}.

  \begin{itemize}
  \tightlist
  \item
    On Line 21-23, the function creates an object \texttt{radio\_button} using \texttt{ttkradiobutton()}. Note that the top-level path name, \texttt{tt} is given as the first argument, the initial radiobutton designation is given in the second argument, the arguments \texttt{text} and \texttt{value} will change levels in \texttt{vaues} change. the function \texttt{callback()} is called using the \texttt{ttkradiobutton()} \texttt{command} argument to respond to the selected radiobutton.
  \item
    On Lines 25-26, radiobuttons are stacked, one row at a time, using \texttt{tkgrid()}, as \texttt{sapply()} cycles through levles in \texttt{values}.
  \end{itemize}
\item
  On Lines 29-30, an aesthetic empty row is created at the bottom of column one create some additional space.
\item
  On Lines 31-32, the object \texttt{l} is created from the function \texttt{ttklabel()} to hold the image object \texttt{cyno}, created on Line 13.\\
\item
  On Lines 33-34, the image is embedded into the entirety of column two.
\end{itemize}

The final form of the GUI is shown in Fig \ref{fig:phylaGUI}.

\index[R]{\texttt{tcltk::tkimage.create()}}
\index[R]{\texttt{tcltk::ttklabel.create()}}
\index[R]{\texttt{tcltk::ttkradiobutton()}}
\index[R]{\texttt{ifelse()}}
\index[R]{\texttt{sapply()}}

\begin{figure}[H]

{\centering \includegraphics[width=0.9\linewidth,]{figs11/phylaGUI} 

}

\caption{A radiobutton \textit{cltk} GUI. Photo from CSIRO, CC BY 3.0, https://commons.wikimedia.org/w/index.php?curid=3548094}\label{fig:phylaGUI}
\end{figure}

\end{example}

\(\blacksquare\)

\subsection{Widget Modifications}\label{widget-modifications}

Tcl/Tk provides shared modification settings for many of its widgets. These include the standard arguments \texttt{foreground} (the widget foreground color, see choices \href{https://www.tcl.tk/man/tcl/TkCmd/colors.htm}{here}), \texttt{background} (the widget background color), \texttt{image} (an image to display in the widget)\footnote{This requires creation with \texttt{tkimage.create()} (see Example \ref{exm:exm-c11pack2}).}, \texttt{relief} (the 3D appearance of the widget), \texttt{font}, and \texttt{text} (a text string to be placed in the widget), among many others. Be aware, however, that standard Tk widget modifiers may not always align with Ttk modifiers.

\begin{example}
\protect\hypertarget{exm:exm-c11mod}{}\label{exm:exm-c11mod}

\(\text{}\)\\

Here is an example of a GUI with working (but non-functional) widgets that allows demonstration of relief, background, and foreground color options for different types of widgets.

\vspace{0.1in}

\begin{Shaded}
\begin{Highlighting}[numbers=left,,]
\NormalTok{types }\OtherTok{=} \FunctionTok{c}\NormalTok{(}\StringTok{"flat"}\NormalTok{, }\StringTok{"groove"}\NormalTok{, }\StringTok{"raised"}\NormalTok{, }\StringTok{"ridge"}\NormalTok{,}
          \StringTok{"solid"}\NormalTok{, }\StringTok{"sunken"}\NormalTok{)}
\NormalTok{bg }\OtherTok{=} \FunctionTok{c}\NormalTok{(}\StringTok{"beige"}\NormalTok{, }\StringTok{"AntiqueWhite1"}\NormalTok{, }\StringTok{"aquamarine4"}\NormalTok{, }\StringTok{"burlywood3"}\NormalTok{)}
\NormalTok{fg }\OtherTok{=} \FunctionTok{c}\NormalTok{(}\StringTok{"BlueViolet"}\NormalTok{,}\StringTok{"aquamarine3"}\NormalTok{, }\StringTok{"white"}\NormalTok{, }\StringTok{"black"}\NormalTok{)}
          
\NormalTok{base }\OtherTok{=} \FunctionTok{tktoplevel}\NormalTok{()}
\FunctionTok{tkwm.title}\NormalTok{(base,}\StringTok{"Widget Styles"}\NormalTok{)}

\NormalTok{cnames }\OtherTok{\textless{}{-}} \FunctionTok{tkframe}\NormalTok{(base)}
\FunctionTok{tkpack}\NormalTok{(}\FunctionTok{tklabel}\NormalTok{(cnames, }\AttributeTok{text =} \StringTok{"Labels"}\NormalTok{, }\AttributeTok{font =} \StringTok{"bold"}\NormalTok{),}
       \AttributeTok{side =} \StringTok{"left"}\NormalTok{, }\AttributeTok{padx =} \DecValTok{3}\NormalTok{)}
\FunctionTok{tkpack}\NormalTok{(}\FunctionTok{tklabel}\NormalTok{(cnames, }\AttributeTok{text =} \StringTok{"Buttons"}\NormalTok{, }\AttributeTok{font =} \StringTok{"bold"}\NormalTok{),}
       \AttributeTok{side =} \StringTok{"left"}\NormalTok{, }\AttributeTok{padx =} \DecValTok{17}\NormalTok{)}
\FunctionTok{tkpack}\NormalTok{(}\FunctionTok{tklabel}\NormalTok{(cnames, }\AttributeTok{text =} \StringTok{"Radiobuttons"}\NormalTok{, }\AttributeTok{font =} \StringTok{"bold"}\NormalTok{),}
       \AttributeTok{side =} \StringTok{"left"}\NormalTok{, }\AttributeTok{padx =} \DecValTok{0}\NormalTok{)}
\FunctionTok{tkpack}\NormalTok{(}\FunctionTok{tklabel}\NormalTok{(cnames, }\AttributeTok{text =} \StringTok{"Sliders"}\NormalTok{, }\AttributeTok{font =} \StringTok{"bold"}\NormalTok{),}
       \AttributeTok{side =} \StringTok{"left"}\NormalTok{, }\AttributeTok{padx =} \DecValTok{22}\NormalTok{)}
\FunctionTok{tkpack}\NormalTok{(cnames, }\AttributeTok{side =} \StringTok{"top"}\NormalTok{, }\AttributeTok{fill =} \StringTok{"both"}\NormalTok{)}

\NormalTok{mkframe }\OtherTok{\textless{}{-}} \ControlFlowTok{function}\NormalTok{(type)\{}
\NormalTok{  fr }\OtherTok{\textless{}{-}} \FunctionTok{tkframe}\NormalTok{(base)}
  \FunctionTok{tkpack}\NormalTok{(}\FunctionTok{tklabel}\NormalTok{(fr, }\AttributeTok{text =}\NormalTok{ type, }\AttributeTok{relief =}\NormalTok{ type, }
                 \AttributeTok{fg =}\NormalTok{ fg[}\DecValTok{1}\NormalTok{], }\AttributeTok{bg =}\NormalTok{ bg[}\DecValTok{1}\NormalTok{]),}
         \AttributeTok{side =} \StringTok{"left"}\NormalTok{, }\AttributeTok{padx =} \DecValTok{5}\NormalTok{)}
  \FunctionTok{tkpack}\NormalTok{(}\FunctionTok{tkbutton}\NormalTok{(fr, }\AttributeTok{text =}\NormalTok{ type, }\AttributeTok{relief =}\NormalTok{ type, }
                 \AttributeTok{fg =}\NormalTok{ fg[}\DecValTok{2}\NormalTok{], }\AttributeTok{bg =}\NormalTok{ bg[}\DecValTok{2}\NormalTok{]),}
         \AttributeTok{side =} \StringTok{"left"}\NormalTok{, }\AttributeTok{padx =} \DecValTok{30}\NormalTok{)}
  \FunctionTok{tkpack}\NormalTok{(}\FunctionTok{tkradiobutton}\NormalTok{(fr, }\AttributeTok{text =}\NormalTok{ type, }\AttributeTok{relief =}\NormalTok{ type, }
                 \AttributeTok{fg =}\NormalTok{ fg[}\DecValTok{3}\NormalTok{], }\AttributeTok{bg =}\NormalTok{ bg[}\DecValTok{3}\NormalTok{]),}
         \AttributeTok{side =} \StringTok{"left"}\NormalTok{, }\AttributeTok{padx =} \DecValTok{10}\NormalTok{)}
  \FunctionTok{tkpack}\NormalTok{(}\FunctionTok{tkscale}\NormalTok{(fr, }\AttributeTok{from =} \DecValTok{0}\NormalTok{, }\AttributeTok{to =} \DecValTok{10}\NormalTok{, }\AttributeTok{showvalue =}\NormalTok{ T,}
                 \AttributeTok{variable =} \DecValTok{1}\NormalTok{, }\AttributeTok{resolution =} \DecValTok{1}\NormalTok{,}
                 \AttributeTok{orient =} \StringTok{"horizontal"}\NormalTok{,}
                 \AttributeTok{relief =}\NormalTok{ type, }\AttributeTok{fg =}\NormalTok{ fg[}\DecValTok{4}\NormalTok{], }
                 \AttributeTok{bg =}\NormalTok{ bg[}\DecValTok{4}\NormalTok{]), }\AttributeTok{side =} \StringTok{"left"}\NormalTok{, }\AttributeTok{padx =} \DecValTok{14}\NormalTok{)}

  \FunctionTok{tkpack}\NormalTok{(fr, }\AttributeTok{side =} \StringTok{"top"}\NormalTok{, }\AttributeTok{pady =} \DecValTok{5}\NormalTok{)}
\NormalTok{\}}

\FunctionTok{sapply}\NormalTok{(types, mkframe)}
\end{Highlighting}
\end{Shaded}

See Fig \ref{fig:styles}.

\begin{figure}[H]

{\centering \includegraphics[width=0.7\linewidth,]{figs11/styles} 

}

\caption{Standard widget modifications. Relief styles are varied within each widget type, and background and foreground colors are varied among widget types.}\label{fig:styles}
\end{figure}

\end{example}

\(\blacksquare\)

\subsection{\texorpdfstring{Additional \emph{tcltk} Packages and Toolkits}{Additional tcltk Packages and Toolkits}}\label{additional-tcltk-packages-and-toolkits}

Several \textbf{R} packages have been developed to streamline and extend the capacities of the \emph{tcltk} package. These include \emph{fgui} \citep{fgui} and \emph{PBSmodelling} \citep{PBS, schnute2013} which provides wrappers for some Tcl/Tk routines to simplify and facilitate GUI creation. The \texttt{gWidgets2} package has ambitiously sought to create \textbf{R} simplifying binding frameworks for several GUI toolkits including GTK, qt, and Tcl/Tk\footnote{GTK (formerly GIMP ToolKit and GTK+) and qt (pronounced `cute') are free, open-source, cross-platform, toolkits for creating GUIs.}. Currently, however, only the \texttt{gWidgets2} port for \texttt{tcltk}, called \emph{gWidgets2tcltk}, is working. The \emph{gWidgets2tcltk} package is currently used to build interactive self test questions for the pedagogic statistics package \emph{asbio} (Fig \ref{fig:typeIandII}).

\begin{Shaded}
\begin{Highlighting}[]
\NormalTok{asbio}\SpecialCharTok{::}\FunctionTok{selftest.typeIISS.tck1}\NormalTok{()}
\end{Highlighting}
\end{Shaded}

\begin{figure}[H]

{\centering \includegraphics[width=0.9\linewidth,]{figs11/typeI_II} 

}

\caption{A self-test GUI using the \textit{gWidgets2tcltk} function \texttt{gcheckboxgroup}. For GUI code type: \texttt{fix(selftest.typeIISS.tck1)}.}\label{fig:typeIandII}
\end{figure}

\index[R]{\texttt{gWidgets::gcheckboxgroup()}}
\index[terms]{R!package!\textit{fgui}}
\index[terms]{R!package!\textit{gWidgets2}}
\index[terms]{R!package!\textit{gWidgets2tcltk}}
\index[terms]{R!package!\textit{PBSmodelling}}

A number of Tcl/Tk extensions for \textbf{R} are available from the \emph{SciViews} family of packages \citep{grosjean}, including \emph{svDialogs} (an attempt at creating standard cross-platform dialog boxes), \emph{svGUI}, and \emph{tcltk2}. These packages, however, have not been updated (at least on CRAN) for several years.

\index[terms]{R!package!\textit{SciViews}}
\index[terms]{R!package!\textit{svDialogs}}
\index[terms]{R!package!\textit{svGUI}}
\index[terms]{R!package!\textit{tcltk2}}

\section{JS and JSON Interactive Apps}\label{JS}

Many newer interactive \textbf{R} applications are designed and implemented using JavaScript\footnote{Java is an OOP language designed to have few dependencies. Once compiled, Java code can run on all platforms that support Java. Additional details for Java web design are given \href{https://developer.mozilla.org/en-US/docs/Learn/JavaScript}{here}. JavaScript, while linguistically similar to Java, has many many important differences. For instance, JavaScript is an interpreted language, whereas Java code is generally compiled.} (JS) or JavaScript Object Notation\footnote{JavaScript Object Notation (JSON) was derived from JavaScript largely to facilitate server-to-browser session communication.} (JSON). GUIs generated from these approaches are often embedded in HTML\footnote{As noted in Section \ref{workflow}, HTML (Hypertext Markup Language) is the standard language for structuring web pages and web content. Basic HTML programming details are available from a number of sources, including this \href{https://developer.mozilla.org/en-US/docs/Learn/Getting_started_with_the_web/HTML_basics}{Mozilla developer site}.} format, and thus can be viewed from web browsers.

There are several widely used \textbf{R} packages for generating these sorts of apps. The two most popular are \emph{plotly} and \emph{shiny}.

\begin{itemize}
\item
  \emph{plotly} creates interactive HTML/web graphics via the \emph{plotly.js} JS library for interactive charts.
\item
  \emph{shiny} is a package designed by RStudio developers that utilizes other packages, chiefly \emph{htmltools} and \emph{htmlwidgets}.\footnote{The packages \emph{htmltools} and \emph{htmlwidgets} were created by RStudio developers for \textbf{R} bindings to JS libraries and HTML code.} to provide direct interfaces between HTML embedded GUIs and \textbf{R}.
\end{itemize}

The next two sections of this Chapter will focus on these packages and approaches.

\index[terms]{R!package!\textit{htmltools}}
\index[terms]{R!package!\textit{htmlwidgets}}
\index[terms]{R!package!\textit{shiny}}
\index[terms]{Java!package!\textit{plotly.js}}
\index[terms]{HTML}
\index[terms]{Java (programming language)}
\index[terms]{JavaScript (JS) (programming language)}
\index[terms]{JavaScript Open Notatation (JSON) (programming language)}

\section{plotly}\label{plotly}

\index[terms]{R!package!\textit{plotly}}
The package \emph{plotly} \citep{sievert2020}, uses the \textbf{R} package \emph{jsonlite}, which provides an \textbf{R} binder for JSON. JSON code is read by the JS library \emph{plotly.js} to create interactive HTML embedded graphics (Fig \ref{fig:plotly1}). Charts resulting from this process are interactive under a standardized \emph{plotly} format, although they don't represent GUIs in a conventional sense.

\begin{figure}[H]

{\centering \includegraphics[width=0.75\linewidth,]{figs11/plotly3} 

}

\caption{A graph from \citet{sievert2020} that shows the process of creating an HTML-embedded plotly chart from a graph generated in \textbf{R}.}\label{fig:plotly1}
\end{figure}

\begin{example}
\protect\hypertarget{exm:exm-c116}{}\label{exm:exm-c116}

\(\text{}\)\\

To provide a simple demonstration I bring in some libraries, including \emph{plotly}, and the \texttt{world.emissions} data from package \emph{asbio}.

\vspace{0.1in}

\begin{Shaded}
\begin{Highlighting}[numbers=left,,]
\FunctionTok{library}\NormalTok{(tidyverse); }\FunctionTok{library}\NormalTok{(plotly)}
\FunctionTok{library}\NormalTok{(asbio); }\FunctionTok{data}\NormalTok{(world.emissions)}

\NormalTok{subset }\OtherTok{\textless{}{-}}\NormalTok{ world.emissions }\SpecialCharTok{|\textgreater{}}
  \FunctionTok{filter}\NormalTok{(country }\SpecialCharTok{\%in\%} \FunctionTok{c}\NormalTok{(}\StringTok{"United States"}\NormalTok{, }\StringTok{"Mexico"}\NormalTok{, }\StringTok{"China"}\NormalTok{,}
                        \StringTok{"Germany"}\NormalTok{, }\StringTok{"Russia"}\NormalTok{, }\StringTok{"Canada"}\NormalTok{)) }\SpecialCharTok{|\textgreater{}}
  \FunctionTok{filter}\NormalTok{(year }\SpecialCharTok{\textgreater{}} \DecValTok{1950} \SpecialCharTok{\&}\NormalTok{ year }\SpecialCharTok{\textless{}=} \DecValTok{2019}\NormalTok{)}
\end{Highlighting}
\end{Shaded}

Plotly graphs are rendered using the function \texttt{plot\_ly()}.
\vspace{0.1in}

\begin{Shaded}
\begin{Highlighting}[numbers=left,,firstnumber=8,]
\FunctionTok{plot\_ly}\NormalTok{(subset, }\AttributeTok{x =} \SpecialCharTok{\textasciitilde{}}\NormalTok{year, }\AttributeTok{y =} \SpecialCharTok{\textasciitilde{}}\NormalTok{co2) }\SpecialCharTok{|\textgreater{}}
  \FunctionTok{add\_lines}\NormalTok{(}\AttributeTok{color =} \SpecialCharTok{\textasciitilde{}}\NormalTok{country) }\SpecialCharTok{|\textgreater{}}
  \FunctionTok{layout}\NormalTok{(}
        \AttributeTok{yaxis =} \FunctionTok{list}\NormalTok{(}\AttributeTok{tickfont =} \FunctionTok{list}\NormalTok{(}\AttributeTok{size =} \DecValTok{20}\NormalTok{),}
                     \AttributeTok{title =}\StringTok{\textquotesingle{}CO\textbackslash{}U2082 (million tonnes)\textquotesingle{}}\NormalTok{,}
                     \AttributeTok{titlefont =} \FunctionTok{list}\NormalTok{(}\AttributeTok{size =} \DecValTok{23}\NormalTok{)),}
        \AttributeTok{xaxis =} \FunctionTok{list}\NormalTok{(}\AttributeTok{tickfont =} \FunctionTok{list}\NormalTok{(}\AttributeTok{size =} \DecValTok{20}\NormalTok{),}
                     \AttributeTok{title=}\StringTok{\textquotesingle{}Year\textquotesingle{}}\NormalTok{,}
                     \AttributeTok{titlefont =} \FunctionTok{list}\NormalTok{(}\AttributeTok{size =} \DecValTok{23}\NormalTok{)),}
        \AttributeTok{legend =} \FunctionTok{list}\NormalTok{(}\AttributeTok{font =} \FunctionTok{list}\NormalTok{(}\AttributeTok{size =} \DecValTok{20}\NormalTok{))}
\NormalTok{        )}
\end{Highlighting}
\end{Shaded}

\begin{itemize}
\tightlist
\item
  On Line 8, I call \texttt{plot\_ly()}. Note the use of the tilde operator to call \emph{x} and \emph{y} axis variables, i.e., \texttt{x\ =\ \textasciitilde{}year,\ y\ =\ \textasciitilde{}co2}.
\item
  On Lines 9-18, I call additional plot characteristics using \emph{tidyverse} pipe operators.

  \begin{itemize}
  \tightlist
  \item
    On Line 9 I use \texttt{add\_lines()} to add a line \emph{trace} to the plot.
  \item
    Plot characteristics can be modified in a large number of ways using lists within the function \texttt{layout} (Lines 10-18).
  \end{itemize}
\end{itemize}

The result is shown in Fig \ref{fig:plotly2a}.
If you are viewing this document as an HTML, the lines in the plot will be interactive, and the plot will contain a menu that allows summarization of single or multiple data points, panning and zooming.
\index[R]{\texttt{plotly::plot\_ly()}}

\vspace{0.1in}

\index[R]{\texttt{plotly::add\_lines()}}
\index[R]{\texttt{plotly::add\_trace()}}
\index[R]{\texttt{plotly::layout()}}

\begin{figure}[H]

{\centering \includegraphics[width=0.9\linewidth,]{figs11/plotly5} 

}

\caption{Simple \textit{plolty} chart with an interactive trace and standard \textit{plotly} menu shown (topright).}\label{fig:plotly2a}
\end{figure}

\end{example}

\(\blacksquare\)

\subsection{\texorpdfstring{\emph{ggplot} and \emph{plotly}}{ggplot and plotly}}\label{ggplot-and-plotly}

\emph{Ggplot2} objects can be converted (imperfectly) to \emph{plotly} objects using the function \texttt{plolty::ggplotly()}.
\index[R]{\texttt{plolty::ggplotly()}}
Unfortunately, a large number \texttt{ggplot} layout characteristics including figure margins and locations of \(x\) and \(y\) axis labels will not translate to \texttt{ggplotly()}. Instead, we must call on potentially exhaustive hierarchically nested list components. This can be a pain, and it may be expedient to build separate list or function objects to facilitate the process.

\begin{example}
\protect\hypertarget{exm:exm-c117}{}\label{exm:exm-c117}

\(\text{}\)\\

To illustrate I extend the previous example. First, I create a nested list object, \texttt{k}, that specifies desired margin and axis characteristics.
\vspace{0.1in}

\begin{Shaded}
\begin{Highlighting}[numbers=left,,]
\NormalTok{k }\OtherTok{\textless{}{-}} \FunctionTok{list}\NormalTok{(}
  \AttributeTok{yaxis =} \FunctionTok{list}\NormalTok{(}\AttributeTok{title =} \FunctionTok{list}\NormalTok{(}\AttributeTok{font =} \FunctionTok{list}\NormalTok{(}\AttributeTok{size =} \DecValTok{18}\NormalTok{)),}
               \AttributeTok{tickfont =} \FunctionTok{list}\NormalTok{(}\AttributeTok{size =} \DecValTok{14}\NormalTok{)),}
  \AttributeTok{xaxis =} \FunctionTok{list}\NormalTok{(}\AttributeTok{title =} \FunctionTok{list}\NormalTok{(}\AttributeTok{font =} \FunctionTok{list}\NormalTok{(}\AttributeTok{size =} \DecValTok{18}\NormalTok{)),}
               \AttributeTok{tickfont =} \FunctionTok{list}\NormalTok{(}\AttributeTok{size =} \DecValTok{15}\NormalTok{)),}
    \AttributeTok{margin =} \FunctionTok{list}\NormalTok{(}\AttributeTok{t =} \DecValTok{20}\NormalTok{, }\AttributeTok{r =} \DecValTok{20}\NormalTok{, }\AttributeTok{b =} \DecValTok{80}\NormalTok{, }\AttributeTok{l =} \DecValTok{80}\NormalTok{))}
\end{Highlighting}
\end{Shaded}

Here I create a simple ggplot boxplot, \texttt{g}. To get the desired characteristics in the mapped plotly graph I call on list components in \texttt{k} within \texttt{plotly::layout()}. Fig \ref{fig:plotly3a} shows the result.

\vspace{0.1in}

\begin{Shaded}
\begin{Highlighting}[numbers=left,,firstnumber=7,]
\NormalTok{g }\OtherTok{\textless{}{-}} \FunctionTok{ggplot}\NormalTok{(subset, }\FunctionTok{aes}\NormalTok{(}\AttributeTok{x =}\NormalTok{ country, }\AttributeTok{y =}\NormalTok{ co2)) }\SpecialCharTok{+}
  \FunctionTok{geom\_boxplot}\NormalTok{(}\FunctionTok{aes}\NormalTok{(}\AttributeTok{fill =}\NormalTok{ country)) }\SpecialCharTok{+}
  \FunctionTok{theme\_classic}\NormalTok{() }\SpecialCharTok{+}
  \FunctionTok{ylab}\NormalTok{(}\StringTok{"CO\textbackslash{}U2082 (million tonnes)"}\NormalTok{) }\SpecialCharTok{+}
  \FunctionTok{xlab}\NormalTok{(}\StringTok{"Country"}\NormalTok{)}

\FunctionTok{ggplotly}\NormalTok{(g) }\SpecialCharTok{\%\textgreater{}\%}
  \FunctionTok{layout}\NormalTok{(}\AttributeTok{showlegend =} \ConstantTok{FALSE}\NormalTok{, }\AttributeTok{xaxis =}\NormalTok{ k}\SpecialCharTok{$}\NormalTok{xaxis, }\AttributeTok{yaxis =}\NormalTok{ k}\SpecialCharTok{$}\NormalTok{yaxis,}
         \AttributeTok{margin =}\NormalTok{ k}\SpecialCharTok{$}\NormalTok{margin)}
\end{Highlighting}
\end{Shaded}

\begin{figure}[H]

{\centering \includegraphics[width=0.9\linewidth,]{figs11/plotly4} 

}

\caption{Simple \textit{plolty} barplot, based on a ggplot. Interactive trace for the US shown.}\label{fig:plotly3a}
\end{figure}

\end{example}

\(\blacksquare\)

\begin{example}
\protect\hypertarget{exm:exm-c118}{}\label{exm:exm-c118}\(\text{}\)\\

Here is another application using the function \texttt{GGally::ggcoef()} to create a coefficient plot\index[terms]{R!graphics!coefficient plot}. A coefficient plot displays statistical model parameter estimates and confidence intervals. We can make the plot interactive (in HTML) using \texttt{ggplotly()} (Fig \ref{fig:plotly4a}).

\vspace{0.1in}

\begin{Shaded}
\begin{Highlighting}[numbers=left,,]
\FunctionTok{library}\NormalTok{(GGally)}
\NormalTok{model }\OtherTok{\textless{}{-}} \FunctionTok{lm}\NormalTok{(}\FunctionTok{log}\NormalTok{(co2) }\SpecialCharTok{\textasciitilde{}}\NormalTok{ country }\SpecialCharTok{+}\NormalTok{ year,}
            \AttributeTok{data =}\NormalTok{ subset)}

\NormalTok{gg }\OtherTok{\textless{}{-}} \FunctionTok{ggcoef}\NormalTok{(model,}
             \AttributeTok{exclude\_intercept =} \ConstantTok{TRUE}\NormalTok{,}
             \AttributeTok{errorbar\_height =}\NormalTok{ .}\DecValTok{1}\NormalTok{,}
             \AttributeTok{color =} \StringTok{"blue"}\NormalTok{) }\SpecialCharTok{+}
  \FunctionTok{theme\_bw}\NormalTok{()}

\FunctionTok{ggplotly}\NormalTok{(gg) }\SpecialCharTok{\%\textgreater{}\%}
  \FunctionTok{layout}\NormalTok{(}\AttributeTok{yaxis =} \FunctionTok{list}\NormalTok{(}\AttributeTok{tickfont =} \FunctionTok{list}\NormalTok{(}\AttributeTok{size =} \DecValTok{15}\NormalTok{),}
                       \AttributeTok{title =}\StringTok{\textquotesingle{}\textquotesingle{}}\NormalTok{,}
                       \AttributeTok{titlefont =} \FunctionTok{list}\NormalTok{(}\AttributeTok{size =} \DecValTok{18}\NormalTok{)),}
          \AttributeTok{xaxis =} \FunctionTok{list}\NormalTok{(}\AttributeTok{tickfont =} \FunctionTok{list}\NormalTok{(}\AttributeTok{size =} \DecValTok{15}\NormalTok{),}
                        \AttributeTok{title=}\StringTok{\textquotesingle{}Parameter estimates\textquotesingle{}}\NormalTok{,}
                        \AttributeTok{titlefont =} \FunctionTok{list}\NormalTok{(}\AttributeTok{size =} \DecValTok{18}\NormalTok{)),}
           \AttributeTok{legend =} \FunctionTok{list}\NormalTok{(}\AttributeTok{font =} \FunctionTok{list}\NormalTok{(}\AttributeTok{size =} \DecValTok{15}\NormalTok{))}
\NormalTok{           )}
\end{Highlighting}
\end{Shaded}

\begin{figure}[H]

{\centering \includegraphics[width=0.8\linewidth,]{figs11/plotly7} 

}

\caption{Coefficient plot from the function \texttt{GGally::ggcoef()}, rendered using \texttt{ggplotly()}.}\label{fig:plotly4a}
\end{figure}

\index[R]{\texttt{GGally:ggcoef()}}
\index[terms]{R!package!\textit{GGally}}
\end{example}

\(\blacksquare\)

\section{\texorpdfstring{\emph{shiny}}{shiny}}\label{shiny}

Probably the easiest and most flexible way to create interactive HTML apps is through tools in the package \emph{shiny}. Unlike \emph{ploty} apps, \emph{shiny} apps will allow real-time interfacing with \textbf{R} for computations. RStudio has internals to facilitate \emph{shiny} app creation for embedding on webpages. Examples given here are often based on apps described in Hadley Wickham's book \emph{Mastering Shiny} \citep{wickham2021mastering}.
\index[terms]{R!package!\textit{shiny}}

A \emph{shiny} app will have three components:

\begin{itemize}
\tightlist
\item
  A user interface (\texttt{ui}) specification that defines how your app \emph{looks}.
\item
  A server function that defines how your app \emph{works}. The function will (generally) have three arguments \texttt{input}, \texttt{output}, and \texttt{session}.
\item
  An app execution call that conjoins the user interface and server functions. This is done with \texttt{shinyApp()}
\end{itemize}

\begin{example}
\protect\hypertarget{exm:exm-c119}{}\label{exm:exm-c119}

As a first example, here is \textbf{R} code for rendering text in an HTML app (Fig \ref{fig:hw1}).
\vspace{0.1in}

\begin{Shaded}
\begin{Highlighting}[numbers=left,,]
\FunctionTok{library}\NormalTok{(shiny)}
\NormalTok{ui }\OtherTok{\textless{}{-}} \FunctionTok{fluidPage}\NormalTok{(}
  \StringTok{"Hello, world!"}
\NormalTok{)}
\NormalTok{server }\OtherTok{\textless{}{-}} \ControlFlowTok{function}\NormalTok{(input, output, session) \{}
\NormalTok{\}}
\FunctionTok{shinyApp}\NormalTok{(ui, server)}
\end{Highlighting}
\end{Shaded}

\begin{figure}[H]

{\centering \includegraphics[width=0.5\linewidth,]{figs11/hello_world} 

}

\caption{A very simple \textit{shiny} app. Code follows \citet{wickham2021mastering}.}\label{fig:hw1}
\end{figure}

\begin{itemize}
\tightlist
\item
  \textbf{\texttt{ui}:} \texttt{shiny::fluidPage()} is a layout function that defines the basic visual structure of the app (Lines 1-4). Among other things the function allows definition of app rows using \texttt{shiny::fluidRow()}, and columns (within rows) using \texttt{shiny::column()}. Fluid pages can rescale their components in real-time to fill the available GUI window width.
\item
  \textbf{\texttt{server}:} For this simple example the \texttt{server} function contains no commands (Lines 5-6). Although, as a matter of convention, the \texttt{server} arguments: \texttt{input,\ output,\ session} are still included.
\item
  \textbf{\texttt{shinyApp}:} The app function pairs the \texttt{ui}/\texttt{server} objects (Line 7).
\end{itemize}

\end{example}

\(\blacksquare\)

\index[R]{\texttt{shiny::fluidPage()}}
\index[R]{\texttt{shiny::fluidRow()}}
\index[R]{\texttt{shiny::column()}}

One can open an \textbf{R} script with a \emph{shiny} app skeleton in RStudio by going to \textbf{File\(>\)New File\(>\)Shiny Web Application}. This will allow RStudio to recognize the script as app code. This, in turn, allows running the app by either sending its code to the console (e.g., using Ctrl + Enter), or by using the \textbf{Run App} button in the \textbf{Shiny Web Application} toolbar.

\begin{example}
\protect\hypertarget{exm:exm-c1110}{}\label{exm:exm-c1110}

Here is a simple app that lists and provides details about datasets in the package \emph{asbio} (Fig \ref{fig:asbiod1}).
\vspace{0.1in}

\begin{Shaded}
\begin{Highlighting}[numbers=left,,]
\NormalTok{ui }\OtherTok{\textless{}{-}} \FunctionTok{fluidPage}\NormalTok{(}
  \FunctionTok{selectInput}\NormalTok{(}\StringTok{"dataset"}\NormalTok{, }\AttributeTok{label =} \StringTok{"Dataset"}\NormalTok{,}
              \AttributeTok{choices =} \FunctionTok{data}\NormalTok{(}\AttributeTok{package =} \StringTok{"asbio"}\NormalTok{)}\SpecialCharTok{$}\NormalTok{results[,}\DecValTok{3}\NormalTok{])}
\NormalTok{)}
\NormalTok{server }\OtherTok{\textless{}{-}} \ControlFlowTok{function}\NormalTok{(input, output, session) \{}
\NormalTok{\}}
\FunctionTok{shinyApp}\NormalTok{(ui, server)}
\end{Highlighting}
\end{Shaded}

\begin{itemize}
\tightlist
\item
  \textbf{\texttt{ui}:} \texttt{fluidPage()} includes \texttt{shiny::selectInput()}, an input control function that provides the user with a select box widget. An appropriate label \texttt{"Dataset"} is defined. Selection box \texttt{choices} are the third column in \texttt{data(package\ =\ "asbio")\$results}, which contains the names of the dataframe object names in \emph{asbio}.
\item
  \textbf{\texttt{server}:} Once again, the \texttt{server} function contains no commands (Lines 5-6).
\item
  \textbf{\texttt{shinyApp}:} The app function again pairs the \texttt{ui}/\texttt{server} objects (Line 7).
\end{itemize}

\index[R]{\texttt{shiny::verbatimTextOutput()}}
\index[R]{\texttt{shiny::tableOutput()}}
\index[R]{\texttt{shiny::selectInput()}}
\vspace{0.1in}

\begin{figure}[H]

{\centering \includegraphics[width=0.5\linewidth,]{figs11/asbio1} 

}

\caption{A \textit{shiny} app to allow scrolling through \textit{asbio} datasets.}\label{fig:asbiod1}
\end{figure}

The app in Fig \ref{fig:asbiod1} has limited usefulness because it provides only the \texttt{asbio} dataframe object names. Indeed, we could get more information by simply running \texttt{data(package\ =\ "asbio")}. Here we insert additional features into the user interface and server to increase functionality (Fig \ref{fig:asbiod2}).
\vspace{0.1in}

\begin{Shaded}
\begin{Highlighting}[numbers=left,,]
\NormalTok{adata }\OtherTok{\textless{}{-}} \FunctionTok{data}\NormalTok{(}\AttributeTok{package =} \StringTok{"asbio"}\NormalTok{)}\SpecialCharTok{$}\NormalTok{results[,}\DecValTok{3}\NormalTok{]}
\FunctionTok{data}\NormalTok{(}\AttributeTok{list =}\NormalTok{ adata[}\DecValTok{1}\SpecialCharTok{:}\FunctionTok{length}\NormalTok{(adata)]) }\CommentTok{\# loads all asbio datasets}

\NormalTok{ui }\OtherTok{\textless{}{-}} \FunctionTok{fluidPage}\NormalTok{(}
  \FunctionTok{selectInput}\NormalTok{(}\StringTok{"dataset"}\NormalTok{, }\AttributeTok{label =} \StringTok{"Dataset"}\NormalTok{, }\AttributeTok{choices =}\NormalTok{ adata),}
  \FunctionTok{verbatimTextOutput}\NormalTok{(}\StringTok{"summary"}\NormalTok{),}
  \FunctionTok{tableOutput}\NormalTok{(}\StringTok{"table"}\NormalTok{)}
\NormalTok{)}

\NormalTok{server }\OtherTok{\textless{}{-}} \ControlFlowTok{function}\NormalTok{(input, output, session)\{}
\NormalTok{    output}\SpecialCharTok{$}\NormalTok{summary }\OtherTok{\textless{}{-}} \FunctionTok{renderPrint}\NormalTok{(\{}
\NormalTok{    dataset }\OtherTok{\textless{}{-}} \FunctionTok{get}\NormalTok{(input}\SpecialCharTok{$}\NormalTok{dataset)}
    \FunctionTok{summary}\NormalTok{(dataset)}
\NormalTok{  \})}

\NormalTok{  output}\SpecialCharTok{$}\NormalTok{table }\OtherTok{\textless{}{-}} \FunctionTok{renderTable}\NormalTok{(\{}
\NormalTok{      dataset }\OtherTok{\textless{}{-}} \FunctionTok{get}\NormalTok{(input}\SpecialCharTok{$}\NormalTok{dataset)}
\NormalTok{    dataset}
\NormalTok{  \})}
\NormalTok{\}}

\FunctionTok{shinyApp}\NormalTok{(ui, server)}
\end{Highlighting}
\end{Shaded}

\begin{itemize}
\item
  The code \texttt{data(list\ =\ adata{[}1:length(adata){]})} loads all the \emph{asbio} datasets into the global environment (Line 2).
\item
  \textbf{\texttt{ui}:} \texttt{fluidPage()} now specifies three features, which will occur from top to bottom app, as they are listed (Lines 4-8).

  \begin{itemize}
  \tightlist
  \item
    The functions\texttt{shiny::verbatimTextOutput()} (Line 6) and \texttt{shiny::tableOutput()} (Line 7) are controls that define how and where output (depending on the order they are specified in \texttt{fluidPage()}) are displayed. Specifically, \texttt{verbatimTextOutput()} displays code, and \texttt{tableOutput()} displays tables.
  \end{itemize}
\item
  \textbf{\texttt{server}:} The \texttt{server} function has been modified to allow interaction with the user interface (Lines 10-20). It allows generations of two objects: \texttt{output\$summary} (Line 11) and \texttt{output\$table} (Line 12), based on \texttt{input\$dataset} from the \texttt{ui}.

  \begin{itemize}
  \tightlist
  \item
    \texttt{output\$summary} (lines 10-14) is a rendered expression from \texttt{shiny::renderPrint()}. In particular, this will be output from \texttt{summary()} (Line 13) for columns in \texttt{input\$dataset} which is made available in the object \texttt{dataset} on Line 12. The output operation is coupled with \texttt{verbatimTextOutput("summary")} in the \texttt{ui}.\\
  \item
    \texttt{output\$table} (Lines 16-19) is a rendered expression from \texttt{shiny::renderTable()}. It will show the raw data in a scrollable table for the dataframe specified in the \texttt{ui}. This output operation is coupled with \texttt{tableOutput("table")} in the \texttt{ui}.
  \end{itemize}
\item
  \textbf{\texttt{shinyApp}:} As before, we generate the app using: \texttt{shinyApp(ui,\ server)} (Line 22).
\end{itemize}

\index[R]{\texttt{shiny::renderPrint()}}
\index[R]{\texttt{shiny::renderTable()}}
\index[R]{\texttt{get()}}

\begin{figure}[H]

{\centering \includegraphics[width=0.5\linewidth,]{figs11/asbio2} 

}

\caption{A modified \textit{shiny} app to provide summaries of scrollable \textit{asbio} datasets.}\label{fig:asbiod2}
\end{figure}

\end{example}

\(\blacksquare\)

\subsection{\texorpdfstring{\texttt{ui} Details}{ui Details}}\label{ui-details}

Input widget functions are specified in the \texttt{ui}. A \texttt{ui} input function, e.g., \texttt{selectInput()} with first argument \texttt{"foo"}, or with \texttt{inputId\ =\ "foo"} can be called by \texttt{server} operations using the script \texttt{input\$foo}. Most input functions have a 2nd argument called \texttt{label} that creates a user-readable label for the control widget on the app. A The 3rd input argument is typically \texttt{value} which creates a starting value for the widget control. Fig \ref{fig:widgets} shows many of the standard \emph{shiny} \texttt{ui} input functions (without output). Important \texttt{ui} import functions are also listed Table \ref{tab:uii}.
\vspace{0.1in}

\begin{Shaded}
\begin{Highlighting}[]
\FunctionTok{source}\NormalTok{(}\StringTok{"shiny\_widgets.R"}\NormalTok{)}
\FunctionTok{shiny\_widgets}\NormalTok{()}
\end{Highlighting}
\end{Shaded}

\begin{figure}[H]

{\centering \includegraphics[width=0.8\linewidth,]{figs11/widgets} 

}

\caption{A variety of \textit{shiny} input widgets and operations that can be specified in the \texttt{ui}. By row, the figure depicts widgets generated by the functions:  \texttt{actionButton()},  \texttt{submitButton()},  \texttt{checkboxInput},  \texttt{checkboxGroupInput()},  \texttt{dateInput()},  \texttt{dateRangeInput()},  \texttt{fileInput()},  \texttt{helpText()},  \texttt{numericInput()},  \texttt{radioButtons()},  \texttt{selectInput()},  \texttt{sliderInput()}, and  \texttt{textInput()}. Also see Table \ref{tab:uii}.}\label{fig:widgets}
\end{figure}

\index[R]{\texttt{shiny::actionButton()}}
\index[R]{\texttt{shiny::submitButton()}}
\index[R]{\texttt{shiny::checkboxInput}}
\index[R]{\texttt{shiny::checkboxGroupInput()}} \index[R]{\texttt{shiny::dateInput()}}
\index[R]{\texttt{shiny::dateRangeInput()}}
\index[R]{\texttt{shiny::fileInput()}}
\index[R]{\texttt{shiny::helpText()}}
\index[R]{\texttt{shiny::numericInput()}}
\index[R]{\texttt{shiny::radioButtons()}}
\index[R]{\texttt{shiny::selectInput()}}
\index[R]{\texttt{shiny::sliderInput()}}
\index[R]{\texttt{shiny::textInput()}}
\index[R]{\texttt{shiny::passwordInput()}}

\begin{table}[H]
\caption{Some important \textit{shiny} \texttt{ui} input functions.}
\begin{center}

\renewcommand{\arraystretch}{1.1}
\begin{tabular}{ll} \hline
Function & Purpose \\ \hline
\begin{tabular}[c]{@{}l@{}}\texttt{actionButton()} \\ \texttt{actionLink()} \end{tabular}                & \begin{tabular}[c]{@{}l@{}} Creates an action button or link. \end{tabular}                  \\
\begin{tabular}[c]{@{}l@{}}\texttt{submitButton()}\end{tabular}  & Create a submit button.     \\
\begin{tabular}[c]{@{}l@{}}\texttt{checkboxInput()}\end{tabular} &  Create a checkbox to specify logical values. \\
\begin{tabular}[c]{@{}l@{}}\texttt{dateInput()}\end{tabular}  & Create a selectable calendar.   \\
\begin{tabular}[c]{@{}l@{}}\texttt{dateRangeInput()} \end{tabular}  & Create a pair of selectable calendars.                      \\
\texttt{fileInput()} & Create a file upload control to upload one or more files.\\
\texttt{helpText()} & \begin{tabular}[c]{@{}l@{}} Create help text which can be added to input \\to provide additional information. \end{tabular} \\
\texttt{numericInput()} & \begin{tabular}[c]{@{}l@{}} Create an input control for entry of numeric values \end{tabular} \\
\texttt{radioButtons()} & \begin{tabular}[c]{@{}l@{}} Create a set of radio buttons to select item from a list. \end{tabular}\\
\texttt{selectInput()} & \begin{tabular}[c]{@{}l@{}} Create a selectable list, from which single or multiple items \\ can be selected.\end{tabular}\\
\texttt{sliderInput()} & \begin{tabular}[c]{@{}l@{}} Constructs a slider widget to elect a number, date, or date-time.\end{tabular}\\
\texttt{passwordInput} & \begin{tabular}[c]{@{}l@{}} Create an control for entry for passwords.\end{tabular}\\
\texttt{textInput()} & \begin{tabular}[c]{@{}l@{}} Create an input control for unstructured text
\end{tabular}\\
\hline
\end{tabular}
\label{tab:uii}
\end{center}
\end{table}

\subsubsection{Output}\label{output}

Output functions in the \texttt{ui} create placeholders that can filled by the \texttt{server} function. As with inputs, outputs require a unique ID. For instance, in Fig \ref{fig:usco21}, which provides a simple summary of the \texttt{asbio::world.emissions} dataset, \texttt{output\$code} and \texttt{output\$text} generated in the \texttt{server} are placed in the fluid page using \texttt{textOutput("text")} and \texttt{verbatimTextOutput("code")}.
\vspace{0.1in}

\begin{Shaded}
\begin{Highlighting}[numbers=left,,]
\FunctionTok{library}\NormalTok{(asbio)}
\FunctionTok{data}\NormalTok{(world.emissions)}

\NormalTok{US }\OtherTok{\textless{}{-}}\NormalTok{ world.emissions }\SpecialCharTok{|\textgreater{}}
  \FunctionTok{filter}\NormalTok{(country }\SpecialCharTok{==} \StringTok{"United States"}\NormalTok{)}

\NormalTok{ui }\OtherTok{\textless{}{-}} \FunctionTok{fluidPage}\NormalTok{(}
  \FunctionTok{textOutput}\NormalTok{(}\StringTok{"text"}\NormalTok{),}
  \FunctionTok{verbatimTextOutput}\NormalTok{(}\StringTok{"code"}\NormalTok{)}
\NormalTok{)}
\NormalTok{server }\OtherTok{\textless{}{-}} \ControlFlowTok{function}\NormalTok{(input, output, session) \{}
\NormalTok{  output}\SpecialCharTok{$}\NormalTok{text }\OtherTok{\textless{}{-}} \FunctionTok{renderText}\NormalTok{(\{}
    \StringTok{"Summary of the US CO\textbackslash{}u2082 data }\SpecialCharTok{\textbackslash{}n}\StringTok{"}
\NormalTok{  \})}
\NormalTok{  output}\SpecialCharTok{$}\NormalTok{code }\OtherTok{\textless{}{-}} \FunctionTok{renderPrint}\NormalTok{(\{}
    \FunctionTok{summary}\NormalTok{(US}\SpecialCharTok{$}\NormalTok{co2)}
\NormalTok{  \})}
\NormalTok{\}}

\FunctionTok{shinyApp}\NormalTok{(ui, server)}
\end{Highlighting}
\end{Shaded}

\begin{figure}[H]

{\centering \includegraphics[width=0.7\linewidth,]{figs11/co2_1} 

}

\caption{A simple app providing a single descriptive statistics summary.}\label{fig:usco21}
\end{figure}

Table \ref{tab:uio} lists some potential \texttt{ui} output functions that can be used in \emph{shiny} apps.

\begin{table}[H]
\caption{Some \textit{shiny} \texttt{ui} output functions.}
\begin{center}

\renewcommand{\arraystretch}{1.1}
\begin{tabular}{ll} \hline
Function & Purpose \\ \hline
\begin{tabular}[c]{@{}l@{}}\texttt{downloadButton()}\\ \texttt{downloadLink()}\end{tabular}         & \begin{tabular}[c]{@{}l@{}} Create a download button or link. \\ To be paired with \texttt{downloadHandler()} in \texttt{server}.\end{tabular}\\
\begin{tabular}[c]{@{}l@{}}\texttt{htmlOutput()} \\ \texttt{uiOutput()} \end{tabular}                & Create an HTML output element.                  \\
\begin{tabular}[c]{@{}l@{}}\texttt{imageOutput()}\\ \texttt{plotOutput()}\end{tabular}              & \begin{tabular}[c]{@{}l@{}} Create a plot or image output element.\\ To be paired with \texttt{renderPlot()} and \texttt{renderImage()}, \\ respectively, in \texttt{server}.\end{tabular}  \\
\texttt{outputOptions()} & Set options for an output object.            \\
\begin{tabular}[c]{@{}l@{}}\texttt{modalDialog()} \\\texttt{modalButton()}\end{tabular}             & Create a modal dialog interface.                       \\
\begin{tabular}[c]{@{}l@{}}\texttt{showNotification()} \\ \texttt{removeNotification()}\end{tabular} & Show or remove a notification. \\
\begin{tabular}[c]{@{}l@{}}\texttt{textOutput()} \\ \texttt{verbatimTextOutput()}\end{tabular}      & \begin{tabular}[c]{@{}l@{}} Create a text output element. \\ To be paired with \texttt{renderText()} and \texttt{renderPrint()},\\ respectively, in \texttt{server}. \end{tabular}                 \\
\texttt{urlModal()}  & Generate a modal dialog that displays a URL.  \\
\hline
\end{tabular}
\label{tab:uio}
\end{center}
\end{table}

\index[R]{\texttt{shiny::downloadButton()}}
\index[R]{\texttt{shiny::downloadLink()}}
\index[R]{\texttt{shiny::htmlOutput()}}
\index[R]{\texttt{shiny::uiOutput()}}
\index[R]{\texttt{shiny::imageOutput()}}
\index[R]{\texttt{shiny::outputOptions()}}
\index[R]{\texttt{shiny::modalDialog()}}
\index[R]{\texttt{shiny::modalButton()}}
\index[R]{\texttt{shiny::showNotification()}}
\index[R]{\texttt{shiny::removeNotification()}}
\index[R]{\texttt{shiny::textOutput()}}
\index[R]{\texttt{shiny::verbatimTextOutput()}}
\index[R]{\texttt{shiny::urlModal()}}

\subsection{\texorpdfstring{\texttt{server} Details}{server Details}}\label{server-details}

As noted earlier, the \texttt{server} function requires three arguments: \texttt{input}, \texttt{output}, and \texttt{session}. The \texttt{input} argument allows assembly of items from the \texttt{ui} front-end to create a list-like object. The \texttt{output} argument in \texttt{server} provides output for \texttt{ui} inputs, often via rendering and handling functions (Table \ref{tab:serveo}).

\vspace{0.1in}

\begin{table}[H]
\caption{Some \textit{shiny} \texttt{server} rendering and handling functions.}
\begin{center}

\renewcommand{\arraystretch}{1.1}
\begin{tabular}{ll} \hline
Function & Purpose \\ \hline
\begin{tabular}[c]{@{}l@{}} \texttt{downloadHandler()} \end{tabular}    & \begin{tabular}[c]{@{}l@{}} Create a download button or link. \\ To be paired with \texttt{downloadButton()} and \texttt{downloadLink()} in \texttt{ui}.\end{tabular}\\
\begin{tabular}[c]{@{}l@{}}\texttt{renderPlot()} \\ \texttt{renderImage()}\end{tabular}              & \begin{tabular}[c]{@{}l@{}} Create a plot or image output element.\\ To be paired with \texttt{imageOutput()} and \texttt{plotOutput()}, \\ respectively, in \texttt{ui}.\end{tabular}  \\
\begin{tabular}[c]{@{}l@{}}\texttt{renderText()} \\ \texttt{renderPrint()}\end{tabular}      & \begin{tabular}[c]{@{}l@{}} Create a text output element. \\ To be paired with \texttt{textOutput()} and \texttt{verbatimTextOutput()},\\ respectively, in \texttt{ui}. \end{tabular}                 \\
\hline
\end{tabular}
\label{tab:serveo}
\end{center}
\end{table}

\index[R]{\texttt{shiny::downloadHandler()}}
\index[R]{\texttt{shiny::renderPlot()}}
\index[R]{\texttt{shiny::renderImage()}}
\index[R]{\texttt{shiny::renderText()}}
\index[R]{\texttt{shiny::renderPrint()}}

Fig \ref{fig:shiny4} shows an app with sliders inputs, generated using the function \texttt{sliderInput()} in the \texttt{ui}, and text (product) output which is provided to the \texttt{ui} from the \texttt{server}, via \texttt{renderText()}.
\vspace{0.1in}

\begin{Shaded}
\begin{Highlighting}[numbers=left,,]
\NormalTok{ui }\OtherTok{\textless{}{-}} \FunctionTok{fluidPage}\NormalTok{(}
  \FunctionTok{sliderInput}\NormalTok{(}\StringTok{"x"}\NormalTok{, }\AttributeTok{label =} \StringTok{"If x is"}\NormalTok{, }\AttributeTok{min =} \DecValTok{1}\NormalTok{, }\AttributeTok{max =} \DecValTok{50}\NormalTok{, }\AttributeTok{value =} \DecValTok{30}\NormalTok{),}
  \FunctionTok{sliderInput}\NormalTok{(}\StringTok{"y"}\NormalTok{, }\AttributeTok{label =} \StringTok{"And y is"}\NormalTok{, }\AttributeTok{min =} \DecValTok{1}\NormalTok{, }\AttributeTok{max =} \DecValTok{50}\NormalTok{, }\AttributeTok{value =} \DecValTok{30}\NormalTok{),}
  \StringTok{"then x times y is"}\NormalTok{,}
  \FunctionTok{textOutput}\NormalTok{(}\StringTok{"product"}\NormalTok{)}
\NormalTok{)}

\NormalTok{server }\OtherTok{\textless{}{-}} \ControlFlowTok{function}\NormalTok{(input, output, session) \{}
\NormalTok{  output}\SpecialCharTok{$}\NormalTok{product }\OtherTok{\textless{}{-}} \FunctionTok{renderText}\NormalTok{(\{}
\NormalTok{    input}\SpecialCharTok{$}\NormalTok{x }\SpecialCharTok{*}\NormalTok{ input}\SpecialCharTok{$}\NormalTok{y}
\NormalTok{  \})}
\NormalTok{\}}

\FunctionTok{shinyApp}\NormalTok{(ui, server)}
\end{Highlighting}
\end{Shaded}

\begin{figure}[H]

{\centering \includegraphics[width=0.7\linewidth,]{figs11/shiny4} 

}

\caption{A simple slider app.}\label{fig:shiny4}
\end{figure}

\subsection{\texorpdfstring{Running \emph{shiny} Apps in \textbf{R} Markdown}{Running shiny Apps in R Markdown}}\label{running-shiny-apps-in-r-markdown}

Unlike \emph{plotly} graphics, \emph{shiny} apps are not automatically interactive in an \textbf{R} markdown rendered document. However, it is easy to a make shiny app interactive in this setting (provided that \textbf{R} is open to run the app). One simply adds \texttt{runtime:shiny} to the \textbf{R} Markdown YAML header.\index[terms]{YAML} Thus, the YAML header in \textbf{R} markdown should have the format of Fig \ref{fig:yaml11}. Note that this approach will now be possible under Bookdown.

\begin{figure}[H]

{\centering \includegraphics[width=0.55\linewidth,]{figs11/YAML} 

}

\caption{YAML header to allow inclusion of \textit{shiny} apps in an \textbf{R} Markdown generated HTML.}\label{fig:yaml11}
\end{figure}

Some adjustments to \textbf{R} and Markdown code may be needed (e.g., figure margins in \texttt{renderPlot()} may need to be changed) to make apps fit nicely on a page. A \emph{shiny} app will only work remotely (outside of an \textbf{R} session) if a server implementing \textbf{R} is used to implement the app's code. This process will be detailed at the end of the chapter.

\subsection{Reactive Programming}\label{reactive-programming}

We can increase efficiency in Shiny apps using reactive programming wherein outputs automatically update as inputs change. Under reactive programming we specify interactive dependencies so that when an input changes, all related outputs are automatically updated. The code below results in the app shown in Fig \ref{fig:react}. Note that the output updates ``reactively'\,' as I type individual characters of my name.

\vspace{0.1in}

\begin{Shaded}
\begin{Highlighting}[numbers=left,,]
\NormalTok{ui }\OtherTok{\textless{}{-}} \FunctionTok{fluidPage}\NormalTok{(}
  \FunctionTok{textInput}\NormalTok{(}\StringTok{"name"}\NormalTok{, }\StringTok{"What\textquotesingle{}s your name?"}\NormalTok{),}
  \FunctionTok{textOutput}\NormalTok{(}\StringTok{"greeting"}\NormalTok{)}
\NormalTok{)}

\NormalTok{server }\OtherTok{\textless{}{-}} \ControlFlowTok{function}\NormalTok{(input, output, session) \{}
\NormalTok{  output}\SpecialCharTok{$}\NormalTok{greeting }\OtherTok{\textless{}{-}} \FunctionTok{renderText}\NormalTok{(\{}
   \FunctionTok{paste0}\NormalTok{(}\StringTok{"Hello "}\NormalTok{, input}\SpecialCharTok{$}\NormalTok{name, }\StringTok{"!"}\NormalTok{)}
\NormalTok{  \})}
\NormalTok{\}}
\FunctionTok{shinyApp}\NormalTok{(ui, server)}
\end{Highlighting}
\end{Shaded}

\begin{figure}[H]

{\centering \includegraphics[width=0.5\linewidth,]{figs11/names} 

}

\caption{Reactive behavior of a simple \textit{shiny} app.}\label{fig:react}
\end{figure}

Reactive programming usually occurs in more complex settings than the previous example and requires the function \texttt{reactive}.

\begin{example}
\protect\hypertarget{exm:exm-c11react}{}\label{exm:exm-c11react}

\(\text{}\)\\
As an extended example, imagine we wish to rapidly examine green house emissions data for the fifty countries in the \texttt{asbio::world.emissions} dataset with the highest current populations. As a first step, we do some data tidying and create a stats summary callback function we will use later.

\vspace{0.1in}

\begin{Shaded}
\begin{Highlighting}[numbers=left,,]
\FunctionTok{library}\NormalTok{(tidyverse)}
\NormalTok{not.redundant }\OtherTok{\textless{}{-}}\NormalTok{ world.emissions }\SpecialCharTok{|\textgreater{}} \FunctionTok{filter}\NormalTok{(continent }\SpecialCharTok{!=} \StringTok{"Redundant"}\NormalTok{)}

\NormalTok{pop.max }\OtherTok{\textless{}{-}}\NormalTok{ not.redundant }\SpecialCharTok{|\textgreater{}}
  \FunctionTok{group\_by}\NormalTok{(country) }\SpecialCharTok{|\textgreater{}}
  \FunctionTok{summarise}\NormalTok{(}\AttributeTok{max.pop =} \FunctionTok{max}\NormalTok{(population)) }\SpecialCharTok{|\textgreater{}}
  \FunctionTok{arrange}\NormalTok{(}\FunctionTok{desc}\NormalTok{(max.pop))}

\CommentTok{\# names of 50 largest countries}
\NormalTok{country\_names }\OtherTok{\textless{}{-}} \FunctionTok{setNames}\NormalTok{(}\AttributeTok{nm =}\NormalTok{ pop.max}\SpecialCharTok{$}\NormalTok{country[}\DecValTok{1}\SpecialCharTok{:}\DecValTok{50}\NormalTok{])}

\CommentTok{\# 50 largest countries data}
\NormalTok{top50 }\OtherTok{\textless{}{-}}\NormalTok{ not.redundant }\SpecialCharTok{|\textgreater{}}
  \FunctionTok{filter}\NormalTok{(country }\SpecialCharTok{\%in\%}\NormalTok{ pop.max}\SpecialCharTok{$}\NormalTok{country[}\DecValTok{1}\SpecialCharTok{:}\DecValTok{50}\NormalTok{])}

\CommentTok{\# summary stats}
\NormalTok{summarize }\OtherTok{\textless{}{-}} \ControlFlowTok{function}\NormalTok{(x, }\AttributeTok{rn =} \FunctionTok{c}\NormalTok{(}\StringTok{"CO2"}\NormalTok{, }\StringTok{"CH4"}\NormalTok{, }\StringTok{"NOx"}\NormalTok{, }\StringTok{"total GHG"}\NormalTok{))\{}
\NormalTok{    mean }\OtherTok{\textless{}{-}} \FunctionTok{apply}\NormalTok{(x, }\DecValTok{2}\NormalTok{, }\ControlFlowTok{function}\NormalTok{(x) }\FunctionTok{mean}\NormalTok{(x, }\AttributeTok{na.rm =}\NormalTok{ T))}
\NormalTok{    max }\OtherTok{\textless{}{-}} \FunctionTok{apply}\NormalTok{(x, }\DecValTok{2}\NormalTok{, }\ControlFlowTok{function}\NormalTok{(x) }\FunctionTok{max}\NormalTok{(x, }\AttributeTok{na.rm =}\NormalTok{ T))}
\NormalTok{    sum }\OtherTok{\textless{}{-}} \FunctionTok{apply}\NormalTok{(x, }\DecValTok{2}\NormalTok{, }\ControlFlowTok{function}\NormalTok{(x) }\FunctionTok{sum}\NormalTok{(x, }\AttributeTok{na.rm =}\NormalTok{ T))}
\NormalTok{    n }\OtherTok{\textless{}{-}} \FunctionTok{apply}\NormalTok{(x, }\DecValTok{2}\NormalTok{, }\ControlFlowTok{function}\NormalTok{(x) }\FunctionTok{length}\NormalTok{(}\FunctionTok{which}\NormalTok{(}\SpecialCharTok{!}\FunctionTok{is.na}\NormalTok{(x))))}
\NormalTok{    df }\OtherTok{\textless{}{-}} \FunctionTok{data.frame}\NormalTok{(}\AttributeTok{mean =}\NormalTok{ mean, }\AttributeTok{max =}\NormalTok{ max, }\AttributeTok{cumulative =}\NormalTok{ sum, }\AttributeTok{n =}\NormalTok{ n)}
    \FunctionTok{row.names}\NormalTok{(df) }\OtherTok{\textless{}{-}}\NormalTok{ rn}
\NormalTok{    df}
\NormalTok{\}}
\end{Highlighting}
\end{Shaded}

\begin{itemize}
\tightlist
\item
  \textbf{\texttt{ui}:} We define a relatively complex \texttt{ui} that will provide sufficient inputs and outputs for the \texttt{server} function.
\end{itemize}

\vspace{0.1in}

\begin{Shaded}
\begin{Highlighting}[numbers=left,,firstnumber=26,]
\NormalTok{ui }\OtherTok{\textless{}{-}} \FunctionTok{fluidPage}\NormalTok{(}
  \FunctionTok{titlePanel}\NormalTok{(}\FunctionTok{h1}\NormalTok{(}\StringTok{"Greenhouse gasses"}\NormalTok{, }\AttributeTok{align =} \StringTok{"center"}\NormalTok{)), }\CommentTok{\#h1 = HTML heading}
  \FunctionTok{fluidRow}\NormalTok{(}
    \FunctionTok{column}\NormalTok{(}\DecValTok{6}\NormalTok{,}
      \FunctionTok{selectInput}\NormalTok{(}\StringTok{"country"}\NormalTok{, }\AttributeTok{choices =}\NormalTok{ country\_names, }\AttributeTok{label =} \StringTok{"Country"}\NormalTok{)}
\NormalTok{    )}
\NormalTok{  ),}
  \FunctionTok{fluidRow}\NormalTok{(}
    \FunctionTok{column}\NormalTok{(}\DecValTok{4}\NormalTok{, }\FunctionTok{tableOutput}\NormalTok{(}\StringTok{"diag"}\NormalTok{)), }\CommentTok{\# table}
\NormalTok{  ),}
  \FunctionTok{br}\NormalTok{(), }\CommentTok{\# line break}
  \FunctionTok{br}\NormalTok{(),}
  \FunctionTok{fluidRow}\NormalTok{(}
    \FunctionTok{column}\NormalTok{(}\DecValTok{12}\NormalTok{, }\FunctionTok{plotOutput}\NormalTok{(}\StringTok{"plot"}\NormalTok{)) }\CommentTok{\# plot}
\NormalTok{  )}
\NormalTok{)}
\end{Highlighting}
\end{Shaded}

\begin{itemize}
\tightlist
\item
  Note that we use the \texttt{fluidRow()} layout function. Shiny app rows, created with \texttt{fluidRow()}, contain twelve columns. These can be divided up in various ways using \texttt{shiny::column()}(Fig \ref{fig:srows}).
\item
  The functions \texttt{htmltools::br()} and \texttt{htmltools::h1()} are HTML tags from the package \emph{htmltools}, which is imported by \emph{shiny}. The \texttt{h1()} function creates a first level heading. Thus, it is equivalent to the HTML operator \texttt{\textless{}h1\textgreater{}}. The \texttt{br()} functions equates to an HTML line break tag, i.e., \texttt{\textless{}br\textgreater{}}. A list of HTML equivalent tags is provided in \texttt{?htmltools::builder}.
\item
  The function \texttt{plotOutput()} allows input of interactive \textbf{R} graphs, including ggplots (see below).
\end{itemize}

\index[terms]{R!package!\textit{htmltools}}
\index[R]{\texttt{shiny::fluidRow()}}
\index[R]{\texttt{shiny::column()}}
\index[R]{\texttt{htmltools::h1()}}
\index[R]{\texttt{htmltools::br()}}
\index[R]{\texttt{shiny::plotOutput()}}

\begin{figure}[H]

{\centering \includegraphics[width=0.5\linewidth,]{figs11/fluidrow} 

}

\caption{Behavior of \textit{shiny} app rows and columns, in the \texttt{ui}. Figure taken from \citet{wickham2021mastering}.}\label{fig:srows}
\end{figure}

\begin{itemize}
\tightlist
\item
  \textbf{\texttt{server}:}: The \texttt{server} function contains several new features, including use of the function \texttt{reactive()}.
\end{itemize}

\vspace{0.1in}

\begin{Shaded}
\begin{Highlighting}[numbers=left,,firstnumber=42,]
\NormalTok{server }\OtherTok{\textless{}{-}} \ControlFlowTok{function}\NormalTok{(input, output, session) \{}
\NormalTok{  selected }\OtherTok{\textless{}{-}} \FunctionTok{reactive}\NormalTok{(top50 }\SpecialCharTok{\%\textgreater{}\%} \FunctionTok{filter}\NormalTok{(country }\SpecialCharTok{==}\NormalTok{ input}\SpecialCharTok{$}\NormalTok{country))}

\NormalTok{  output}\SpecialCharTok{$}\NormalTok{diag }\OtherTok{\textless{}{-}} \FunctionTok{renderTable}\NormalTok{(}
  \FunctionTok{summarize}\NormalTok{(}\FunctionTok{select}\NormalTok{(}\FunctionTok{selected}\NormalTok{(), co2, methane, nitrous\_oxide, total\_ghg)),}
  \AttributeTok{colnames =} \ConstantTok{TRUE}\NormalTok{, }\AttributeTok{rownames =} \ConstantTok{TRUE}
\NormalTok{  )}

\NormalTok{  output}\SpecialCharTok{$}\NormalTok{plot }\OtherTok{\textless{}{-}} \FunctionTok{renderPlot}\NormalTok{(\{}
  \FunctionTok{selected}\NormalTok{() }\SpecialCharTok{\%\textgreater{}\%}
       \FunctionTok{ggplot}\NormalTok{(}\FunctionTok{aes}\NormalTok{(year, co2)) }\SpecialCharTok{+}
       \FunctionTok{geom\_line}\NormalTok{() }\SpecialCharTok{+}
       \FunctionTok{labs}\NormalTok{(}\AttributeTok{x =} \StringTok{"Year"}\NormalTok{,}
            \AttributeTok{y =} \FunctionTok{expression}\NormalTok{(}\FunctionTok{paste}\NormalTok{(CO[}\DecValTok{2}\NormalTok{], }\StringTok{" emissions ("}\NormalTok{, }\DecValTok{10}\SpecialCharTok{\^{}}\DecValTok{6}\NormalTok{, }\StringTok{" tonnes)"}\NormalTok{)))}
\NormalTok{  \})}
\NormalTok{\}}
\end{Highlighting}
\end{Shaded}

\begin{itemize}
\tightlist
\item
  The code:
  \vspace{0.1in}
\end{itemize}

\begin{Shaded}
\begin{Highlighting}[]
\NormalTok{selected }\OtherTok{\textless{}{-}} \FunctionTok{reactive}\NormalTok{(top50 }\SpecialCharTok{\%\textgreater{}\%} \FunctionTok{filter}\NormalTok{(country }\SpecialCharTok{==}\NormalTok{ input}\SpecialCharTok{$}\NormalTok{country))}
\end{Highlighting}
\end{Shaded}

provides a data subset for a particular country that only needs to be calculated once, and then re-used. This also allows spontaneous (as possible) interaction with the \texttt{ui} with respect to this choice. The reactive object \texttt{selected} is called several times, as a function, in the \texttt{server} function.
- The object \texttt{output\$plot\ \textless{}-\ renderPlot()} will be a ggplot generated from \texttt{selected()} which will be called by \texttt{plotOutput("plot")} in the \texttt{ui}.

\index[R]{\texttt{shiny::reactive()}}
\index[R]{\texttt{shiny::renderPlot()}}

\begin{itemize}
\tightlist
\item
  \textbf{\texttt{shinyApp}:} As before, we generate the app using:
  \vspace{0.1in}
\end{itemize}

\begin{Shaded}
\begin{Highlighting}[numbers=left,,firstnumber=58,]
\FunctionTok{shinyApp}\NormalTok{(ui, server)}
\end{Highlighting}
\end{Shaded}

The final form of the app is shown in Fig \ref{fig:gg11}.

\begin{figure}[H]

{\centering \includegraphics[width=0.5\linewidth,]{figs11/gg} 

}

\caption{A \textit{shiny} app to graphically depict changing CO$_2$ levels over time for a user-selected country.}\label{fig:gg11}
\end{figure}

\end{example}

\(\blacksquare\)

\begin{example}
\protect\hypertarget{exm:exm-c11react2}{}\label{exm:exm-c11react2}

\(\text{}\)\\
\citet{wickham2021mastering} used \(t\)-test computations to demonstrate reactive programming as shown (with some modifications) below. We first delineate two callback functions we wish to use in the app.
\vspace{0.1in}

\begin{Shaded}
\begin{Highlighting}[numbers=left,,]
\NormalTok{freqpoly }\OtherTok{\textless{}{-}} \ControlFlowTok{function}\NormalTok{(x1, x2, }\AttributeTok{binwidth =} \FloatTok{0.1}\NormalTok{, }\AttributeTok{xlim =} \FunctionTok{c}\NormalTok{(}\SpecialCharTok{{-}}\DecValTok{3}\NormalTok{, }\DecValTok{3}\NormalTok{)) \{}
  \FunctionTok{require}\NormalTok{(ggplot2)}
\NormalTok{  df }\OtherTok{\textless{}{-}} \FunctionTok{data.frame}\NormalTok{(}
    \AttributeTok{x =} \FunctionTok{c}\NormalTok{(x1, x2),}
    \AttributeTok{group =} \FunctionTok{c}\NormalTok{(}\FunctionTok{rep}\NormalTok{(}\StringTok{"x1"}\NormalTok{, }\FunctionTok{length}\NormalTok{(x1)), }\FunctionTok{rep}\NormalTok{(}\StringTok{"x2"}\NormalTok{, }\FunctionTok{length}\NormalTok{(x2)))}
\NormalTok{  )}
  \FunctionTok{ggplot}\NormalTok{(df, }\FunctionTok{aes}\NormalTok{(x, }\AttributeTok{colour =}\NormalTok{ group)) }\SpecialCharTok{+}
    \FunctionTok{geom\_freqpoly}\NormalTok{(}\AttributeTok{binwidth =}\NormalTok{ binwidth, }\AttributeTok{linewidth =} \DecValTok{1}\NormalTok{) }\SpecialCharTok{+}
    \FunctionTok{coord\_cartesian}\NormalTok{(}\AttributeTok{xlim =}\NormalTok{ xlim)}
\NormalTok{\}}

\NormalTok{t\_test }\OtherTok{\textless{}{-}} \ControlFlowTok{function}\NormalTok{(x1, x2) \{}
\NormalTok{  test }\OtherTok{\textless{}{-}} \FunctionTok{t.test}\NormalTok{(x1, x2)}
  \FunctionTok{sprintf}\NormalTok{(}
    \StringTok{"p{-}value: \%0.3f }\SpecialCharTok{\textbackslash{}n}\StringTok{CI for µ1 {-} µ2: [\%0.2f, \%0.2f]"}\NormalTok{,}
\NormalTok{     test}\SpecialCharTok{$}\NormalTok{p.value, test}\SpecialCharTok{$}\NormalTok{conf.int[}\DecValTok{1}\NormalTok{], test}\SpecialCharTok{$}\NormalTok{conf.int[}\DecValTok{2}\NormalTok{]}
\NormalTok{  )}
\NormalTok{\}}
\end{Highlighting}
\end{Shaded}

The function \texttt{t.test()} runs \(t\)-tests for true normal population means. In particular, assuming \(X_1 \sim N(\mu_1, \sigma^2_1)\), \(X_2 \sim N(\mu_2, \sigma^2_2)\) we generally consider the hypotheses:

\begin{align*}
\text{H}_0: \mu_1 &= \mu_2 \\
\text{H}_\text{A}: \mu_1 &\neq \mu_2
\end{align*}

By default, \texttt{t.test()} does not assume homoscedasticty (that is, it allows \(\sigma^2_1 \neq \sigma^2_2\)). Thus, it uses the Satterthwaite method to estimate degrees of freedom for the null \(t\)-distribution of the test statistic \citep{aho2014}. The GUI we will create will run \(t\)-tests on randomly generated data from two user-specified normal distributions \texttt{x1} and \texttt{x2}.

The function \texttt{sprintf()} in \texttt{t\_test()} uses C code to return a formatted combination of text and variable outcomes. The code below combines text and inputs for double precision values (indicated with \texttt{f}) for \(p\)-values, and bounds for a 95\% confidence interval for a true mean difference. The code \texttt{\%0.3f} indicates rounding to three significant digits. As before, the code \texttt{\textbackslash{}n} creates a text line break.

\begin{itemize}
\tightlist
\item
  \textbf{\texttt{ui}:} we use the function \texttt{numericInput()} to specify characteristics of the normal distributions under consideration. The \texttt{sliderInput()} function is used to specify \emph{x}-limits in app-rendered ggplot frequency plot.
\end{itemize}

\vspace{0.1in}
\small

\begin{Shaded}
\begin{Highlighting}[numbers=left,,firstnumber=19,]
\NormalTok{ui }\OtherTok{\textless{}{-}} \FunctionTok{fluidPage}\NormalTok{(}
  \FunctionTok{fluidRow}\NormalTok{(}
    \FunctionTok{column}\NormalTok{(}\DecValTok{4}\NormalTok{,}
      \StringTok{"Distribution 1"}\NormalTok{,}
      \FunctionTok{numericInput}\NormalTok{(}\StringTok{"n1"}\NormalTok{, }\AttributeTok{label =} \StringTok{"n"}\NormalTok{, }\AttributeTok{value =} \DecValTok{200}\NormalTok{, }\AttributeTok{min =} \DecValTok{1}\NormalTok{),}
      \FunctionTok{numericInput}\NormalTok{(}\StringTok{"mean1"}\NormalTok{, }\AttributeTok{label =} \StringTok{"µ"}\NormalTok{, }\AttributeTok{value =} \DecValTok{0}\NormalTok{, }\AttributeTok{step =} \FloatTok{0.1}\NormalTok{),}
      \FunctionTok{numericInput}\NormalTok{(}\StringTok{"sd1"}\NormalTok{, }\AttributeTok{label =} \StringTok{"\textbackslash{}u03c3"}\NormalTok{, }\AttributeTok{value =} \FloatTok{0.5}\NormalTok{, }\AttributeTok{min =} \FloatTok{0.1}\NormalTok{, }\AttributeTok{step =} \FloatTok{0.1}\NormalTok{)}
\NormalTok{    ),}
    \FunctionTok{column}\NormalTok{(}\DecValTok{4}\NormalTok{,}
      \StringTok{"Distribution 2"}\NormalTok{,}
      \FunctionTok{numericInput}\NormalTok{(}\StringTok{"n2"}\NormalTok{, }\AttributeTok{label =} \StringTok{"n"}\NormalTok{, }\AttributeTok{value =} \DecValTok{200}\NormalTok{, }\AttributeTok{min =} \DecValTok{1}\NormalTok{),}
      \FunctionTok{numericInput}\NormalTok{(}\StringTok{"mean2"}\NormalTok{, }\AttributeTok{label =} \StringTok{"µ"}\NormalTok{, }\AttributeTok{value =} \DecValTok{0}\NormalTok{, }\AttributeTok{step =} \FloatTok{0.1}\NormalTok{),}
      \FunctionTok{numericInput}\NormalTok{(}\StringTok{"sd2"}\NormalTok{, }\AttributeTok{label =} \StringTok{"\textbackslash{}u03c3"}\NormalTok{, }\AttributeTok{value =} \FloatTok{0.5}\NormalTok{, }\AttributeTok{min =} \FloatTok{0.1}\NormalTok{, }\AttributeTok{step =} \FloatTok{0.1}\NormalTok{)}
\NormalTok{    ),}
    \FunctionTok{column}\NormalTok{(}\DecValTok{4}\NormalTok{,}
      \StringTok{"Frequency polygon"}\NormalTok{,}
      \FunctionTok{numericInput}\NormalTok{(}\StringTok{"binwidth"}\NormalTok{, }\AttributeTok{label =} \StringTok{"Bin width"}\NormalTok{, }\AttributeTok{value =} \FloatTok{0.1}\NormalTok{, }\AttributeTok{step =} \FloatTok{0.1}\NormalTok{),}
      \FunctionTok{sliderInput}\NormalTok{(}\StringTok{"range"}\NormalTok{, }\AttributeTok{label =} \StringTok{"range"}\NormalTok{, }\AttributeTok{value =} \FunctionTok{c}\NormalTok{(}\SpecialCharTok{{-}}\DecValTok{3}\NormalTok{, }\DecValTok{3}\NormalTok{), }\AttributeTok{min =} \SpecialCharTok{{-}}\DecValTok{5}\NormalTok{, }\AttributeTok{max =} \DecValTok{5}\NormalTok{)}
\NormalTok{    )}
\NormalTok{  ),}
  \FunctionTok{fluidRow}\NormalTok{(}
    \FunctionTok{column}\NormalTok{(}\DecValTok{12}\NormalTok{, }\FunctionTok{plotOutput}\NormalTok{(}\StringTok{"hist"}\NormalTok{))}
\NormalTok{  ),}
  \FunctionTok{fluidRow}\NormalTok{(}
    \FunctionTok{column}\NormalTok{(}\DecValTok{1}\NormalTok{),}
    \FunctionTok{column}\NormalTok{(}\DecValTok{5}\NormalTok{, }\FunctionTok{verbatimTextOutput}\NormalTok{(}\StringTok{"ttest"}\NormalTok{)),}
    \FunctionTok{column}\NormalTok{(}\DecValTok{2}\NormalTok{),}
    \FunctionTok{column}\NormalTok{(}\DecValTok{3}\NormalTok{, }\FunctionTok{actionButton}\NormalTok{(}\StringTok{"simulate"}\NormalTok{, }\StringTok{"Simulate!"}\NormalTok{)),}
    \FunctionTok{column}\NormalTok{(}\DecValTok{1}\NormalTok{)}
\NormalTok{  )}
\NormalTok{)}
\end{Highlighting}
\end{Shaded}

\normalsize

\begin{itemize}
\tightlist
\item
  \textbf{\texttt{server}:} In the server we use reactive programming to generate random samples from a normal distribution. Specifically, for the object \texttt{x1} we obtain a random sample of size \texttt{input\$n1} from a normal distribution with a mean of \texttt{input\$mean1} and a standard deviation of \texttt{input\$sd1}. These parameter values are specified in the \texttt{ui}.
\end{itemize}

\vspace{0.1in}

\begin{Shaded}
\begin{Highlighting}[numbers=left,,firstnumber=50,]
\NormalTok{server }\OtherTok{\textless{}{-}} \ControlFlowTok{function}\NormalTok{(input, output, session) \{}
\NormalTok{  x1 }\OtherTok{\textless{}{-}} \FunctionTok{reactive}\NormalTok{(\{input}\SpecialCharTok{$}\NormalTok{simulate}
        \FunctionTok{rnorm}\NormalTok{(input}\SpecialCharTok{$}\NormalTok{n1, input}\SpecialCharTok{$}\NormalTok{mean1, input}\SpecialCharTok{$}\NormalTok{sd1)\})}
\NormalTok{  x2 }\OtherTok{\textless{}{-}} \FunctionTok{reactive}\NormalTok{(\{input}\SpecialCharTok{$}\NormalTok{simulate}
        \FunctionTok{rnorm}\NormalTok{(input}\SpecialCharTok{$}\NormalTok{n2, input}\SpecialCharTok{$}\NormalTok{mean2, input}\SpecialCharTok{$}\NormalTok{sd2)\})}

\NormalTok{  output}\SpecialCharTok{$}\NormalTok{hist }\OtherTok{\textless{}{-}} \FunctionTok{renderPlot}\NormalTok{(\{}
    \FunctionTok{freqpoly}\NormalTok{(}\FunctionTok{x1}\NormalTok{(), }\FunctionTok{x2}\NormalTok{(), }\AttributeTok{binwidth =}\NormalTok{ input}\SpecialCharTok{$}\NormalTok{binwidth, }\AttributeTok{xlim =}\NormalTok{ input}\SpecialCharTok{$}\NormalTok{range)}
\NormalTok{  \}, }\AttributeTok{res =} \DecValTok{96}\NormalTok{)}

\NormalTok{  output}\SpecialCharTok{$}\NormalTok{ttest }\OtherTok{\textless{}{-}} \FunctionTok{renderText}\NormalTok{(\{}
    \FunctionTok{t\_test}\NormalTok{(}\FunctionTok{x1}\NormalTok{(), }\FunctionTok{x2}\NormalTok{())}
\NormalTok{  \})}
\NormalTok{\}}
\end{Highlighting}
\end{Shaded}

\index[R]{\texttt{ggplot2::freqpoly()}}
- \textbf{\texttt{shinyApp}:} As before, we generate the app using:
\vspace{0.1in}

\begin{Shaded}
\begin{Highlighting}[numbers=left,,firstnumber=64,]
\FunctionTok{shinyApp}\NormalTok{(ui, server)}
\end{Highlighting}
\end{Shaded}

The final form of the app is shown in Fig \ref{fig:gg2}.

\index[R]{\texttt{shiny::numericInput()}}
\index[R]{\texttt{sprintf()}}
\index[R]{\texttt{t.test()}}
\index[terms]{C (programming language)}

\begin{figure}[H]

{\centering \includegraphics[width=0.7\linewidth,]{figs11/gg2} 

}

\caption{A \textit{shiny} app to demonstrate the mechanism of $t$-tests.}\label{fig:gg2}
\end{figure}

\end{example}

\(\blacksquare\)

\subsection{Additional Layout Control}\label{additional-layout-control}

We have already learned about techniques like \texttt{fluidRow()} to control single page layouts in \texttt{fluidPage()}. Another popular HTML layout uses side panels. These can be implemented in the \emph{shiny} \texttt{ui} using the functions \texttt{sidebarLayout()} and \texttt{sidebarPanel()}. Sidebar formatting is summarized in Fig \ref{fig:sb}.

\index[R]{\texttt{shiny::sidebarLayout()}}
\index[R]{\texttt{shiny::sidebarPanel()}}

\begin{figure}[H]

{\centering \includegraphics[width=0.6\linewidth,]{figs11/fluidpage} 

}

\caption{Sidebar formatting for \textit{shiny}. Figure taken from \citet{wickham2021mastering}.}\label{fig:sb}
\end{figure}

\begin{example}
\protect\hypertarget{exm:exm-c11norm}{}\label{exm:exm-c11norm}\(\text{}\)\\
Here is an example for displaying a normal distribution using sliders in sidebars.

\begin{itemize}
\tightlist
\item
  \textbf{\texttt{ui}:} The user interface specifies a sidebar layout, using \texttt{sidebarLayout()}, that contains a sidebar panel designated with \texttt{sidebarPanel}, and a main panel, designated with \texttt{mainPanel()}.
\end{itemize}

\index[R]{\texttt{shiny::mainPanel()}}
\vspace{0.1in}

\begin{Shaded}
\begin{Highlighting}[numbers=left,,]
\NormalTok{ui }\OtherTok{\textless{}{-}} \FunctionTok{fluidPage}\NormalTok{(}
   \FunctionTok{titlePanel}\NormalTok{(}\FunctionTok{h1}\NormalTok{(}\StringTok{"Normal Distribution"}\NormalTok{, }\AttributeTok{align =} \StringTok{"center"}\NormalTok{)),}
   \FunctionTok{sidebarLayout}\NormalTok{(}
      \FunctionTok{sidebarPanel}\NormalTok{(}
         \FunctionTok{sliderInput}\NormalTok{(}\StringTok{"mu"}\NormalTok{, }\StringTok{"\textbackslash{}u03BC"}\NormalTok{, }\AttributeTok{step =} \FloatTok{0.2}\NormalTok{, }\AttributeTok{min =} \SpecialCharTok{{-}}\DecValTok{3}\NormalTok{,}
                     \AttributeTok{max =} \DecValTok{3}\NormalTok{, }\AttributeTok{value =} \DecValTok{0}\NormalTok{),}
         \FunctionTok{sliderInput}\NormalTok{(}\StringTok{"sigma"}\NormalTok{, }\StringTok{"\textbackslash{}u03C3"}\NormalTok{, }\AttributeTok{min =} \FloatTok{0.5}\NormalTok{, }\AttributeTok{max =} \DecValTok{3}\NormalTok{,}
                     \AttributeTok{value =} \DecValTok{1}\NormalTok{), }\AttributeTok{width =} \DecValTok{4}
\NormalTok{      ),}
    \FunctionTok{mainPanel}\NormalTok{(}\FunctionTok{plotOutput}\NormalTok{(}\StringTok{"plot"}\NormalTok{))}
\NormalTok{))}
\end{Highlighting}
\end{Shaded}

\begin{itemize}
\tightlist
\item
  \textbf{\texttt{server}:} The only output from the \texttt{server} is a base \textbf{R} plot of the normal PDF.
  \vspace{0.1in}
\end{itemize}

\begin{Shaded}
\begin{Highlighting}[numbers=left,,firstnumber=12,]
\NormalTok{server }\OtherTok{\textless{}{-}} \ControlFlowTok{function}\NormalTok{(input, output, session) (\{}
\NormalTok{   xmin }\OtherTok{\textless{}{-}} \SpecialCharTok{{-}}\DecValTok{4}\NormalTok{; xmax }\OtherTok{\textless{}{-}} \DecValTok{4}\NormalTok{; ymin }\OtherTok{\textless{}{-}} \DecValTok{0}\NormalTok{; ymax }\OtherTok{\textless{}{-}} \FloatTok{0.8}
\NormalTok{   xx }\OtherTok{\textless{}{-}} \FunctionTok{seq}\NormalTok{(xmin, xmax, }\AttributeTok{length =} \DecValTok{100}\NormalTok{)}

\NormalTok{   output}\SpecialCharTok{$}\NormalTok{plot }\OtherTok{\textless{}{-}} \FunctionTok{renderPlot}\NormalTok{(\{}
\NormalTok{       yy }\OtherTok{\textless{}{-}} \FunctionTok{dnorm}\NormalTok{(xx, input}\SpecialCharTok{$}\NormalTok{mu, input}\SpecialCharTok{$}\NormalTok{sigma)}
       \FunctionTok{plot}\NormalTok{(xx, yy, }\AttributeTok{type =} \StringTok{"l"}\NormalTok{, }\AttributeTok{xlim =} \FunctionTok{c}\NormalTok{(xmin, xmax), }\AttributeTok{ylim =} \FunctionTok{c}\NormalTok{(ymin, ymax),}
            \AttributeTok{xlab =} \FunctionTok{expression}\NormalTok{(}\FunctionTok{italic}\NormalTok{(x)),}
            \AttributeTok{ylab =} \FunctionTok{expression}\NormalTok{(}\FunctionTok{paste}\NormalTok{(}\FunctionTok{italic}\NormalTok{(f), }\StringTok{"("}\NormalTok{, }\FunctionTok{italic}\NormalTok{(x), }\StringTok{")"}\NormalTok{, }\AttributeTok{sep =} \StringTok{""}\NormalTok{)),}
            \AttributeTok{cex.axis =} \FloatTok{1.2}\NormalTok{, }\AttributeTok{cex.lab =} \FloatTok{1.2}\NormalTok{, }\AttributeTok{lwd =} \FloatTok{1.4}\NormalTok{)}
\NormalTok{     \})}
\NormalTok{\})}
\end{Highlighting}
\end{Shaded}

\}
- \textbf{\texttt{shinyApp}:} As before, we use \texttt{shinyApp()} to generate the app.
\vspace{0.1in}

\begin{Shaded}
\begin{Highlighting}[numbers=left,,firstnumber=24,]
\FunctionTok{shinyApp}\NormalTok{(ui, server)}
\end{Highlighting}
\end{Shaded}

The final form of the app is shown in Fig \ref{fig:normdist}.

\begin{figure}[H]

{\centering \includegraphics[width=0.8\linewidth,]{figs11/normdist} 

}

\caption{A \textit{shiny} app for demonstrating the normal distribution.}\label{fig:normdist}
\end{figure}

Interestingly, the app appears less reactive than an analogous plot GUI generated with \emph{tcltk}. Compare the app from Fig \ref{fig:normdist} to \texttt{asbio::see.norm.tck()}.
\end{example}

\(\blacksquare\)

\subsubsection{Multi-page Apps}\label{multi-page-apps}

Complex apps may be impossible to fit onto a single page. In \emph{shiny}, the simplest way to break a app page into multiple pages is to use \texttt{tabsetPanel()} and \texttt{tabPanel()}.
\citet{wickham2021mastering} that does not provide widget output because of its empty \texttt{server} (Fig \ref{fig:multipanel}). In the \texttt{ui} a tabset panel is generated using \texttt{tabsetPanel()}. This entity has three panels, each is generated using \texttt{tabPanel()}. Only the first panel \texttt{"Import\ data"} currently contains content.
\index[R]{\texttt{shiny::tabsetPanel()}}\index[R]{\texttt{shiny::tabPanel()}}
\vspace{0.1in}

\begin{Shaded}
\begin{Highlighting}[numbers=left,,]
\NormalTok{ui }\OtherTok{\textless{}{-}} \FunctionTok{fluidPage}\NormalTok{(}
  \FunctionTok{tabsetPanel}\NormalTok{(}
    \FunctionTok{tabPanel}\NormalTok{(}\StringTok{"Import data"}\NormalTok{,}
      \FunctionTok{fileInput}\NormalTok{(}\StringTok{"file"}\NormalTok{, }\StringTok{"Data"}\NormalTok{, }\AttributeTok{buttonLabel =} \StringTok{"Upload..."}\NormalTok{),}
      \FunctionTok{textInput}\NormalTok{(}\StringTok{"delim"}\NormalTok{, }\StringTok{"Delimiter"}\NormalTok{, }\StringTok{""}\NormalTok{),}
      \FunctionTok{numericInput}\NormalTok{(}\StringTok{"skip"}\NormalTok{, }\StringTok{"Rows to skip"}\NormalTok{, }\DecValTok{0}\NormalTok{, }\AttributeTok{min =} \DecValTok{0}\NormalTok{),}
      \FunctionTok{numericInput}\NormalTok{(}\StringTok{"rows"}\NormalTok{, }\StringTok{"Rows to preview"}\NormalTok{, }\DecValTok{10}\NormalTok{, }\AttributeTok{min =} \DecValTok{1}\NormalTok{)}
\NormalTok{    ),}
    \FunctionTok{tabPanel}\NormalTok{(}\StringTok{"Set parameters"}\NormalTok{),}
    \FunctionTok{tabPanel}\NormalTok{(}\StringTok{"Visualise results"}\NormalTok{)}
\NormalTok{  )}
\NormalTok{)}

\NormalTok{server }\OtherTok{\textless{}{-}} \ControlFlowTok{function}\NormalTok{(input, output, session) \{}
\NormalTok{\}}
\FunctionTok{shinyApp}\NormalTok{(ui, server)}
\end{Highlighting}
\end{Shaded}

\begin{figure}[H]

{\centering \includegraphics[width=0.6\linewidth,]{figs11/multipanel} 

}

\caption{A \textit{shiny} multipanel example.}\label{fig:multipanel}
\end{figure}

A tabset can be an input when its \texttt{id} argument is used.
This allows an app to behave differently depending on which tab is currently visible (Fig \ref{fig:tabs2}).
\vspace{0.1in}

\begin{Shaded}
\begin{Highlighting}[numbers=left,,]
\NormalTok{ui }\OtherTok{\textless{}{-}} \FunctionTok{fluidPage}\NormalTok{(}
  \FunctionTok{sidebarLayout}\NormalTok{(}
    \FunctionTok{sidebarPanel}\NormalTok{(}
      \FunctionTok{textOutput}\NormalTok{(}\StringTok{"panel"}\NormalTok{)}
\NormalTok{    ),}
    \FunctionTok{mainPanel}\NormalTok{(}
      \FunctionTok{tabsetPanel}\NormalTok{(}
        \AttributeTok{id =} \StringTok{"tabset"}\NormalTok{,}
        \FunctionTok{tabPanel}\NormalTok{(}\StringTok{"panel 1"}\NormalTok{),}
        \FunctionTok{tabPanel}\NormalTok{(}\StringTok{"panel 2"}\NormalTok{),}
        \FunctionTok{tabPanel}\NormalTok{(}\StringTok{"panel 3"}\NormalTok{)}
\NormalTok{      )}
\NormalTok{    )}
\NormalTok{  )}
\NormalTok{)}
\NormalTok{server }\OtherTok{\textless{}{-}} \ControlFlowTok{function}\NormalTok{(input, output, session) \{}
\NormalTok{  output}\SpecialCharTok{$}\NormalTok{panel }\OtherTok{\textless{}{-}} \FunctionTok{renderText}\NormalTok{(\{}
    \FunctionTok{paste}\NormalTok{(}\StringTok{"Current panel: "}\NormalTok{, input}\SpecialCharTok{$}\NormalTok{tabset)}
\NormalTok{  \})}
\NormalTok{\}}
\FunctionTok{shinyApp}\NormalTok{(ui, server)}
\end{Highlighting}
\end{Shaded}

\begin{figure}[H]

{\centering \includegraphics[width=1\linewidth,]{figs11/tabs2} 

}

\caption{Tabs for a multi-page app.}\label{fig:tabs2}
\end{figure}

Because tabs are displayed horizontally, there is a limit to their number. The functions \texttt{navlistPanel()}, \texttt{navbarPage()}, and \texttt{navbarMenu()} provide vertical layouts that allow more tabs with longer titles (Fig \ref{fig:tabs3}).
\index[R]{\texttt{shiny::navbarPage()}}
\index[R]{\texttt{shiny::navbarMenu()}}
\vspace{0.1in}

\begin{Shaded}
\begin{Highlighting}[numbers=left,,]
\NormalTok{ui }\OtherTok{\textless{}{-}} \FunctionTok{fluidPage}\NormalTok{(}
  \FunctionTok{navlistPanel}\NormalTok{(}
    \AttributeTok{id =} \StringTok{"tabset"}\NormalTok{,}
    \StringTok{"Heading 1"}\NormalTok{,}
    \FunctionTok{tabPanel}\NormalTok{(}\StringTok{"panel 1"}\NormalTok{, }\StringTok{"Panel one contents"}\NormalTok{),}
    \StringTok{"Heading 2"}\NormalTok{,}
    \FunctionTok{tabPanel}\NormalTok{(}\StringTok{"panel 2"}\NormalTok{, }\StringTok{"Panel two contents"}\NormalTok{),}
    \FunctionTok{tabPanel}\NormalTok{(}\StringTok{"panel 3"}\NormalTok{, }\StringTok{"Panel three contents"}\NormalTok{)}
\NormalTok{  )}
\NormalTok{)}

\NormalTok{server }\OtherTok{\textless{}{-}} \ControlFlowTok{function}\NormalTok{(input, output, session) \{}
\NormalTok{\}}
\FunctionTok{shinyApp}\NormalTok{(ui, server)}
\end{Highlighting}
\end{Shaded}

\begin{figure}[H]

{\centering \includegraphics[width=1\linewidth,]{figs11/tabs2} 

}

\caption{A multi-page app with vertical tabs.}\label{fig:tabs3}
\end{figure}

\subsubsection{Layout Themes}\label{layout-themes}

Customization of the general \emph{shiny} layout can be obtained by utilizing or modifying \textbf{Bootstrap}\footnote{Bootstrap is a collection of HTML conventions, Cascading Style Sheets (CSS) styles (CSS is a language used to style HTML documents, including colors and fonts), and java script snippets bundled into a convenient form.} themes and classes.
These can include layouts specific to mobile apps (see package \emph{RInterface}) and Google's material design frame (see package \emph{shinymaterial}).

\index[terms]{Bootstrap (HTML tools)}
\index[terms]{CSS (HTML tools)}
\index[terms]{R!package!\textit{RInterface}}
\index[terms]{R!package!\textit{shinymaterial}}

Here we use the \texttt{"darkly"} bootswatch (Fig \ref{fig:bstheme}). Other choices include \texttt{"sandstone"}, \texttt{"flatly"}, and \texttt{"united"}.
\vspace{0.1in}

\begin{Shaded}
\begin{Highlighting}[numbers=left,,]
\NormalTok{ui }\OtherTok{\textless{}{-}} \FunctionTok{fluidPage}\NormalTok{(}
  \AttributeTok{theme =}\NormalTok{ bslib}\SpecialCharTok{::}\FunctionTok{bs\_theme}\NormalTok{(}\AttributeTok{bootswatch =} \StringTok{"darkly"}\NormalTok{),}
  \FunctionTok{sidebarLayout}\NormalTok{(}
    \FunctionTok{sidebarPanel}\NormalTok{(}
      \FunctionTok{textInput}\NormalTok{(}\StringTok{"txt"}\NormalTok{, }\StringTok{"Text input:"}\NormalTok{, }\StringTok{"text here"}\NormalTok{),}
      \FunctionTok{sliderInput}\NormalTok{(}\StringTok{"slider"}\NormalTok{, }\StringTok{"Slider input:"}\NormalTok{, }\DecValTok{1}\NormalTok{, }\DecValTok{100}\NormalTok{, }\DecValTok{30}\NormalTok{)}
\NormalTok{    ),}
    \FunctionTok{mainPanel}\NormalTok{(}
      \FunctionTok{h1}\NormalTok{(}\FunctionTok{paste0}\NormalTok{(}\StringTok{"Theme: darkly"}\NormalTok{)),}
      \FunctionTok{h2}\NormalTok{(}\StringTok{"Header 2"}\NormalTok{),}
      \FunctionTok{p}\NormalTok{(}\StringTok{"Some text"}\NormalTok{)}
\NormalTok{    )}
\NormalTok{  )}
\NormalTok{)}

\NormalTok{server }\OtherTok{\textless{}{-}} \ControlFlowTok{function}\NormalTok{(input, output, session) \{}
\NormalTok{\}}
\FunctionTok{shinyApp}\NormalTok{(ui, server)}
\end{Highlighting}
\end{Shaded}

\begin{figure}[H]

{\centering \includegraphics[width=0.8\linewidth,]{figs11/darkly} 

}

\caption{A \textit{shiny} app using the \texttt{darkly} bootswatch from Bootstrap.}\label{fig:bstheme}
\end{figure}

Further control of \emph{shiny} apps can be achieved by programming directly in HTML, CSS, and Java\footnote{for more information, check this \textbf{R}-studio help \href{https://shiny.rstudio.com/articles/html-ui.html}{link} and this \href{https://unleash-shiny.rinterface.com/index.html}{link} to a book by David Granjon}. In fact, HTML code in \texttt{ui}s is revealed by running \texttt{ui} functions directly in the \textbf{R} console (Fig \ref{fig:HTMLcode}).

\begin{figure}[H]

{\centering \includegraphics[width=1\linewidth,]{figs11/HTML_code} 

}

\caption{Representation of \textit{shiny} code as HTML code.}\label{fig:HTMLcode}
\end{figure}

\subsection{\texorpdfstring{\texttt{plotOutput} Interactives}{plotOutput Interactives}}\label{plotoutput-interactives}

One of the perks of \texttt{plotOutput()} is that it can be an input that responds to mouse pointer events. Such controls are also possible with \emph{tcltk} GUIs. A \emph{shiny} plot can respond to four different mouse events: \texttt{click}, \texttt{dblclick} (double click), \texttt{hover} (i.e., the mouse stays in the same place), and \texttt{brush} (a rectangular selection tool).

\begin{example}
\protect\hypertarget{exm:exm-c11mouse}{}\label{exm:exm-c11mouse}

\(\text{}\)\\
Consider the following simple example:
\vspace{0.1in}

\begin{Shaded}
\begin{Highlighting}[numbers=left,,]
\NormalTok{US }\OtherTok{\textless{}{-}}\NormalTok{ world.emissions }\SpecialCharTok{\%\textgreater{}\%} \FunctionTok{filter}\NormalTok{(country }\SpecialCharTok{==} \StringTok{"United States"}\NormalTok{)}

\NormalTok{ui }\OtherTok{\textless{}{-}} \FunctionTok{fluidPage}\NormalTok{(}
  \FunctionTok{plotOutput}\NormalTok{(}\StringTok{"plot"}\NormalTok{, }\AttributeTok{click =} \StringTok{"plot\_click"}\NormalTok{),}
  \FunctionTok{verbatimTextOutput}\NormalTok{(}\StringTok{"info"}\NormalTok{)}
\NormalTok{)}

\NormalTok{server }\OtherTok{\textless{}{-}} \ControlFlowTok{function}\NormalTok{(input, output) \{}
\NormalTok{  output}\SpecialCharTok{$}\NormalTok{plot }\OtherTok{\textless{}{-}} \FunctionTok{renderPlot}\NormalTok{(\{}
    \FunctionTok{par}\NormalTok{(}\AttributeTok{mar =} \FunctionTok{c}\NormalTok{(}\DecValTok{5}\NormalTok{,}\DecValTok{5}\NormalTok{,}\DecValTok{2}\NormalTok{,}\DecValTok{2}\NormalTok{))}
    \FunctionTok{plot}\NormalTok{(US}\SpecialCharTok{$}\NormalTok{year, US}\SpecialCharTok{$}\NormalTok{co2, }\AttributeTok{xlab =} \StringTok{"Year"}\NormalTok{,}
         \AttributeTok{ylab =} \FunctionTok{expression}\NormalTok{(}\FunctionTok{paste}\NormalTok{(CO[}\DecValTok{2}\NormalTok{], }\StringTok{" ("}\NormalTok{,}\DecValTok{10}\SpecialCharTok{\^{}}\DecValTok{6}\NormalTok{, }\StringTok{" tonnes)"}\NormalTok{)), }\AttributeTok{type =} \StringTok{"l"}\NormalTok{)}
\NormalTok{  \}, }\AttributeTok{res =} \DecValTok{96}\NormalTok{)}

\NormalTok{  output}\SpecialCharTok{$}\NormalTok{info }\OtherTok{\textless{}{-}} \FunctionTok{renderPrint}\NormalTok{(\{}
    \FunctionTok{req}\NormalTok{(input}\SpecialCharTok{$}\NormalTok{plot\_click)}
\NormalTok{    x }\OtherTok{\textless{}{-}} \FunctionTok{round}\NormalTok{(input}\SpecialCharTok{$}\NormalTok{plot\_click}\SpecialCharTok{$}\NormalTok{x, }\DecValTok{2}\NormalTok{)}
\NormalTok{    y }\OtherTok{\textless{}{-}} \FunctionTok{round}\NormalTok{(input}\SpecialCharTok{$}\NormalTok{plot\_click}\SpecialCharTok{$}\NormalTok{y, }\DecValTok{2}\NormalTok{)}
    \FunctionTok{cat}\NormalTok{(}\StringTok{"[year = "}\NormalTok{, x, }\StringTok{", CO2 = "}\NormalTok{, y, }\StringTok{" million tonnes]"}\NormalTok{, }\AttributeTok{sep =} \StringTok{""}\NormalTok{)}
\NormalTok{  \})}
\NormalTok{\}}
\FunctionTok{shinyApp}\NormalTok{(ui, server)}
\end{Highlighting}
\end{Shaded}

Note the use of \texttt{req()}, to ensure the app doesn't do anything before the first click. The resulting app is shown in Fig \ref{fig:mouse1}.

\begin{figure}[H]

{\centering \includegraphics[width=1\linewidth,]{figs11/mouse1} 

}

\caption{A mouse interactive \textit{shiny} app.}\label{fig:mouse1}
\end{figure}

Here we use \texttt{nearPoints()} to return a dataframe for a point \emph{near} a mouse click.
\vspace{0.1in}

\begin{Shaded}
\begin{Highlighting}[numbers=left,,]
\NormalTok{US.ghg }\OtherTok{\textless{}{-}} \FunctionTok{data.frame}\NormalTok{(US[,}\FunctionTok{c}\NormalTok{(}\DecValTok{3}\NormalTok{,}\DecValTok{4}\NormalTok{,}\DecValTok{10}\NormalTok{,}\DecValTok{11}\NormalTok{,}\DecValTok{12}\NormalTok{,}\DecValTok{14}\NormalTok{,}\DecValTok{15}\NormalTok{)])}

\NormalTok{ui }\OtherTok{\textless{}{-}} \FunctionTok{fluidPage}\NormalTok{(}
  \FunctionTok{plotOutput}\NormalTok{(}\StringTok{"plot"}\NormalTok{, }\AttributeTok{click =} \StringTok{"plot\_click"}\NormalTok{),}
  \FunctionTok{tableOutput}\NormalTok{(}\StringTok{"data"}\NormalTok{)}
\NormalTok{)}
\NormalTok{server }\OtherTok{\textless{}{-}} \ControlFlowTok{function}\NormalTok{(input, output, session) \{}
\NormalTok{  output}\SpecialCharTok{$}\NormalTok{plot }\OtherTok{\textless{}{-}} \FunctionTok{renderPlot}\NormalTok{(\{}
    \FunctionTok{par}\NormalTok{(}\AttributeTok{mar =} \FunctionTok{c}\NormalTok{(}\DecValTok{5}\NormalTok{,}\DecValTok{5}\NormalTok{,}\DecValTok{2}\NormalTok{,}\DecValTok{2}\NormalTok{))}
    \FunctionTok{plot}\NormalTok{(US.ghg}\SpecialCharTok{$}\NormalTok{year, US.ghg}\SpecialCharTok{$}\NormalTok{co2, }\AttributeTok{xlab =} \StringTok{"Year"}\NormalTok{,}
         \AttributeTok{ylab =} \FunctionTok{expression}\NormalTok{(}\FunctionTok{paste}\NormalTok{(CO[}\DecValTok{2}\NormalTok{], }\StringTok{" ("}\NormalTok{,}\DecValTok{10}\SpecialCharTok{\^{}}\DecValTok{6}\NormalTok{, }\StringTok{" tonnes)"}\NormalTok{)),}
         \AttributeTok{type =} \StringTok{"l"}\NormalTok{)}
\NormalTok{  \}, }\AttributeTok{res =} \DecValTok{96}\NormalTok{)}

\NormalTok{  output}\SpecialCharTok{$}\NormalTok{data }\OtherTok{\textless{}{-}} \FunctionTok{renderTable}\NormalTok{(\{}
    \FunctionTok{nearPoints}\NormalTok{(US.ghg, input}\SpecialCharTok{$}\NormalTok{plot\_click,}
               \AttributeTok{xvar =} \StringTok{"year"}\NormalTok{, }\AttributeTok{yvar =} \StringTok{"co2"}\NormalTok{)}
\NormalTok{  \})}
\NormalTok{\}}
\FunctionTok{shinyApp}\NormalTok{(ui, server)}
\end{Highlighting}
\end{Shaded}

The resulting app is shown in Fig \ref{fig:mouse2}.

\begin{figure}[H]

{\centering \includegraphics[width=1\linewidth,]{figs11/mouse2} 

}

\caption{Another mouse interactive \textit{shiny} app.}\label{fig:mouse2}
\end{figure}

\end{example}

\(\blacksquare\)

\subsection{Putting Your App Online}\label{putting-your-app-online}

A \emph{shiny} app will only work remotely (outside of an \textbf{R} session) if a server implementing \textbf{R} is used to call the app's code. RStudio helps with this by housing a shiny server site shinyapps.io (Fig \ref{fig:shinyweb}). The site is currently free of charge for a relatively small number of personal applications.

\begin{figure}[H]

{\centering \includegraphics[width=1\linewidth,]{figs11/shiny_web} 

}

\caption{The \textit{shiny} apps website \url{https://www.shinyapps.io/}.}\label{fig:shinyweb}
\end{figure}

My personal shinyapps.io account is shown in Fig \ref{fig:shinyweb1}.

\begin{figure}[H]

{\centering \includegraphics[width=1\linewidth,]{figs11/shiny_web1} 

}

\caption{My personal \textit{shiny} apps website, with three apps.}\label{fig:shinyweb1}
\end{figure}

The account houses links for some apps summarizing the \href{https://ahoken.shinyapps.io/Examples_of_Interactives_v1/}{green house gas data}, and the \href{https://ahoken.shinyapps.io/seeHW/}{Hardy Weinberg equilibrium}.

\section{Comparison of GUI-generating Approaches}\label{comparison-of-gui-generating-approaches}

Three \textbf{R} GUI building approaches were described in this chapter. The package \emph{tcltk} uses the Tcl/Tk GUI building tools alongside the native windowing capacities of Windows, Unix-like and Mac operating systems. The \emph{plotly} and \emph{shiny} libraries render GUIs under an HTML framework. A final comparative summary of the three approaches is given in Table \ref{tab:guimethods}.

\begin{landscape}

\begin{table}[H]
\caption{Comparison of the three approaches for GUI generation in \textbf{R} introduced in this chapter.}
\begin{center}
\renewcommand{\arraystretch}{1.1}
\begin{tabular}{l|l|l|l} \hline
          &  Mechanics   &  \multicolumn{1}{c}{Strengths}   &  \multicolumn{1}{c}{Weaknesses}   \\ \hline
\textit{tcltk}  &    \begin{tabular}[c]{@{}l@{}} Package provides binding for \\ Tcl/Tk GUI building tools. \end{tabular}        &  \begin{tabular}[c]{@{}l@{}} 1) Direct interfacing with \textbf{R} \\ 2) Excellent GUI reactivity \\ 3) Wide range of widgets \end{tabular}          &      \begin{tabular}[c]{@{}l@{}} 1) Limited to \textbf{R} environment \\ 2) GUIs may have poor aesthetics \\ 3) Awkward coding frameworks \\ 4) Poor support online or otherwise  \end{tabular}   \\

\textit{plotly}    &  \begin{tabular}[c]{@{}l@{}}Package provides language \\interfacing from \textbf{R} to JSON \\to HTML. \end{tabular}        &   \begin{tabular}[c]{@{}l@{}} 1) After generation, does not require \textbf{R} \\ 2) Some built-in ggplot compatibility \end{tabular}         &    \begin{tabular}[c]{@{}l@{}} 1) GUI capabilities limited\\ to plot interactives   \end{tabular}      \\

\textit{shiny} &  \begin{tabular}[c]{@{}l@{}}Language interfacing from \textbf{R} \\to JSON to HTML. Maintains \\connection to \textbf{R} environment  \end{tabular} &  \begin{tabular}[c]{@{}l@{}} 1) Good support online and \\otherwise. \\ 2) High level of RStudio compatibility. \\ 3) Potentially aesthetic GUIs.\\ 4) Straightforward coding \\ 5) Wide range of widgets\end{tabular}         &    \begin{tabular}[c]{@{}l@{}} 1) Requires direct connection to an \\  \textbf{R} session, or server connection \\ to an \textbf{R} environment \\
2) Potentially poor reactivity  \end{tabular}  \\
\hline
\end{tabular}
\label{tab:guimethods}
\end{center}
\end{table}

\end{landscape}

\section*{Exercises}\label{exercises-10}
\addcontentsline{toc}{section}{Exercises}

\begin{enumerate}
\def\labelenumi{\arabic{enumi}.}
\tightlist
\item
  Make a \emph{tcltk} GUI that solves and reports the solutions to differential equations.
  \vspace{0.1in}
\item
  Make a \texttt{plotly} graph of any \texttt{gglot2} graph using \texttt{ggplotly}.
  \vspace{0.1in}
\item
  Make a \texttt{shiny} app to greet someone. Hint: place the two code chunks below in the \texttt{ui} and the \texttt{server} function, respectively.
  \vspace{0.1in}
\end{enumerate}

\begin{Shaded}
\begin{Highlighting}[]
\FunctionTok{textInput}\NormalTok{(}\StringTok{"name"}\NormalTok{, }\StringTok{"What\textquotesingle{}s your name?"}\NormalTok{)}
\end{Highlighting}
\end{Shaded}

\vspace{0.1in}

\begin{Shaded}
\begin{Highlighting}[]
\NormalTok{output}\SpecialCharTok{$}\NormalTok{greeting }\OtherTok{\textless{}{-}} \FunctionTok{renderText}\NormalTok{(\{}\FunctionTok{paste0}\NormalTok{(}\StringTok{"Hello "}\NormalTok{, input}\SpecialCharTok{$}\NormalTok{name)\})}
\end{Highlighting}
\end{Shaded}

~~~~~~Make the app interactive inside an \textbf{R} Markdown rendered document.\\
\strut ~~~~~~Along with the code, include a snapshot of the app in action.

\hfill\break
\vspace{0.1in}
4. Your friend has designed an app that solves the exponential growth function for a population with an initial population size of 10, and an intrinsic growth rate of 2, for times, \(t\), from 1 to 50:
\[f(t) = 10 \times \exp(2 \times t).\]
\vspace{0.1in}

\begin{Shaded}
\begin{Highlighting}[]
\NormalTok{ui }\OtherTok{\textless{}{-}} \FunctionTok{fluidPage}\NormalTok{(}
  \FunctionTok{sliderInput}\NormalTok{(}\StringTok{"t"}\NormalTok{, }\AttributeTok{label =} \StringTok{"If t is"}\NormalTok{, }\AttributeTok{min =} \DecValTok{1}\NormalTok{, }\AttributeTok{max =} \DecValTok{50}\NormalTok{, }\AttributeTok{value =} \DecValTok{30}\NormalTok{),}
  \StringTok{"then the population size is"}\NormalTok{,}
  \FunctionTok{textOutput}\NormalTok{(}\StringTok{"exp.growth"}\NormalTok{)}
\NormalTok{)}
\NormalTok{server }\OtherTok{\textless{}{-}} \ControlFlowTok{function}\NormalTok{(input, output, session) \{}
\NormalTok{  output}\SpecialCharTok{$}\NormalTok{exp.growth }\OtherTok{\textless{}{-}} \FunctionTok{renderText}\NormalTok{(\{}\DecValTok{10} \SpecialCharTok{*} \FunctionTok{exp}\NormalTok{(}\DecValTok{2} \SpecialCharTok{*}\NormalTok{ t)\})}
\NormalTok{\}}
\FunctionTok{shinyApp}\NormalTok{(ui, server)}
\end{Highlighting}
\end{Shaded}

~~~~~~Does the function generate an error? Why?\\
\strut ~~~~~~Fix the code and provide a snapshot of the app in action.

\chapter{R and Your Computer}\label{ch12}

\begin{quote}
\emph{``Those who can imagine anything, can create the impossible.''}
\end{quote}

- \textbf{Alan Turing}, \emph{(1912--1954)}

\section{How Do Computers Work?}\label{how-do-computers-work}

To better understand \textbf{R} we need to understand the underlying constraints of computer systems we use to run \textbf{R}.
Computers accept data, process data, produce output, and store processed results. This is generally accomplished through through the generation, integration and storage of electrical signals at microscopic scales. A list of (current but often changing) computer hardware terms are given below.

\begin{itemize}
\item
  \emph{Power supply}: Converts alternating current (AC) electric power to low-voltage direct current (DC) power.
\item
  \emph{Motherboard}: A circuit board connecting computer components including the CPU, RAM and memory disk drives.
\item
  \emph{Central Processing Unit} (\emph{CPU}): A microprocessor that performs most of the calculations that allow a computer to function. Specifically, the CPU processes program instructions and sends the results on for further processing and execution by other computer components.
\item
  \emph{Graphics Processing Unit} (\emph{GPU}): An electronic circuit originally designed to accelerate computer graphics, but now widely applied for non-graphic, but highly parallel, calculations.
\item
  \emph{Chipset}: Mediates communication between the CPU and the other computer components.
\item
  \emph{Random Access Memory} (\emph{RAM}): Stores code and data in \emph{primary memory} to allow it to be directly accessed by the CPU. RAM is volatile memory which requires power to retain stored information. Thus, when power is interrupted, RAM data can be lost. RAM types include \emph{dynamic random access memory} (\emph{DRAM}) and \emph{static random-access memory} (\emph{SRAM}). DRAM constitutes modern computer \emph{main memory} and \emph{graphics cards}. DRAM typically takes the form of an \emph{integrated circuit} chip that can consist of up to billions of memory cells, with each cell consisting of a pairing of a tiny capacitor\footnote{A capacitor stores electrical energy by ``accumulating electric charges on two closely spaced surfaces that are insulated from each other'' \citep{wikicapacitor}.} and transistor\footnote{A transistor
    is a semiconductor device (a material with an intermediate electrical conductivity, e.g., silicon) that is used to amplify or switch electrical signals and power'' \citep{wikitransistor}.}, allowing each cell to store or read or write one bit of information (Fig \ref{fig:dram}). SRAM uses latching circuitry that holds data permanently in the presence of power, whereas DRAM decays in seconds and must be periodically refreshed. Memory access via SRAM is much faster than DRAM, although DRAM circuits are much less expensive to construct.
\end{itemize}

\begin{figure}[H]

{\centering \includegraphics[width=0.8\linewidth,]{figs12/DRAM} 

}

\caption{Sixteen DRAM memory cells each representing a bit of information for computational storage, reading, or writing. To read the binary word line \texttt{0101...} in row two of the circuit, binary signals are sent down the bit lines to sense amplifiers.}\label{fig:dram}
\end{figure}

\begin{itemize}
\item
  \emph{Disk drives}: including CD, DVD, \emph{hard disk} (\emph{HDD}), and \emph{solid state disk} (\emph{SSD}) are used for secondary memory. That is, memory that is not directly accessible from the CPU. Secondary memory can be accessed or retrieved even if the computer is off. Secondary memory is also non-volatile and thus can be used to store data and programs for extended periods. User files and software (like \textbf{R}) are generally stored on HDDs or SSDs. \emph{Flash memory}, which uses modified \emph{metal--oxide--semiconductor field-effect transistors} (\emph{MOSFETs}), is typically used on USB and SSD devices to provide secondary memory that can be erased and reprogrammed. Flash memory can also be used in RAM applications.
\item
  \emph{Read-Only Memory} (\emph{ROM}): Stores the BIOS (see below) that runs when the computer is powered on (cold boot) or restarted (warm boot or \emph{reboot}). ROM constitutes primary memory.
\item
  \emph{Basic Input Output System} (\emph{BIOS}): Basic boot (startup) and power management firmware (software that provides low level control for computer hardware). Newer motherboards use the so-called \emph{Unified Extensible Firmware Interface} (\emph{UEFI}) to address BIOS limitations, including restrictive 16 bit addresses.
\item
  \emph{Video card}: Processes computer graphics.
\end{itemize}

\index[terms]{Basic input output system (BIOS)}
\index[terms]{Unified extensible firmware interface (UEFI)}
\index[terms]{Memory! read-only memory (ROM)}
\index[terms]{Memory! random access memory (RAM)}
\index[terms]{Memory! primary memory}
\index[terms]{Memory! secondary memory}
\index[terms]{Memory! disk drives}
\index[terms]{Chipset}
\index[terms]{Transistor}
\index[terms]{Capacitor}
\index[terms]{Boot (computer startup)}
\index[terms]{Video card}
\index[terms]{Central processing unit (CPU)}
\index[terms]{Graphics processing unit (GPU)}

\section{Base-2 and Base-10}\label{base-2-and-base-10}

To understand computer processes, it is important to distinguish base-2 (\emph{binary})\index[terms]{Binary (numerical system)} and base-10 (\emph{decimal}) numerical systems\index[terms]{Decimal (numerical system)}. In both cases, the \emph{base} refers to the number of unique digits. Thus, base-2 systems can have two unique digits, commonly \(0\) and \(1\), and the base-10 system has 10 unique digits: \(0,1,2,3,4,5,6,7,8,9\). The latter --more widely used system-- probably arose because we have ten fingers for counting\footnote{A base-20 system used by Pre-Columbian Mesoamerican cultures probably arose because we have twenty fingers and toes \citep{wikidec}.}. A \emph{radix}\index[terms]{Radix} (commonly a decimal symbol) is used to distinguish the \emph{integer part} of a number from its \emph{fractional part} (Fig \ref{fig:bin}). The radix convention is used by both base-2 and base-10 systems. For example, the decimal number \(4\frac{3}{4}\), has integer component \(4\) and fractional component \(\frac{3}{4}\), can be expressed as \(4.75\). The binary equivalent of \(4\frac{3}{4}\) is 100.110.

Traditionally, a base-10 number could only be expressed as a rational fraction whose denominator was a power of ten (Fig \ref{fig:bin}). However, the decimal system can be extended to any real number, by allowing a conceptual infinite sequence of digits following the radix \citep{wikidec}.

\begin{figure}[H]

{\centering \includegraphics[width=1\linewidth,]{figs12/Base10} 

}

\caption{A decimal place value chart. A radix (decimal) is placed between the ones and thenths columns to distinguish decimal number components greater than one (to the left), and components less than one but greater than zero (to the right).}\label{fig:bin}
\end{figure}

\section{Bits and Bytes}\label{bits-and-bytes}

Computers are designed around bits and bytes. A \emph{bit} is a binary (base-2) unit of digital information. Specifically, a bit will represent a \texttt{0} or a \texttt{1}. This convention occurs because computer systems typically use electronic circuits that exist in only one of two states, on or off. For instance, in DRAM memory cells (Fig \ref{fig:dram}) convert electrical low and high voltages into binary \texttt{0} and \texttt{1} responses, respectively, for the purpose of reading, writing, and storing data. Although bits are used by all software in all conventional computer operating systems, these mechanisms are easily revealed in \textbf{R}\footnote{Non-binary operating systems are rarely implemented because: 1) they are less efficient, and 2) currently no IEEE standards have been specified\index[terms]{IEEE (Institute of Electrical and Electronics Engineers)}. In order of increasing precision and decreasing efficiency, alternative systems include: Limited-Precision Decimal, Arbitrary-Precision Decimal, and Symbolic Calculation systems.}.

Bits are generally counted in units of \emph{bytes}. For somewhat arbitrary historical reasons, a byte equals eight bits. Two major systems exist for counting bytes. The \emph{decimal method}, the most common system, uses powers of 10, allowing implementation of SI prefixes (i.e., kilo = \(10^3 = 1000\), mega = \(10^6\) = \(1000^2\), giga = \(10^9\) = \(1000^3\), etc.) (Table \ref{tab:mem}). A computer hard drive with 1 gigabyte (1 billion bytes) of memory will have \(1 \times 10^9\) bytes = \(8 \times 10^9\) bits of memory. The \emph{binary system}, used frequently by Windows to describe RAM, defines byte units in multiples of 1024.

\index[terms]{Bit (unit of digital information)}
\index[terms]{Byte (unit of digital information)}
\index[terms]{Decimal method (digital storage system)}
\index[terms]{Binary method (digital storage system)}

\begin{table}[h]
\caption{Frequently used byte units.}
\begin{center}

\renewcommand{\arraystretch}{1.5}
\label{tab:mem}
\begin{tabular}{ll|ll}
\hline
\multicolumn{2}{c}{Decimal} & \multicolumn{2}{c}{Binary} \\
\hline
Bytes    & Name             & Bytes   & Name (IEC)       \\ \hline
$1000$     & kB (kilobyte)  & $1024$  & KiB (kibibyte)   \\
$1000^2$   & MB (megabyte)    & $1024^2$   & MiB (mebibyte)   \\
$1000^3$   & GB (gigabyte)    & $1024^3$   & GiB (gibibyte)   \\
$1000^4$    & TB (terabyte)    & $1024^4$   & TiB (tebibyte)   \\
$1000^5$    & PB (petabyte)    & $1024^5$  & PeB (pebibyte)  \\ \hline
\end{tabular}
\end{center}
\end{table}

With a single bit we can describe only \(2^1 = 2\) distinct digital objects. These will be an entity represented by a 0, and an entity represented by a 1. It follows that \(2^2 = 4\) distinct objects can be described with two bits, \(2^3 = 8\) entities can be described with three bits, and so on\footnote{For instance, images often contain eight bit (one byte) variables describing the colors red, green, and blue. Thus, the color red would be a number between 0 and 255 (i.e., red could have \(2^8 = 256\) distinct values). Given that the colors blue and green were also eight bit, there would be \(256^3 = 16,777,216\) color possibilities (combinations) for any pixel in an image.}.

\section{Decimal to Binary}\label{decimal-to-binary}

We count to ten in binary using: \texttt{0} = 0, \texttt{1} = 1, \texttt{10} = 2, \texttt{11} = 3, \texttt{100} = 4, \texttt{101} = 5, \texttt{110} = 6, \texttt{111} = 7, \texttt{1000} = 8, \texttt{1001} = 9, and \texttt{1010} = 10. Thus, we require four bits to count to ten. Note that the binary sequences for all positive integers greater than or equal to one, start with one.

\subsection{Positive Integers}\label{positive-integers}

We can obtain the binary expression of the integer part of any decimal number by iteratively performing \emph{integer division} by two, and cataloging each \emph{modulus}. The iterations are stopped when a quotient of one is reached. The modulus sequence is read from right to left (backwards). If the whole number of interest is greater than one (i.e., the whole number is not 0 or 1) we place a one in front of the reversed sequence, because all binary sequences for numbers greater than or equal to one must start with one.

\begin{example}
\protect\hypertarget{exm:exm-c12ex1}{}\label{exm:exm-c12ex1}

\(\text{}\)\\
Consider the number 23:

\[
\begin{aligned}
&\text{Modulus (remainder)     }& 1&           & 1&          & 1&         & 0&\\
&\text{Integer Quotient        }            & 23/2 = 11& & 11/2 = 5& & 5/2 = 2& & 2/2 = 1&
\end{aligned}
\]

The reversed sequence is \texttt{0111}. We place a one in front to get the binary representation for 23: \texttt{10111}. The function \texttt{dec2bin()} from \emph{asbio} does the work for us:
\index[R]{\texttt{asbio::dec2bin()}}
\vspace{0.1in}

\begin{Shaded}
\begin{Highlighting}[]
\FunctionTok{library}\NormalTok{(asbio)}
\FunctionTok{dec2bin}\NormalTok{(}\DecValTok{23}\NormalTok{)}
\end{Highlighting}
\end{Shaded}

\begin{verbatim}
[1] 10111
\end{verbatim}

\end{example}

\(\blacksquare\)

\subsection{Positive Fractions}\label{sec:pf}

The fractional part of a decimal number can be converted to binary in a similar fashion.

\begin{itemize}
\tightlist
\item
  To identify the fractional expression as a non integer, start the binary sequence with \texttt{0.} (a zero followed by a decimal symbol).
\item
  Double the fraction to be converted, and record a \texttt{1} if the product is \(\ge 1\), and \texttt{0} otherwise.
\item
  For subsequent binary digits, multiply two by the fractional part of the previous multiplication. If the product is \(\ge 1\), record a \texttt{1}. If not, record a \texttt{0}.
\end{itemize}

\begin{example}
\protect\hypertarget{exm:exm-c12ex2}{}\label{exm:exm-c12ex2}\(\text{}\)\\
Consider the fraction \(\frac{1}{4}\). We have:

\[
\begin{aligned}
&\text{Binary outcome}& 0              & & 1          && 0         && 0\\
&\text{Product}& 1/4 \times 2 = 1/2 < 1 & & 1/2 \times 2 = 1\ge 1 && 0 \times 2 = 0 < 1&& 0 \times 2 = 0 < 1 \end{aligned}
\]
\end{example}

\(\blacksquare\)

We have a clear repeating sequence of zeroes, due to a product of two in the second step. This allows us to stop the growth of the binary expression. For fractions, the binary sequence is read conventionally, from left to right. Thus, the binary expression for \(\frac{1}{4}\) is \texttt{0.01} .
\vspace{0.1in}

\begin{Shaded}
\begin{Highlighting}[]
\FunctionTok{dec2bin}\NormalTok{(}\FloatTok{0.25}\NormalTok{)}
\end{Highlighting}
\end{Shaded}

\begin{verbatim}
[1] 0.01
\end{verbatim}

\section{Binary to Decimal}\label{binary-to-decimal}

The addition of a binary digit (i.e., a bit) represents a doubling of information storage. For instance, as we increase from two bits to three bits, the number of describable integers increases from four (integers 0 to 3) to eight (integers 0 to 7). As a result we say that the rightmost digit in a set of binary digits represents \(2^0\), the next represents \(2^1\), then \(2^2\), and so on. This can be defined with an equation based on Horner's method \citep{horner1815} that allows conversion of binary to decimal numbers:

\begin{equation}
\sum_{\kappa=\min(\kappa)}^{\max(\kappa)}\alpha\beta^\kappa
\label{eq:bin}
\end{equation}

where \(\alpha\) is a quantity known as the \emph{significand}\index[terms]{Significand}, that contains bit (0, 1) outcomes. For the purpose of binary expressions, the modifying base, \(\beta\), is 2. The term \(\kappa\) is called (appropriately) \emph{the exponent}\index[terms]{Exponent (binary expression equation)}.

The maximum and minimum values of \(\kappa\) are determined by counting the number of placeholder digits in the binary expression represented by the significand, with respect to a binary radix point (Fig \ref{fig:bin2dec}). Note that counting starts with respect to 0 (the first digit to the left of the radix) for both positive (bits to the left of the radix) and negative (bits to the right of the radix) values of the exponent, \(\kappa\). The radix reference has prompted this method to be called \emph{floating point arithmetic}.\index[terms]{Floating point arithmetic}

\begin{figure}[H]

{\centering \includegraphics[width=1\linewidth,]{figs12/Fig12_1} 

}

\caption{Conceptualization of binary to decimal conversion, as given in Eq \ref{eq:bin}.}\label{fig:bin2dec}
\end{figure}

\subsection{Positive Integers}\label{positive-integers-1}

For positive integers the entirety of the corresponding binomial expression will be to the left of the radix point (Fig \ref{fig:bin2dec}). Thus, the minimum value of \(\kappa\) will be zero and the maximum value of \(\kappa\) will be the number of digits (bits) in the binary expression, minus one.

Equation \eqref{eq:bin} represents a \emph{dot product}. That is, the equation is a sum of the element-wise multiplication of two vectors. For instance, to find the integers represented by a single binary bit, we multiply the binary digit value, \texttt{0} or \texttt{1}, by the power of two it represents. Because the single bit signature would occur at the right-most address to the left of the radix, the value of exponent would be 0 (Fig \ref{fig:bin2dec}). That is, \(\min(\kappa)\) = \(\max(\kappa)\) = 0 in Eq. \eqref{eq:bin}.\\

If the single bit equals \texttt{0} we have:

\[0 \times 2^0 = 0,\]

and if the single bit equals \texttt{1} we have:

\[1 \times 2^0 = 1.\]

Accordingly, to find the decimal version of a set of binary values, we take the sum of the products of the binary digits and their corresponding (decreasing) powers of base 2.

\begin{example}
\protect\hypertarget{exm:exm-c12ex3}{}\label{exm:exm-c12ex3}\(\text{}\)\\
For example, the binary number \texttt{010101} equals:

\[\begin{aligned}
(0 \times 2^5) + (1 \times 2^4) + (0 \times 2^3) + (1 \times 2^2) + (0 \times 2^1) + (1 \times 2^0) &=
0 + 16 + 0 + 4 + 0 +1 \\&= 21.
\end{aligned}
\]

The function \texttt{bin2dec} in \emph{asbio} does the calculation for us.
\vspace{0.1in}

\begin{Shaded}
\begin{Highlighting}[]
\FunctionTok{bin2dec}\NormalTok{(}\DecValTok{010101}\NormalTok{)}
\end{Highlighting}
\end{Shaded}

\begin{verbatim}
[1] 21
\end{verbatim}

\index[R]{\texttt{asbio::bin2dec()}}
\end{example}

\(\blacksquare\)

\subsection{Positive Fractions}\label{positive-fractions}

For positive fractions, values of the \(\kappa\) exponent will decrease by minus one as bits increase by one (Fig \ref{fig:bin2dec}). Thus, to obtain decimal fractions from binary fractions we multiply a bit's binary value by decreasing negative powers of base two, starting at 0, and find the sum, as shown in Eq \eqref{eq:bin}.

\begin{example}
\protect\hypertarget{exm:exm-c12ex4}{}\label{exm:exm-c12ex4}

\(\text{}\)\\
For example, the binary value \texttt{0.01} equals:
\[(0 \times 2^0) + (0 \times 2^{-1}) + (1 \times 2^{-2}) = 0.25\]
\vspace{0.1in}

\begin{Shaded}
\begin{Highlighting}[]
\FunctionTok{bin2dec}\NormalTok{(}\FloatTok{0.01}\NormalTok{)}
\end{Highlighting}
\end{Shaded}

\begin{verbatim}
[1] 0.25
\end{verbatim}

\end{example}

\(\blacksquare\)

\subsubsection{Terminality}\label{terminality}

Most decimal fractions will not have a clear \emph{terminal} binary sequence. That is, a binary representation of a decimal fraction with a finite number of digits will not exist. This results in mere binary approximations of decimal numbers \citep{goldberg1991}. For instance, the 10 bit binary expression of \(\frac{1}{10}\) is
\vspace{0.1in}

\begin{Shaded}
\begin{Highlighting}[]
\FunctionTok{dec2bin}\NormalTok{(}\FloatTok{0.1}\NormalTok{)}
\end{Highlighting}
\end{Shaded}

\begin{verbatim}
[1] 0.0001100110
\end{verbatim}

But translating this back to decimal we find:
\vspace{0.1in}

\begin{Shaded}
\begin{Highlighting}[]
\FunctionTok{bin2dec}\NormalTok{(}\FloatTok{0.0001100110}\NormalTok{)}
\end{Highlighting}
\end{Shaded}

\begin{verbatim}
[1] 0.0996
\end{verbatim}

Increasing the number of bits in the binary expression increases precision,
\vspace{0.1in}

\begin{Shaded}
\begin{Highlighting}[]
\FunctionTok{dec2bin}\NormalTok{(}\FloatTok{0.1}\NormalTok{, }\AttributeTok{max.bits =} \DecValTok{14}\NormalTok{)}
\end{Highlighting}
\end{Shaded}

\begin{verbatim}
[1] 0.00011001100110
\end{verbatim}

but the decimal approximation remains imperfect.
\vspace{0.1in}

\begin{Shaded}
\begin{Highlighting}[]
\FunctionTok{options}\NormalTok{(}\AttributeTok{digits =} \DecValTok{20}\NormalTok{)}
\FunctionTok{bin2dec}\NormalTok{(}\FloatTok{0.00011001100110}\NormalTok{)}
\end{Highlighting}
\end{Shaded}

\begin{verbatim}
[1] 0.10000000000000000555
\end{verbatim}

Note that the imperfect conversion above gives the actual result of the division \(\frac{1}{10}\) for all software on all current conventional computers (not just \textbf{R})!
\vspace{0.1in}

\begin{Shaded}
\begin{Highlighting}[]
\FunctionTok{options}\NormalTok{(}\AttributeTok{digits =} \DecValTok{20}\NormalTok{)}
\DecValTok{1}\SpecialCharTok{/}\DecValTok{10}
\end{Highlighting}
\end{Shaded}

\begin{verbatim}
[1] 0.10000000000000000555
\end{verbatim}

\vspace{0.1in}

It may seem surprising that rational fractions like \(\frac{1}{10}\) may have non-terminating binary expressions. \emph{Terminality}, however, will only occur for a decimal fraction if a product of 2 results from the successive multiplication steps described in Section \ref{sec:pf}. This product does not occur for \(\frac{1}{10}\).
\index[terms]{Terminality (of binary expressions)}

Lack of terminality for binary expressions prompts the need for quantifying imprecision in computers systems. This can be obtained from Eq \eqref{eq:bin}. In particular, the exponent in Eq \eqref{eq:bin} determines minimum and maximum possible encoded numeric values, and the number of digits in the significand determines numeric precision. Indeed, by changing the base from 2 to 10, Eq \eqref{eq:bin} can be used to quantify the precision of binary and decimal numbers.

\begin{example}
\protect\hypertarget{exm:exm-c12ex5}{}\label{exm:exm-c12ex5}\(\text{}\)\\
For instance, the decimal number \(1,245.42\) has the \emph{scientific notation}: \(1.24542 \times 10^3\). The expression has a the precision of six digits, because under Eq \eqref{eq:bin} the significand has six digits. Note that applying these digits in Eq. \eqref{eq:bin} we have:

\[
\begin{aligned}
1 \times 10^3 & + & 2 \times 10^2 & + & 4 \times 10^1 & + & 5 \times 10^0 & + & 4 \times 10^{-1} & + &2 \times 10^{-2} & = \\
         1000 & + & 200 & + & 40 & + & 5 & + & 0.4 & + &0.02 & = 1245.42
\end{aligned}
\]
\end{example}

\(\blacksquare\)

\section{Double Precision}\label{double-precision}

\index[terms]{Double precision}
In most programs, on most workstations, the results of computations are stored as 32 bits (i.e., 4 bytes) or as 64 bits (8 bytes) of information. The 64 bit double precision format allows high precision representations of both positive and negative integers and their fractional components. Under this framework, one bit is allocated to the sign of the stored item, 53 bits are assigned to the significand, and 11 bits are given to the exponent (Fig \ref{fig:sfour}).

\begin{figure}[H]

{\centering \includegraphics[width=1\linewidth,]{figs12/sixtyfourbit} 

}

\caption{The IEEE 754 double-precision binary floating-point format Figure taken from \url{https://commons.wikimedia.org/w/index.php?curid=3595583}.}\label{fig:sfour}
\end{figure}

This can be represented mathematically as a more complex form of Eq \eqref{eq:bin}:

\begin{equation}
(-1)^{\text{sign}}\left(1 + \sum_{i=1}^{52} b_{52-i} 2^{-i} \right)\times 2^{e-1023}
\label{eq:dp}
\end{equation}

which gives the assumed numeric value for a 64-bit double-precision datum with \emph{exponent bias}\index[terms]{Double precision!exponent bias}.\textbackslash{}

\begin{example}
\protect\hypertarget{exm:exm-c12ex6}{}\label{exm:exm-c12ex6}\(\text{}\)\\
The function \texttt{bit64()} below is taken from the Examples of the documentation for the \emph{base} function \texttt{numToBits()}, which converts digital numbers to 64 bits. The function distinguishes:

\begin{itemize}
\tightlist
\item
  The single bit giving the sign of the number (\texttt{0} = positive, \texttt{1} = negative).
\item
  The 11 bit exponent.
\item
  A 52 bit significand (without the implicit leading \texttt{1}).
\end{itemize}

\index[R]{\texttt{numToBits()}}
\vspace{0.1in}

\begin{Shaded}
\begin{Highlighting}[]
\NormalTok{bit64 }\OtherTok{\textless{}{-}} \ControlFlowTok{function}\NormalTok{(x)}
  \FunctionTok{noquote}\NormalTok{(}\FunctionTok{vapply}\NormalTok{(}\FunctionTok{as.double}\NormalTok{(x),}
                 \ControlFlowTok{function}\NormalTok{(x) \{}
\NormalTok{                   b }\OtherTok{\textless{}{-}} \FunctionTok{substr}\NormalTok{(}\FunctionTok{as.character}\NormalTok{(}\FunctionTok{rev}\NormalTok{(}\FunctionTok{numToBits}\NormalTok{(x))), }\DecValTok{2}\DataTypeTok{L}\NormalTok{, }\DecValTok{2}\DataTypeTok{L}\NormalTok{)}
                   \FunctionTok{paste0}\NormalTok{(}\FunctionTok{c}\NormalTok{(b[}\DecValTok{1}\DataTypeTok{L}\NormalTok{], }\StringTok{" "}\NormalTok{, b[}\DecValTok{2}\SpecialCharTok{:}\DecValTok{12}\NormalTok{], }\StringTok{" | "}\NormalTok{, b[}\DecValTok{13}\SpecialCharTok{:}\DecValTok{64}\NormalTok{]), }\AttributeTok{collapse =} \StringTok{""}\NormalTok{)}
\NormalTok{                   \}, }\StringTok{""}\NormalTok{)}
\NormalTok{  )}
\end{Highlighting}
\end{Shaded}

Here is the double precision representation of \(\frac{1}{3}\)

\footnotesize
\vspace{0.1in}

\begin{Shaded}
\begin{Highlighting}[]
\FunctionTok{bit64}\NormalTok{(}\DecValTok{1}\SpecialCharTok{/}\DecValTok{3}\NormalTok{)}
\end{Highlighting}
\end{Shaded}

\begin{verbatim}
[1] 0 01111111101 | 0101010101010101010101010101010101010101010101010101
\end{verbatim}

\normalsize

We see this follows the form of Eq \eqref{eq:dp}. The exponent \texttt{01111111101} represents the decimal number 1021:
\vspace{0.1in}

\begin{Shaded}
\begin{Highlighting}[]
\FunctionTok{bin2dec}\NormalTok{(}\DecValTok{01111111101}\NormalTok{)}
\end{Highlighting}
\end{Shaded}

\begin{verbatim}
[1] 1021
\end{verbatim}

And one plus the dot product of the significand and base-2 raised to the sequence -1 to -52, multiplied by \(2^{1021 - 1023}\), is:

\footnotesize
\vspace{0.1in}

\begin{Shaded}
\begin{Highlighting}[]
\NormalTok{sigd }\OtherTok{\textless{}{-}} \FunctionTok{strsplit}\NormalTok{(}\StringTok{"0101010101010101010101010101010101010101010101010101"}\NormalTok{, }\ConstantTok{NULL}\NormalTok{)}
\NormalTok{sigd }\OtherTok{\textless{}{-}} \FunctionTok{as.numeric}\NormalTok{(}\FunctionTok{unlist}\NormalTok{(sigd))}
\NormalTok{base2 }\OtherTok{\textless{}{-}} \DecValTok{2}\SpecialCharTok{\^{}}\NormalTok{(}\SpecialCharTok{{-}}\DecValTok{1}\SpecialCharTok{:{-}}\DecValTok{52}\NormalTok{)}
\NormalTok{(}\DecValTok{1} \SpecialCharTok{+} \FunctionTok{sum}\NormalTok{(sigd }\SpecialCharTok{*}\NormalTok{ base2)) }\SpecialCharTok{*} \DecValTok{2}\SpecialCharTok{\^{}{-}}\DecValTok{2}
\end{Highlighting}
\end{Shaded}

\begin{verbatim}
[1] 0.3333
\end{verbatim}

\normalsize

That is, we have:

\[
\begin{aligned}
value &= (-1)^{\text{sign}}\left(1 + \sum_{i=1}^{52} b_{52-i} 2^{-i} \right)\times 2^{e-1023}\\
&= -1^{0} \times (1 + 2^{-2} + 2^{-4} + \dots + 2^{-52}) \times 2^{1021-1023}\\
&\approx 1.33\bar{3} \times 2^{-2}\\
&\approx \frac{1}{3}
\end{aligned}
\]
\end{example}

\(\blacksquare\)

The 11 bit width of the double precision exponent allows the expression of numbers between \(10^{-308}\) and \(10^{308}\), with full 15--17 decimal digits precision. This is clearly demonstrated in \textbf{R}. Specifically, imprecision problems with non-terminal fractions become evident for decimal numbers with greater than 16 displayed digital digits.
\vspace{0.1in}

\begin{Shaded}
\begin{Highlighting}[]
\FunctionTok{options}\NormalTok{(}\AttributeTok{digits =} \DecValTok{18}\NormalTok{)}
\DecValTok{1}\SpecialCharTok{/}\DecValTok{3}
\end{Highlighting}
\end{Shaded}

\begin{verbatim}
[1] 0.333333333333333315
\end{verbatim}

\vspace{0.1in}

Additionally, the current upper numerical limit in \textbf{R} (ver 4.3.2) is somewhere between:
\vspace{0.1in}

\begin{Shaded}
\begin{Highlighting}[]
\FloatTok{1.8} \SpecialCharTok{*} \DecValTok{10}\SpecialCharTok{\^{}}\DecValTok{307}
\end{Highlighting}
\end{Shaded}

\begin{verbatim}
[1] 1.8e+307
\end{verbatim}

and
\vspace{0.1in}

\begin{Shaded}
\begin{Highlighting}[]
\FloatTok{1.8} \SpecialCharTok{*} \DecValTok{10}\SpecialCharTok{\^{}}\DecValTok{308}
\end{Highlighting}
\end{Shaded}

\begin{verbatim}
[1] Inf
\end{verbatim}

The \emph{subnormal representation}\footnote{*Subnormal numbers* fill the underflow gap around zero in floating-point arithmetic [@wikisubnormal]. For subnormal numbers, $e$ in Eq:\\eqref{eq:dp} is taken to be zero. *Underflow* occurs when the result of a calculation is a number with greater precision than the computer can actually represent in its CPU memory [@wikiunderflow].
\index[terms]{Subnormal number}\index[terms]{Underflow (arithmetic)}} compromises precision, but allows allows fractional representations approaching \(5 \times 10^{-324}\). This approach is used by \textbf{R}, whose smallest represented fraction is between:
\vspace{0.1in}

\begin{Shaded}
\begin{Highlighting}[]
\FloatTok{5.0} \SpecialCharTok{*} \DecValTok{10}\SpecialCharTok{\^{}{-}}\DecValTok{323}
\end{Highlighting}
\end{Shaded}

\begin{verbatim}
[1] 4.94065645841246544e-323
\end{verbatim}

and
\vspace{0.1in}

\begin{Shaded}
\begin{Highlighting}[]
\FloatTok{5.0} \SpecialCharTok{*} \DecValTok{10}\SpecialCharTok{\^{}{-}}\DecValTok{324}
\end{Highlighting}
\end{Shaded}

\begin{verbatim}
[1] 0
\end{verbatim}

Binary fractional numbers are expressed with respect to a decimal, and the number of digits will (often) be dictated by the significand. Given 13 bits we have the following binary translations to decimal numbers: \texttt{1} = 1/1, \texttt{0.1} = 1/2, \texttt{0.01010101\ldots} = 1/3, \texttt{0.01} = 1/4, \texttt{0.00110011} = 1/5, \texttt{0.0010101\ldots} = 1/6, \texttt{0.001001\ldots} = 1/7, \texttt{0.001} = 1/8, \texttt{0.000111000111\ldots} = 1/9, \texttt{0.000110011\ldots} = 1/10.

\section{Binary Characters}\label{binary-characters}

\index[terms]{ASCII}
\index[terms]{Unicode}
\index[terms]{UTF-8}
\index[terms]{UTF-16}
Characters can also be expressed in binary. The American Standard Code for Information Interchange (ASCII) consists of 128 characters, and requires one byte = eight bits\footnote{Originally developed from telegraph code, ASCII has only 128 code points, of which only 95 are printable characters \citep{wikiascii}.}. The newer eight bit Unicode Transformation Format (\href{https://en.wikipedia.org/wiki/UTF-8}{UTF-8}) system --the most widely used Unicode system-- can represent 1,112,064 valid code points, using between 1 to 4 bytes = 8 to 32 bits \citep{wikutf8}. Specifically, from the perspective of the \href{https://en.wikipedia.org/wiki/UTF-16\#U+D800_to_U+DFFF}{UTF-16} system, the UTF-8 system uses portions of seventeen \emph{planes} \footnote{In Unicode, a plane is a group of \(2^{16} = 65,536\) code points. There are 17 planes because UTF-16, can encode \(2^{20}\) code points (16 planes) as pairs of words, plus the so-called Basic Multilingual Plane (UTF-16 plane 0) as a single word.}, each consisting of sixteen bits (and, thus, \(2^{16} = 65,536\) code variants). This results in the quantity:

\[(17 \times 2^{16}) - 2^{11} = 1,112,064\]

The \(2^{11} = 2048\) subtraction acknowledges that there are 2048 technically-invalid Unicode surrogates \citep{wikutf16}. The first 128 UTF-8 characters are the ASCII characters, allowing back-comparability with ASCII.

\begin{example}
\protect\hypertarget{exm:exm-c12ex7}{}\label{exm:exm-c12ex7}\(\text{}\)\\
\textbf{R} uses the UTF-8 system of characters. We can observe the process of binary character assignment using the functions \texttt{as.raw()}, \texttt{rawToChar()}, and \texttt{rawToBits()}. The base type \texttt{raw} (Section \ref{bt}) is intended to hold raw byte information.
Here is a list of the 128 ASCII characters.
\index[terms]{R!base type!raw}
\index[R]{\texttt{as.raw()}}
\index[R]{\texttt{rawToChar()}}
\index[R]{\texttt{rawToBits()}}
\vspace{0.1in}

\begin{Shaded}
\begin{Highlighting}[]
\FunctionTok{rawToChar}\NormalTok{(}\FunctionTok{as.raw}\NormalTok{(}\DecValTok{1}\SpecialCharTok{:}\DecValTok{128}\NormalTok{), }\AttributeTok{multiple =} \ConstantTok{TRUE}\NormalTok{)}
\end{Highlighting}
\end{Shaded}

\begin{verbatim}
  [1] "\001" "\002" "\003" "\004" "\005" "\006" "\a"   "\b"   "\t"    "\n"  
 [11] "\v"   "\f"   "\r"   "\016" "\017" "\020" "\021" "\022" "\023" "\024"
 [21] "\025" "\026" "\027" "\030" "\031" "\032" "\033" "\034" "\035" "\036"
 [31] "\037" " "    "!"    "\""   "#"    "$"    "%"    "&"    "'"    "("   
 [41] ")"    "*"    "+"    ","    "-"    "."    "/"    "0"    "1"    "2"   
 [51] "3"    "4"    "5"    "6"    "7"    "8"    "9"    ":"    ";"    "<"   
 [61] "="    ">"    "?"    "@"    "A"    "B"    "C"    "D"    "E"    "F"   
 [71] "G"    "H"    "I"    "J"    "K"    "L"    "M"    "N"    "O"    "P"   
 [81] "Q"    "R"    "S"    "T"    "U"    "V"    "W"    "X"    "Y"    "Z"   
 [91] "["    "\\"   "]"    "^"    "_"    "`"    "a"    "b"    "c"    "d"   
[101] "e"    "f"    "g"    "h"    "i"    "j"    "k"    "l"    "m"    "n"   
[111] "o"    "p"    "q"    "r"    "s"    "t"    "u"    "v"    "w"    "x"   
[121] "y"    "z"    "{"    "|"    "}"    "~"    "\177" "\200"   
\end{verbatim}

Note that the exclamation point is character number 33. Its 16 bit binary code is:
\vspace{0.1in}

\begin{Shaded}
\begin{Highlighting}[]
\FunctionTok{rawToBits}\NormalTok{(}\FunctionTok{as.raw}\NormalTok{(}\DecValTok{33}\NormalTok{))}
\end{Highlighting}
\end{Shaded}

\begin{verbatim}
[1] 01 00 00 00 00 01 00 00
\end{verbatim}

From the output above, codes 1-31 and 127-128 are not printable characters. Thus, there are only 128 - 33 = 95 printable ASCII characters. Note that codes 7-13 are command characters. For instance, character 10, \texttt{"\textbackslash n"} indicates ``make a new line'\,' within a character string.
\index[terms]{Command character!\texttt{\textbackslash n}}
\end{example}

\(\blacksquare\)

\section{Optimizing R}\label{optimizing-r}

Because attention was given to computational efficiency in several earlier sections in this chapter, here I briefly consider several methods for optimizing \textbf{R}. In particular, I consider the use of \textbf{R}-interfaces, including scripting from command line OS shells to implement high performance computers (HPCs) and parallel computing.

\subsection{Calling Linux-driven HPCs}\label{calling-linux-driven-hpcs}

\index[terms]{High performance computers (HPCs)}
Under construction

\subsection{Parallel Computing}\label{parallel-computing}

\index[terms]{Parellel computing}
Under construction

\section*{Exercises}\label{exercises-11}
\addcontentsline{toc}{section}{Exercises}

\begin{enumerate}
\def\labelenumi{\arabic{enumi}.}
\item
  Define the following terms:

  \begin{enumerate}
  \def\labelenumii{(\alph{enumii})}
  \tightlist
  \item
    Motherboard
  \item
    Central processing unit (CPU)
  \item
    Random access memory
  \item
    Primary memory
  \item
    Secondary memory
  \item
    Volatile memory
  \item
    Non-volatile memory\\
    \strut ~\\
    \vspace{0.1in}
  \end{enumerate}
\item
  What is the level of trustworthy precision (in number of digits) for decimal fractional components in \textbf{R} (and all software that use 64 bit double precision)?
  \vspace{0.1in}
\item
  Obtain the five bit binary sequence for the number 21 by hand. Check your answer using \texttt{dec2bin()}.
  \vspace{0.1in}
\item
  Find the decimal number corresponding to the five bit binary sequence \texttt{11111}. Check your answer using \texttt{bin2dec()}.
\item
  Find the 64 bit expression for the decimal number \(-2\) (minus 2) using the function \texttt{bit64()}, as shown in this chapter. Back-transform this binary representation to the decimal number by hand using Eq. \eqref{eq:dp}. Use \textbf{R} functions like \texttt{strsplit()} \texttt{unlist()}, etc., to help.
\end{enumerate}

  \bibliography{book.bib}

\printindex[terms]
\printindex[R]

\end{document}
